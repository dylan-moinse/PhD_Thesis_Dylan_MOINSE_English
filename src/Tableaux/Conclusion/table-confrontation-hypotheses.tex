% Table comparing research hypotheses

    \begin{table}[h!]
    \centering
    \renewcommand{\arraystretch}{1.5}
    \resizebox{\columnwidth}{!}{
    \begin{tabular}{p{0.01\columnwidth}p{0.495\columnwidth}p{0.495\columnwidth}}
        %\hline
    \rule{0pt}{15pt} \small{\textbf{\textcolor{blue}{}}} & \small{\textbf{\textcolor{blue}{Expected Results}}} & \small{\textbf{\textcolor{blue}{Observed Results}}}\\
        \hline
\cellcolor{green!20} & \textbf{Hypothesis 1} \small{(\hyperref[hypothese-1]{\(H_{1}\)}, page~\pageref{hypothese-1})} & \cellcolor{green!20}\textbf{\small{Corroborated}}\\
\cellcolor{green!20} & \small{\textsl{The emergence of new research themes on Transit-Oriented Development and light individual mobility calls for a joint analysis.}} & \small{While these two study topics are gaining increasing interest, they are still often addressed separately, even though their synergy provides valuable insights.}\\
    \hdashline
\cellcolor{orange!20} & \textbf{Hypothesis 2} \small{(\hyperref[hypothese-2]{\(H_{2}\)}, page~\pageref{hypothese-2})} & \cellcolor{orange!20}\textbf{\small{Partially invalidated}}\\
\cellcolor{orange!20} & \small{\textsl{Research on this intermodal synergy is still conditioned by the association between bicycles and trains, and is rarely linked to the concept of urban planning.}} & \small{While \textsl{Transit-Oriented Development} is often referenced in studies, it is inadequately utilized, and the integration of new mobility solutions is becoming an increasingly important research topic, except in European contexts.}\\
    \hdashline
\cellcolor{orange!20} & \textbf{Hypothesis 3} \small{(\hyperref[hypothese-3]{\(H_{3}\)}, page~\pageref{hypothese-3})} & \cellcolor{orange!20}\textbf{\small{Partially invalidated}}\\
\cellcolor{orange!20} & \small{\textsl{The complexity of interactions between networks and territories calls for a systemic methodology capable of producing results that go beyond a simple juxtaposition.}} & \small{The complementarity of approaches has not only generated new questions but has also structured a multidimensional and multiscalar approach, albeit with risks of overlap and temporal constraints.}\\
    \hdashline
\cellcolor{green!20} & \textbf{Hypothesis 4} \small{(\hyperref[hypothese-4]{\(H_{4}\)}, page~\pageref{hypothese-4})} & \cellcolor{green!20}\textbf{\small{Corroborated}}\\
\cellcolor{green!20} & \small{\textsl{Intermodal practices, involving the use of light individual mobility, are intensifying due to the rise of micro-mobility, despite unequal adoption by certain social groups.}} & \small{Light individual mobility, as a mode of transport, is experiencing an \Commas{emergence} mainly driven by the adoption of electric scooters for personal use. However, its intermodal use is more unequal than in a monomodal context, although interventions through urban planning play a moderating role on gender inequalities.} \\
    \hdashline
\cellcolor{green!20} & \textbf{Hypothesis 5} \small{(\hyperref[hypothese-5]{\(H_{5}\)}, page~\pageref{hypothese-5})} & \cellcolor{green!20}\textbf{\small{Corroborated}}\\
\cellcolor{green!20} & \small{\textsl{With the integration of light individual mobility, multi-scalar accessibility by public transport becomes significantly more efficient, resilient, and competitive.}} & \small{Train station areas accessible by light individual mobility benefit from a local extension of their perimeter and an increase in opportunities to access regional resources. Route choices are influenced by detour and pause strategies, which, thanks to the reach and flexibility of these vehicles, optimize intermodal travel.}\\
    \hdashline
\cellcolor{green!20} & \textbf{Hypothesis 6} \small{(\hyperref[hypothese-6]{\(H_{6}\)}, page~\pageref{hypothese-6})} & \cellcolor{green!20}\textbf{\small{Corroborated}}\\
\cellcolor{green!20} & \small{\textsl{Associating light individual mobility with urban planning strategies represents an opportunity to integrate and extend train station areas, as well as to stimulate public transport usage.}} & \small{Among the main action levers for rail-oriented urban planning is the improvement of local connectivity, with a particular focus on investments in the \Commas{bike system} to boost train station usage, supporting the \acrshort{TOD} framework.}\\
        \hline
        \end{tabular}}
    \caption{Testing research hypotheses based on research conclusions.}
    \label{table-conclusion:confrontation-hypotheses}
        \vspace{5pt}
        \begin{flushright}\scriptsize{
        Author: \textcolor{blue}{Dylan Moinse (2025)}
        }\end{flushright}
        \end{table}