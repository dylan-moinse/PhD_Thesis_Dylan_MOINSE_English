% Studied Stations
%%Translated%%
        \begin{table}[h!]
  \centering
  \renewcommand{\arraystretch}{1.5}
  \resizebox{\columnwidth}{!}{
  \begin{tabular}{p{0.2\columnwidth}p{0.48\columnwidth}p{0.15\columnwidth}p{0.17\columnwidth}}
    % \hline
    \rule{0pt}{15pt} \small{\textcolor{blue}{\textbf{Station or Stop}}} & \small{\textcolor{blue}{\textbf{\acrshort{EPCI}}}} & \small{\textcolor{blue}{\textbf{Department}}} & \small{\textcolor{blue}{\textbf{Bicycle* (2020)}}}\\
        \hline
    \multicolumn{4}{l}{\small{\textbf{Lille Flandres Station} (\(S_1\))}}\\
\small{Lille} & \small{\acrfull{MEL}} & \small{Nord} & \small{6.6\% (3.8\%)}\\
        \hdashline
    \multicolumn{4}{l}{\small{\textbf{Dunkerque Station} (\(S_2\))}}\\
\small{Dunkerque} & \small{\acrfull{CUD}} & \small{Nord} & \small{3.5\% (3.2\%)}\\
        \hdashline
    \multicolumn{4}{l}{\small{\textbf{Béthune Station} (\(S_3\))}}\\
\multirow{1.5}{*}{\small{Béthune}} & \small{\acrfull{CABBALR}} & \multirow{1.5}{*}{\small{Pas-de-Calais}} & \multirow{1.5}{*}{\small{2.9\% (1.4\%)}}\\
        \hdashline
    \multicolumn{4}{l}{\small{\textbf{Armentières Station} (\(S_4\))}}\\
\small{Armentières} & \small{\acrfull{MEL}} & \small{Nord} & \small{3.4\% (3.8\%)}\\
        \hdashline
    \multicolumn{4}{l}{\small{\textbf{Creil Station} (\(S_5\))}}\\
\small{Creil} & \small{\acrfull{ACSO}} & \small{Oise} & \small{1.1\% (1.1\%)}\\
        \hdashline
    \multicolumn{4}{l}{\small{\textbf{Lille CHR Stop} (\(S_6\))}}\\
\small{Lille} & \small{\acrfull{MEL}} & \small{Nord} & \small{6.6\% (3.8\%)}\\
        \hdashline
    \multicolumn{4}{l}{\small{\textbf{Lesquin Station} (\(S_7\))}}\\
\small{Lesquin} & \small{\acrfull{MEL}} & \small{Nord} & \small{2.2\% (3.8\%)}\\
        \hdashline
    \multicolumn{4}{l}{\small{\textbf{Le Poirier Université Stop} (\(S_8\))}}\\
\multirow{1.5}{*}{\small{Trith-Saint-Léger}} & \small{\acrfull{Porte du Hainaut}} & \multirow{1.5}{*}{\small{Nord}} & \multirow{1.5}{*}{\small{1.0\% (1.8\%)}}\\
        \hdashline
    \multicolumn{4}{l}{\small{\textbf{Vis-à-Marles Stop} (\(S_9\))}}\\
\multirow{1.5}{*}{\small{Marles-les-Mines}} & \small{\acrfull{CABBALR}} & \multirow{1.5}{*}{\small{Pas-de-Calais}} & \multirow{1.5}{*}{\small{0.9\% (1.4\%)}}\\
        \hline
        \end{tabular}}
    \caption{Modal distribution of bicycle usage in the railway station intermunicipalities.}
    \label{table-chap3:part-modale-velo-gares-examinees}
        \vspace{5pt}
        \begin{flushleft}\scriptsize{
        \textcolor{blue}{Note:} The last column of the table refers to the modal share of bicycle usage at the municipal level, followed by that of the \acrshort{EPCI} in parentheses.
        \\
        \textcolor{blue}{Reading:} Among the nine stations examined, exclusive bicycle trips are much more developed in Lille, follow the national average in Dunkerque, Armentières, and Béthune, and are marginal in the municipalities of Lesquin, Creil, Trith-Saint-Léger, and Marles-les-Mines.
        }\end{flushleft}
        \begin{flushright}\scriptsize{
        Data sets related to municipal and inter-municipal modal share of bicycles: \textsl{Regional Bicycle Atlas Hauts-de-France} by \textcolor{blue}{\textcite{velo__territoires_atlas_2023}}, sourced from the \textsl{Professional Mobility (MOBPro) Census Data of 2020} by \textcolor{blue}{\textcite{insee_documentation_2023}}
        \\
        Data sets related to bicycle service availability: \textcolor{blue}{\textcite{sncf_voyageurs_stationnement_2023}}, \textcolor{blue}{\textcite{ilevia_abris_nodate}} and \textcolor{blue}{\textcite{openstreetmap_openstreetmap_2023}} 
        \\
        Author: \textcolor{blue}{Dylan Moinse (2023)}
        }\end{flushright}
        \end{table}