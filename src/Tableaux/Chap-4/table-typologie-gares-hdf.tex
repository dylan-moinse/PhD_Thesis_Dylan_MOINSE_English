% Table Typology of Stations in Hauts-de-France  
%%Translated%%  
    \begin{table}[h!]  
    \centering  
    \renewcommand{\arraystretch}{1.5}  
    \resizebox{\columnwidth}{!}{  
    \begin{tabular}{p{0.27\columnwidth}p{0.13\columnwidth}p{0.60\columnwidth}}  
        %\hline  
    \rule{0pt}{15pt} \small{\textbf{\textcolor{blue}{Station or Halt}}} & \small{\textbf{\textcolor{blue}{\acrshort{DRG}}}} & \small{\textbf{\textcolor{blue}{Contextualization of the Class}}}\\  
        \hline  
\small{Creil} & \multirow{2}{*}{\small{Profile~\(a\)}} & \multirow{2}{*}{\small{\textsl{Stations of regional hubs}}}\\  
    \small{Lille Flandres} & & \\  
        \hdashline  
\small{Armentières} & \multirow{3}{*}{\small{Profile~\(b\)}} & \multirow{3}{*}{\small{\textsl{Stations of intermediate hubs}}}\\  
    \small{Béthune} & & \\  
    \small{Dunkerque} & & \\  
        \hdashline 
\small{Le Poirier Université} & \multirow{4}{*}{\small{Profile~\(c\)}} & \multirow{4}{*}{\small{\textsl{Stations feeding into urban centers}}}\\
\small{Lesquin} & & \\  
\small{Lille CHR} & & \\  
\small{Vis-à-Marles} & & \\  
        \hline  
        \end{tabular}}  
    \caption{Reuse of the \Commas{Segment DRG} reference applied to the nine stations explored in the Hauts-de-France region.}  
    \label{table-chap4:typologie-gares-hdf}  
        \vspace{5pt}  
        \begin{flushleft}\scriptsize{  
        \textcolor{blue}{Reading Guide~:} This table applies the segment typology to the nine stations in the Hauts-de-France region explored in our investigation, classifying them into three distinct profiles: profiles \(a\), \(b\), and \(c\).  
        }\end{flushleft}  
        \begin{flushright}\scriptsize  
        Datasets~: \textcolor{blue}{\textcite{sncf_gares__connexions_gares_2024}}\index{SNCF Gares \& Connexions@\textsl{SNCF Gares \& Connexions}|pagebf}  
        \\
        Author~: \textcolor{blue}{Dylan Moinse (2022)}  
        \end{flushright}  
        \end{table}