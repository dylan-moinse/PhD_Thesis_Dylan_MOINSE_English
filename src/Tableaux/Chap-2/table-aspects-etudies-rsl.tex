% Table of studied aspects
%%Translated%%
        \begin{table}[h!]
    \centering
    \renewcommand{\arraystretch}{1.5}
    \resizebox{\columnwidth}{!}{
    \begin{tabular}{p{0.5\columnwidth}p{0.5\columnwidth}}
        %\hline
    \rule{0pt}{15pt} \small{\textbf{\textcolor{blue}{Studied Aspects}}} & \small{\textbf{\textcolor{blue}{Sub-sections}}}\\
        \hline
    \multicolumn{2}{l}{\small{\textbf{\textcolor{blue}{Metadata and Networks}}}}\\
\small{Author(s), institutions, scientific journals, citations} & \small{\hyperref[chap2:etat-litterature-scientifique-internationale-btod]{sub-section 2.1} (page~\pageref{chap2:etat-litterature-scientifique-internationale-btod})}\\
    \hdashline
    \multicolumn{2}{l}{\small{\textbf{\textcolor{blue}{Terminology}}}}\\
\small{Title, keywords, abstract, content} & \small{\hyperref[chap2:analyse-textuelle]{sub-section 2.1.2} (page~\pageref{chap2:analyse-textuelle})}\\
    \hdashline
    \multicolumn{2}{l}{\small{\textbf{\textcolor{blue}{Study Objects}}}}\\
\small{light individual mobility, public transport, forms of intermodal integration} & \small{\hyperref[chap2:evolution-recherches-tc-mobilite-individuelle-legere]{sub-section 2.1.3} (page~\pageref{chap2:evolution-recherches-tc-mobilite-individuelle-legere})}\\
    \hdashline
    \multicolumn{2}{l}{\small{\textbf{\textcolor{blue}{Geographical Areas}}}}\\
\small{Case studies, geographical scales, international comparisons} & \small{\hyperref[chap2:exploration-terrains-geographiques]{sub-section 2.1.4} (page~\pageref{chap2:exploration-terrains-geographiques})}\\
    \hdashline
    \multicolumn{2}{l}{\small{\textbf{\textcolor{blue}{Concepts}}}}\\
\small{Theoretical frameworks, place of \acrshort{TOD}} & \small{\hyperref[chap2:fondements-theoriques]{sub-section 2.2.1} (page~\pageref{chap2:fondements-theoriques})}\\
    \hdashline
    \multicolumn{2}{l}{\small{\textbf{\textcolor{blue}{Methodology}}}}\\
\small{Research methods, data sources, sampling, types of analysis} & \small{\hyperref[chap2:methodes-collecte-donnees]{sub-sections 2.2.2} and \hyperref[chap2:demarches-types-analyses]{2.2.3} (pages \pageref{chap2:methodes-collecte-donnees} and \pageref{chap2:demarches-types-analyses})}\\
    \hdashline
    \multicolumn{2}{l}{\small{\textbf{\textcolor{blue}{TOD Principles (\Commas{\acrshort{7Ds}})}}}}\\
\small{Density, diversity, design, accessibility of destinations, distance to public transport stations, demand management, and social inclusion} & \small{\hyperref[chap2:densite-population]{sub-sections 3.1} (page~\pageref{chap2:densite-population}), \hyperref[chap2:diversite-fonctionnelle]{3.2} (page~\pageref{chap2:diversite-fonctionnelle}), \hyperref[chap2:traitement-espaces-publics]{3.3} (page~\pageref{chap2:traitement-espaces-publics}), \hyperref[chap2:accessibilite-intermodale]{3.4} (page~\pageref{chap2:accessibilite-intermodale}), \hyperref[chap2:distances-premiers-derniers-km]{3.5} (page~\pageref{chap2:distances-premiers-derniers-km}), \hyperref[chap2:gestion-demande-mobilite]{3.6} (page~\pageref{chap2:gestion-demande-mobilite}) and \hyperref[chap2:sociodemographie-usagers]{3.7} (page~\pageref{chap2:sociodemographie-usagers})}\\
    \hdashline
    \multicolumn{2}{l}{\small{\textbf{\textcolor{blue}{Mobility Behaviors}}}}\\
\small{Reasons, experience, social representations} & \small{\hyperref[chap2:comportements-mobilite]{sub-section 3.8} (page~\pageref{chap2:comportements-mobilite})}\\
    \hdashline
    \multicolumn{2}{l}{\small{\textbf{\textcolor{blue}{Impacts}}}}\\
\small{Mobility, urban planning, economy, environment} & \small{\hyperref[chap2:impacts-systemes-urbain-mobilite]{sub-section 3.9} (page~\pageref{chap2:impacts-systemes-urbain-mobilite})}\\
        \hline
        \end{tabular}}
    \caption{Analysis grid of the systematic literature review on a \textsl{Micromobility-friendly Transit-Oriented Development}.}
    \label{table-chap2:aspects-etudies-rsl}
        \vspace{5pt}
        \begin{flushleft}\scriptsize{
        \textcolor{blue}{Note:} The aspects examined are not confined to a single theme and appear throughout the chapter.
        \\
        \textcolor{blue}{Reading:} The systematic literature review on a transit-oriented urbanism supported by light individual mobility is based on metadata, terminology, study objects, geographical contexts, concepts and techniques mobilized, planning model principles, mobility behaviors, and observed impacts.
        }\end{flushleft}
        \begin{flushright}\scriptsize{
        Author: \textcolor{blue}{Dylan Moinse (2023)}
        }\end{flushright}
        \end{table}