% Table of analyzed territories RSL
%%Translated%%
        \begin{table}[h!]
    \centering
    \renewcommand{\arraystretch}{1.5}
    \resizebox{\columnwidth}{!}{
    \begin{tabular}{p{1\columnwidth}}
        %\hline
    \rule{0pt}{15pt} \small{\textbf{\textcolor{blue}{Agglomerations and Municipalities}}}\\
        \hline
    \small{\textbf{\textcolor{blue}{More than 10,000,000 inhabitants (88 references)}}}\\
\small{Beijing (21), Nanjing (19), Shenzhen (9), Shanghai (9), Seoul (6), New York City (4), Chengdu (3), Los Angeles (3), New Delhi (3), Xi'an (3), Chicago (2), Île-de-France (2), Mumbai (2), Bogotá (1), Johannesburg-Pretoria (1), Osaka (1), Manila (1), Rio de Janeiro (1)}\\
        \hdashline
    \small{\textbf{\textcolor{blue}{Between 3,000,000 and 10,000,000 inhabitants (53 references)}}}\\
\small{Washington D.C. (7), Berlin (4), Boston (4), Taipei (4), San Francisco (3), Seattle (3), Suzhou (3), Kaohsiung (2), Melbourne (2), Minneapolis (2), Montreal (2), Philadelphia (2), Toronto (2), Accra (1), Ahmedabad (1), Athens (1), Atlanta (1), Birmingham (1), Boulder (1), Cape Town (1), Nanchang (1), Porto Alegre (1), Rome (1), Surat (1), Sydney (1), Vienna (1)}\\
        \hdashline
    \small{\textbf{\textcolor{blue}{Between 1,000,000 and 3,000,000 inhabitants (33 references)}}}\\
\small{Rotterdam-The Hague (6), Amsterdam (4), Austin (3), Cincinnati (2), Cleveland (2), Copenhagen (2), Oslo (2), Auckland (1), Bristol (1), Columbus (1), Helsinki (1), Gutenberg (1), Aix-Marseille-Provence (1), Nashville (1), Orlando (1), Poznań (1), Seville (1), Tucson (1), Turin (1)}\\
        \hdashline
    \small{\textbf{\textcolor{blue}{Between 250,000 and 1,000,000 inhabitants (16 references)}}}\\
\small{Hamilton (3), Portland (3), Utrecht (3), Delft (2), Amstelland-Meerlanden (1), Eindhoven (1), Malmö (1), Mamelodi (1), Tarnow (1)}\\
        \hdashline
    \small{\textbf{\textcolor{blue}{Less than 250,000 inhabitants (8 references)}}}\\
\small{Amboise (3), Bayeux (1), El Monte (1), Ithaca (1), Longmont (1), Belgian municipalities between 30,000 and 200,000 inhabitants (1)}\\
        \hline
        \end{tabular}}
    \caption{Size of agglomerations and municipalities studied in the systematic literature review.}
    \label{table-chap2:tailles-territoires-rsl}
        \vspace{5pt}
        \begin{flushleft}\scriptsize{
        \textcolor{blue}{Reading:} Based on 198 studies including a geographical area, the systematic literature review primarily consists of agglomerations with more than 3,000,000 inhabitants.
        }\end{flushleft}
        \begin{flushright}\scriptsize
        Author: \textcolor{blue}{Dylan Moinse (2023)}
        \end{flushright}
        \end{table}