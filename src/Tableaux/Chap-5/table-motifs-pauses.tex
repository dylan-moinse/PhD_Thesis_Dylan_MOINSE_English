% Pause Typology Table
%%Translated%%
        \begin{table}[h!]
        \centering
        \renewcommand{\arraystretch}{1.5}
        \resizebox{\columnwidth}{!}{
        \begin{tabular}{p{0.58\columnwidth}p{0.21\columnwidth}p{0.21\columnwidth}}
        %\hline
    \rule{0pt}{15pt} \textcolor{blue}{\textbf{Break Typology}} & \textcolor{blue}{\textbf{Count}} & \textcolor{blue}{\textbf{Share}}\\
        \hline
\small{Purchases} & \small{82} & \small{74.5\%} \\
\small{Social meetings and accompaniment} & \small{30} & \small{27.3\%} \\
\small{Professional activities} & \small{27} & \small{24.5\%} \\
\small{Leisure} & \small{23} & \small{20.9\%} \\
\small{Administrative tasks} & \small{17} & \small{15.5\%} \\
\small{Visits and walks} & \small{16} & \small{14.5\%} \\
        \hline
        \end{tabular}}
    \caption{Reasons for the 110 breaks reported by intermodal cyclists during intermodal journeys.}
    \label{table-chap5:motifs-pauses}
        \vspace{5pt}
        \begin{flushleft}\scriptsize{
        \textcolor{blue}{Note:} As the question allowed multiple choices, the total share exceeds 100\%.
        \\
        \textcolor{blue}{Reading Guide:} Purchases are the primary reason for breaks during intermodal trips, followed by social meetings and accompaniment, and professional activities. This table highlights the diversity of reasons associated with trip interruptions.
        }\end{flushleft}
        \begin{flushright}\scriptsize{
        Author: \textcolor{blue}{Dylan Moinse (2023)}
        }\end{flushright}
        \end{table}