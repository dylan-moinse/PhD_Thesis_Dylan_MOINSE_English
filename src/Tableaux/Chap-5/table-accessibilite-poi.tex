% Table accessibility to POIs
%%Translated%%
    \begin{table}[h!]
    \centering
    \renewcommand{\arraystretch}{1.5}
    \resizebox{\columnwidth}{!}{
    \begin{tabular}{p{0.35\columnwidth}p{0.12\columnwidth}p{0.17\columnwidth}p{0.12\columnwidth}p{0.12\columnwidth}p{0.12\columnwidth}}
        %\hline
    \rule{0pt}{15pt} \small{\textbf{\textcolor{blue}{Facilities and Services}}} & \small{\textbf{\textcolor{blue}{\textless1 km.}}} & \small{\textbf{\textcolor{blue}{{[}1~;~4{[} km.}}} & \small{\textbf{\textcolor{blue}{\textless4 km.}}} & \small{\textbf{\textcolor{blue}{\geq 4 km.}}} & \small{\textbf{\textcolor{blue}{Region}}}\\
        \hline
    \multicolumn{6}{l}{\textbf{All types of \acrshort{POIs}}}\\
\small{Count} & \small{31,743} & \small{36,689} & \small{68,432} & \small{36,009} & \small{104,441}\\
\small{Share} & \small{30.39\%} & \small{35.13\%} & \small{65.52\%} & \small{34.48\%} & \small{100\%}\\
\small{Density (km\textsuperscript{2})} & \small{63.23} & \small{23.21} & \small{32.87} & \small{9.96} & \small{18.35}\\
        \hdashline
    \multicolumn{6}{l}{\textbf{\acrshort{POIs} of \Commas{proximity}}}\\
\small{Count} & \small{18,976} & \small{23,171} & \small{42,147} & \small{26,371} & \small{68,518}\\
\small{Share} & \small{27.69\%} & \small{33.82\%} & \small{61.51\%} & \small{38.49\%} & \small{100\%}\\
\small{Density (km\textsuperscript{2})} & \small{37.80} & \small{14.66} & \small{20.24} & \small{7.29} & \small{12.04}\\
        \hdashline
    \multicolumn{6}{l}{\textbf{\acrshort{POIs} \Commas{intermediate}}}\\
\small{Count} & \small{9,392} & \small{9,651} & \small{19,043} & \small{7,716} & \small{26,759}\\
\small{Share} & \small{35.10\%} & \small{36.07\%} & \small{71.16\%} & \small{28.84\%} & \small{100\%}\\
\small{Density (km\textsuperscript{2})} & \small{18.71} & \small{6.11} & \small{9.15} & \small{2.13} & \small{4.70}\\
        \hdashline
    \multicolumn{6}{l}{\textbf{\acrshort{POIs} \Commas{superior}}}\\
\small{Count} & \small{3,375} & \small{3,867} & \small{7,242} & \small{1,922} & \small{9,164}\\
\small{Share} & \small{36.83\%} & \small{42.20\%} & \small{79.03\%} & \small{20.97\%} & \small{100\%}\\
\small{Density (km\textsuperscript{2})} & \small{6.72} & \small{2.45} & \small{3.48} &	\small{0.53} & \small{1.61}\\
        \hline
        \end{tabular}}
    \caption{Accessibility to points of interest, centered on train station areas at the reach of pedestrian and cycling levels.}
    \label{table-chap5:accessibilite-poi}
        \vspace{5pt}
        \begin{flushleft}\scriptsize{
        \textcolor{blue}{Note:} The train station districts accessible by bike and micromobility extend up to three kilometers, except for the influence areas of the six multimodal exchange hubs, which have a radius of four kilometers.
        \\
        \textcolor{blue}{Reading Guide:} The \Commas{proximity} and \Commas{intermediate} points of interest are mostly accessible within limited radii, while larger-scale facilities are more concentrated at distances reachable by bike. This highlights a hierarchy of facilities based on their accessibility distance.
        }\end{flushleft}
        \begin{flushright}\scriptsize
        Data sources: \acrfull{BPE} from \textcolor{blue}{\textcite{insee_base_2021}}\index{Insee@\textsl{Insee}|pagebf}
        \\
        Author: \textcolor{blue}{Dylan Moinse (2024)}
        \end{flushright}
        \end{table}