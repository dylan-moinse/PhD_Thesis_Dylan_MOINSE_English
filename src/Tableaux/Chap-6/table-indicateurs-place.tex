% Table indicator grid (place)
%%Translated%%
    \begin{table}[h!]
    \centering
    \renewcommand{\arraystretch}{1.5}
    \resizebox{\columnwidth}{!}{
    \begin{tabular}{p{0.43\columnwidth}p{0.08\columnwidth}p{0.49\columnwidth}}
        %\hline
    \rule{0pt}{15pt} \small{\textbf{\textcolor{blue}{Category}}} & \small{\textbf{\textcolor{blue}{ID}}} & \small{\textbf{\textcolor{blue}{Indicator}}}\\
        \hline
    \multirow{2}{*}{\textbf{Density}} & \small{\(P_{1}\)} & \small{Population density}\\
& \small{\(P_{2}\)} & \small{Employment density}\\
        \hdashline
    \multirow{5.5}{*}{\textbf{Functional Diversity}} & \small{\multirow{1.5}{*}{\(P_{3}\)}} & \small{Residential-dominant land use}\\
& \small{\multirow{1.5}{*}{\(P_{4}\)}} & \small{Commercial and public service-dominant land use}\\
& \small{\multirow{1.5}{*}{\(P_{5}\)}} & \small{Office and industrial-dominant land use}\\
& \small{\multirow{1}{*}{\(P_{6}\)}} & \small{Green space-dominant land use}\\
        \hdashline
    \multirow{3}{*}{\textbf{Attraction Locations}} & \small{\(P_{7}\)} & \small{Proximity points of interest (\acrshort{POIs})}\\
& \small{\multirow{1}{*}{\(P_{8}\)}} & \small{Intermediate points of interest (\acrshort{POIs})}\\
& \small{\(P_{9}\)} & \small{Superior points of interest (\acrshort{POIs})}\\
        \hdashline
    \multirow{2.5}{*}{\textbf{Land Value}} & \small{\(P_{10}\)} & \small{Land value of residential properties}\\
& \small{\multirow{1.5}{*}{\(P_{11}\)}} & \small{Land value of industrial, commercial, and office properties}\\
        \hdashline
    \multirow{2}{*}{\textbf{Social Stratification}} & \small{\(P_{12}\)} & \small{Proportion of social housing}\\
& \small{\(P_{13}\)} & \small{Average household income}\\
        \hline
        \end{tabular}}
    \caption{Indicator grid grouped within the dimension related to \Commas{urban development degree} (\textsl{place}).}
    \label{table-chap6:indicateurs-place}
        \vspace{5pt}
        \begin{flushleft}\scriptsize{
        \textcolor{blue}{Reading Guide:} Thirteen indicators are grouped within the dimension \(P\). 
        }\end{flushleft}
        \begin{flushright}\scriptsize{
        Realization: \textcolor{blue}{Dylan Moinse (2024)}
        \\
        Authors: \acrshort{NPART} Research Project
        }\end{flushright}
        \end{table}