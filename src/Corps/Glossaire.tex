%------------------------------%
%% ✎ Dylan (V1) %%%%%%%%% ✅ %%
%% ✎ Alain (V2) %%%%%%%%% ✅ %%
%% ✎ Dylan (V3) %%%%%%%%% ✅ %%
%------------------------------%

%%%%%%%%%%%%%%%%%%%%%%%%%%%%%%%%
% Sigles et acronymes

\newacronym[description={Seven key aspects of \textsl{Transit-Oriented Development}.}]{7Ds}{7Ds}{seven key aspects of \textsl{Transit-Oriented Development}}%%Translated%%

\newacronym[description={\textsl{Urban Attraction Areas}.}]{UAA}{UAA}{\textsl{urban attraction areas}}%%Translated%%
% AAV

\newacronym[description={Cost-Benefit Econometric Analysis.}]{CBEA}{CBEA}{Cost-Benefit Econometric Analysis}%%Translated%%
% ACB

\newacronym[description={\textsl{Accessibility and Connectivity knowledge hub for Urban Transformation in Europe}.}]{ACUTE}{ACUTE}{\textsl{Accessibility and Connectivity knowledge hub for Urban Transformation in Europe}}%%Translated%%

\newacronym[description={Creil Sud Oise Metropolitan Community.}]{ACSO}{ACSO}{Creil Sud Oise Metropolitan Community}%%Translated%%

\newacronym[description={\textsl{French Standardization Association}.}]{AFNOR}{AFNOR}{\textsl{French Standardization Association}}%%Translated%%

\newacronym[description={Béthune Bruay Artois Lys Romane Metropolitan Community.}]{CABBALR}{CABBALR}{Béthune Bruay Artois Lys Romane Metropolitan Community}%%Translated%%

\newacronym[description={\textsl{AGglomerative NESting}.}]{AGNES}{AGNES}{\textsl{AGglomerative NESting}}%%Translated%%

\newacronym[description={Multi-Criteria Analysis.}]{MCA}{MCA}{Multi-Criteria Analysis}%%Translated%%
% AMC

\newacronym[description={\textsl{Artificial Neural Networks}.}]{ANN}{ANN}{\textsl{Artificial Neural Networks}}%%Translated%%

\newacronym[description={\textsl{Application Programming Interface}.}]{API}{API}{\textsl{Application Programming Interface}}%%Translated%%

\newacronym[description={\textsl{Bicycle-based Transit-Oriented Development}.}]{B-TOD}{B-TOD}{\textsl{Bicycle-based Transit-Oriented Development}}%%Translated%%

\newacronym[description={Adjusted Percentage Reduction in Error.}]{APRE}{APRE}{\textsl{Adjusted Percentage Reduction in Error}}%%Translated%%

\newacronym[description={\textsl{Bay Area Rapid Transit}.}]{BART}{BART}{\textsl{Bay Area Rapid Transit}}%%Translated%%

\newacronym[description={\textsl{Vocational Studies Certificate}.}]{BEP}{BEP}{\textsl{Vocational Studies Certificate}}%%Translated%%

\newacronym[description={Bus Rapid Transit.}]{BRT}{BRT}{Bus Rapid Transit}%%Translated%%
% BHNS

\newacronym[description={\textsl{Permanent Facilities Database}.}]{BPE}{BPE}{\textsl{Permanent Facilities Database}}%%Translated%%

\newacronym[description={\textsl{Advanced Technician's Certificate}.}]{BTS}{BTS}{\textsl{Advanced Technician's Certificate}}%%Translated%%

\newacronym[description={\textsl{University Bachelor of Technology}.}]{BUT}{BUT}{\textsl{University Bachelor of Technology}}%%Translated%%

\newacronym[description={Best Worst Method.}]{BWM}{BWM}{Best Worst Method}%%Translated%%

\newacronym[description={\textsl{Metropolitan Community}.}]{CA}{CA}{\textsl{Metropolitan Community}}%%Translated%%

\newacronym[description={\textsl{Certificate of Professional Competence}.}]{CAP}{CAP}{\textsl{Certificate of Professional Competence}}%%Translated%%

\newacronym[description={Central Business District.}]{CBD}{CBD}{Central Business District}%%Translated%%

\newacronym[description={\textsl{Community of Municipalities}.}]{CC}{CC}{\textsl{Community of Municipalities}}%%Translated%%

\newacronym[description={\textsl{Centre for Direct Scientific Communication}.}]{CCSD}{CCSD}{\textsl{Centre for Direct Scientific Communication}}%%Translated%%

\newacronym[description={\textsl{Centre for Studies and Expertise on Risks, Environment, Mobility, and Planning}.}]{Cerema}{Cerema}{\textsl{Centre for Studies and Expertise on Risks, Environment, Mobility, and Planning}}%%Translated%%

\newacronym[description={Chlorofluorocarbons.}]{CFC}{CFC}{chlorofluorocarbons}%%Translated%%

\newacronym[description={Methane.}]{CH4}{CH4}{methane}%%Translated%%

\newacronym[description={University Hospital Center.}]{CHU}{CHU}{University Hospital Center}%%Translated%%

\newacronym[description={Carbon dioxide.}]{CO2}{CO2}{carbon dioxide}%%Translated%%

\newacronym[description={Overseas Collectivities.}]{COM}{COM}{Overseas Collectivities}%%Translated%%

\newacronym[description={\textsl{National Commission on Informatics and Liberties}.}]{CNIL}{CNIL}{\textsl{National Commission on Informatics and Liberties}}%%Translated%%

\newacronym[description={\textsl{National Center for Scientific Research}.}]{CNRS}{CNRS}{\textsl{National Center for Scientific Research}}%%Translated%%

\newacronym[description={Dunkirk Urban Community.}]{CUD}{CUD}{Dunkirk Urban Community}%%Translated%%

\newacronym[description={\textsl{Urban Community}.}]{CU}{CU}{\textsl{Urban Community}}%%Translated%%

\newacronym[description={Central lane shared roadway.}]{CVCB}{CVCB}{central lane shared roadway}%%Translated%%

\newacronym[description={\textsl{General University Studies Diploma}.}]{DEUG}{DEUG}{\textsl{General University Studies Diploma}}%%Translated%%

\newacronym[description={\textsl{University Diploma in Scientific and Technical Studies}.}]{DEUST}{DEUST}{\textsl{University Diploma in Scientific and Technical Studies}}%%Translated%%

\newacronym[description={Data Protection Officers.}]{DPO}{DPO}{Data Protection Officers}%%Translated%%

\newacronym[description={\textsl{Reference Document for Train Stations}.}]{DRG}{DRG}{\textsl{Reference Document for Train Stations}}%%Translated%%

\newacronym[description={Overseas Departments and Regions.}]{DROM}{DROM}{Overseas Departments and Regions}%%Translated%%

\newacronym[description={Overseas Departments, Regions, and Collectivities.}]{DROM-COM}{DROM-COM}{Overseas Departments, Regions, and Collectivities}%%Translated%%

\newacronym[description={\textsl{University Diploma in Technology}.}]{DUT}{DUT}{\textsl{University Diploma in Technology}}%%Translated%%

\newacronym[description={\textsl{Equitable-TOD}.}]{e-TOD}{e-TOD}{\textsl{Equitable-TOD}}%%Translated%%

\newacronym[description={\textsl{Extended Transit-Oriented Development}.}]{E-TOD}{E-TOD}{\textsl{Extended Transit-Oriented Development}}%%Translated%%

\newacronym[description={\textsl{Corridor Transit-Oriented Development}.}]{C-TOD}{C-TOD}{\textsl{Corridor Transit-Oriented Development}}%%Translated%%

\newacronym[description={\textsl{Escaping Transit Voronoi Station}.}]{E-TVS}{E-TVS}{\textsl{Escaping Transit Voronoi Station}}%%Translated%%

\newacronym[description={Personal Mobility Device.}]{PMD}{PMD}{Personal Mobility Device}%%Translated%%
% EDP

\newacronym[description={Electric Personal Mobility Device.}]{ePMD}{MPMD}{Electric Personal Mobility Device}%%Translated%%
% EDPM

\newacronym[description={\textsl{Paris School of Urban Planning}.}]{EUP}{EUP}{\textsl{Paris School of Urban Planning}}%%Translated%%

\newacronym[description={\textsl{Architecture Research Engagement Post-carbon}.}]{AREP}{AREP}{\textsl{Architecture Research Engagement Post-carbon}}%%Translated%%

\newacronym[description={\textsl{Établissement Public de Coopération Intercommunale}.}]{EPCI}{EPCI}{\textsl{Établissement Public de Coopération Intercommunale}}%%Translated%%

\newacronym[description={\textsl{Fundacion para la Educacion y el Desarrollo Social}.}]{FES}{FES}{\textsl{Fundacion para la Educacion y el Desarrollo Social}}%%Translated%%

\newacronym[description={\textsl{Fédération des Professionnels de la Micro-Mobilité.}.}]{FP2M}{FP2M}{\textsl{Fédération des Professionnels de la Micro-Mobilité.}}%%Translated%%

\newacronym[description={General Bikeshare Feed Specification.}]{GBFS}{GBFS}{General Bikeshare Feed Specification}%%Translated%%

\newacronym[description={Greenhouse Gases.}]{GHG}{GHG}{Greenhouse Gases}%%Translated%%
% GES

\newacronym[description={Gaussian Mixture Model.}]{GMM}{GMM}{Gaussian Mixture Model}%%Translated%%

\newacronym[description={Geographical Route Directness Index.}]{GRDI}{GRDI}{Geographical Route Directness Index}%%Translated%%

\newacronym[description={General Transit Feed Specification.}]{GTFS}{GTFS}{General Transit Feed Specification}%%Translated%%

\newacronym[description={\textsl{Accreditation to Supervise Research}.}]{HDR}{HDR}{\textsl{Accreditation to Supervise Research}}%%Translated%%

\newacronym[description={Artificial Intelligence.}]{AI}{AI}{Artificial Intelligence}%%Translated%%
% IA

\newacronym[description={\textsl{Lille Institute of Urban Planning and Geography}.}]{IAUGL}{IAUGL}{\textsl{Lille Institute of Urban Planning and Geography}}%%Translated%%

\newacronym[description={\textsl{Preliminary Joint Inspection}.}]{ICP}{ICP}{\textsl{Preliminary Joint Inspection}}%%Translated%%

\newacronym[description={Information Entropy Weight.}]{IEW}{IEW}{Information Entropy Weight}%%Translated%%

\newacronym[description={Bibliometric Hirsch Index.}]{h-index}{h-index}{Bibliometric Hirsch Index}%%Translated%%
% indice~h

\newacronym[description={k-Nearest Neighbors.}]{KNN}{KNN}{k-Nearest Neighbors}%%Translated%%

\newacronym[description={High-Speed Rail Line.}]{HSR}{HSR}{High-Speed Rail Line}%%Translated%%
% LGV

\newacronym[description={\textsl{City Mobility Transport Laboratory}.}]{LVMT}{LVMT}{\textsl{City Mobility Transport Laboratory}}%%Translated%%

\newacronym[description={\textsl{Micromobility-friendly Transit-Oriented Development}.}]{M-TOD}{M-TOD}{\textsl{Micromobility-friendly Transit-Oriented Development}}%%Translated%%

\newacronym[description={Mobility as a Service.}]{MaaS}{MaaS}{Mobility as a Service}%%Translated%%

\newacronym[description={European Metropolis of Lille.}]{MEL}{MEL}{European Metropolis of Lille}%%Translated%%

\newacronym[description={Multiple Linear Regression.}]{MLR}{MLR}{Multiple Linear Regression}%%Translated%%

\newacronym[description={Mean Squared Error.}]{MSE}{MSE}{Mean Squared Error}%%Translated%%

\newacronym[description={Nitrous oxide.}]{N2O}{N2O}{nitrous oxide}%%Translated%%

\newacronym[description={Nitrogen dioxide.}]{NO2}{NO2}{nitrogen dioxide}%%Translated%%

\newacronym[description={Nitrogen oxides.}]{NOx}{NOx}{nitrogen oxides}%%Translated%%

\newacronym[description={\textsl{Node Place Accessibility Ridership per Time}.}]{NPART}{NPART}{\textsl{Node Place Accessibility Ridership per Time}}%%Translated%%

\newacronym[description={Node-Place Model.}]{NPM}{NPM}{Node-Place Model}%%Translated%%

\newacronym[description={Information and Communication Technologies.}]{ICT}{ICT}{Information and Communication Technologies}%%Translated%%
% NTIC

\newacronym[description={New Individual Electric Vehicles.}]{NIEV}{NIEV}{New Individual Electric Vehicles}%%Translated%%
% NVEI

\newacronym[description={Lightweight Personal Electric Vehicles.}]{LPEV}{LPEV}{Lightweight Personal Electric Vehicles}%%Translated%%
% VLEU

\newacronym[description={Personal Light Electric Vehicles.}]{PLEV}{PLEV}{Personal Light Electric Vehicles}%%Translated%%
% VLEP

\newacronym[description={Ozone.}]{O3}{O3}{ozone}%%Translated%%

\newacronym[description={\textsl{Land Use in 2~Dimensions}.}]{OCS2D}{OCS2D}{\textsl{Land Use in 2~Dimensions}}%%Translated%%

\newacronym[description={\textsl{Sustainable Development Goals}.}]{SDGs}{SDGs}{\textsl{Sustainable Development Goals}}%%Translated%%
% ODD

\newacronym[description={Ordinary Least Squares.}]{OLS}{OLS}{Ordinary Least Squares}%%Translated%%

\newacronym[description={\textsl{Onderzoek Verplaatsingen in Nederland}.}]{OViN}{OViN}{\textsl{Onderzoek Verplaatsingen in Nederland}}%%Translated%%

\newacronym[description={Park-and-Ride.}]{PnR}{PnR}{park-and-ride}%%Translated%%
% P+R

\newacronym[description={\textsl{Professions and Socio-Professional Categories}.}]{PCS}{PCS}{\textsl{Professions and Socio-Professional Categories}}%%Translated%%

\newacronym[description={Points of Interest.}]{POIs}{POIs}{Points of Interest}%%Translated%%

\newacronym[description={Porte du Hainaut Metropolitan Community.}]{Porte du Hainaut}{Porte du Hainaut}{Porte du Hainaut Metropolitan Community}%%Translated%%

\newacronym[description={\textsl{Gross disposable income}.}]{GDI}{GDI}{\textsl{gross disposable income}}%%Translated%%
% RDB

\newacronym[description={Route Directness Index.}]{RDI}{RDI}{Route Directness Index}%%Translated%%

\newacronym[description={\textsl{Mobility Plan}.}]{PDM}{PDM}{\textsl{Mobility Plan}}%%Translated%%

\newacronym[description={\textsl{Regional Express Network}.}]{RER}{RER}{\textsl{Regional Express Network}}%%Translated%%

\newacronym[description={\textsl{General Data Protection Regulation}.}]{GDPR}{GDPR}{\textsl{General Data Protection Regulation}}%%Translated%%
% RGPD

\newacronym[description={\textsl{Directory of Social Landlords’ Rental Housing}.}]{RPLS}{RPLS}{\textsl{Directory of Social Landlords’ Rental Housing}}%%Translated%%

\newacronym[description={\textsl{National Directory of Professional Certifications}.}]{RNCP}{RNCP}{\textsl{National Directory of Professional Certifications}}%%Translated%%

\newacronym[description={Systematic Literature Review.}]{SLR}{SLR}{Systematic Literature Review}%%Translated%%
% RSL

\newacronym[description={\textsl{Multi-Agent Transport Simulation Toolkit}.}]{MATSim}{MATSim}{\textsl{Multi-Agent Transport Simulation Toolkit}}%%Translated%%

\newacronym[description={\textsl{Bibliographic research based on Cascading Citations}.}]{CC~Search}{CC~Search}{\textsl{bibliographic research based on Cascading Citations}}%%Translated%%
% Recherche CC

\newacronym[description={\textsl{Bibliographic research based on Scientific Journal Ranking}.}]{JR~Search}{JR~Search}{\textsl{bibliographic research based on Scientific Journal Ranking}}%%Translated%%
% Recherche CR

\newacronym[description={\textsl{Bibliographic research in English}.}]{EN~Search}{EN~Search}{\textsl{bibliographic research in English}}%%Translated%%
% Recherche EN

\newacronym[description={\textsl{Bibliographic research in French}.}]{FR~Search}{FR~Search}{\textsl{bibliographic research in French}}%%Translated%%
% Recherche FR

\newacronym[description={\textsl{Territorial Coherence Scheme}.}]{SCoT}{SCoT}{\textsl{Territorial Coherence Scheme}}%%Translated%%

\newacronym[description={\textsl{Metropolitan Regional Express Services}.}]{SERM}{SERM}{\textsl{Metropolitan Regional Express Services}}%%Translated%%

\newacronym[description={Human and Social Sciences.}]{HSS}{HSS}{Human and Social Sciences}%%Translated%%
% SHS

\newacronym[description={\textsl{National System for Business and Establishment Identification and Directory}.}]{Sirene}{Sirene}{\textsl{National System for Business and Establishment Identification and Directory}}%%Translated%%

\newacronym[description={Geographic Information System.}]{GIS}{GIS}{Geographic Information System}%%Translated%%
% SIG

\newacronym[description={Sulfur dioxide.}]{SO2}{SO2}{sulfur dioxide}%%Translated%%

\newacronym[description={Sulfur trioxide.}]{SO3}{SO3}{sulfur trioxide}%%Translated%%

\newacronym[description={Sulfur oxides.}]{SOx}{SOx}{sulfur oxides}%%Translated%%

\newacronym[description={\textsl{Local Public Company}.}]{SPL}{SPL}{\textsl{Local Public Company}}%%Translated%%

\newacronym[description={\textsl{Objectives and Performance Contract}.}]{COP}{COP}{\textsl{Objectives and Performance Contract}}%%Translated%%

\newacronym[description={\textsl{Transit-Adjacent Development}.}]{TAD}{TAD}{\textsl{Transit-Adjacent Development}}%%Translated%%

\newacronym[description={Dockless e-Scooter Sharing System.}]{DESS}{DESS}{Dockless e-Scooter Sharing System.}%%Translated%%
% TEFF

\newacronym[description={Personal Electric Scooter.}]{PeS}{PeS}{Personal Electric Scooter}%%Translated%%
% TEP

\newacronym[description={\textsl{Regional Express Transport}.}]{TER}{TER}{\textsl{Regional Express Transport}}%%Translated%%

\newacronym[description={\textsl{High-Speed Regional Express Transport}.}]{TERGV}{TERGV}{\textsl{High-Speed Regional Express Transport}}%%Translated%%

\newacronym[description={High-Speed Train.}]{HST}{HST}{High-Speed Train}%%Translated%%
% TGV

\newacronym[description={\textsl{Economic Theory, Modeling, and Applications}.}]{ThéMA}{ThéMA}{\textsl{Economic Theory, Modeling, and Applications}}%%Translated%%

\newacronym[description={\textsl{Transit-Oriented Development}.}]{TOD}{TOD}{\textsl{Transit-Oriented Development}}%%Translated%%

\newacronym[description={\textsl{University College Dublin}.}]{UCD}{UCD}{\textsl{University College Dublin}}%%Translated%%

\newacronym[description={Electric-Assisted Bicycle.}]{e-Bike}{e-Bike}{Electric-Assisted Bicycle}%%Translated%%
% VAE

\newacronym[description={Dockless Bicycle Sharing System.}]{DBS}{DBS}{Dockless Shared Bicycle}%%Translated%%
% VFF

\newacronym[description={Public Station-based Bicycle System.}]{PBS}{PBS}{Public Station-based Bicycle}%%Translated%%
% VLS

\newacronym[description={Ride-Hailing Service}]{RHS}{RHS}{Ride-Hailing Service}%%Translated%%
% VTC

\newacronym[description={Within-Cluster Sum of Squares.}]{WCSS}{WCSS}{Within-Cluster Sum of Squares}%%Translated%%

\newacronym[description={\textsl{Zero Net Land Take}.}]{ZAN}{ZAN}{\textsl{Zero Net Land Take}}%%Translated%%

\newacronym[description={World Geodetic System 1984.}]{WGS84}{WGS84}{World Geodetic System 1984}%%Translated%%

\newacronym[description={Intergovernmental Panel on Climate Change.}]{IPCC}{IPCC}{Intergovernmental Panel on Climate Change}%%Translated%%
% GIEC

\newacronym[description={\textsl{Concerted Development Zone}.}]{ZAC}{ZAC}{\textsl{Concerted Development Zone}}%%Translated%%

\newacronym[description={\textsl{Zones urbaines fonctionnelles}.}]{ZUF}{ZUF}{\textsl{Zones urbaines fonctionnelles}}%%Translated%%

\newacronym[description={\textsl{Loi d’Orientation des Transports Intérieurs}.}]{LOTI}{LOTI}{\textsl{Loi d’Orientation des Transports Intérieurs}}%%Translated%%

\newacronym[description={\textsl{Nouvelle Organisation Territoriale de la République}.}]{NOTRe}{NOTRe}{\textsl{Nouvelle Organisation Territoriale de la République}}%%Translated%%

\newacronym[description={\textsl{Loi d’Orientation des Mobilités}.}]{LOM}{LOM}{\textsl{Loi d’Orientation des Mobilités}}%%Translated%%

\newacronym[description={\textsl{Autorité Organisatrice de Transport}.}]{AOT}{AOT}{\textsl{Autorité Organisatrice de Transport}}%%Translated%%

\newacronym[description={\textsl{Autorité Organisatrice de la Mobilité}.}]{AOM}{AOM}{\textsl{Autorité Organisatrice de la Mobilité}}%%Translated%%

\newacronym[description={\textsl{Schéma Régional d’Aménagement, de Développement Durable et d’Égalité des Territoires}.}]{SRADDET}{SRADDET}{\textsl{Schéma Régional d’Aménagement, de Développement Durable et d’Égalité des Territoires}}%%Translated%%

\newacronym[description={\textsl{Loi de Modernisation de l'Action Publique Territoriale et d'Affirmation des Métropoles}.}]{MAPTAM}{MAPTAM}{\textsl{Loi de Modernisation de l'Action Publique Territoriale et d'Affirmation des Métropoles}}%%Translated%%

\newacronym[description={\textsl{Loi d’Orientation pour l’Aménagement et le Développement du Territoire}.}]{LOADT}{LOADT}{\textsl{Loi d’Orientation pour l’Aménagement et le Développement du Territoire}}%%Translated%%

\newacronym[description={\textsl{Loi d’Orientation pour l’Aménagement et le Développement Durable du Territoire}.}]{LOADDT}{LOADDT}{\textsl{Loi d’Orientation pour l’Aménagement et le Développement Durable du Territoire}}%%Translated%%

\newacronym[description={\textsl{Loi relative à la Solidarité et au Renouvellement Urbains}.}]{SRU}{SRU}{\textsl{Loi relative à la Solidarité et au Renouvellement Urbains}}%%Translated%%

\newacronym[description={\textsl{Schéma Régional de l’Intermodalité}.}]{SRI}{SRI}{\textsl{Schéma Régional de l’Intermodalité}}%%Translated%%

\newacronym[description={European Grouping of Territorial Cooperation.}]{EGTC}{EGTC}{European Grouping of Territorial Cooperation}%%Translated%%
% GECT

\newacronym[description={\textsl{Chamber of Commerce and Industry}.}]{CCI}{CCI}{\textsl{Chamber of Commerce and Industry}}%%Translated%%

\newacronym[description={Gross Domestic Product.}]{GDP}{GDP}{Gross Domestic Product}%%Translated%%
% PIB

\newacronym[description={\textsl{National Survey on People's Mobility}.}]{EMP}{EMP}{\textsl{National Survey on People's Mobility}}%%Translated%%

\newacronym[description={\textsl{Droit au Vélo.}.}]{ADAV}{ADAV}{\textsl{Droit au Vélo}}%%Translated%%

\newacronym[description={\textsl{Regional Scheme for Cycle Routes and Greenways}.}]{SR3V}{SR3V}{\textsl{Regional Scheme for Cycle Routes and Greenways}}%%Translated%%

\newacronym[description={\textsl{Club des villes et territoires cyclables et marchables}.}]{CVTCM}{CVTCM}{\textsl{Club des villes et territoires cyclables et marchables}.}%%Translated%%

\newacronym[description={\textsl{Fédération française des Usagers de la Bicyclette}.}]{FUB}{FUB}{\textsl{Fédération française des Usagers de la Bicyclette.}}%%Translated%%

\newacronym[description={Life Cycle Assessment.}]{LCA}{LCA}{Life Cycle Assessment}%%Translated%%
% ACV

\newacronym[description={\textsl{SCImago Journal Rank}.}]{SJR}{SJR}{\textsl{SCImago Journal Rank}}%%Translated%%

\newacronym[description={GPS Exchange Format.}]{GPX}{GPX}{GPS Exchange Format}%%Translated%%

\newacronym[description={\textsl{SiChuan University}.}]{SCU}{SCU}{\textsl{SiChuan University}}%%Translated%%

\newacronym[description={\textsl{What You See Is What You Get}.}]{WYSIWYG}{WYSIWYG}{\textsl{What You See Is What You Get}}%%Translated%%

\newacronym[description={\textsl{Sichuan University of Science \& Engineering}.}]{SUSE}{SUSE}{\textsl{Sichuan University of Science \& Engineering}}%%Translated%%

\newacronym[description={\textsl{Institute for Energy Transition}.}]{ITE}{ITE}{\textsl{Institute for Energy Transition}}%%Translated%%

\newacronym[description={Global Positioning System.}]{GPS}{GPS}{Global Positioning System}%%Translated%%

\newacronym[description={\textsl{National Transport and Travel Survey}.}]{ENDT}{ENDT}{\textsl{National Transport and Travel Survey}}%%Translated%%

\newacronym[description={\textsl{Household Travel Survey}.}]{EMD}{EMD}{\textsl{Household Travel Survey}}%%Translated%%

\newacronym[description={\textsl{Planification Régionale de l'Intermodalité}.}]{PRI}{PRI}{\textsl{Planification Régionale de l'Intermodalité}}%%Translated%%

\newacronym[description={\textsl{Planification Régionale des Infrastructures de Transports}.}]{PRIT}{PRIT}{\textsl{Planification Régionale des Infrastructures de Transports}}%%Translated%%

\newacronym[description={\textsl{State-Region Plan Contract}.}]{CPER}{CPER}{\textsl{State-Region Plan Contract}}%%Translated%%

\newacronym[description={\textsl{Agency for the Environment and Energy Management}.}]{ADEME}{ADEME}{\textsl{Agency for the Environment and Energy Management}}%%Translated%%

\newacronym[description={\textsl{Commercial Urban Planning Master Plan}.}]{SDUC}{SDUC}{\textsl{Commercial Urban Planning Master Plan}}%%Translated%%

\newacronym[description={\textsl{Urban Transport Perimeter}.}]{PTU}{PTU}{\textsl{Urban Transport Perimeter}}%%Translated%%

\newacronym[description={\textsl{Regional Cycle Route Scheme}.}]{SRV}{SRV}{\textsl{Regional Cycle Route Scheme}}%%Translated%%

\newacronym[description={\textsl{Regional Climate, Air, and Energy Plan}.}]{SRCAE}{SRCAE}{\textsl{Regional Climate, Air, and Energy Plan}}%%Translated%%

\newacronym[description={Higher Education and Research.}]{ESR}{ESR}{Higher Education and Research}%%Translated%%

\newacronym[description={\textsl{National Institute of Geographic and Forestry Information}.}]{IGN}{IGN}{\textsl{National Institute of Geographic and Forestry Information}}%%Translated%%

\newacronym[description={\textsl{Transport Infrastructure Master Plan}.}]{SDIT}{SDIT}{\textsl{Transport Infrastructure Master Plan}}%%Translated%%

\newacronym[description={\textsl{Plan de Déplacements Urbains}.}]{PDU}{PDU}{\textsl{Plan de Déplacements Urbains}}%%Translated%%

\newacronym[description={Lille Metropolitan Urban Community.}]{LMCU}{LMCU}{Lille Metropolitan Urban Community}%%Translated%%

\newacronym[description={\textsl{Public Land Establishment}.}]{EPF}{EPF}{\textsl{Public Land Establishment}}%%Translated%%

\newacronym[description={\textsl{Pedestrian Accessibility Zones}.}]{ZAP}{ZAP}{\textsl{Pedestrian Accessibility Zones}}%%Translated%%

\newacronym[description={\textsl{Bicycle Accessibility Zones}.}]{ZAV}{ZAV}{\textsl{Bicycle Accessibility Zones}}%%Translated%%

\newacronym[description={Sport Utility Vehicle.}]{SUV}{SUV}{Sport Utility Vehicle}%%Translated%%

\newacronym[description={Light Automated Vehicle.}]{VAL}{VAL}{Light Automated Vehicle}%%Translated%%

\newacronym[description={\textsl{Montpellier Geography and Urban Planning Laboratory}.}]{LAGAM}{LAGAM}{\textsl{Montpellier Geography and Urban Planning Laboratory}}%%Translated%%

\newacronym[description={\textsl{Google Popular Times}.}]{GPT}{GPT}{\textsl{Google Popular Times}}%%Translated%%

\newacronym[description={\textsl{Europe ENvironnement Ville Aménagement Réseau}.}]{ENVAR}{ENVAR}{\textsl{Europe ENvironnement Ville Aménagement Réseau}}%%Translated%%

\newacronym[description={\textsl{Congrès International d'Architecture Moderne}.}]{CIAM}{CIAM}{\textsl{Congrès International d'Architecture Moderne}}%%Translated%%


\newacronym[description={Traditional Neighborhood Design.}]{TND}{TND}{Traditional Neighborhood Design}%%Translated%%

\newacronym[description={Public-Private Partnership.}]{PPP}{PPP}{Public-Private Partnership}%%Translated%%

\newacronym[description={\textsl{Établissement Public d'Aménagement}.}]{EPA}{EPA}{\textsl{Établissement Public d'Aménagement}}%%Translated%%

\newacronym[description={\textsl{Plan Local d'Urbanisme intercommunal}.}]{PLUi}{PLUi}{\textsl{Plan Local d'Urbanisme intercommunal}}%%Translated%%

\newacronym[description={\textsl{Plan Local d'Urbanisme}.}]{PLU}{PLU}{\textsl{Plan Local d'Urbanisme}}%%Translated%%

\newacronym[description={\textsl{Programme Local de l'Habitat}.}]{PLH}{PLH}{\textsl{Programme Local de l'Habitat}}%%Translated%%

\newacronym[description={Transit Related Development.}]{TRD}{TRD}{Transit Related Development}%%Translated%%

\newacronym[description={Transportation Systems Management.}]{TSM}{TSM}{Transportation Systems Management}%%Translated%%

\newacronym[description={Transportation Demand Management.}]{TDM}{TDM}{Transportation Demand Management}%%Translated%%

\newacronym[description={\textsl{Master Plan for the Île-de-France Region}.}]{SDRIF}{SDRIF}{\textsl{Master Plan for the Île-de-France Region}}%%Translated%%

\newacronym[description={\textsl{Transport Axis Enhancement Discs}.}]{DIVAT}{DIVAT}{\textsl{Transport Axis Enhancement Discs}}%%Translated%%

\newacronym[description={Development-Oriented Transit.}]{DOT}{DOT}{Development-Oriented Transit}%%Translated%%

\newacronym[description={Automated Personal Transport.}]{PRT}{PRT}{Automated Personal Transport}%%Translated%%

\newacronym[description={Feeder-Distributor-Circulator Network.}]{F-D-C}{F-D-C}{Feeder-Distributor-Circulator Network}%%Translated%%

\newacronym[description={Mass Rapid Transit.}]{MRT}{MRT}{Mass Rapid Transit}%%Translated%%

\newacronym[description={\textsl{Housing Development Board}.}]{HDB}{HDB}{\textsl{Housing Development Board}}%%Translated%%

\newacronym[description={Mountain Bike.}]{VTT}{VTT}{Mountain Bike}%%Translated%%

\newacronym[description={\textsl{High Council for Research and Higher Education Assessment}.}]{Hcéres}{Hcéres}{\textsl{High Council for Research and Higher Education Assessment}}%%Translated%%

\newacronym[description={Nickel-Cadmium.}]{NiCd}{NiCd}{Nickel-Cadmium}%%Translated%%

\newacronym[description={Nickel-Metal Hydride.}]{NiMH}{NiMH}{Nickel-Metal Hydride}%%Translated%%

\newacronym[description={Lithium-ion.}]{Li-ion}{Li-ion}{Lithium-ion}%%Translated%%

\newacronym[description={\textsl{Organisation for Economic Co-operation and Development}.}]{OCDE}{OCDE}{\textsl{Organisation for Economic Co-operation and Development}}%%Translated%%

\newacronym[description={\textsl{French Institute of Science and Technology for Transport, Development and Networks}.}]{IFSTTAR}{IFSTTAR}{\textsl{French Institute of Science and Technology for Transport, Development and Networks}}%%Translated%%

\newacronym[description={\textsl{Public Establishment of a Scientific, Cultural, and Professional Nature}.}]{EPSCP}{EPSCP}{\textsl{Public Establishment of a Scientific, Cultural, and Professional Nature}}%%Translated%%

\newacronym[description={\textsl{Paris-Est Marne-la-Vallée University}.}]{UPEM}{UPEM}{\textsl{Paris-Est Marne-la-Vallée University}}%%Translated%%

\newacronym[description={\textsl{Joint Research Unit}.}]{UMR}{UMR}{\textsl{Joint Research Unit}}%%Translated%%

\newacronym[description={\textsl{National School of Bridges and Roads}.}]{ENPC}{ENPC}{\textsl{National School of Bridges and Roads}}%%Translated%%

\newacronym[description={\textsl{Nederlandse Spoorwegen}.}]{NS}{NS}{\textsl{Nederlandse Spoorwegen}}%%Translated%%

\newacronym[description={Nitric oxide.}]{NO}{NO}{nitric oxide}%%Translated%%

\newacronym[description={Shared Transport Services.}]{STP}{STP}{Shared Transport Services}%%Translated%%

\newacronym[description={European Union.}]{EU}{EU}{European Union}%%Translated%%
% UE

\newacronym[description={Internet of Things.}]{IoT}{IoT}{Internet of Things}%%Translated%%

\newacronym[description={\textsl{New Urbanist Memes for Transit-Oriented Teens}.}]{NUMTOT}{NUMTOT}{\textsl{New Urbanist Memes for Transit-Oriented Teens}}%%Translated%%

\newacronym[description={Demand-Responsive Transport.}]{DRT}{DRT}{Demand-Responsive Transport}%%Translated%%
% TaD

\newacronym[description={\textsl{Aménagement de l’espace URbain et mobilités à Faible impact Environnemental}.}]{URFé}{URFé}{Aménagement de l’espace URbain et mobilités à Faible impact Environnemental}%%Translated%%

\newacronym[description={\textsl{International Association of Public Transport}.}]{UITP}{UITP}{\textsl{International Association of Public Transport}}%%Translated%%
% UITP

\newacronym[description={Carbon monoxide.}]{CO}{CO}{carbon monoxide}%%Translated%%

\newacronym[description={\textsl{French Institute of Public Opinion}.}]{IFOP}{IFOP}{\textsl{French Institute of Public Opinion}}%%Translated%%

\newacronym[description={\textsl{What, Who, Where, When, How, How Much, Why}.}]{QQOQCCP}{QQOQCCP}{\textsl{What, Who, Where, When, How, How Much, Why}}%%Translated%%

\newacronym[description={User Governance.}]{MUS}{MUS}{User Governance}%%Translated%%

\newacronym[description={Project Management.}]{MOE}{MOE}{Project Management}%%Translated%%

\newacronym[description={Project Ownership.}]{MOU}{MOU}{Project Ownership}%%Translated%%

%%____________________________
% Glossaire
    \begin{refsegment}

% Accessibilité
\newglossaryentry{accessibility}{
    name={Accessibility~:},
    text={accessibility},
description={\scriptsize{\Commas{\textsl{The possibility of access to or from a place. Accessibility characterizes the level of service and strongly influences the level of property values. Accessibility can be measured from a point (residential location) in several ways~: by \Commas{all-or-nothing} [\dots] using isochrone curves [\dots] by averaging the generalized travel costs to different destinations [\dots] depending on transport supply and the system of activities [\dots] The latter formulation is the most satisfactory. Accessibility can be weighted for different types of opportunities (jobs, shopping locations, leisure spots, etc.).}} \textcolor{blue}{\autocite[5]{merlin_accessibilite_2023}}\index{Merlin, Pierre|pagebf}\index{Choay, Françoise|pagebf}. \Commas{\textsl{Thus, the concept always refers to the type of place to be reached, in other words, what one can access in that place. It also involves a measurement aspect. [\dots] It implies the elimination of barriers limiting access possibilities, regardless of their nature. [\dots] but also the transport systems and their intermodality [\dots]}} \textcolor{blue}{\autocite[11-12]{demailly_accessibilite_2021}}\index{Demailly, Kaduna-Eve|pagebf}\index{Monnet, Jérôme|pagebf}\index{Scapino, Julie|pagebf}\index{Deraëve, Sophie|pagebf}\index{Alauzet, Aline|pagebf}\index{Raton, Gwenaëlle|pagebf}.}
}}%%Translated%%

% Cartographie
\newglossaryentry{cartography}{
    name={Cartography~:},
    text={cartography},
description={\scriptsize{\Commas{\textsl{Cartography is a mode of communication, a \Commas{language}, that articulates concepts, graphic signs (diagrams, colors, symbol sizes, etc.), and a reference framework of geographical shapes, sometimes schematic but generally constructed using advanced technologies and mathematics (geodesy, projection, metrology, etc.).}} \textcolor{blue}{\autocite[55]{demailly_cartographie_2021}}\index{Demailly, Kaduna-Eve|pagebf}\index{Monnet, Jérôme|pagebf}\index{Scapino, Julie|pagebf}\index{Deraëve, Sophie|pagebf}\index{Hubert, Jean-Paul|pagebf}\index{Lefèvre, Quentin|pagebf}.}
}}%%Translated%%

% Coupure urbaine
\newglossaryentry{urban barrier}{
    name={Urban Barriers~:},
    text={urban barrier},
description={\scriptsize{\Commas{\textsl{An urban barrier is a linear or surface-based feature that disrupts the relationships between surrounding populations. It can be a natural discontinuity (rivers, steep slopes) or human-made, such as a transport infrastructure (highway, railway, canal, etc.), a large parcel open during the day and closed at night (parks, cemeteries, etc.), or a large block (industrial zone, shopping mall, dead-end housing development, etc). [\dots] Four types of simple barriers can be distinguished based on the hindrance to walking in the city~: impassable linear barriers (highway, river, railway, etc.), \Commas{traffic barriers} difficult to cross (a main road, etc.), dangerous roads to travel (a peripheral road without a sidewalk, etc.), and surface-based barriers (an airport, an inaccessible industrial zone, etc).}} \textcolor{blue}{\autocite[89]{demailly_coupures_2021}}\index{Demailly, Kaduna-Eve|pagebf}\index{Monnet, Jérôme|pagebf}\index{Scapino, Julie|pagebf}\index{Deraëve, Sophie|pagebf}\index{Héran, Frédéric|pagebf}.}
}}%%Translated%%

% Déplacement
\newglossaryentry{journey}{
    name={Journey~:},
    text={journey},
description={\scriptsize{\Commas{\textsl{Journey is the action of changing location, explained by a reason~: the need to go to another place to perform an activity that cannot be done at the original location. Journey uses one or more modes of transport, which can be characterized in an economic calculation by the financial cost and travel time specific to each mode to determine the utility of the journey.}} \textcolor{blue}{\autocite[96]{demailly_deplacement_2021}}\index{Demailly, Kaduna-Eve|pagebf}\index{Monnet, Jérôme|pagebf}\index{Scapino, Julie|pagebf}\index{Deraëve, Sophie|pagebf}. \Commas{\textsl{Movement of a person from an origin to a destination. The path taken using a specific mode of transport is called a trip. A movement may therefore require one or more trips.}} \textcolor{blue}{\autocite[243-244]{merlin_deplacement_2023}}\index{Merlin, Pierre|pagebf}\index{Choay, Françoise|pagebf}.}
}}%%Translated%%

% Détour
\newglossaryentry{detour}{
    name={Detour~:},
    text={detour},
    description={\scriptsize{\Commas{\textsl{A detour [\dots] is defined as any deviation from the most direct path connecting the origin to the destination of the movement. The observable realities of the detour and the direct path are closely related to the abstract notion of the straight line, which serves as a cognitive reference for movements. From this perspective, even the shortest path can be seen as marked by some form of detour, extending the concept to a relationship between the actual path and the abstraction of the straight line. [\dots] We thus propose to define three components of the detour~: an intrinsic component that is unavoidable in urban mobility, a component chosen by the pedestrian and therefore favorable to walking, and finally a component imposed on the pedestrian by urban designs beyond human scale, which are unfavorable to walking.}} \textcolor{blue}{\autocite[98]{demailly_detour_2021}}\index{Demailly, Kaduna-Eve|pagebf}\index{Monnet, Jérôme|pagebf}\index{Scapino, Julie|pagebf}\index{Deraëve, Sophie|pagebf}\index{L'Hostis, Alain|pagebf}.}
}}%%Translated%%

% Espace public
\newglossaryentry{public space}{
    name={Public Space~:},
    text={public space},
    description={\scriptsize{\Commas{\textsl{The concept of public space encompasses two main meanings, one more often used by philosophy and the social sciences, and the other referring more to the regulations governing circulation routes and places that accommodate the public.}} \textcolor{blue}{\autocite[128]{demailly_espace_2021}}\index{Demailly, Kaduna-Eve|pagebf}\index{Monnet, Jérôme|pagebf}\index{Scapino, Julie|pagebf}\index{Deraëve, Sophie|pagebf}\index{Monnet, Jérôme|pagebf}.}
}}%%Translated%%

% Genre
\newglossaryentry{gender}{
    name={Gender~:},
    text={gender},
    description={\scriptsize{\Commas{\textsl{The term \Commas{gender} refers to the meaning individuals assign to different categories of sex (beyond biological characteristics). It consists of a wide variety of attributes commonly associated with masculinity or femininity in a given society. [\dots] In intersectionality with age, ethnicity, culture, sexual orientation, and social class, it creates social and cognitive asymmetry.}} \textcolor{blue}{\autocite[160]{demailly_genre_2021}}\index{Demailly, Kaduna-Eve|pagebf}\index{Monnet, Jérôme|pagebf}\index{Scapino, Julie|pagebf}\index{Deraëve, Sophie|pagebf}\index{Faure, Emmanuelle|pagebf}\index{Granié, Marie-Axelle|pagebf}\index{Hernández González|pagebf}\index{Edna|pagebf}.}
}}%%Translated%%

% Itinéraire
\newglossaryentry{itinerary}{
    name={Itinerary~:},
    text={itinerary},
    description={\scriptsize{\Commas{\textsl{The itinerary, or the route, most commonly used, may correspond to a selected path, taken more or less frequently, thus habitually, though not necessarily fixed. Its utilitarian dimension, often highlighted, lies in the fact that it allows reaching a destination generally associated with a reason for travel. The path seems more random, undecided, and evokes an exploration of possibilities, leading to a form of urban serendipity. [\dots] Studying routes can reveal the sensitive link to the environment, particularly the numerous factors influencing the choices made when defining travel paths.}} \textcolor{blue}{\autocite[184]{demailly_itineraire_2021}}\index{Demailly, Kaduna-Eve|pagebf}\index{Monnet, Jérôme|pagebf}\index{Scapino, Julie|pagebf}\index{Deraëve, Sophie|pagebf}\index{Piombini, Arnaud|pagebf}\index{Meissonnier, Joël|pagebf}.}
}}%%Translated%%

% Marchabilité
\newglossaryentry{walkability}{
    name={Walkability~:},
    text={walkability},
    description={\scriptsize{\Commas{\textsl{Walkability, a neologism translating the English term, is considered in the scientific literature as an indicator to measure the potential for walking based on the characteristics and quality of a given built environment. The expression \Commas{pedestrian potential} is also used.}} \textcolor{blue}{\autocite[216]{demailly_marchabilite_2021}}\index{Demailly, Kaduna-Eve|pagebf}\index{Monnet, Jérôme|pagebf}\index{Scapino, Julie|pagebf}\index{Deraëve, Sophie|pagebf}\index{Huguenin-Richard, Florence|pagebf}\index{Cloutier, Marie-Soleil|pagebf}.}
}}%%Translated%%

% Métrique
\newglossaryentry{metric}{
    name={Metric~:},
    text={metric},
    description={\scriptsize{\Commas{\textsl{Each mode of movement defines a metric used to evaluate proximity or distance within a territory, whether urban or not. The metric depends on the speed allowed by a mode of transport, but it varies depending on the location, time (off-peak or peak), the loads being carried, and evolves with technologies, network shapes, etc. A pedestrian, automotive, or railway metric does not produce an objective ahistorical measure~: it mixes physiological or social temporalities with geographical frameworks, whether local or regional, such as public transport networks, urban blocks, or suburban housing developments. It is a measure of land use, to help orient oneself and, most importantly, to project oneself within it.}} \textcolor{blue}{\autocite[231]{demailly_metrique_2021}}\index{Demailly, Kaduna-Eve|pagebf}\index{Monnet, Jérôme|pagebf}\index{Scapino, Julie|pagebf}\index{Deraëve, Sophie|pagebf}\index{Hubert, Jean-Paul|pagebf}.}
}}%%Translated%%

% Micro-véhicule
\newglossaryentry{micro-vehicles}{
    name={Portable Micro-Vehicles~:},
    text={micro-vehicles},
    description={\scriptsize{\Commas{\textsl{Since the early 2010s, new \Commas{unidentified rolling objects}, small in size and designed to carry one person, have appeared in the public spaces of major cities around the world. This proliferation coincided with technological innovations in electric motorization~: on one hand, gyroscopic motors that can be housed within the wheels, and on the other hand, smaller batteries with enhanced power and duration performance. Alongside older forms (scooter, skateboard, rollerblades, bicycle) equipped with motors, new motorized devices (gyropod, unicycle) have appeared, and hybrids have been created (smartboard, gyroskate). It is likely that we are only at the beginning of an invention cycle with both technological and economic, social, and political stakes [\dots] they differ from the bicycle by their \Commas{portable} nature, which gives them a significant advantage for intermodality.}} \textcolor{blue}{\autocite[233]{demailly_micro-vehicules_2021}}\index{Demailly, Kaduna-Eve|pagebf}\index{Monnet, Jérôme|pagebf}\index{Scapino, Julie|pagebf}\index{Deraëve, Sophie|pagebf}\index{Monnet, Jérôme|pagebf}.}
}}%%Translated%%

% Modes actifs
\newglossaryentry{active modes}{
    name={Active Mobility~:},
    text={active modes},
    description={\scriptsize{\Commas{\textsl{It is therefore preferable to refer to active mobility, rather than subsuming walking under it, as it provides a clear, unequivocal definition, thus expressing not a judgment but a description, and ensuring that we do not inadvertently reintroduce mobilities that were previously dismissed. Once this terminological preference for active mobility is established, it is crucial to remember that it encompasses two main components, namely walking and cycling, components whose characteristics are distinctly different—thus, cycling (unlike walking) is a mechanized mode, and the speed and average range of these two modes vary from 1 to 4; we thus prefer, to avoid overlooking the clearly differentiated constraints and opportunities of these two modes, to speak of active mobilities, which nonetheless share their desirability both in urban, environmental, and health terms.}} \textcolor{blue}{\autocite[240-241]{demailly_mobilite_2021}}\index{Demailly, Kaduna-Eve|pagebf}\index{Monnet, Jérôme|pagebf}\index{Scapino, Julie|pagebf}\index{Deraëve, Sophie|pagebf}\index{Demade, Julien|pagebf}.}
}}%%Translated%%

% Multimodalité
\newglossaryentry{multimodality}{
    name={Multimodality~:},
    text={multimodality},
    description={\scriptsize{\Commas{\textsl{Multimodality, as a practice, refers to the use of different modes of transport or movement depending on the circumstances of the journey.}} \textcolor{blue}{\autocite[253]{demailly_multimodalite_2021}}\index{Demailly, Kaduna-Eve|pagebf}\index{Monnet, Jérôme|pagebf}\index{Scapino, Julie|pagebf}\index{Deraëve, Sophie|pagebf}\index{Chrétien, Julie|pagebf}. Multimodality is defined as \Commas{\textsl{the possibility of alternately using several modes of transport on the same route. It is also referred to as alternative intermodality. It is based on the notion of choice, and the multimodal customer will direct their mode choice differently depending on the day, time, or reason for their journey. They seek to optimize the use of the available transport options by leveraging the inherent performance advantages of each mode.}} \textcolor{blue}{\autocite[7]{souchon_intermodalite_2006}}\index{Souchon, Aurélie|pagebf}.}
}}%%Translated%%

% Intermodalité
\newglossaryentry{intermodality}{
    name={Intermodality~:},
    text={intermodality},
    description={\scriptsize{\Commas{\textsl{Intermodality refers to [\dots] the possibility of combining several modes of transport during a single journey. This practice allows for the aggregation of users towards major public transport routes, thus making them more cost-effective. However, it is seen as a constraint for users due to the interruptions caused by transfers and waiting times.}} \textcolor{blue}{\autocite[253]{demailly_multimodalite_2021}}\index{Demailly, Kaduna-Eve|pagebf}\index{Monnet, Jérôme|pagebf}\index{Scapino, Julie|pagebf}\index{Deraëve, Sophie|pagebf}\index{Chrétien, Julie|pagebf}. \Commas{\textsl{Intermodality was initially perceived by decision-makers as simply a physical proximity of different modes in the same location. But it is much more than that. It involves offering users integrated and efficient intermodal transport services in both space and time in order to compete with single-mode chains, particularly road-based ones. The capacity of transport hubs to effectively connect different modes and services is a central element in the success or failure of intermodal mobility solutions. [\dots] Intermodality, through the accessibility gains it provides, emerges as a key determinant of mobility. It is, therefore, at the heart of transport policies.}} \textcolor{blue}{\autocite[107-108]{chapelon_evaluation_2016}}\index{Chapelon, Laurent|pagebf}.}
}}%%Translated%%

% Pause
\newglossaryentry{break}{
    name={Break~:},
    text={break},
    description={\scriptsize{\Commas{\textsl{A break, defined as a stop for a limited duration during a journey [\dots] refers to \Commas{lateralization}, meaning the need to step away from the journey for a break, a pause, which then provides the necessary energy to continue the trip. [\dots] The forms of a break are multiple. First, like all movement, walking consumes energy that needs to be replenished. A break is a moment suitable for eating, drinking, or catching one's breath. [\dots] The break can also be mental. [\dots] The break is also facilitated by the \Commas{gamification} of the city [\dots] Through this urban function of the break, there is also the possibility of an aesthetic appropriation of the space.}} \textcolor{blue}{\autocite[279]{demailly_pause_2021}}\index{Demailly, Kaduna-Eve|pagebf}\index{Monnet, Jérôme|pagebf}\index{Scapino, Julie|pagebf}\index{Deraëve, Sophie|pagebf}\index{L'Hostis, Alain|pagebf}.}
}}%%Translated%%

% Perception
\newglossaryentry{perception}{
    name={Perception~:},
    text={perception},
    description={\scriptsize{\Commas{\textsl{Perception is a process of receiving or collecting stimuli and information from the environment, developed physiologically, cognitively, and emotionally. The terminology is often confused with that of representation and evaluation of an object, situation, or space. The perception of a place is the product of the subjective processing of a set of sensory information and the sociospatial representations of the place.}} \textcolor{blue}{\autocite[285]{demailly_perception_2021}}\index{Demailly, Kaduna-Eve|pagebf}\index{Monnet, Jérôme|pagebf}\index{Scapino, Julie|pagebf}\index{Deraëve, Sophie|pagebf}\index{Marchand, Dorothée|pagebf}\index{Urrutia, Enric Pol|pagebf}.}
}}%%Translated%%

% Périurbain
\newglossaryentry{peri-urban}{
    name={Peri-urban~:},
    text={peri-urban},
    description={\scriptsize{\Commas{\textsl{Peri-urban life is organized within a mosaic of natural, agricultural, or built spaces, forming an urban-rural network system. The mobility between places of residence, services, leisure, and employment shapes a complex and multifaceted peri-urban form, far from being solely characterized by commuting trips.}} \textcolor{blue}{\autocite[288]{demailly_periurbain_2021}}\index{Demailly, Kaduna-Eve|pagebf}\index{Monnet, Jérôme|pagebf}\index{Scapino, Julie|pagebf}\index{Deraëve, Sophie|pagebf}\index{Mancebo, François|pagebf}\index{Salles, Sylvie|pagebf}. \Commas{\textsl{Peri-urbanization refers to peripheral urbanization around urban agglomerations. [\dots] It is often mistakenly confused with rurbanization, which is the movement of urban centers toward predominantly rural areas. [\dots] Peri-urbanization primarily corresponds to a physical and economic necessity.}} \textcolor{blue}{\autocite[549]{merlin_periurbanisation_2023}}\index{Merlin, Pierre|pagebf}\index{Choay, Françoise|pagebf}.}
}}%%Translated%%

% Design
\newglossaryentry{design}{
    name={Urban Design~:},
    text={design},
    description={\scriptsize{\Commas{\textsl{The English term \Commas{urban design} is sometimes used as is in French due to the lack of a satisfactory equivalent that captures both the notion of \Commas{design} (the conception of a solution aimed at fulfilling a goal) and \Commas{drawing} (the formalization, often graphic, of this solution). Related terms are often used, such as \Commas{urban composition}, \Commas{urban art}, \Commas{urban project}, though they are not true synonyms. [\dots] This involves working on the relationships between public space and the buildings that adjoin it, in order to create a human-scale landscape, in terms of its dimensions, qualities, and ambiance. While the Modern Movement viewed the architectural building as an isolated object within the urban fabric, urban design, on the other hand, considers the placement of buildings in a way that produces both formal quality and functionality of public space.}} \textcolor{blue}{\autocite[386-387]{demailly_urban_2021}}\index{Demailly, Kaduna-Eve|pagebf}\index{Monnet, Jérôme|pagebf}\index{Scapino, Julie|pagebf}\index{Deraëve, Sophie|pagebf}\index{Hernandez, Frédérique|pagebf}\index{Pinson, Daniel|pagebf}.}
}}%%Translated%%

% Urbanisme tactique
\newglossaryentry{tactical urbanism}{
    name={Tactical Urbanism~:},
    text={tactical urbanism},
    description={\scriptsize{\Commas{\textsl{Tactical urbanism is a concept that encompasses a diverse set of small-scale interventions, often temporary, reversible, and experimental. [\dots] Their main goal is to reclaim public spaces for greater conviviality. [\dots] This diversity is semantically reflected in the proliferation of terms (do-it-yourself urbanism, makeshift, pop-up, guerrilla, insurgent, everyday, etc.). The subtle differences between these concepts depend on the project actors (citizen or institutional initiative), the relationship of interventions to the norm (authorized and legal or not), and the project process (prefiguration or not), as well as the intended objectives (functional, aesthetic, etc.). Overall, tactical urbanism reflects a critique of functionalist urban planning, which is seen as hostile to urban sociability practices where walking is an essential condition.}} \textcolor{blue}{\autocite[391-392]{demailly_urbanisme_2021}}\index{Demailly, Kaduna-Eve|pagebf}\index{Monnet, Jérôme|pagebf}\index{Scapino, Julie|pagebf}\index{Deraëve, Sophie|pagebf}\index{Gomes, Pedro|pagebf}\index{Demailly, Kaduna-Eve|pagebf}.}
}}%%Translated%%

% Vélo
\newglossaryentry{bicycle}{
    name={Bicycle~:},
    text={bicycle},
    description={\scriptsize{\Commas{\textsl{As with walking, its driving force is human, so it shares with walking the designation of an active mode of transport. However, the bicycle is considered a vehicle under the Highway Code. Like all vehicles, its rightful place is on the roadway. [\dots] After the war, bicycle use collapsed in France due to fierce competition from manufacturers of motorized two-wheelers who explicitly aimed to motorize cyclists. [\dots] In the 1970s-1990s, cyclists became rare in cities, and pedestrians as well as motorists even forgot their existence.}} \textcolor{blue}{\autocite[406]{demailly_velo_2021}}\index{Demailly, Kaduna-Eve|pagebf}\index{Monnet, Jérôme|pagebf}\index{Scapino, Julie|pagebf}\index{Deraëve, Sophie|pagebf}\index{Hiron, Benoît|pagebf}.}
}}%%Translated%%

% Sustainable
\newglossaryentry{sustainable}{
    name={Sustainable City~:},
    text={sustainable},
    description={\scriptsize{\Commas{\textsl{The deployment of the sustainable city, a city that is resilient to the challenges of the energy transition and climate change, and that allows its inhabitants to reduce their environmental impact, is often associated with the assignment of labels, technical standards, and all sorts of eco-technologies (energy efficiency, smart grids, vertical farming, real-time flow management, etc.). However, these systems, designed according to an engineering logic, do not integrate urbanity, the roughness of the city, or its real uses. [\dots] A truly sustainable urban approach also involves helping the collective construction, by the inhabitants, of a memory of the places. This also concerns relationships between city dwellers. How do we share space? What do we have in common? What challenges and constraints must we face and adapt to?}} \textcolor{blue}{\autocite[411]{demailly_ville_2021}}\index{Demailly, Kaduna-Eve|pagebf}\index{Monnet, Jérôme|pagebf}\index{Scapino, Julie|pagebf}\index{Deraëve, Sophie|pagebf}\index{Mancebo, François|pagebf}\index{Salles, Sylvie|pagebf}.}
}}%%Translated%%

% Transport en commun
\newglossaryentry{public transport}{
    name={Public Transport~:},
    text={public transport},
    description={\scriptsize{\Commas{\textsl{A transport system made available to the public in urban centers, which uses vehicles designed to accommodate multiple passengers simultaneously, with pricing, schedules, and routes planned and known in advance. Public transport is typically provided by buses, subways, trams, and commuter trains.}} \textcolor{blue}{\autocite[8]{boisclair_retisser_2013}}\index{Boisclair, Catherine|pagebf}. It should be differentiated from \Commas{\textsl{collective transport [which also represents] all modes of transport using vehicles designed to accommodate multiple passengers simultaneously. However, it can take various forms, including public transport [\dots]}} \textcolor{blue}{\autocite[14]{boucher_lamenagement_2011}}\index{Boucher, Isabelle|pagebf}\index{Fontaine, Nicolas|pagebf}.}
}}%%Translated%%

% Rabattement
\newglossaryentry{access}{
    name={Access~:},
    text={access},
    description={\scriptsize{\Commas{\textsl{A complementary transport mode that gathers users to use a primary mode of transport. Cars and bicycles are often used for access a railway station or subway station, where a parking lot, called a park-and-ride, is provided. It is mixed transport, primarily for commuting between the suburbs and the city center, intended to combine the advantages of individual transport [\dots] and public transport [\dots]}} \textcolor{blue}{\autocite[653]{merlin_rabattement_2023}}\index{Merlin, Pierre|pagebf}\index{Choay, Françoise|pagebf}. \Commas{\textsl{The access trip to the station corresponds to the first mode(s) of transport used to get there~: it is made between the origin of the journey (usually home) and the first station.}} \textcolor{blue}{\autocites[15]{hasiak_estimation_2018}[24]{hasiak_estimation_2023}}\index{Hasiak, Fabrice|pagebf}\index{Verdier, Laurent|pagebf}\index{Lannoy, Arnaud|pagebf}\index{Bodard, Géraldine|pagebf}\index{Palmier, Patrick|pagebf}.}
}}%%Translated%%

% Diffusion
\newglossaryentry{egress}{
    name={Egress~:},
    text={egress},
    description={\scriptsize{\Commas{\textsl{The egress trip corresponds to the mode(s) of transport used after the train journey~: it is made between the train station and the final destination.}} \textcolor{blue}{\autocites[15]{hasiak_estimation_2018}[24]{hasiak_estimation_2023}}\index{Hasiak, Fabrice|pagebf}\index{Verdier, Laurent|pagebf}\index{Lannoy, Arnaud|pagebf}\index{Bodard, Géraldine|pagebf}\index{Palmier, Patrick|pagebf}.}
}}%%Translated%%

% Conurbation
\newglossaryentry{conurbation}{
    name={Conurbation~:},
    text={conurbation},
    description={\scriptsize{\Commas{\textsl{A conurbation arises from the coalescence of urbanized areas [\dots] they have merged due to the abundance of certain resources (coal for the most part, but also iron ore or hydraulic power in less spectacular cases) and through the diffusion, step by step, of successful industrial forms. The conurbation consists of a proliferation of very little hierarchical built spaces, with no overall plan. The term [\dots] is perfectly applied to industrial complexes that emerged in the "black countries" of France, Belgium, or Germany in the 19th century, as well as to certain regions of the northeastern United States. Elsewhere, it is not used; it is better to speak of an urbanized region or a regional city, depending on the case.}} \textcolor{blue}{\autocite[201]{merlin_conurbation_2023}}\index{Merlin, Pierre|pagebf}\index{Choay, Françoise|pagebf}\index{Claval, Paul|pagebf}.}
}}%%Translated%%

% Accessibilité intermodale
\newglossaryentry{intermodal accessibility}{
    name={Intermodal Accessibility~:},
    text={intermodal accessibility},
    description={\scriptsize{\Commas{\textsl{Intermodal accessibility describes how one reaches places by combining several modes of transport for each trip, while multimodal accessibility considers access to places globally through all modes or combinations of modes. [\dots] Intermodal accessibility raises the issue of matching modes within a transport chain (intermodality), thus improving an existing transport system. Multimodal accessibility allows for the comparison of performance between different modal chains and focuses on modal shift from one type of chain (car-dominant) to another (public transport-dominant) deemed more satisfactory from a sustainable development perspective.}} \textcolor{blue}{\autocite[7]{lhostis_definir_2010}}\index{L'Hostis, Alain|pagebf}\index{Conesa, Alexis|pagebf}.}
}}%%Translated%%

% Micro-mobilité
\newglossaryentry{micromobility}{
    name={Micromobility~:},
    text={micromobility},
    description={\scriptsize{\Commas{\textsl{Micromobility refers to all forms of mobility based on motorized vehicles powered by electricity. The vehicles used for micromobility, often referred to as \Commas{Electric Personal Mobility Devices} (ePMDs), are distinguished by their portability compared to other modes. ePMDs are designed without a seat for the transport of a single person and are equipped with a non-thermal motor whose maximum speed, by construction, does not exceed 25 km/h. They are intended for short trips in urban and peri-urban environments. These modes of transport are innovative, low-carbon alternatives to thermal vehicles and, in combination with public transport, can provide a solution to the \Commas{last mile} mobility challenges (for example, between a public transport station and a workplace).}} \textcolor{blue}{\autocite{france_mobilites_observation_nodate}}\index{France Mobilités@\textsl{France Mobilités}|pagebf}.}
}}%%Translated%%

    \end{refsegment}