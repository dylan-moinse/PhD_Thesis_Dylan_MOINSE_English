% ___________________________________________
    % Detour angles
    %\newpage
    \setcounter{section}{0}
\chapterheader{Supplementary Notes on Angle Measurement}
\chapter{Appendices on the Determination of Angles in Detour Geometry}
    \label{annexes:calcul-detours}

    % Reference
\hyperref[annexes:calcul-detours]{Appendix~\ref{annexes:calcul-detours}} refers to the \hyperref[chap5:detours-pauses-optimisation]{section dedicated to detours and pauses} (page \pageref{chap5:detours-pauses-optimisation}), within \hyperref[chap5:titre]{Chapter~5} (page \pageref{chap5:titre}), and provides detailed calculations to determine the angles of intermodal movements involving geometric detours.%%Translated%%

    % ___________________________________________
    % Mini-table of contents
    \setcounter{tocdepth}{2}
    % Redefine the title of the local table of contents
    \renewcommand{\localcontentsname}{Structure of Appendix~\ref{annexes:calcul-detours}}
\localtableofcontents

    \newpage
    % Questionnaire structure
    \newpage
    \needspace{1\baselineskip} % Reserve space
    \sectionheader{Angle Calculation}
\section{Angle Calculation}
    \label{annexes:calcul-detours-angles}

    % Cartesian coordinates
\subsection{Cartesian Coordinates}
    \label{annexes:calcul-detours-angles-cartesien}

    % Cartesian coordinates
Using the geographical coordinates of each point \(A\), \(B\), and \(C\), expressed in latitude and longitude using the \acrshort{WGS84} reference system, we converted them into Cartesian coordinates (\(x\), \(y\), \(z\)) using the conversion formulas for a sphere.%%Translated%%

    \begin{equation*}
    \begin{array}{lclclclclcl}
    \displaystyle x_{(A, B, C)} &=& R* cos(latitude) * cos(longitude)\\\\
    \displaystyle y_{(A, B, C)} &=& R* cos(latitude) * sin(longitude)\\\\
    \displaystyle z_{(A, B, C)} &=& R* sin(latitude)\\\\
    \end{array}
    \end{equation*}

    \begin{align*}
    &\text{where:} \\
    &R \text{ represents the Earth's radius.}
    \end{align*}

    % Vectors AB and BC
\subsection{Vectors}
    \label{annexes:calcul-detours-angles-vecteurs}

    % Vectors AB and BC
Using the determined Cartesian coordinates \(x_{(A, B, C)}\), \(y_{(A, B, C)}\), and \(z_{(A, B, C)}\), we then calculated the vectors \(AB\) and \(BC\) by subtracting the Cartesian coordinates of points \(A\) and \(B\), and then \(B\) and \(C\).%%Translated%%

    \begin{equation*}
    \begin{array}{lclclclclcl}
    \displaystyle AB &=& (x_B - x_A, y_B - y_A, z_B - z_A)\\\\
    \displaystyle BC &=& (x_C - x_B, y_C - y_B, z_C - z_B)\\\\
    \end{array}
    \end{equation*}

    % Dot product and norm of vectors
\subsection{Dot Product and Norm of Vectors}
    \label{annexes:calcul-detours-angles-produit-scalaire}

    % Dot product and norm of vectors
From the obtained vectors \(AB\) and \(BC\), we calculated their dot product (\(AB \cdot BC\)) and their norms (\(||{AB}||\)) and (\(||{BC}||\)).%%Translated%%

    \begin{equation*}
    \begin{array}{lclclclclcl}
    \displaystyle AB \cdot BC &=& (x_{AB} * x_{BC}) + (y_{AB} * y_{BC}) + (z_{AB} * z_{BC})\\
    \end{array}
    \end{equation*}

    \begin{align*}
    &\text{where:} \\
    &x/y/z_{AB} \text{ represent the components of the vector AB;}\\
    &x/y/z_{BC} \text{ represent the components of the vector BC.}
    \end{align*}

    \begin{equation*}
    \begin{array}{lclclclclcl}
    \displaystyle ||AB|| &=& \sqrt{(x_{AB}^2 + y_{AB}^2 + z_{AB}^2)}\\
    \displaystyle ||BC|| &=& \sqrt{(x_{BC}^2 + y_{BC}^2 + z_{BC}^2)}\\
    \end{array}
    \end{equation*}

    % Angle
\subsection{Determination of Angles in Radians and Degrees}
    \label{annexes:calcul-detours-angles-produit-angle}

    % Angle
The last step involved measuring the angle at point \(B\) between the vectors \(AB\) and \(BC\), expressed in radians \(\delta\) and in degrees \(\alpha\).%%Translated%%

    \begin{equation*}
    \begin{array}{lclcl}
    \displaystyle \delta &=& \displaystyle\frac{AB \cdot BC}{||AB|| * ||BC||}\\\\
    \end{array}
    \end{equation*}

    \begin{align*}
    &\text{where:} \\
    &acos \text{ represents the $arccosine$ function.}\\
    \end{align*}

    \begin{equation*}
    \begin{array}{lclclclclcl}
    \displaystyle \alpha &=& \delta * (\displaystyle\frac{180}{\pi})\\\\
    \end{array}
    \end{equation*}