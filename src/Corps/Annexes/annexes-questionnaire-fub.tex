% ___________________________________________
    % French Cycling Users Federation Questionnaire
    %\newpage
    \setcounter{section}{0}
\chapterheader{French Cycling Users Federation Questionnaire}
\chapter{Appendices on the questionnaire used for the \textsl{Baromètre des Villes Cyclables}}
    \label{annexes:structure-questionnaire-fub}

    % Cross-reference
\hyperref[annexes:structure-questionnaire-fub]{Appendix~\ref{annexes:structure-questionnaire-fub}} refers to the \hyperref[section-chap4:cyclabilite-territoires-genre]{section dedicated to modeling the cyclability of territories in relation to the gendered practice of light individual mobility} (page \pageref{section-chap4:cyclabilite-territoires-genre}), within the framework of \hyperref[chap4:titre]{Chapter~4} (page \pageref{chap4:titre}), and aims to detail the structure of the questionnaire conducted by \textcolor{blue}{\textcite{fub_barometre_2021}}\index{FUB@\textsl{FUB}|pagebf}.%%Translated%%

    % ___________________________________________
    % Mini-sommaire
    \setcounter{tocdepth}{2}
    % Redefine the title of the local table of contents
    \renewcommand{\localcontentsname}{Structure of Appendix~\ref{annexes:structure-questionnaire-fub}}
\localtableofcontents

    % Description questions
    \newpage
    \needspace{1\baselineskip} % Reserve space
    \sectionheader{Detailed Questions from the Questionnaire}
\section{Detailed Questions from the Questionnaire}
    \label{annexes:structure-questionnaire-fub-questions}

    % Detailed Questions: T1
The first thematic category, labeled \Commas{Overall Perception} (\(T_{5}\)), explores a series of questions related to the appreciation of cycling (\(Q_{14}\)), the continuity of the cycling network (\(Q_{15}\)), potential conflicts between cyclists and pedestrians (\(Q_{16}\)), interactions with motor vehicles (\(Q_{17}\)), the density and speed of car traffic (\(Q_{18}\)), and the democratization of bicycle use (\(Q_{19}\)).

    % Detailed Questions: T2
The second section, focusing on \Commas{Safety} (\(T_{2}\)), delves into the feeling of safety while cycling (\(Q_{20}\)), both on main roads (\(Q_{21}\)), and in residential streets (\(Q_{22}\)), at junctions with neighboring municipalities (\(Q_{23}\)), and during intersection crossings (\(Q_{24}\)), while also addressing inclusivity towards vulnerable users, namely children and the elderly (\(Q_{25}\)).

    % Detailed Questions: T3
The third section is dedicated to \Commas{Comfort} (\(T_{3}\)), focusing on the quality level of cycling routes (\(Q_{26}\)), their maintenance (\(Q_{27}\)), the presence of road signage (\(Q_{28}\)), the establishment of temporary routes during construction works (\(Q_{29}\)), and the availability of one-way cycling paths (\(Q_{30}\)).

    % Detailed Questions: T4
The fourth thematic area, focusing on the \Commas{City's Efforts} (\(T_{4}\)), examines actions taken by the municipality to promote bicycle use (\(Q_{31}\)), the city's communication efforts (\(Q_{32}\)), the integration of cyclists in urban planning and mobility projects (\(Q_{33}\)), and issues related to obstructive car parking (\(Q_{34}\)).

    % Detailed Questions: T5
Finally, the last category, dedicated to \Commas{Services and Parking} (\(T_{5}\)), focuses on general bicycle parking facilities (\(Q_{35}\)), specifically around public transport stations (\(Q_{36}\)), bike rental services for short and long durations (\(Q_{37}\)), the availability of bike shops and repair workshops (\(Q_{38}\)), and the risk of bicycle thefts (\(Q_{39}\)).%%Translated%%

    % Questionnaire FUB Table
    \newpage
    \needspace{1\baselineskip} % Reserve space
    \sectionheader{Questionnaire Structure}
\section{Questionnaire Structure}
    \label{annexes:structure-questionnaire-fub-tableau}

    % FUB questionnaire structure table
% Table T1
%%Translated%%
  \begin{table}[h!]
    \centering
    \renewcommand{\arraystretch}{1.5}
    \resizebox{\columnwidth}{!}{
    \begin{tabular}{p{0.1\columnwidth}p{0.5\columnwidth}p{0.4\columnwidth}}
      % \hline
      \rule{0pt}{15pt} \textcolor{blue}{\textbf{\small{ID}}} & \textcolor{blue}{\textbf{\small{Question Title}}} & \textcolor{blue}{\textbf{\small{Responses}}}\\
      \hline
        \multicolumn{3}{l}{\textsl{\textbf{Theme~1: Overall feeling}} (\(T_{1}\))}\\
            \hdashline
    \small{\(Q_{14}\)} & \small{\textsl{Cycling in your municipality is\dots}} & \small{\textsl{Unpleasant / Pleasant}}\\
        \hdashline
    \small{\(Q_{15}\)} & \small{\textsl{The cycling route network in my municipality allows me to go anywhere quickly and directly}} & \small{\textsl{Not at all / Completely}}\\
        \hdashline
    \small{\(Q_{16}\)} & \small{\textsl{Conflicts between cyclists and pedestrians are\dots}} & \small{\textsl{Very frequent / Very rare}}\\
        \hdashline
    \small{\(Q_{17}\)} & \small{\textsl{When cycling, motor vehicle drivers respect me}} & \small{\textsl{Not at all / Completely}}\\
        \hdashline
    \small{\(Q_{18}\)} & \small{\textsl{When cycling, I find motor traffic (volume and speed) to be\dots}} & \small{\textsl{Intolerable / Not at all bothersome}}\\
        \hdashline
    \small{\(Q_{19}\)} & \small{\textsl{In my opinion, cycling in my municipality is\dots}} & \small{\textsl{Limited to some / Very widespread}}\\
        \hline
    \end{tabular}}
    \caption*{}
    \vspace{5pt}
        \begin{flushright}\scriptsize
        Source: \textcolor{blue}{\textcite{fub_barometre_2021}}
        \end{flushright}
        \end{table}

% Table T2
%%Translated%%
  \begin{table}[h!]
    \centering
    \renewcommand{\arraystretch}{1.5}
    \resizebox{\columnwidth}{!}{
    \begin{tabular}{p{0.1\columnwidth}p{0.5\columnwidth}p{0.4\columnwidth}}
      % \hline
      \rule{0pt}{15pt} \textcolor{blue}{\textbf{\small{ID}}} & \textcolor{blue}{\textbf{\small{Question Title}}} & \textcolor{blue}{\textbf{\small{Responses}}}\\
      \hline
        \multicolumn{3}{l}{\textsl{\textbf{Theme~2: Safety}} (\(T_{2}\))}\\
            \hdashline
    \small{\(Q_{20}\)} & \small{\textsl{In general, when I cycle in my municipality, I feel\dots}} & \small{\textsl{Safe / In danger}}\\
        \hdashline
    \small{\(Q_{21}\)} & \small{\textsl{I can cycle safely on the major roads in my municipality}} & \small{\textsl{Not at all / Completely}}\\
        \hdashline
    \small{\(Q_{22}\)} & \small{\textsl{I can cycle safely on residential streets}} & \small{\textsl{Not at all / Completely}}\\
        \hdashline
    \small{\(Q_{23}\)} & \small{\textsl{I can cycle safely to nearby municipalities}} & \small{\textsl{Not at all / Completely}}\\
        \hdashline
    \small{\(Q_{24}\)} & \small{\textsl{In my opinion, crossing an intersection or roundabout is\dots}} & \small{\textsl{Always dangerous / Never dangerous}}\\
        \hdashline
    \small{\(Q_{25}\)} & \small{\textsl{For children and seniors, cycling is\dots}} & \small{\textsl{Very dangerous / Very safe}}\\
        \hline
    \end{tabular}}
    \caption*{}
    \vspace{5pt}
        \begin{flushright}\scriptsize
        Source: \textcolor{blue}{\textcite{fub_barometre_2021}}
        \end{flushright}
        \end{table}

% Table T3
%%Translated%%
  \begin{table}[h!]
    \centering
    \renewcommand{\arraystretch}{1.5}
    \resizebox{\columnwidth}{!}{
    \begin{tabular}{p{0.1\columnwidth}p{0.5\columnwidth}p{0.4\columnwidth}}
      % \hline
      \rule{0pt}{15pt} \textcolor{blue}{\textbf{\small{ID}}} & \textcolor{blue}{\textbf{\small{Question Title}}} & \textcolor{blue}{\textbf{\small{Responses}}}\\
      \hline
        \multicolumn{3}{l}{\textsl{\textbf{Theme~3: Comfort}} (\(T_{3}\))}\\
            \hdashline
    \small{\(Q_{26}\)} & \small{\textsl{In my opinion, the cycling routes are\dots}} & \small{\textsl{Not at all comfortable / Very comfortable}}\\
        \hdashline
    \small{\(Q_{27}\)} & \small{\textsl{The maintenance of the cycling routes is\dots}} & \small{\textsl{Very poor / Very good}}\\
        \hdashline
    \small{\(Q_{28}\)} & \small{\textsl{Cycling directions are correctly indicated by signs}} & \small{\textsl{Not at all / Completely}}\\
        \hdashline
    \small{\(Q_{29}\)} & \small{\textsl{When there is construction on the cycling routes, a safe alternative is provided}} & \small{\textsl{Never / Always}}\\
        \hdashline
    \small{\(Q_{30}\)} & \small{\textsl{When cycling, I am allowed to use one-way streets in the opposite direction}} & \small{\textsl{Never / Always}}\\
        \hline
    \end{tabular}}
    \caption*{}
    \vspace{5pt}
        \begin{flushright}\scriptsize
        Source: \textcolor{blue}{\textcite{fub_barometre_2021}}
        \end{flushright}
        \end{table}

% Table T4
%%Translated%%
  \begin{table}[h!]
    \centering
    \renewcommand{\arraystretch}{1.5}
    \resizebox{\columnwidth}{!}{
    \begin{tabular}{p{0.1\columnwidth}p{0.5\columnwidth}p{0.4\columnwidth}}
      % \hline
      \rule{0pt}{15pt} \textcolor{blue}{\textbf{\small{ID}}} & \textcolor{blue}{\textbf{\small{Question Title}}} & \textcolor{blue}{\textbf{\small{Responses}}}\\
      \hline
        \multicolumn{3}{l}{\textsl{\textbf{Theme~4: Efforts of the municipality}} (\(T_{4}\))}\\
            \hdashline
    \small{\(Q_{31}\)} & \small{\textsl{In my opinion, the efforts made by the city in favor of cycling are\dots}} & \small{\textsl{Nonexistent / Significant}}\\
        \hdashline
    \small{\(Q_{32}\)} & \small{\textsl{Communication in favor of cycling is\dots}} & \small{\textsl{Nonexistent / Important}}\\
        \hdashline
    \small{\(Q_{33}\)} & \small{\textsl{The city hall listens to the needs of cyclists, involving them in its mobility reflections and planning projects}} & \small{\textsl{Never / Always}}\\
        \hdashline
    \small{\(Q_{34}\)} & \small{\textsl{In my opinion, parking of motorized vehicles (cars, trucks, motorcycles\dots) on cycling routes is\dots}} & \small{\textsl{Very frequent / Very rare}}\\
        \hline
    \end{tabular}}
    \caption*{}
    \vspace{5pt}
        \begin{flushright}\scriptsize
        Source: \textcolor{blue}{\textcite{fub_barometre_2021}}
        \end{flushright}
        \end{table}

% Table T5
%%Translated%%
  \begin{table}[h!]
    \centering
    \renewcommand{\arraystretch}{1.5}
    \resizebox{\columnwidth}{!}{
    \begin{tabular}{p{0.1\columnwidth}p{0.5\columnwidth}p{0.4\columnwidth}}
      % \hline
      \rule{0pt}{15pt} \textcolor{blue}{\textbf{\small{ID}}} & \textcolor{blue}{\textbf{\small{Question Title}}} & \textcolor{blue}{\textbf{\small{Responses}}}\\
      \hline
        \multicolumn{3}{l}{\textsl{\textbf{Theme 5: Services and parking}} (\(T_{5}\))}\\ 
            \hdashline
    \small{\(Q_{35}\)} & \small{\textsl{In the municipality or nearby, finding a parking space for my bicycle that suits my needs (duration, security\dots) is\dots}} & \small{\textsl{Impossible / Very easy}}\\
        \hdashline
    \small{\(Q_{36}\)} & \small{\textsl{Parking my bicycle at a train station or public transport station is\dots}} & \small{\textsl{Impossible / Very easy}}\\
        \hdashline
    \small{\(Q_{37}\)} & \small{\textsl{Renting a bicycle for a few hours or for several months is\dots}} & \small{\textsl{Impossible / Very easy}}\\
        \hdashline
    \small{\(Q_{38}\)} & \small{\textsl{In the municipality or nearby, finding a bike shop or repair workshop is\dots}} & \small{\textsl{Impossible / Very easy}}\\
        \hdashline
    \small{\(Q_{39}\)} & \small{\textsl{In my opinion, bicycle thefts are\dots}} & \small{\textsl{Very frequent / Very rare}}\\
      \hline
    \end{tabular}}
    \caption*{}
    \vspace{5pt}
        \begin{flushleft}\scriptsize
        \textcolor{blue}{Reading Guide:} In the \textsl{Baromètre des Villes Cyclables}, 26 questions (from \(Q_{14}\) to \(Q_{39}\)) are used to determine the average cycling rating of municipalities, across 5 themes (from \(T_{1}\) to \(T_{5}\)).
        \end{flushleft}
        \begin{flushright}\scriptsize
        Source: \textcolor{blue}{\textcite{fub_barometre_2021}}
        \end{flushright}
        \end{table}%%Translated%%