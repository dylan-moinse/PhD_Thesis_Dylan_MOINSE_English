%%%%%%%%%%%%%%%%%%%%%%%%%%%%%%%%
% Données ouvertes

%\begin{refsegment}
%%%% ___________________________________________________________________________________
% ANNEXES
    \newpage
\part*{Données ouvertes}
    \markboth{Données ouvertes}{}
    \addcontentsline{toc}{part}{Données ouvertes}
    \label{donnees-ouvertes:titre}
    \renewcommand{\thesection}{\Alph{section}}
    \setcounter{section}{0}

    % ___________________________________________
    % ANNEXE B~: Types de mobilité individuelle légère dans la RSL
    \newpage
\section{Listes des publications scientifiques intégrées dans la \acrshort{RSL} classées selon le type de \gls{mobilité individuelle légère} examinée}
    \label{donnees-ouvertes:rsl_publications_type_mobilite_individuelle_legere}
    \markboth{Listes des publications scientifiques intégrées dans la RSL classées selon le type de mobilité individuelle légère examinée}{}
    \markright{Listes des publications scientifiques intégrées dans la RSL classées selon le type de mobilité individuelle légère examinée}{}

    % Référence
Cette section s'attache à décrire la documentation analysée dans le cadre de la \acrfull{RSL} et présentée dans le \hyperref[chap2:titre]{chapitre 2} (page \pageref{chap2:titre}).\par

    % Annexe B.1
\subsection{Corpus examinant le vélo à usage personnel}
    \label{donnees-ouvertes:rsl_publications_velo_personnel}

% Tableau B.1
        \begin{longtable}{p{0.5cm}p{5.5cm}p{3cm}p{4cm}}
        \hline
        \small{\textcolor{blue}{\textbf{ID}}} & \small{\textcolor{blue}{\textbf{Références scientifiques (vélo personnel)}}} & \small{\textcolor{blue}{\textbf{Autres options}}} & \small{\textcolor{blue}{\textbf{Transports en commun}}}\\
        \hline
        \endhead
    \small{1} & \small{\textcite{abours_rapport_2015}}\index{Abours, Sylvie|pagebf} & \small{vélo pliant, VAE} & \small{Train, Métro, Tramway, Bus}\\
    \small{2} & \small{\textcite{advani_bicycle_2006}}\index{Advani, Mukti|pagebf} & \small{-} & \small{Bus}\\
    \small{3} & \small{\textcite{ann_examination_2019}}\index{Ann, Sangeetha|pagebf} & \small{-} & \small{Métro}\\
    \small{4} & \small{\textcite{arbis_analysis_2016}}\index{Arbis, David|pagebf} & \small{-} & \small{Train}\\
    \small{5} & \small{\textcite{arias_molinares_bike_2018}}\index{Arias Molinares, Daniela|pagebf} & \small{-} & \small{Train, Métro, Tramway}\\
    \small{6} & \small{\textcite{balya_integration_2016}}\index{Balya, Manjurali|pagebf} & \small{-} & \small{Bus}\\
    \small{7} & \small{\textcite{bauer_influence_2021}}\index{Bauer, Marek|pagebf} & \small{-} & \small{Train, Bus}\\
    \small{8} & \small{\textcite{bearn_adaption_2018}}\index{Bearn, Cary|pagebf} & \small{-} & \small{Métro}\\
    \small{9} & \small{\textcite{bechstein_cycling_2010}}\index{Bechstein, Eva|pagebf} & \small{-} & \small{Train}\\
    \small{10} & \small{\textcite{bopp_examining_2015}}\index{Bopp, Melissa|pagebf} & \small{-} & \small{Train, Métro, Tramway}\\
    \small{11} & \small{\textcite{brand_assessing_2015}}\index{Brand, Judith Caroline|pagebf} & \small{-} & \small{Bus}\\
    \small{12} & \small{\textcite{brand_modelling_2017}}\index{Brand, Judith Caroline|pagebf} & \small{-} & \small{Bus}\\
    \small{13} & \small{\textcite{cervero_bike-and-ride_2013}}\index{Cervero, Robert|pagebf} & \small{-} & \small{Métro}\\
    \small{14} & \small{\textcite{cervero_influences_2009}}\index{Cervero, Robert|pagebf} & \small{-} & \small{Bus}\\
    \small{15} & \small{\textcite{chan_factors_2020}}\index{Chan, Kevin|pagebf} & \small{-} & \small{Train}\\
    \small{16} & \small{\textcite{chen_determinants_2012}}\index{Chen, Lijun|pagebf} & \small{-} & \small{Métro}\\
    \small{17} & \small{\textcite{chen_study_2013}}\index{Chen, Wan|pagebf} & \small{-} & \small{Métro}\\
    \small{18} & \small{\textcite{cheng_evaluating_2012}}\index{Cheng, Yung-Hsiang|pagebf} & \small{-} & \small{Métro}\\
    \small{19} & \small{\textcite{cooke_relationship_2018}}\index{Cooke, Sean|pagebf} & \small{-} & \small{Bus}\\
    \small{20} & \small{\textcite{cottrell_transforming_2007}}\index{Cottrell, Wayne D.|pagebf} & \small{-} & \small{Bus}\\
    \small{21} & \small{\textcite{coue_embarq_2021}}\index{Coué, Antoine|pagebf} & \small{-} & \small{Train}\\
    \small{22} & \small{\textcite{souza_modelling_2017}}\index{Souza, Flavia de|pagebf} & \small{-} & \small{Train, Métro, Tramway, Bus}\\
    \small{23} & \small{\textcite{debrezion_modelling_2009}}\index{Debrezion, Ghebreegziabiher|pagebf} & \small{-} & \small{Train}\\
    \small{24} & \small{\textcite{djurhuus_building_2016}}\index{Djurhuus, Sune|pagebf} & \small{-} & \small{Train, Ferry, Métro, Tramway, Bus}\\
    \small{25} & \small{\textcite{doolittle_jr_integration_1994}}\index{Doolittle Jr, John T.|pagebf} & \small{-} & \small{Train, Ferry, Métro, Tramway, Bus}\\
    \small{26} & \small{\textcite{ensor_forecasting_2010}}\index{Ensor, Matt|pagebf} & \small{-} & \small{Train, Ferry, Métro, Tramway, Bus}\\
    \small{27} & \small{\textcite{fillone_i_2018}}\index{Fillone, Alexis|pagebf} & \small{-} & \small{Tramway, Bus}\\
    \small{28} & \small{\textcite{flamm_determinants_2013}}\index{Flamm, Bradley J.|pagebf} & \small{-} & \small{Bus}\\
    \small{29} & \small{\textcite{flamm_public_2014}}\index{Flamm, Bradley J.|pagebf} & \small{-} & \small{Train, Métro, Tramway, Bus}\\
    \small{30} & \small{\textcite{flamm_changes_2014}}\index{Flamm, Bradley J.|pagebf} & \small{-} & \small{Train, Métro, Tramway, Bus}\\
    \small{31} & \small{\textcite{flamm_perceptions_2014}}\index{Flamm, Bradley J.|pagebf} & \small{-} & \small{Train, Ferry, Métro, Tramway, Bus}\\
    \small{32} & \small{\textcite{fournier_continuous_2021}}\index{Fournier, Nicholas|pagebf} & \small{-} & \small{Train}\\
    \small{33} & \small{\textcite{gan_associations_2021}}\index{Gan, Zuoxian|pagebf} & \small{-} & \small{Métro}\\
    \small{34} & \small{\textcite{garcia-bello_methodological_2019}}\index{Marques, R.|pagebf} & \small{-} & \small{Train}\\
    \small{35} & \small{\textcite{geurs_multi-modal_2016}}\index{Geurs, Karst T.|pagebf} & \small{-} & \small{Train}\\
    \small{36} & \small{\textcite{giansoldati_train-feeder_2021}}\index{Giansoldati, Marco|pagebf} & \small{-} & \small{Train}\\
    \small{37} & \small{\textcite{givoni_access_2007}}\index{Givoni, Moshe|pagebf} & \small{-} & \small{Train}\\
    \small{38} & \small{\textcite{hagelin_return_2005}}\index{Hagelin, Christopher A.|pagebf} & \small{-} & \small{Bus}\\
    \small{39} & \small{\textcite{halldorsdottir_home-end_2017}}\index{Halldórsdóttir, Katrín|pagebf} & \small{-} & \small{Train}\\
    \small{40} & \small{\textcite{hamidi_shaping_2020}}\index{Hamidi, Zahra|pagebf} & \small{-} & \small{Train, Métro, Tramway, Bus}\\
    \small{41} & \small{\textcite{hamidi_inequalities_2019}}\index{Hamidi, Zahra|pagebf} & \small{-} & \small{Train, Bus}\\
    \small{42} & \small{\textcite{hasiak_access_2019}}\index{Hasiak, Sophie|pagebf} & \small{-} & \small{Train}\\
    \small{43} & \small{\textcite{heinen_multimodal_2014}}\index{Heinen, Eva|pagebf} & \small{-} & \small{Train, Métro, Tramway, Bus}\\
    \small{44} & \small{\textcite{hochmair_assessment_2015}}\index{Hochmair, Hartwig H.|pagebf} & \small{-} & \small{Train, Métro, Tramway, Bus}\\
    \small{45} & \small{\textcite{jansson_almeida_alternativas_2022}}\index{Jansson Almeida, Bárbara|pagebf} & \small{-} & \small{Métro}\\
    \small{46} & \small{\textcite{chen_demand_2013}}\index{Chen, Jingxu|pagebf} & \small{-} & \small{Métro}\\
    \small{47} & \small{\textcite{jonkeren_bicycle_2021}}\index{Jonkeren, Olaf|pagebf} & \small{-} & \small{Train}\\
    \small{48} & \small{\textcite{jonkeren_bicycle-train_2021}}\index{Jonkeren, Olaf|pagebf} & \small{-} & \small{Train}\\
    \small{49} & \small{\textcite{kager_synergies_2017}}\index{Kager, Roland|pagebf} & \small{-} & \small{Train}\\
    \small{50} & \small{\textcite{kager_characterisation_2016}}\index{Kager, Roland|pagebf} & \small{-} & \small{Train}\\
    \small{51} & \small{\textcite{keijer_how_2000}}\index{Keijer, Majanka|pagebf} & \small{-} & \small{Train}\\
    \small{52} & \small{\textcite{krizek_bicycling_2010}}\index{Krizek, Kevin J.|pagebf} & \small{-} & \small{Bus}\\
    \small{53} & \small{\textcite{krizek_assessing_2011}}\index{Krizek, Kevin|pagebf} & \small{-} & \small{Tramway, Bus}\\
    \small{54} & \small{\textcite{krizek_detailed_2007}}\index{Krizek, Kevin J.|pagebf} & \small{-} & \small{Tramway}\\
    \small{55} & \small{\textcite{krygsman_multimodal_2004}}\index{Krygsman, Stephan|pagebf} & \small{-} & \small{Train, Métro, Tramway, Bus}\\
    \small{56} & \small{\textcite{la_paix_puello_modelling_2015}}\index{La Paix Puello, Lissy|pagebf} & \small{-} & \small{Train}\\
    \small{57} & \small{\textcite{la_paix_puello_integration_2016}}\index{La Paix Puello, Lissy|pagebf} & \small{-} & \small{Train}\\
    \small{58} & \small{\textcite{la_paix_puello_Train_2016}}\index{La Paix Puello, Lissy|pagebf} & \small{-} & \small{Train}\\
    \small{59} & \small{\textcite{la_paix_puello_role_2021}}\index{La Paix Puello, Lissy|pagebf} & \small{-} & \small{Train}\\
    \small{60} & \small{\textcite{lee_bicycle-based_2016}}\index{Lee, Jaeyeong|pagebf} & \small{-} & \small{Métro}\\
    \small{61} & \small{\textcite{lee_strategies_2010}}\index{Lee, Jaeyeong|pagebf} & \small{-} & \small{Métro}\\
    \small{62} & \small{\textcite{li_exploring_2017}}\index{Li, Wenxiang|pagebf} & \small{-} & \small{Train, Bus}\\
    \small{63} & \small{\textcite{luan_better_2020}}\index{Luan, Xin|pagebf} & \small{-} & \small{Métro}\\
    \small{64} & \small{\textcite{marques_potential_2017}}\index{Marques, R.|pagebf} & \small{-} & \small{Train, Métro, Tramway}\\
    \small{65} & \small{\textcite{martens_bicycle_2004}}\index{Martens, Karel|pagebf} & \small{-} & \small{Train, Métro, Tramway, Bus}\\
    \small{66} & \small{\textcite{martens_promoting_2007}}\index{Martens, Karel|pagebf} & \small{-} & \small{Train, Ferry, Métro, Tramway, Bus}\\
    \small{67} & \small{\textcite{meng_influence_2016}}\index{Meng, Meng|pagebf} & \small{-} & \small{Métro}\\
    \small{68} & \small{\textcite{midenet_modal_2018}}\index{Midenet, Sophie|pagebf} & \small{-} & \small{Train}\\
    \small{69} & \small{\textcite{mohanty_effect_2017}}\index{Mohanty, Sudatta|pagebf} & \small{-} & \small{Train, Bus}\\
    \small{70} & \small{\textcite{molin_bicycle_2015}}\index{Molin, Eric|pagebf} & \small{-} & \small{Train}\\
    \small{71} & \small{\textcite{molino_les_2015}}\index{Molino, Marie|pagebf} & \small{-} & \small{Train, Métro, Tramway, Bus}\\
    \small{72} & \small{\textcite{nielsen_bikeability_2018}}\index{Nielsen, Thomas Alexander Sick|pagebf} & \small{-} & \small{Train, Bus}\\
    \small{73} & \small{\textcite{nigro_land_2019}}\index{Nigro, Antonio|pagebf} & \small{-} & \small{Train}\\
    \small{74} & \small{\textcite{oostendorp_combining_2018}}\index{Oostendorp, Rebekka|pagebf} & \small{-} & \small{Train, Métro, Tramway, Bus}\\
    \small{75} & \small{\textcite{pan_intermodal_2010}}\index{Pan, Haixiao|pagebf} & \small{-} & \small{Métro}\\
    \small{76} & \small{\textcite{papon_evaluation_2017}}\index{Papon, Francis|pagebf} & \small{-} & \small{Train}\\
    \small{77} & \small{\textcite{papon_rapport_2015}}\index{Papon, Francis|pagebf} & \small{-} & \small{Train}\\
    \small{78} & \small{\textcite{park_finding_2014}}\index{Park, Sungjin|pagebf} & \small{-} & \small{Train}\\
    \small{79} & \small{\textcite{pucher_integrating_2009}}\index{Pucher, John|pagebf} & \small{-} & \small{Train, Métro, Tramway, Bus}\\
    \small{80} & \small{\textcite{quarshie_integrating_2007}}\index{Quarshie, Magnus|pagebf} & \small{-} & \small{Bus}\\
    \small{81} & \small{\textcite{rastogi_willingness_2010}}\index{Rastogi, Rajat|pagebf} & \small{-} & \small{Train}\\
    \small{82} & \small{\textcite{rastogi_travel_2003}}\index{Rastogi, Rajat|pagebf} & \small{-} & \small{Train}\\
    \small{83} & \small{\textcite{ravensbergen_biking_2018}}\index{Ravensbergen, Léa|pagebf} & \small{-} & \small{Train}\\
    \small{84} & \small{\textcite{richer_service_2017}}\index{Richer, Cyprien|pagebf} & \small{-} & \small{Train}\\
    \small{85} & \small{\textcite{rietveld_accessibility_2000}}\index{Rietveld, Piet|pagebf} & \small{-} & \small{Train}\\
    \small{86} & \small{\textcite{rijsman_walking_2019}}\index{Rijsman, Lotte|pagebf} & \small{-} & \small{Tramway}\\
    \small{87} & \small{\textcite{risimati_spatial_2021}}\index{Risimati, Brightnes|pagebf} & \small{-} & \small{Train, Métro, Tramway, Bus}\\
    \small{88} & \small{\textcite{schneider_integration_2005}}\index{Schneider, Robert|pagebf} & \small{-} & \small{Ferry, transport à la demande}\\
    \small{89} & \small{\textcite{sebban_complementarite_2003}}\index{Sebban, Annie-Claude|pagebf} & \small{-} & \small{Train, Ferry, Métro, Tramway, Bus}\\
    \small{90} & \small{\textcite{shelat_analysing_2018}}\index{Shelat, Sanmay|pagebf} & \small{-} & \small{Train}\\
    \small{91} & \small{\textcite{sherwin_practices_2011}}\index{Sherwin, Henrietta|pagebf} & \small{-} & \small{Train}\\
    \small{92} & \small{\textcite{singleton_exploring_2014}}\index{Singleton, Patrick A.|pagebf} & \small{-} & \small{Métro}\\
    \small{93} & \small{\textcite{song_accessibility_2021}}\index{Song, Mingzhu|pagebf} & \small{-} & \small{Bus}\\
    \small{94} & \small{\textcite{staricco_implementing_2020}}\index{Staricco, Luca|pagebf} & \small{-} & \small{Train}\\
    \small{95} & \small{\textcite{stransky_quartiers_2017}}\index{Stransky, Václav|pagebf} & \small{-} & \small{Train}\\
    \small{96} & \small{\textcite{stransky_periurbain_2019}}\index{Stransky, Václav|pagebf} & \small{-} & \small{Train, Bus}\\
    \small{97} & \small{\textcite{taylor_analysis_1996}}\index{Taylor, Dean|pagebf} & \small{-} & \small{Bus}\\
    \small{98} & \small{\textcite{ton_understanding_2020}}\index{Ton, Danique|pagebf} & \small{-} & \small{Tramway}\\
    \small{99} & \small{\textcite{waerden_relation_2018}}\index{Waerden, Peter|pagebf} & \small{-} & \small{Train}\\
    \small{100} & \small{\textcite{kampen_understanding_2020}}\index{van Kampen, Jullian|pagebf} & \small{-} & \small{Train}\\
   \small{101} & \small{\textcite{kampen_bicycle_2021}}\index{van Kampen, Jullian|pagebf} & \small{-} & \small{Métro}\\
    \small{102} & \small{\textcite{kampen_understanding_2021}}\index{van Kampen, Jullian|pagebf} & \small{-} & \small{Train}\\
    \small{103} & \small{\textcite{van_mil_insights_2020}}\index{van Mil, Joeri F.P.|pagebf} & \small{-} & \small{Train}\\
    \small{104} & \small{\textcite{wang_bicycle-transit_2013}}\index{Wang, Rui|pagebf} & \small{-} & \small{Train, Bus}\\
    \small{105} & \small{\textcite{wang_interchange_2016}}\index{Wang, Zi-jia|pagebf} & \small{-} & \small{Métro}\\
    \small{106} & \small{\textcite{welch_long-term_2016}}\index{Welch, Timothy F.|pagebf} & \small{-} & \small{Tramway}\\
    \small{107} & \small{\textcite{weliwitiya_factors_2017}}\index{Weliwitiya, Hesara|pagebf} & \small{-} & \small{Train}\\
    \small{108} & \small{\textcite{weliwitiya_bicycle_2019}}\index{Weliwitiya, Hesara|pagebf} & \small{-} & \small{Train}\\
    \small{109} & \small{\textcite{zhang_make_2023}}\index{Zhang, Mengyuan|pagebf} & \small{-} & \small{Train}\\
    \small{110} & \small{\textcite{zhu_improved_2021}}\index{Zhu, Zhenjun|pagebf} & \small{-} & \small{Métro}\\
    \small{111} & \small{\textcite{zuo_bikeway_2019}}\index{Zuo, Ting|pagebf} & \small{-} & \small{Bus}\\
    \small{112} & \small{\textcite{zuo_promote_2020}}\index{Zuo, Ting|pagebf} & \small{-} & \small{Bus}\\
    \small{113} & \small{\textcite{zuo_incorporating_2021}}\index{Zuo, Ting|pagebf} & \small{-} & \small{Bus}\\
    \small{114} & \small{\textcite{zuo_determining_2018}}\index{Zuo, Ting|pagebf} & \small{-} & \small{Bus}\\
    \small{115} & \small{\textcite{zuo_first-and-last_2020}}\index{Zuo, Ting|pagebf} & \small{-} & \small{Bus}\\
        \hline
        \caption*{}
        \label{Annexe RSL sur le vélo personnel}
        \begin{flushright}
        \scriptsize
    Auteur~: \textcopyright~Moinse 2023
        \end{flushright}
        \end{longtable}

    \newpage
    % Annexe B.2
\subsection{Corpus examinant le \acrfull{VLS}}
    \label{donnees-ouvertes:rsl_publications_vls}

% Tableau B.2
        \begin{longtable}{p{0.5cm}p{5.5cm}p{3cm}p{4cm}}
        \hline
        \small{\textcolor{blue}{\textbf{ID}}} & \small{\textcolor{blue}{\textbf{Références scientifiques (VLS)}}} & \small{\textcolor{blue}{\textbf{Autres options}}} & \small{\textcolor{blue}{\textbf{Transports en commun}}}\\
        \hline
        \endhead
    \small{1} & \small{\textcite{adnan_last-mile_2019}}\index{Adnan, Muhammad|pagebf} & \small{-} & \small{Train, Métro, Tramway, Bus}\\
    \small{2} & \small{\textcite{aljeri_impacts_2020}}\index{Aljeri, Moathe|pagebf} & \small{-} & \small{Métro}\\
    \small{3} & \small{\textcite{andersson_neighbourhood_2021}}\index{Andersson, David Emanuel|pagebf} & \small{-} & \small{Métro}\\
    \small{4} & \small{\textcite{ashraf_impacts_2021}}\index{Ashraf, Md Tanvir|pagebf} & \small{-} & \small{Métro}\\
    \small{5} & \small{\textcite{bachand-marleau_much-anticipated_2011}}\index{Bachand-Marleau, Julie|pagebf} & \small{-} & \small{Train, Métro, Tramway, Bus}\\
    \small{6} & \small{\textcite{basu_planning_2021}}\index{Basu, Rounaq|pagebf} & \small{-} & \small{Train, Métro, Tramway, Bus}\\
    \small{7} & \small{\textcite{bi_analysis_2021}}\index{Bi, Hui|pagebf} & \small{-} & \small{Métro}\\
    \small{8} & \small{\textcite{bocker_bike_2020}}\index{Böcker, Lars|pagebf} & \small{-} & \small{Métro}\\
    \small{9} & \small{\textcite{caggiani_equality-based_2020}}\index{Caggiani, Leonardo|pagebf} & \small{Vélo personnel} & \small{Bus}\\
    \small{10} & \small{\textcite{cheng_exploring_2022}}\index{Cheng, Long|pagebf} & \small{-} & \small{Métro}\\
    \small{11} & \small{\textcite{cheng_expanding_2018}}\index{Cheng, Yung-Hsiang|pagebf} & \small{-} & \small{Métro}\\
    \small{12} & \small{\textcite{cho_estimation_2022}}\index{Cho, Shin-Hyung|pagebf} & \small{-} & \small{Métro}\\
    \small{13} & \small{\textcite{glass_role_2020}}\index{Glass, Caroline|pagebf} & \small{-} & \small{Bus}\\
    \small{14} & \small{\textcite{griffin_planning_2016}}\index{Griffin, Greg|pagebf} & \small{-} & \small{Train, Bus}\\
    \small{15} & \small{\textcite{gu_measuring_2019}}\index{Gu, Tianqi|pagebf} & \small{-} & \small{Métro}\\
    \small{16} & \small{\textcite{hua_transfer_2022}}\index{Hua, Mingzhuang|pagebf} & \small{-} & \small{Métro}\\
    \small{17} & \small{\textcite{iacobucci_transit_2017}}\index{Iacobucci, Joe|pagebf} & \small{-} & \small{Train, Bus}\\
    \small{18} & \small{\textcite{itf_integrating_2018}}\index{International Transport Forum|pagebf} & \small{-} & \small{Tramway, Bus}\\
    \small{19} & \small{\textcite{jappinen_modelling_2013}}\index{Jäppinen, Sakari|pagebf} & \small{-} & \small{Train, Ferry, Métro, Tramway, Bus}\\
    \small{20} & \small{\textcite{ji_public_2017}}\index{Ji, Yanjie|pagebf} & \small{-} & \small{Métro}\\
    \small{21} & \small{\textcite{ji_exploring_2018}}\index{Ji, Yanjie|pagebf} & \small{-} & \small{Métro}\\
    \small{22} & \small{\textcite{kim_analysis_2021}}\index{Kim, Minjun|pagebf} & \small{-} & \small{Métro, Bus}\\
    \small{23} & \small{\textcite{kong_deciphering_2020}}\index{Kong, Hui|pagebf} & \small{-} & \small{Train, Métro, Tramway, Bus}\\
    \small{24} & \small{\textcite{leth_competition_2017}}\index{Leth, Ulrich|pagebf} & \small{-} & \small{Train, Tramway, Bus}\\
    \small{25} & \small{\textcite{li_investigating_2022}}\index{Li, Xiaofeng|pagebf} & \small{-} & \small{Tramway, Bus}\\
    \small{26} & \small{\textcite{lin_built_2018}}\index{Lin, Jen-Jia|pagebf} & \small{-} & \small{Métro}\\
    \small{27} & \small{\textcite{liu_optimizing_2019}}\index{Liu, Lumei|pagebf} & \small{-} & \small{Métro}\\
    \small{28} & \small{\textcite{liu_understanding_2020}}\index{Liu, Yang|pagebf} & \small{-} & \small{Métro}\\
    \small{29} & \small{\textcite{liu_simultaneous_2015}}\index{Liu, Yang|pagebf} & \small{-} & \small{Bus}\\
    \small{30} & \small{\textcite{liu_solving_2012}}\index{Liu, Zhili|pagebf} & \small{-} & \small{Métro, Bus}\\
    \small{31} & \small{\textcite{lu_improving_2018}}\index{Lu, Miaojia|pagebf} & \small{-} & \small{Métro, Bus}\\
    \small{32} & \small{\textcite{ma_estimating_2019}}\index{Ma, Ting|pagebf} & \small{-} & \small{Métro}\\
    \small{33} & \small{\textcite{ma_bicycle_2015}}\index{Ma, Ting|pagebf} & \small{-} & \small{Métro}\\
    \small{34} & \small{\textcite{ma_understanding_2018}}\index{Ma, Xinwei|pagebf} & \small{-} & \small{Métro}\\
    \small{35} & \small{\textcite{ma_measuring_2018}}\index{Ma, Xinwei|pagebf} & \small{-} & \small{Métro}\\
    \small{36} & \small{\textcite{martin_evaluating_2014}}\index{Martin, Elliot W.|pagebf} & \small{-} & \small{Métro, Tramway}\\
    \small{37} & \small{\textcite{montes_shared_2023}}\index{Montes, Alejandro|pagebf} & \small{Vélo personnel} & \small{Métro, Tramway, Bus}\\
    \small{38} & \small{\textcite{nam_designing_2018}}\index{Nam, Daisik|pagebf}\index{Nam, Daisik|pagebf} & \small{-} & \small{Métro}\\
    \small{39} & \small{\textcite{radzimski_exploring_2021}}\index{Radzimski, Adam|pagebf} & \small{-} & \small{Tramway, Bus}\\
    \small{40} & \small{\textcite{romm_differences_2022}}\index{Romm, Daniel|pagebf} & \small{-} & \small{Métro}\\
    \small{41} & \small{\textcite{shah_b-t_2016}}\index{Shah, Neelkumar|pagebf} & \small{-} & \small{Bus}\\
    \small{42} & \small{\textcite{song_investigating_2020}}\index{Song, Ying|pagebf} & \small{-} & \small{Métro}\\
    \small{43} & \small{\textcite{tamakloe_determinants_2021}}\index{Tamakloe, Reuben|pagebf} & \small{-} & \small{Métro, Bus}\\
    \small{44} & \small{\textcite{tang_uncovering_2021}}\index{Tang, Jinjun|pagebf} & \small{-} & \small{Métro, Bus}\\
    \small{45} & \small{\textcite{tarpin-pitre_typology_2020}}\index{Tarpin-Pitre, Léandre|pagebf} & \small{-} & \small{Métro}\\
    \small{46} & \small{\textcite{tomita_demand_2017}}\index{Tomita, Yasuo|pagebf}\index{Tomita, Yasuo|pagebf} & \small{-} & \small{Train}\\
    \small{47} & \small{\textcite{kuijk_preferences_2022}}\index{van Kuijk, R.J.|pagebf} & \small{-} & \small{Tramway, Bus}\\
    \small{48} & \small{\textcite{wu_identification_2023}}\index{Wu, Hao|pagebf}\index{Wu, Hao|pagebf} & \small{-} & \small{Métro}\\
    \small{49} & \small{\textcite{wu_optimal_2020}}\index{Wu, Liyu|pagebf} & \small{-} & \small{Métro, Bus}\\
    \small{50} & \small{\textcite{yang_bike-and-ride_2014}}\index{Yang, Liu|pagebf}\index{Yang, Liu|pagebf} & \small{Vélo personnel} & \small{Train, Métro, Tramway, Bus}\\
    \small{51} & \small{\textcite{yang_empirical_2016}}\index{Yang, Min|pagebf} & \small{-} & \small{Métro}\\
    \small{52} & \small{\textcite{yang_metro_2015}}\index{Yang, Min|pagebf} & \small{Vélo personnel, VAE} & \small{Métro}\\
    \small{53} & \small{\textcite{yen_how_2023}}\index{Yen, Barbara T.H.|pagebf} & \small{-} & \small{Métro}\\
    \small{54} & \small{\textcite{yu_policy_2021}}\index{Yu, Qing|pagebf} & \small{-} & \small{Métro}\\
    \small{55} & \small{\textcite{zhao_bicycle-metro_2017}}\index{Zhao, Pengjun|pagebf} & \small{Vélo personnel} & \small{Métro}\\
    \small{56} & \small{\textcite{zhao_public_2022}}\index{Zhao, Pengjun|pagebf} & \small{-} & \small{Métro}\\
    \small{57} & \small{\textcite{zhao_impacts_2019}}\index{Zhao, Xing|pagebf} & \small{Vélo personnel} & \small{Métro}\\
        \hline
        \caption*{}
        \label{Annexe RSL sur le VLS}
        \begin{flushright}
        \scriptsize
    Auteur~: \textcopyright~Moinse 2023
        \end{flushright}
        \end{longtable}

    \newpage
    % Annexe B.3
\subsection{Corpus examinant le \acrfull{VFF}}
    \label{donnees-ouvertes:rsl_publications_vff}

% Tableau B.3
        \begin{longtable}{p{0.5cm}p{5.5cm}p{3cm}p{4cm}}
        \hline
        \small{\textcolor{blue}{\textbf{ID}}} & \small{\textcolor{blue}{\textbf{Références scientifiques (VFF)}}} & \small{\textcolor{blue}{\textbf{Autres options}}} & \small{\textcolor{blue}{\textbf{Transports en commun}}}\\
        \hline
        \endhead
    \small{1} & \small{\textcite{cai_system_2021}}\index{Cai, Jianming|pagebf} & \small{-} & \small{Train, Métro, Tramway}\\
    \small{2} & \small{\textcite{chen_what_2022}}\index{Chen, Wendong|pagebf} & \small{VLS} & \small{Métro}\\
    \small{3} & \small{\textcite{cheng_comparison_2023}}\index{Cheng, Long|pagebf} & \small{VLS} & \small{Métro}\\
    \small{4} & \small{\textcite{cheng_exploring_2022}}\index{Cheng, Long|pagebf} & \small{-} & \small{Métro}\\
    \small{5} & \small{\textcite{chu_last_2021}}\index{Chu, Junhong|pagebf} & \small{-} & \small{Métro}\\
    \small{6} & \small{\textcite{fan_how_2019}}\index{Fan, Aihua|pagebf} & \small{-} & \small{Train, Métro, Tramway, Bus}\\
    \small{7} & \small{\textcite{fan_dockless_2020}}\index{Fan, Yichun|pagebf} & \small{-} & \small{Métro}\\
    \small{8} & \small{\textcite{van_goeverden_potential_2018}}\index{van Goeverden, Kees|pagebf} & \small{VLS} & \small{Train}\\
    \small{9} & \small{\textcite{guo_exploring_2023}}\index{Guo, Dongbo|pagebf} & \small{-} & \small{Métro, Bus}\\
    \small{10} & \small{\textcite{guo_built_2020}}\index{Guo, Yuanyuan|pagebf} & \small{-} & \small{Métro}\\
    \small{11} & \small{\textcite{guo_role_2021}}\index{Guo, Yuanyuan|pagebf} & \small{-} & \small{Métro}\\
    \small{12} & \small{\textcite{guo_dockless_2021}}\index{Guo, Yuanyuan|pagebf} & \small{-} & \small{Métro}\\
    \small{13} & \small{\textcite{hu_study_2019}}\index{Hu, Li|pagebf} & \small{-} & \small{Métro}\\
    \small{14} & \small{\textcite{hu_examining_2022}}\index{Hu, Songhua|pagebf} & \small{-} & \small{Métro}\\
    \small{15} & \small{\textcite{jin_competition_2019}}\index{Jin, Haitao|pagebf} & \small{-} & \small{Métro, Bus}\\
    \small{16} & \small{\textcite{li_integration_2020}}\index{Li, Jie|pagebf} & \small{-} & \small{Métro}\\
    \small{17} & \small{\textcite{li_unbalanced_2022}}\index{Li, Lili|pagebf} & \small{-} & \small{Métro}\\
    \small{18} & \small{\textcite{li_exploring_2021}}\index{Li, Wei|pagebf} & \small{-} & \small{Métro}\\
    \small{19} & \small{\textcite{li_factors_2020}}\index{Li, Xuefeng|pagebf} & \small{-} & \small{Métro}\\
    \small{20} & \small{\textcite{li_operating_2019}} & \small{-} & \small{Métro}\\
    \small{21} & \small{\textcite{lin_analysis_2019}}\index{Lin, Diao|pagebf} & \small{-} & \small{Métro}\\
    \small{22} & \small{\textcite{liu_mode_2022}}\index{Liu, Lumei|pagebf} & \small{-} & \small{Métro, Bus}\\
    \small{23} & \small{\textcite{liu_temporal_2022}}\index{Liu, Siyang|pagebf} & \small{-} & \small{Métro}\\
    \small{24} & \small{\textcite{liu_concordance_2022}}\index{Liu, Siyang|pagebf} & \small{-} & \small{Métro}\\
    \small{25} & \small{\textcite{ma_impacts_2019}}\index{Ma, Xiaolei|pagebf} & \small{-} & \small{Bus}\\
    \small{26} & \small{\textcite{ni_exploring_2020}}\index{Ni, Ying|pagebf} & \small{-} & \small{Métro}\\
    \small{27} & \small{\textcite{pan_reposition_2023}}\index{Pan, Xiaoyi|pagebf} & \small{-} & \small{Métro}\\
    \small{28} & \small{\textcite{qiu_interplay_2021}}\index{Qiu, Waishan|pagebf} & \small{-} & \small{Bus}\\
    \small{29} & \small{\textcite{van_der_nat_bicycle_2018}}\index{van der Nat, Johanna Debóra|pagebf} & \small{VLS} & \small{Train}\\
    \small{30} & \small{\textcite{wang_relationship_2020}}\index{Wang, Ruoyu|pagebf} & \small{-} & \small{Métro}\\
    \small{31} & \small{\textcite{wang_spatiotemporal_2020}}\index{Wang, Zijia|pagebf} & \small{-} & \small{Métro}\\
    \small{32} & \small{\textcite{wu_measuring_2019}}\index{Wu, Xueying|pagebf} & \small{-} & \small{Métro}\\
    \small{33} & \small{\textcite{yang_spatiotemporal_2019}}\index{Yang, Yuanxuan|pagebf} & \small{-} & \small{Métro}\\
    \small{34} & \small{\textcite{yu_understanding_2021}}\index{Yu, Senbin|pagebf} & \small{-} & \small{Métro}\\
    \small{35} & \small{\textcite{zhang_bicyclemetro_2019}}\index{Zhang, Ze|pagebf} & \small{-} & \small{Métro}\\
    \small{36} & \small{\textcite{zhong_layout_2021}}\index{Zhong, Hongming|pagebf} & \small{-} & \small{Métro}\\
    \small{37} & \small{\textcite{zhou_spatially_2023}}\index{Zhou, X.|pagebf} & \small{-} & \small{Métro, Bus}\\
        \hline
        \caption*{}
        \label{Annexe RSL sur le VFF}
        \begin{flushright}
        \scriptsize
    Auteur~: \textcopyright~Moinse 2023
        \end{flushright}
        \end{longtable}

    \newpage
    % Annexe B.4
\subsection{Corpus examinant la \acrfull{TEP}}
    \label{donnees-ouvertes:rsl_publications_tep}

% Tableau B.4
        \begin{longtable}{p{0.5cm}p{5.5cm}p{3cm}p{4cm}}
        \hline
        \small{\textcolor{blue}{\textbf{ID}}} & \small{\textcolor{blue}{\textbf{Références scientifiques (TEP)}}} & \small{\textcolor{blue}{\textbf{Autres options}}} & \small{\textcolor{blue}{\textbf{Transports en commun}}}\\
        \hline
        \endhead
    \small{1} & \small{\textcite{ensor_mode_2021}}\index{Ensor, Matt|pagebf} & \small{VAE} & \small{Train, Ferry, Métro, Tramway, Bus}\\
    \small{2} & \small{\textcite{moinse_intermodal_2022}}\index{Moinse, Dylan|pagebf} & \small{Trottinette mécanique, vélo personnel} & \small{Train}\\
    \small{3} & \small{\textcite{pages_les_2021}}\index{Pages, Thibaud|pagebf} & \small{-} & \small{Train, Métro, Tramway, Bus}\\
    \small{4} & \small{\textcite{rabaud_quand_2022}}\index{Rabaud, Mathieu|pagebf} & \small{-} & \small{Train, Métro, Tramway, Bus}\\
    \small{5} & \small{\textcite{tzouras_describing_2023}}\index{Tzouras, Panagiotis|pagebf} & \small{-} & \small{}\\
        \hline
        \caption*{}
        \label{Annexe RSL sur la TEP}
        \begin{flushright}
        \scriptsize
    Auteur~: \textcopyright~Moinse 2023
        \end{flushright}
        \end{longtable}

    \newpage
    % Annexe B.5
\subsection{Corpus examinant la \acrfull{TEFF}} \label{Corpus examinant la trottinette électrique en free-floating}
    \label{donnees-ouvertes:rsl_publications_teff}

% Tableau B.5
        \begin{longtable}{p{0.5cm}p{5.5cm}p{3cm}p{4cm}}
        \hline
        \small{\textcolor{blue}{\textbf{ID}}} & \small{\textcolor{blue}{\textbf{Références scientifiques (TEFF)}}} & \small{\textcolor{blue}{\textbf{Autres options}}} & \small{\textcolor{blue}{\textbf{Transports en commun}}}\\
        \hline
        \endhead
    \small{1} & \small{\textcite{baek_electric_2021}}\index{Baek, Kwangho|pagebf} & \small{-} & \small{Métro}\\
    \small{2} & \small{\textcite{beale_integrating_2023}}\index{Beale, Kirsten|pagebf} & \small{VFF} & \small{Métro, Tramway}\\
    \small{3} & \small{\textcite{cervero_transit_2020}}\index{Cervero, Robert|pagebf} & \small{VFF} & \small{Train, Métro, Tramway}\\
    \small{4} & \small{\textcite{grosshuesch_solving_2020}}\index{Grosshuesch, Kelly|pagebf} & \small{VFF} & \small{Train, Métro, Tramway, Bus}\\
    \small{5} & \small{\textcite{park_first-last-mile_2021}}\index{Park, Keunhyun|pagebf} & \small{VFF} & \small{Train, Métro, Tramway, Bus}\\
    \small{6} & \small{\textcite{cao_e-scooter_2021}}\index{Cao, Zhejing|pagebf} & \small{-} & \small{Métro}\\
    \small{7} & \small{\textcite{tyndall_complementarity_2022}}\index{Tyndall, Justin|pagebf} & \small{-} & \small{Train}\\
    \small{8} & \small{\textcite{fearnley_delte_2020}}\index{Fearnley, Nils|pagebf} & \small{-} & \small{Train, Métro, Tramway, Bus}\\
    \small{9} & \small{\textcite{ganesh_multi-modal_2020}}\index{Ganesh, Aditya|pagebf} & \small{-} & \small{Tramway, Bus}\\
    \small{10} & \small{\textcite{heumann_spatiotemporal_2021}}\index{Heumann, Maximilian|pagebf} & \small{-} & \small{Train, Métro, Tramway}\\
    \small{11} & \small{\textcite{lee_forecasting_2021}}\index{Lee, Mina|pagebf} & \small{-} & \small{Métro}\\
    \small{12} & \small{\textcite{liu_measuring_2022}}\index{Liu, Lumei|pagebf} & \small{-} & \small{Bus}\\
    \small{13} & \small{\textcite{ma_connecting_2022}}\index{Ma, Qingyu|pagebf} & \small{VLS} & \small{Métro}\\
    \small{14} & \small{\textcite{mcqueen_assessing_2022}}\index{McQueen, Michael|pagebf} & \small{-} & \small{Tramway}\\
    \small{15} & \small{\textcite{mohammadian_analyzing_2022}}\index{Mohammadian, Abolfazl|pagebf} & \small{-} & \small{Train, Métro, Tramway, Bus}\\
    \small{16} & \small{\textcite{schlueter_langdon_how_2021}}\index{Schlueter Langdon, Chris|pagebf} & \small{VLS, vélo personnel} & \small{Train, Bus}\\
    \small{17} & \small{\textcite{vinagre_diaz_blind_2023}}\index{Vinagre Díaz, Juan José|pagebf} & \small{-} & \small{Train, Métro, Tramway, Bus}\\
    \small{18} & \small{\textcite{yan_spatiotemporal_2021}}\index{Yan Xiang|pagebf} & \small{-} & \small{Métro, Bus}\\
    \small{19} & \small{\textcite{yan_evaluating_2023}}\index{Yan, Xiang|pagebf} & \small{-} & \small{Métro, Bus}\\
    \small{20} & \small{\textcite{ziedan_complement_2021}}\index{Ziedan, Abubakr|pagebf} & \small{-} & \small{Bus}\\
    \small{21} & \small{\textcite{zuniga-garcia_evaluation_2022}}\index{Zuniga-Garcia, Natalia|pagebf} & \small{-} & \small{Bus}\\
        \hline
        \caption*{}
        \label{Annexe RSL sur la TEFF}
        \begin{flushright}
        \scriptsize
    Auteur~: \textcopyright~Moinse 2023
        \end{flushright}
        \end{longtable}

    % ___________________________________________
    % ANNEXE C~: Sources de données RSL
    \newpage
\section{Classification des méthodes de collecte des données employées au sein de la littérature scientifique en lien avec le \acrshort{B-TOD}}
    \label{donnees-ouvertes:rsl_publications_methodes_collecte_donnees}
    \markboth{Classification des méthodes de collecte des données employées au sein de la littérature scientifique en lien avec le B-TOD}{}
    \markriht{Classification des méthodes de collecte des données employées au sein de la littérature scientifique en lien avec le B-TOD}{}

    % Référence
Cette section détaille les méthodes de recherche employées au sein des études intégrées dans la \acrshort{RSL} et analysées dans la \hyperref[Évaluation des méthodes de collecte de données]{deuxième section du chapitre 2} (page \pageref{Évaluation des méthodes de collecte de données}).\par

    % Annexe C.1
\subsection{Répartition des sources de données selon le contexte géographique}
    \label{donnees-ouvertes:rsl_publications_sources_donnees_par_continent}

% Tableau C.1
        \begin{table}[h]
        \centering
        \renewcommand{\arraystretch}{1.5}
        \begin{tabular}{p{3.9cm}p{1.1cm}p{1.1cm}p{1.1cm}p{1.1cm}p{1.1cm}p{1.1cm}p{1.1cm}}
        \hline
    \rule{0pt}{15pt} \textcolor{blue}{\textbf{Sources des données}} & \textcolor{blue}{\textbf{Afrique}} & \textcolor{blue}{\textbf{AdN*}} & \textcolor{blue}{\textbf{AdS*}} & \textcolor{blue}{\textbf{Asie}} & \textcolor{blue}{\textbf{Europe}} & \textcolor{blue}{\textbf{Océanie}} & \textbf{Total}
    \\
            \hline
        & \multicolumn{6}{l}{\textbf{\textsl{Big data}}} & \textbf{200}\\
        \textcolor{blue}{\textbf{\textsl{Open data}}} & 3 & 21 & 0 & 42 & 16 & 2 & 84\\
        \textcolor{blue}{\textbf{API}} & 0 & 9 & 0 & 31 & 2 & 0 & 42\\
        \textcolor{blue}{\textbf{GBFS}} & 0 & 15 & 0 & 16 & 3 & 0 & 34\\
        \textcolor{blue}{\textbf{GTFS}} & 0 & 9 & 0 & 14 & 2 & 0 & 25\\
        \textcolor{blue}{\textbf{Carte multimodale}} & 0 & 0 & 0 & 8 & 0 & 0 & 8\\
        \textcolor{blue}{\textbf{GPS}} & 3 & 0 & 0 & 0 & 4 & 0 & 7\\
            \hline
        & \multicolumn{6}{l}{\textbf{Enquête par questionnaire}} & \textbf{80}\\
        \textcolor{blue}{\textbf{En ligne}} & 0 & 14 & 0 & 15 & 17 & 1 & 47\\
        \textcolor{blue}{\textbf{En face-à-face}} & 0 & 3 & 2 & 16 & 7 & 0 & 28\\
        \textcolor{blue}{\textbf{Voie postale}} & 0 & 4 & 0 & 1 & 0 & 0 & 5\\
            \hline
        & \multicolumn{6}{l}{\textbf{Analyse secondaire d'une enquête publique}} & \textbf{62}\\
        \textcolor{blue}{\textbf{Analyse secondaire}} & 2 & 14 & 2 & 5 & 34 & 5 & 62\\
            \hline
        & \multicolumn{6}{l}{\textbf{État de l'art}} & \textbf{20}\\
        \textcolor{blue}{\textbf{État de l'art}} & 0 & 8 & 1 & 2 & 9 & 0 & 20\\
            \hline
        & \multicolumn{6}{l}{\textbf{Enquête par entretien}} & \textbf{12}\\
        \textcolor{blue}{\textbf{Individuel}} & 0	& 4 & 0 & 0 & 3 & 0 & 7\\
        \textcolor{blue}{\textbf{Collectif}} & 0 & 1 & 3 & 0 & 0 & 0 & 4\\
        \textcolor{blue}{\textbf{En ligne}} & 0 & 1 & 0 & 0 & 0 & 0 & 1\\
            \hline
        & \multicolumn{6}{l}{\textbf{Observation}} & \textbf{11}\\
        \textcolor{blue}{\textbf{Directe}} & 0 & 3 & 0 & 0 & 5 & 0 & 8\\
        \textcolor{blue}{\textbf{Secondaire}} & 0 & 1 & 0 & 1 & 1 & 0 & 3\\
             \hline
        & \multicolumn{6}{l}{\textbf{Algorithme}} & \textbf{1}\\
        \textcolor{blue}{\textbf{Algorithme}} & 0 & 1 & 0 & 0 & 0 & 0 & 1\\
            \hline
        \textbf{Total} & 8 & 108 & 8 & 151 & 103 & 8 & \textbf{386}\\
        \end{tabular}
    \caption*{}
    \label{Annexe RSL tableau sources de données continent}
        \begin{flushright}
    \scriptsize
    * Les continents nord-américain et sud-américain sont respectivement notés «~AdN~» et «~AdS~».
    \\
    Auteur~: \textcopyright~Moinse 2023
        \end{flushright}
        \end{table}

    \newpage
    % Annexe C.2
\subsection{Répartition des sources de données selon la taille démographique de l'agglomération}
    \label{donnees-ouvertes:rsl_publications_sources_donnees_par_taille_agglomeration}

% Tableau C.2
        \begin{table}[h]
        \centering
        \renewcommand{\arraystretch}{1.5}
        \begin{tabular}{p{3.9cm}p{1.3cm}p{1.3cm}p{1.3cm}p{1.3cm}p{1.3cm}p{1.2cm}}
        \hline
    \rule{0pt}{15pt} \textcolor{blue}{\textbf{Sources des données}} & \textcolor{blue}{\textbf{≥10M}} & \textcolor{blue}{\textbf{≥3M}} & \textcolor{blue}{\textbf{≥1M}} & \textcolor{blue}{\textbf{≥250.000}} & \textcolor{blue}{\textbf{<250.000}} & \textbf{Total}
    \\
            \hline
        & \multicolumn{5}{l}{\textbf{\textsl{Big data}}} & \textbf{190}\\
        \textcolor{blue}{\textbf{\textsl{Open data}}} & 38 & 17 & 13 & 6 & 2 & 76\\
        \textcolor{blue}{\textbf{API}} & 30 & 10 & 0 & 0 & 2 & 42\\
        \textcolor{blue}{\textbf{GBFS}} & 18 & 9 & 5 & 0 & 0 & 32\\
        \textcolor{blue}{\textbf{GTFS}} & 10 & 5 & 8 & 2 & 0 & 25\\
        \textcolor{blue}{\textbf{Carte multimodale}} & 7 & 1 & 0 & 0 & 0 & 8\\
        \textcolor{blue}{\textbf{GPS}} & 3 & 0 & 4 & 0 & 0 & 7\\
            \hline
        & \multicolumn{5}{l}{\textbf{Enquête par questionnaire}} & \textbf{64}\\
        \textcolor{blue}{\textbf{En ligne}} & 9 & 7 & 11 & 6 & 4 & 37\\
        \textcolor{blue}{\textbf{En face-à-face}} & 12 & 7 & 3 & 4 & 0 & 26\\
        \textcolor{blue}{\textbf{Voie postale}} & 1 & 0 & 0 & 0 & 0 & 1\\
            \hline
        & \multicolumn{5}{l}{\textbf{Analyse secondaire d'une enquête publique}} & \textbf{36}\\
        \textcolor{blue}{\textbf{Analyse secondaire}} & 11 & 7 & 7 & 3 & 8 & 36\\
            \hline
        & \multicolumn{5}{l}{\textbf{État de l'art}} & \textbf{8}\\
        \textcolor{blue}{\textbf{État de l'art}} & 1 & 6 & 1 & 0 & 0 & 8\\
            \hline
        & \multicolumn{5}{l}{\textbf{Observation}} & \textbf{10}\\
        \textcolor{blue}{\textbf{Directe}} & 0 & 1 & 3 & 0 & 3 & 7\\
        \textcolor{blue}{\textbf{Secondaire}} & 1 & 0 & 1 & 0 & 1 & 3\\
             \hline
        & \multicolumn{5}{l}{\textbf{Enquête par entretien}} & \textbf{9}\\
        \textcolor{blue}{\textbf{Individuel}} & 0 & 3 & 1 & 0 & 0 & 4\\
        \textcolor{blue}{\textbf{Collectif}} & 3 & 1 & 0 & 0 & 0 & 4\\
        \textcolor{blue}{\textbf{En ligne}} & 0 & 1 & 0 & 0 & 0 & 1\\
            \hline
        & \multicolumn{5}{l}{\textbf{Algorithme}} & \textbf{1}\\
        \textcolor{blue}{\textbf{Algorithme}} & 0 & 0 & 0 & 1 & 0 & 1\\
            \hline
        \textbf{Total} & 144 & 75 & 57 & 22 & 20 & 318\\
        \end{tabular}
    \caption*{}
    \label{Annexe RSL tableau sources de données EPCI}
        \begin{flushright}
    \scriptsize
    Note~: les études de cas qui ne reposent pas sur un périmètre administratif à l'échelle intercommunale sont exclues de l'analyse statistique.
    \\
    Auteur~: \textcopyright~Moinse 2023
        \end{flushright}
        \end{table}

    \newpage
    % Annexe C.3
\subsection{Répartition des sources de données selon le type de mobilité individuelle légère}
    \label{donnees-ouvertes:rsl_publications_sources_donnees_par_type_mobilite_individuelle_legere}

% Tableau C.3
        \begin{table}[h]
        \centering
        \renewcommand{\arraystretch}{1.5}
        \begin{tabular}{p{3.9cm}p{0.83cm}p{0.83cm}p{0.83cm}p{0.83cm}p{0.83cm}p{0.83cm}p{0.83cm}p{0.95cm}}
        \hline
    \rule{0pt}{15pt} \textcolor{blue}{\textbf{Sources des données}} & \textcolor{blue}{\textbf{Vélo}} & \textcolor{blue}{\textbf{VAE}} & \textcolor{blue}{\textbf{VLS}} & \textcolor{blue}{\textbf{VFF}} & \textcolor{blue}{\textbf{TEP}} & \textcolor{blue}{\textbf{TM}} & \textcolor{blue}{\textbf{TEFF}} & \textbf{Total}
    \\
            \hline
        & \multicolumn{7}{l}{\textbf{\textsl{Big data}}} & \textbf{211}\\
        \textcolor{blue}{\textbf{\textsl{Open data}}} & 33 & 0 & 38 & 18 & 1& 0 & 0 & 90\\
        \textcolor{blue}{\textbf{API}} & 0 & 0 & 3 & 34 & 0 & 0 & 9 & 46\\
        \textcolor{blue}{\textbf{GBFS}} & 2 & 0 & 30 & 2 & 0 & 0 & 0 & 34\\
        \textcolor{blue}{\textbf{GTFS}} & 11 & 0 & 6 & 3 & 0 & 0 & 6 & 26\\
        \textcolor{blue}{\textbf{Carte multimodale}} & 0 & 0 & 6 & 2 & 0 & 0 & 0 & 8\\
        \textcolor{blue}{\textbf{GPS}} & 7 & 0 & 0 & 0 & 0 & 0 & 0 & 7\\
            \hline
        & \multicolumn{7}{l}{\textbf{Enquête par questionnaire}} & \textbf{81}\\
        \textcolor{blue}{\textbf{En ligne}} & 17 & 0 & 10 & 4 & 0 & 0 & 10 & 41\\
        \textcolor{blue}{\textbf{En face-à-face}} & 24 & 2 & 2 & 2 & 0 & 0 & 1 & 31\\
        \textcolor{blue}{\textbf{Voie postale}} & 2 & 0 & 1 & 3 & 0 & 0 & 3 & 9\\
            \hline
        & \multicolumn{7}{l}{\textbf{Analyse secondaire d'une enquête publique}} & \textbf{62}\\
        \textcolor{blue}{\textbf{Analyse secondaire}} & 51 & 1 & 3 & 1 & 2 & 1 & 3 & 62\\
            \hline
        & \multicolumn{7}{l}{\textbf{État de l'art}} & \textbf{23}\\
        \textcolor{blue}{\textbf{État de l'art}} & 12 & 1 & 3 & 3 & 0 & 0 & 4 & 23\\
            \hline
        & \multicolumn{7}{l}{\textbf{Observation}} & \textbf{10}\\
        \textcolor{blue}{\textbf{Directe}} & 6 & 0 & 0 & 0 & 0 & 1 & 0 & 7\\
        \textcolor{blue}{\textbf{Secondaire}} & 2 & 0 & 0 & 1 & 0 & 0 & 0 & 3\\
             \hline
        & \multicolumn{7}{l}{\textbf{Enquête par entretien}} & \textbf{10}\\
        \textcolor{blue}{\textbf{Individuel}} & 2 & 0 & 3 & 0 & 0 & 0 & 0 & 5\\
        \textcolor{blue}{\textbf{Collectif}} & 4 & 0 & 0 & 0 & 0 & 0 & 0 & 4\\
        \textcolor{blue}{\textbf{En ligne}} & 1 & 0 & 0 & 0 & 0 & 0 & 0 & 1\\
            \hline
        & \multicolumn{7}{l}{\textbf{Algorithme}} & \textbf{4}\\
        \textcolor{blue}{\textbf{Algorithme}} & 1 & 0 & 3 & 0 & 0 & 0 & 0 & 4\\
            \hline
        \textbf{Total} & 175 & 4 & 108 & 73 & 3 & 2 & 36 & 401\\
        \end{tabular}
    \caption*{}
    \label{Annexe RSL tableau sources de données EDP}
        \begin{flushright}
    \scriptsize
    Note~: vélo à usage personnel (Vélo), \acrfull{VAE}, \acrfull{VLS}, \acrfull{VFF}, \acrfull{TEP}, trottinette mécanique (TM), \acrfull{TEFF}.
    \\
    Auteur~: \textcopyright~Moinse 2023
        \end{flushright}
        \end{table}

    % ___________________________________________
    % ANNEXE D~: Résultats RSL
    \newpage
\section{Tableau récapitulatif des résultats identifiés et catégorisés dans le cadre de la \acrshort{RSL} sur le \acrshort{B-TOD}}
    \label{donnees-ouvertes:rsl_resultats}
    \markboth{Tableau récapitulatif des résultats identifiés et catégorisés dans le cadre de la RSL sur le B-TOD}{}
    \markright{Tableau récapitulatif des résultats identifiés et catégorisés dans le cadre de la RSL sur le B-TOD}{}

    % Référence
Cette section se réfère à la \hyperref[Caractérisation du B-TOD au prisme des caractéristiques de l'environnement urbain]{sous-section dédiée aux résultats regroupés par thématiques issus de la RSL} (page \pageref{Caractérisation du B-TOD au prisme des caractéristiques de l'environnement urbain}), dans le cadre du \hyperref[chap2:titre]{chapitre 2} (page \pageref{chap2:titre}). Le tableau ci-dessous se compose de deux valeurs distinctes~: le «~\textcolor{blue}{\textbf{+}}~» signifie que la publication scientifique analysée rend compte d'un résultat lié à la catégorie référencée dans la colonne, tandis que le «~-~» représente une absence de résultats empiriques à ce sujet.\par

    % Variables
Liste des catégories relatives aux résultats empiriques (excluant les revues de littérature) dans le tableau suivant~:
\begin{itemize}
    \item \textcolor{blue}{$D1$}~: Densité de la population et des emplois ;
    \item \textcolor{blue}{$D2$}~: Mixité urbaine et sociale (Diversité) ;
    \item \textcolor{blue}{$D3$}~: Traitement des espaces publics (\textsl{Design}) ;
    \item \textcolor{blue}{$D4$}~: Accessibilité à la destination ;
    \item \textcolor{blue}{$D5$}~: Distance aux réseaux de transport en commun ;
    \item \textcolor{blue}{$D6$}~: Management de la demande ;
    \item \textcolor{blue}{$D7$}~: Inclusion sociale (\textsl{Demographics}) ;
    \item \textcolor{blue}{$I$}~: Impacts sur la mobilité, l'environnement, les territoires et l'économie ;
    \item \textcolor{blue}{$CP$}~: Comportements et pratiques de mobilité ;
     \item \textcolor{blue}{$G$}~: Gouvernance.
\end{itemize}

% Tableau C
        \begin{longtable}{p{0.5cm}p{4.6cm}p{0.3cm}p{0.3cm}p{0.3cm}p{0.3cm}p{0.3cm}p{0.3cm}p{0.3cm}p{0.3cm}p{0.3cm}p{0.3cm}p{1cm}}
        \hline
        \small{\textcolor{blue}{\textbf{ID}}} & \small{\textcolor{blue}{\textbf{Corpus de la \acrshort{RSL} (empirique)}}} & \small{\textcolor{blue}{\textbf{$D1$}}} & \small{\textcolor{blue}{\textbf{$D2$}}} & \small{\textcolor{blue}{\textbf{$D3$}}} & \small{\textcolor{blue}{\textbf{$D4$}}} & \small{\textcolor{blue}{\textbf{$D5$}}} & \small{\textcolor{blue}{\textbf{$D6$}}} & \small{\textcolor{blue}{\textbf{$D7$}}} & \small{\textcolor{blue}{\textbf{$I$}}} & \small{\textcolor{blue}{\textbf{$CP$}}} & \small{\textcolor{blue}{\textbf{$G$}}} & \small{\textcolor{blue}{\textbf{TOTAL}}}
        \hline
        \endhead
    \small{2} & \small{\textcite{adnan_last-mile_2019}}\index{Adnan, Muhammad|pagebf} & - & - & - & - & \textcolor{blue}{\textbf{+}} & - & \textcolor{blue}{\textbf{+}} & \textcolor{blue}{\textbf{+}} & \textcolor{blue}{\textbf{+}} & - & \textbf{4}\\
    \small{3} & \small{\textcite{advani_bicycle_2006}}\index{Advani, Mukti|pagebf} & - & - & \textcolor{blue}{\textbf{+}} & - & \textcolor{blue}{\textbf{+}} & - & \textcolor{blue}{\textbf{+}} & \textcolor{blue}{\textbf{+}} & - & - & \textbf{4}\\
    \small{5} & \small{\textcite{andersson_neighbourhood_2021}}\index{Andersson, David Emanuel|pagebf} & - & - & - & - & \textcolor{blue}{\textbf{+}} & - & - & - & - & - & \textbf{1}\\
    \small{6} & \small{\textcite{ann_examination_2019}}\index{Ann, Sangeetha|pagebf} & - & - & - & - & \textcolor{blue}{\textbf{+}} & - & - & - & - & - & \textbf{1}\\
    \small{7} & \small{\textcite{arbis_analysis_2016}}\index{Arbis, David|pagebf} & - & - & \textcolor{blue}{\textbf{+}} & \textcolor{blue}{\textbf{+}} & - & - & \textcolor{blue}{\textbf{+}} & - & - & - & \textbf{3}\\
    \small{9} & \small{\textcite{ashraf_impacts_2021}}\index{Ashraf, Md Tanvir|pagebf} & - & - & \textcolor{blue}{\textbf{+}} & - & - & - & - & \textcolor{blue}{\textbf{+}} & - & - & \textbf{2}\\
    \small{10} & \small{\textcite{bachand-marleau_much-anticipated_2011}}\index{Bachand-Marleau, Julie|pagebf} & - & - & - & \textcolor{blue}{\textbf{+}} & \textcolor{blue}{\textbf{+}} & - & \textcolor{blue}{\textbf{+}} & \textcolor{blue}{\textbf{+}} & \textcolor{blue}{\textbf{+}} & - & \textbf{5}\\
    \small{11} & \small{\textcite{baek_electric_2021}}\index{Baek, Kwangho|pagebf} & - & - & - & - & \textcolor{blue}{\textbf{+}} & - & - & - & \textcolor{blue}{\textbf{+}} & - & \textbf{2}\\
    \small{12} & \small{\textcite{balya_integration_2016}}\index{Balya, Manjurali|pagebf} & - & - & \textcolor{blue}{\textbf{+}} & - & - & \textcolor{blue}{\textbf{+}} & \textcolor{blue}{\textbf{+}} & - & - & - & \textbf{3}\\
    \small{13} & \small{\textcite{basu_planning_2021}}\index{Basu, Rounaq|pagebf} & - & - & - & - & \textcolor{blue}{\textbf{+}} & - & \textcolor{blue}{\textbf{+}} & \textcolor{blue}{\textbf{+}} & - & - & \textbf{3}\\
    \small{14} & \small{\textcite{bauer_influence_2021}}\index{Bauer, Marek|pagebf} & - & - & - & - & \textcolor{blue}{\textbf{+}} & - & - & - & - & - & \textbf{1}\\
    \small{15} & \small{\textcite{beale_integrating_2023}}\index{Beale, Kirsten|pagebf} & - & - & - & - & - & \textcolor{blue}{\textbf{+}} & - & - & - & - & \textbf{1}\\
    \small{16} & \small{\textcite{bearn_adaption_2018}}\index{Bearn, Cary|pagebf} & - & - & - & - & \textcolor{blue}{\textbf{+}} & - & \textcolor{blue}{\textbf{+}} & \textcolor{blue}{\textbf{+}} & - & - & \textbf{3}\\
    \small{17} & \small{\textcite{bechstein_cycling_2010}}\index{Bechstein, Eva|pagebf} & - & - & \textcolor{blue}{\textbf{+}} & - & - & - & \textcolor{blue}{\textbf{+}} & - & - & - & \textbf{2}\\
    \small{18} & \small{\textcite{bi_analysis_2021}}\index{Bi, Hui|pagebf} & - & \textcolor{blue}{\textbf{+}} & - & \textcolor{blue}{\textbf{+}} & \textcolor{blue}{\textbf{+}} & - & - & - & \textcolor{blue}{\textbf{+}} & - & \textbf{4}\\
    \small{19} & \small{\textcite{bocker_bike_2020}}\index{Böcker, Lars|pagebf} & \textcolor{blue}{\textbf{+}} & \textcolor{blue}{\textbf{+}} & \textcolor{blue}{\textbf{+}} & \textcolor{blue}{\textbf{+}} & - & - & \textcolor{blue}{\textbf{+}} & \textcolor{blue}{\textbf{+}} & \textcolor{blue}{\textbf{+}} & - & \textbf{7}\\
    \small{20} & \small{\textcite{bopp_examining_2015}}\index{Bopp, Melissa|pagebf} & - & - & \textcolor{blue}{\textbf{+}} & - & - & \textcolor{blue}{\textbf{+}} & - & \textcolor{blue}{\textbf{+}} & \textcolor{blue}{\textbf{+}} & - & \textbf{4}\\
    \small{25} & \small{\textcite{cao_e-scooter_2021}}\index{Cao, Zhejing|pagebf} & - & - & - & - & \textcolor{blue}{\textbf{+}} & - & \textcolor{blue}{\textbf{+}} & \textcolor{blue}{\textbf{+}} & - & - & \textbf{3}\\
    \small{27} & \small{\textcite{cervero_bike-and-ride_2013}}\index{Cervero, Robert|pagebf} & - & - & \textcolor{blue}{\textbf{+}} & - & \textcolor{blue}{\textbf{+}} & \textcolor{blue}{\textbf{+}} & \textcolor{blue}{\textbf{+}} & \textcolor{blue}{\textbf{+}} & - & \textcolor{blue}{\textbf{+}} & \textbf{5}\\
    \small{28} & \small{\textcite{cervero_influences_2009}}\index{Cervero, Robert|pagebf} & - & - & \textcolor{blue}{\textbf{+}} & - & \textcolor{blue}{\textbf{+}} & - & \textcolor{blue}{\textbf{+}} & - & - & - & \textbf{3}\\
    \small{29} & \small{\textcite{chan_factors_2020}}\index{Chan, Kevin|pagebf} & \textcolor{blue}{\textbf{+}} & \textcolor{blue}{\textbf{+}} & \textcolor{blue}{\textbf{+}} & - & \textcolor{blue}{\textbf{+}} & \textcolor{blue}{\textbf{+}} & \textcolor{blue}{\textbf{+}} & - & - & \textcolor{blue}{\textbf{+}} & \textbf{6}\\
    \small{30} & \small{\textcite{chen_determinants_2012}}\index{Chen, Lijun|pagebf} & - & - & - & - & \textcolor{blue}{\textbf{+}} & \textcolor{blue}{\textbf{+}} & \textcolor{blue}{\textbf{+}} & - & \textcolor{blue}{\textbf{+}} & - & \textbf{4}\\ 
    \small{31} & \small{\textcite{chen_study_2013}}\index{Chen, Wan|pagebf} & - & - & \textcolor{blue}{\textbf{+}} & \textcolor{blue}{\textbf{+}} & \textcolor{blue}{\textbf{+}} & - & - & \textcolor{blue}{\textbf{+}} & - & \textcolor{blue}{\textbf{+}} & \textbf{4}\\
    \small{32} & \small{\textcite{chen_what_2022}}\index{Chen, Wendong|pagebf} & \textcolor{blue}{\textbf{+}} & \textcolor{blue}{\textbf{+}} & \textcolor{blue}{\textbf{+}} & \textcolor{blue}{\textbf{+}} & \textcolor{blue}{\textbf{+}} & \textcolor{blue}{\textbf{+}} & - & - & \textcolor{blue}{\textbf{+}} & - & \textbf{7}\\
    \small{33} & \small{\textcite{cheng_comparison_2023}}\index{Cheng, Long|pagebf} & \textcolor{blue}{\textbf{+}} & \textcolor{blue}{\textbf{+}} & \textcolor{blue}{\textbf{+}} & \textcolor{blue}{\textbf{+}} & \textcolor{blue}{\textbf{+}} & \textcolor{blue}{\textbf{+}} & \textcolor{blue}{\textbf{+}} & - & - & - & \textbf{7}\\
    \small{34} & \small{\textcite{cheng_promoting_2022}} & \textcolor{blue}{\textbf{+}} & \textcolor{blue}{\textbf{+}} & \textcolor{blue}{\textbf{+}} & \textcolor{blue}{\textbf{+}} & - & - & - & - & \textcolor{blue}{\textbf{+}} & - & \textbf{5}\\
    \small{35} & \small{\textcite{chen_exploring_2022}} & \textcolor{blue}{\textbf{+}} & \textcolor{blue}{\textbf{+}} & \textcolor{blue}{\textbf{+}} & \textcolor{blue}{\textbf{+}} & - & - & - & - & - & - & \textbf{4}\\
    \small{36} & \small{\textcite{cheng_expanding_2018}}\index{Cheng, Yung-Hsiang|pagebf} & \textcolor{blue}{\textbf{+}} & - & - & - & \textcolor{blue}{\textbf{+}} & \textcolor{blue}{\textbf{+}} & \textcolor{blue}{\textbf{+}} & \textcolor{blue}{\textbf{+}} & \textcolor{blue}{\textbf{+}} & - & \textbf{6}\\
    \small{37} & \small{\textcite{cheng_evaluating_2012}}\index{Cheng, Yung-Hsiang|pagebf} & - & - & - & - & \textcolor{blue}{\textbf{+}} & - & \textcolor{blue}{\textbf{+}} & - & \textcolor{blue}{\textbf{+}} & - & \textbf{3}\\
    \small{38} & \small{\textcite{cho_estimation_2022}}\index{Cho, Shin-Hyung|pagebf} & - & - & \textcolor{blue}{\textbf{+}} & - & \textcolor{blue}{\textbf{+}} & - & - & \textcolor{blue}{\textbf{+}} & - & - & \textbf{3}\\
    \small{39} & \small{\textcite{chu_last_2021}}\index{Chu, Junhong|pagebf} & - & \textcolor{blue}{\textbf{+}} & \textcolor{blue}{\textbf{+}} & \textcolor{blue}{\textbf{+}} & \textcolor{blue}{\textbf{+}} & - & - & \textcolor{blue}{\textbf{+}} & \textcolor{blue}{\textbf{+}} & - & \textbf{6}\\
    \small{40} & \small{\textcite{cooke_relationship_2018}}\index{Cooke, Sean|pagebf} & \textcolor{blue}{\textbf{+}} & - & - & - & - & - & - & - & - & - & \textbf{1}\\
    \small{41} & \small{\textcite{cottrell_transforming_2007}}\index{Cottrell, Wayne D.|pagebf} & - & - & \textcolor{blue}{\textbf{+}} & - & \textcolor{blue}{\textbf{+}} & - & - & - & - & - & \textbf{2}\\
    \small{43} & \small{\textcite{souza_modelling_2017}}\index{Souza, Flavia de|pagebf} & - & - & - & - & - & - & \textcolor{blue}{\textbf{+}} & - & - & - & \textbf{1}\\
    \small{44} & \small{\textcite{debrezion_modelling_2009}}\index{Debrezion, Ghebreegziabiher|pagebf} & - & - & - & - & \textcolor{blue}{\textbf{+}} & \textcolor{blue}{\textbf{+}} & \textcolor{blue}{\textbf{+}} & - & - & - & \textbf{3}\\
    \small{45} & \small{\textcite{djurhuus_building_2016}}\index{Djurhuus, Sune|pagebf} & - & - & - & \textcolor{blue}{\textbf{+}} & \textcolor{blue}{\textbf{+}} & - & - & - & - & - & \textbf{2}\\
    \small{49} & \small{\textcite{fan_how_2019}}\index{Fan, Aihua|pagebf} & - & - & \textcolor{blue}{\textbf{+}} & - & \textcolor{blue}{\textbf{+}} & \textcolor{blue}{\textbf{+}} & \textcolor{blue}{\textbf{+}} & \textcolor{blue}{\textbf{+}} & \textcolor{blue}{\textbf{+}} & \textcolor{blue}{\textbf{+}} & \textbf{6}\\
    \small{50} & \small{\textcite{fan_dockless_2020}}\index{Fan, Yichun|pagebf} & - & - & - & \textcolor{blue}{\textbf{+}} & - & - & - & \textcolor{blue}{\textbf{+}} & \textcolor{blue}{\textbf{+}} & - & \textbf{3}\\
    \small{51} & \small{\textcite{fearnley_patterns_2020}}\index{Fearnley, Nils|pagebf} & - & - & - & \textcolor{blue}{\textbf{+}} & \textcolor{blue}{\textbf{+}} & \textcolor{blue}{\textbf{+}} & \textcolor{blue}{\textbf{+}} & \textcolor{blue}{\textbf{+}} & - & - & \textbf{5}\\
    \small{52} & \small{\textcite{fillone_i_2018}}\index{Fillone, Alexis|pagebf} & - & - & \textcolor{blue}{\textbf{+}} & - & \textcolor{blue}{\textbf{+}} & - & \textcolor{blue}{\textbf{+}} & - & \textcolor{blue}{\textbf{+}} & - & \textbf{4}\\
    \small{53} & \small{\textcite{flamm_determinants_2013}}\index{Flamm, Bradley J.|pagebf} & - & - & - & \textcolor{blue}{\textbf{+}} & \textcolor{blue}{\textbf{+}} & - & \textcolor{blue}{\textbf{+}} & - & \textcolor{blue}{\textbf{+}} & - & \textbf{4}\\
    \small{54} & \small{\textcite{flamm_public_2014}}\index{Flamm, Bradley J.|pagebf} & - & - & \textcolor{blue}{\textbf{+}} & - & \textcolor{blue}{\textbf{+}} & - & \textcolor{blue}{\textbf{+}} & - & \textcolor{blue}{\textbf{+}} & - & \textbf{4}\\
    \small{55} & \small{\textcite{flamm_changes_2014}}\index{Flamm, Bradley J.|pagebf} & - & - & - & \textcolor{blue}{\textbf{+}} & - & - & - & - & \textcolor{blue}{\textbf{+}} & - & \textbf{2}\\
    \small{57} & \small{\textcite{fournier_continuous_2021}}\index{Fournier, Nicholas|pagebf} & - & - & - & - & - & \textcolor{blue}{\textbf{+}} & - & - & \textcolor{blue}{\textbf{+}} & - & \textbf{2}\\
    \small{58} & \small{\textcite{gan_associations_2021}}\index{Gan, Zuoxian|pagebf} & \textcolor{blue}{\textbf{+}} & \textcolor{blue}{\textbf{+}} & - & - & \textcolor{blue}{\textbf{+}} & - & - & - & - & - & \textbf{3}\\
    \small{60} & \small{\textcite{garcia-bello_methodological_2019}}\index{Marques, R.|pagebf} & - & - & \textcolor{blue}{\textbf{+}} & - & \textcolor{blue}{\textbf{+}} & - & - & - & - & - & \textbf{2}\\
    \small{61} & \small{\textcite{geurs_multi-modal_2016}}\index{Geurs, Karst T.|pagebf} & - & - & \textcolor{blue}{\textbf{+}} & \textcolor{blue}{\textbf{+}} & - & - & - & \textcolor{blue}{\textbf{+}} & - & - & \textbf{3}\\
    \small{62} & \small{\textcite{giansoldati_train-feeder_2021}}\index{Giansoldati, Marco|pagebf} & - & - & \textcolor{blue}{\textbf{+}} & - & - & \textcolor{blue}{\textbf{+}} & - & - & - & - & \textbf{2}\\
    \small{63} & \small{\textcite{givoni_access_2007}}\index{Givoni, Moshe|pagebf} & - & - & - & \textcolor{blue}{\textbf{+}} & \textcolor{blue}{\textbf{+}} & \textcolor{blue}{\textbf{+}} & \textcolor{blue}{\textbf{+}} & - & - & - & \textbf{4}\\
    \small{64} & \small{\textcite{glass_role_2020}}\index{Glass, Caroline|pagebf} & \textcolor{blue}{\textbf{+}} & - & - & - & \textcolor{blue}{\textbf{+}} & - & - & \textcolor{blue}{\textbf{+}} & \textcolor{blue}{\textbf{+}} & - & \textbf{4}\\
    \small{65} & \small{\textcite{van_goeverden_potential_2018}}\index{van Goeverden, Kees|pagebf} & - & - & \textcolor{blue}{\textbf{+}} & \textcolor{blue}{\textbf{+}} & - & - & \textcolor{blue}{\textbf{+}} & - & \textcolor{blue}{\textbf{+}} & - & \textbf{4}\\
    \small{66} & \small{\textcite{griffin_planning_2016}}\index{Griffin, Greg|pagebf} & \textcolor{blue}{\textbf{+}} & - & - & - & - & - & - & - & \textcolor{blue}{\textbf{+}} & \textcolor{blue}{\textbf{+}} & \textbf{2}\\
    \small{68} & \small{\textcite{gu_measuring_2019}}\index{Gu, Tianqi|pagebf} & - & - & - & - & - & - & - & \textcolor{blue}{\textbf{+}} & \textcolor{blue}{\textbf{+}} & - & \textbf{2}\\
    \small{69} & \small{\textcite{guo_exploring_2023}}\index{Guo, Dongbo|pagebf} & - & - & - & - & \textcolor{blue}{\textbf{+}} & \textcolor{blue}{\textbf{+}} & \textcolor{blue}{\textbf{+}} & - & \textcolor{blue}{\textbf{+}} & - & \textbf{4}\\
    \small{70} & \small{\textcite{guo_built_2020}}\index{Guo, Yuanyuan|pagebf} & \textcolor{blue}{\textbf{+}} & \textcolor{blue}{\textbf{+}} & \textcolor{blue}{\textbf{+}} & \textcolor{blue}{\textbf{+}} & \textcolor{blue}{\textbf{+}} & - & \textcolor{blue}{\textbf{+}} & - & - & - & \textbf{6}\\
    \small{71} & \small{\textcite{guo_role_2021}}\index{Guo, Yuanyuan|pagebf} & \textcolor{blue}{\textbf{+}} & \textcolor{blue}{\textbf{+}} & \textcolor{blue}{\textbf{+}} & \textcolor{blue}{\textbf{+}} & \textcolor{blue}{\textbf{+}} & - & \textcolor{blue}{\textbf{+}} & - & - & - & \textbf{6}\\
    \small{72} & \small{\textcite{guo_dockless_2021}}\index{Guo, Yuanyuan|pagebf} & \textcolor{blue}{\textbf{+}} & \textcolor{blue}{\textbf{+}} & \textcolor{blue}{\textbf{+}} & \textcolor{blue}{\textbf{+}} & \textcolor{blue}{\textbf{+}} & \textcolor{blue}{\textbf{+}} & \textcolor{blue}{\textbf{+}} & - & - & - & \textbf{7}\\
    \small{74} & \small{\textcite{halldorsdottir_home-end_2017}}\index{Halldórsdóttir, Katrín|pagebf} & - & - & \textcolor{blue}{\textbf{+}} & \textcolor{blue}{\textbf{+}} & \textcolor{blue}{\textbf{+}} & \textcolor{blue}{\textbf{+}} & \textcolor{blue}{\textbf{+}} & - & \textcolor{blue}{\textbf{+}} & - & \textbf{6}\\
    \small{75} & \small{\textcite{hamidi_shaping_2020}}\index{Hamidi, Zahra|pagebf} & - & - & \textcolor{blue}{\textbf{+}} & - & \textcolor{blue}{\textbf{+}} & - & \textcolor{blue}{\textbf{+}} & - & \textcolor{blue}{\textbf{+}} & - & \textbf{4}\\
    \small{76} & \small{\textcite{hamidi_inequalities_2019}}\index{Hamidi, Zahra|pagebf} & - & \textcolor{blue}{\textbf{+}} & \textcolor{blue}{\textbf{+}} & \textcolor{blue}{\textbf{+}} & - & - & \textcolor{blue}{\textbf{+}} & - & \textcolor{blue}{\textbf{+}} & - & \textbf{5}\\
    \small{77} & \small{\textcite{hasiak_access_2019}}\index{Hasiak, Sophie|pagebf} & - & - & - & - & \textcolor{blue}{\textbf{+}} & - & \textcolor{blue}{\textbf{+}} & - & - & - & \textbf{2}\\
    \small{78} & \small{\textcite{heinen_multimodal_2014}}\index{Heinen, Eva|pagebf} & - & - & - & - & \textcolor{blue}{\textbf{+}} & \textcolor{blue}{\textbf{+}} & \textcolor{blue}{\textbf{+}} & - & \textcolor{blue}{\textbf{+}} & - & \textbf{4}\\
    \small{79} & \small{\textcite{heumann_spatiotemporal_2021}}\index{Heumann, Maximilian|pagebf} & - & \textcolor{blue}{\textbf{+}} & - & \textcolor{blue}{\textbf{+}} & - & - & - & - & \textcolor{blue}{\textbf{+}} & - & \textbf{3}\\
    \small{80} & \small{\textcite{hochmair_assessment_2015}}\index{Hochmair, Hartwig H.|pagebf} & - & - & \textcolor{blue}{\textbf{+}} & - & \textcolor{blue}{\textbf{+}} & - & - & - & - & - & \textbf{2}\\
    \small{82} & \small{\textcite{hu_examining_2022}}\index{Hu, Songhua|pagebf} & \textcolor{blue}{\textbf{+}} & \textcolor{blue}{\textbf{+}} & \textcolor{blue}{\textbf{+}} & \textcolor{blue}{\textbf{+}} & \textcolor{blue}{\textbf{+}} & - & \textcolor{blue}{\textbf{+}} & - & - & - & \textbf{6}\\
    \small{83} & \small{\textcite{hua_transfer_2022}}\index{Hua, Mingzhuang|pagebf} & - & - & - & - & - & \textcolor{blue}{\textbf{+}} & - & - & - & \textcolor{blue}{\textbf{+}} & \textbf{1}\\
    \small{87} & \small{\textcite{jappinen_modelling_2013}}\index{Jäppinen, Sakari|pagebf} & - & - & - & \textcolor{blue}{\textbf{+}} & \textcolor{blue}{\textbf{+}} & - & - & \textcolor{blue}{\textbf{+}} & - & \textcolor{blue}{\textbf{+}} & \textbf{3}\\
    \small{88} & \small{\textcite{ji_public_2017}}\index{Ji, Yanjie|pagebf} & \textcolor{blue}{\textbf{+}} & - & - & - & \textcolor{blue}{\textbf{+}} & - & \textcolor{blue}{\textbf{+}} & - & \textcolor{blue}{\textbf{+}} & - & \textbf{4}\\
    \small{89} & \small{\textcite{ji_exploring_2018}}\index{Ji, Yanjie|pagebf} & - & \textcolor{blue}{\textbf{+}} & \textcolor{blue}{\textbf{+}} & \textcolor{blue}{\textbf{+}} & \textcolor{blue}{\textbf{+}} & \textcolor{blue}{\textbf{+}} & \textcolor{blue}{\textbf{+}} & - & - & \textcolor{blue}{\textbf{+}} & \textbf{6}\\
    \small{90} & \small{\textcite{jin_competition_2019}}\index{Jin, Haitao|pagebf} & - & - & - & \textcolor{blue}{\textbf{+}} & \textcolor{blue}{\textbf{+}} & \textcolor{blue}{\textbf{+}} & - & \textcolor{blue}{\textbf{+}} & - & - & \textbf{4}\\
    \small{91} & \small{\textcite{chen_demand_2013}}\index{Chen, Jingxu|pagebf} & - & - & - & - & \textcolor{blue}{\textbf{+}} & - & \textcolor{blue}{\textbf{+}} & - & - & - & \textbf{2}\\
    \small{92} & \small{\textcite{jonkeren_bicycle_2021}}\index{Jonkeren, Olaf|pagebf} & - & - & \textcolor{blue}{\textbf{+}} & - & - & \textcolor{blue}{\textbf{+}} & \textcolor{blue}{\textbf{+}} & - & \textcolor{blue}{\textbf{+}} & - & \textbf{4}\\
    \small{93} & \small{\textcite{jonkeren_bicycle-train_2021}}\index{Jonkeren, Olaf|pagebf} & - & - & \textcolor{blue}{\textbf{+}} & \textcolor{blue}{\textbf{+}} & - & - & \textcolor{blue}{\textbf{+}} & \textcolor{blue}{\textbf{+}} & \textcolor{blue}{\textbf{+}} & - & \textbf{5}\\
    \small{95} & \small{\textcite{kager_characterisation_2016}}\index{Kager, Roland|pagebf} & \textcolor{blue}{\textbf{+}} & - & - & \textcolor{blue}{\textbf{+}} & \textcolor{blue}{\textbf{+}} & - & - & - & \textcolor{blue}{\textbf{+}} & - & \textbf{4}\\
    \small{96} & \small{\textcite{keijer_how_2000}}\index{Keijer, Majanka|pagebf} & - & - & - & - & \textcolor{blue}{\textbf{+}} & - & - & - & - & - & \textbf{1}\\
    \small{97} & \small{\textcite{kim_analysis_2021}}\index{Kim, Minjun|pagebf} & - & \textcolor{blue}{\textbf{+}} & \textcolor{blue}{\textbf{+}} & - & \textcolor{blue}{\textbf{+}} & - & - & \textcolor{blue}{\textbf{+}} & \textcolor{blue}{\textbf{+}} & - & \textbf{5}\\
    \small{98} & \small{\textcite{kong_deciphering_2020}}\index{Kong, Hui|pagebf} & \textcolor{blue}{\textbf{+}} & \textcolor{blue}{\textbf{+}} & - & \textcolor{blue}{\textbf{+}} & \textcolor{blue}{\textbf{+}} & - & - & \textcolor{blue}{\textbf{+}} & \textcolor{blue}{\textbf{+}} & - & \textbf{6}\\
    \small{99} & \small{\textcite{kostrzewska_towards_2017}} & - & - & \textcolor{blue}{\textbf{+}} & - & \textcolor{blue}{\textbf{+}} & - & \textcolor{blue}{\textbf{+}} & \textcolor{blue}{\textbf{+}} & \textcolor{blue}{\textbf{+}} & - & \textbf{5}\\
    \small{100} & \small{\textcite{krizek_bicycling_2010}}\index{Krizek, Kevin J.|pagebf} & - & - & - & - & \textcolor{blue}{\textbf{+}} & - & - & - & - & - & \textbf{1}\\
    \small{101} & \small{\textcite{krizek_assessing_2011}}\index{Krizek, Kevin|pagebf} & - & - & \textcolor{blue}{\textbf{+}} & - & \textcolor{blue}{\textbf{+}} & - & \textcolor{blue}{\textbf{+}} & - & - & - & \textbf{3}\\
    \small{102} & \small{\textcite{krizek_detailed_2007}}\index{Krizek, Kevin J.|pagebf} & - & - & \textcolor{blue}{\textbf{+}} & - & \textcolor{blue}{\textbf{+}} & - & - & - & - & - & \textbf{2}\\
    \small{103} & \small{\textcite{krygsman_multimodal_2004}}\index{Krygsman, Stephan|pagebf} & \textcolor{blue}{\textbf{+}} & - & \textcolor{blue}{\textbf{+}} & - & \textcolor{blue}{\textbf{+}} & - & \textcolor{blue}{\textbf{+}} & - & - & - & \textbf{4}\\
    \small{104} & \small{\textcite{la_paix_puello_modelling_2015}}\index{La Paix Puello, Lissy|pagebf} & \textcolor{blue}{\textbf{+}} & - & \textcolor{blue}{\textbf{+}} & - & \textcolor{blue}{\textbf{+}} & \textcolor{blue}{\textbf{+}} & - & - & - & - & \textbf{4}\\
    \small{105} & \small{\textcite{la_paix_puello_integration_2016}}\index{La Paix Puello, Lissy|pagebf} & \textcolor{blue}{\textbf{+}} & - & \textcolor{blue}{\textbf{+}} & \textcolor{blue}{\textbf{+}} & - & - & - & - & - & - & \textbf{3}\\
    \small{106} & \small{\textcite{la_paix_puello_Train_2016}}\index{La Paix Puello, Lissy|pagebf} & - & - & \textcolor{blue}{\textbf{+}} & - & \textcolor{blue}{\textbf{+}} & - & - & \textcolor{blue}{\textbf{+}} & \textcolor{blue}{\textbf{+}} & - & \textbf{4}\\
    \small{107} & \small{\textcite{la_paix_puello_role_2021}}\index{La Paix Puello, Lissy|pagebf} & - & - & \textcolor{blue}{\textbf{+}} & \textcolor{blue}{\textbf{+}} & - & - & \textcolor{blue}{\textbf{+}} & - & \textcolor{blue}{\textbf{+}} & - & \textbf{4}\\
    \small{108} & \small{\textcite{lee_bicycle-based_2016}}\index{Lee, Jaeyeong|pagebf} & - & - & \textcolor{blue}{\textbf{+}} & \textcolor{blue}{\textbf{+}} & \textcolor{blue}{\textbf{+}} & - & - & - & \textcolor{blue}{\textbf{+}} & - & \textbf{4}\\
    \small{109} & \small{\textcite{lee_strategies_2010}}\index{Lee, Jaeyeong|pagebf} & - & - & \textcolor{blue}{\textbf{+}} & - & \textcolor{blue}{\textbf{+}} & - & - & - & \textcolor{blue}{\textbf{+}} & - & \textbf{3}\\
    \small{110} & \small{\textcite{lee_forecasting_2021}}\index{Lee, Mina|pagebf} & \textcolor{blue}{\textbf{+}} & - & - & \textcolor{blue}{\textbf{+}} & \textcolor{blue}{\textbf{+}} & - & - & \textcolor{blue}{\textbf{+}} & - & - & \textbf{4}\\
    \small{113} & \small{\textcite{li_unbalanced_2022}}\index{Li, Lili|pagebf} & - & - & - & \textcolor{blue}{\textbf{+}} & \textcolor{blue}{\textbf{+}} & - & - & - & \textcolor{blue}{\textbf{+}} & - & \textbf{3}\\
    \small{114} & \small{\textcite{li_exploring_2017}}\index{Li, Wenxiang|pagebf} & \textcolor{blue}{\textbf{+}} & \textcolor{blue}{\textbf{+}} & \textcolor{blue}{\textbf{+}} & \textcolor{blue}{\textbf{+}} & - & - & \textcolor{blue}{\textbf{+}} & \textcolor{blue}{\textbf{+}} & - & \textcolor{blue}{\textbf{+}} & \textbf{6}\\
    \small{115} & \small{\textcite{li_exploring_2021}}\index{Li, Wei|pagebf} & \textcolor{blue}{\textbf{+}} & - & - & \textcolor{blue}{\textbf{+}} & \textcolor{blue}{\textbf{+}} & - & - & - & - & - & \textbf{3}\\
    \small{116} & \small{\textcite{li_factors_2020}}\index{Li, Xuefeng|pagebf} & - & \textcolor{blue}{\textbf{+}} & - & \textcolor{blue}{\textbf{+}} & \textcolor{blue}{\textbf{+}} & - & - & - & - & - & \textbf{3}\\
    \small{117} & \small{\textcite{li_investigating_2022}}\index{Li, Xiaofeng|pagebf} & \textcolor{blue}{\textbf{+}} & \textcolor{blue}{\textbf{+}} & \textcolor{blue}{\textbf{+}} & \textcolor{blue}{\textbf{+}} & - & - & - & \textcolor{blue}{\textbf{+}} & - & - & \textbf{5}\\
    \small{118} & \small{\textcite{li_operating_2019}} & \textcolor{blue}{\textbf{+}} & - & - & - & \textcolor{blue}{\textbf{+}} & - & - & - & - & - & \textbf{2}\\
    \small{119} & \small{\textcite{lin_analysis_2019}}\index{Lin, Diao|pagebf} & \textcolor{blue}{\textbf{+}} & - & \textcolor{blue}{\textbf{+}} & \textcolor{blue}{\textbf{+}} & \textcolor{blue}{\textbf{+}} & \textcolor{blue}{\textbf{+}} & - & - & \textcolor{blue}{\textbf{+}} & - & \textbf{6}\\
    \small{120} & \small{\textcite{lin_built_2018}}\index{Lin, Jen-Jia|pagebf} & \textcolor{blue}{\textbf{+}} & \textcolor{blue}{\textbf{+}} & \textcolor{blue}{\textbf{+}} & - & \textcolor{blue}{\textbf{+}} & - & \textcolor{blue}{\textbf{+}} & - & \textcolor{blue}{\textbf{+}} & \textcolor{blue}{\textbf{+}} & \textbf{6}\\
    \small{121} & \small{\textcite{liu_measuring_2022}}\index{Liu, Lumei|pagebf} & - & \textcolor{blue}{\textbf{+}} & - & \textcolor{blue}{\textbf{+}} & \textcolor{blue}{\textbf{+}} & \textcolor{blue}{\textbf{+}} & - & - & - & \textcolor{blue}{\textbf{+}} & \textbf{4}\\
    \small{122} & \small{\textcite{liu_mode_2022}}\index{Liu, Lumei|pagebf} & \textcolor{blue}{\textbf{+}} & - & \textcolor{blue}{\textbf{+}} & - & \textcolor{blue}{\textbf{+}} & \textcolor{blue}{\textbf{+}} & - & - & \textcolor{blue}{\textbf{+}} & - & \textbf{5}\\
    \small{124} & \small{\textcite{liu_temporal_2022}}\index{Liu, Siyang|pagebf} & \textcolor{blue}{\textbf{+}} & \textcolor{blue}{\textbf{+}} & - & - & - & \textcolor{blue}{\textbf{+}} & - & - & - & - & \textbf{3}\\
    \small{125} & \small{\textcite{liu_concordance_2022}}\index{Liu, Siyang|pagebf} & - & - & - & \textcolor{blue}{\textbf{+}} & - & - & - & - & \textcolor{blue}{\textbf{+}} & - & \textbf{2}\\
    \small{126} & \small{\textcite{liu_use_2020}}\index{Liu, Yang|pagebf} & - & \textcolor{blue}{\textbf{+}} & \textcolor{blue}{\textbf{+}} & \textcolor{blue}{\textbf{+}} & \textcolor{blue}{\textbf{+}} & - & \textcolor{blue}{\textbf{+}} & - & \textcolor{blue}{\textbf{+}} & - & \textbf{6}\\
    \small{127} & \small{\textcite{liu_understanding_2020}}\index{Liu, Yang|pagebf} & - & - & - & \textcolor{blue}{\textbf{+}} & \textcolor{blue}{\textbf{+}} & \textcolor{blue}{\textbf{+}} & \textcolor{blue}{\textbf{+}} & - & \textcolor{blue}{\textbf{+}} & - & \textbf{5}\\
    \small{128} & \small{\textcite{liu_simultaneous_2015}}\index{Liu, Yang|pagebf} & - & - & - & \textcolor{blue}{\textbf{+}} & - & - & - & - & - & \textcolor{blue}{\textbf{+}} & \textbf{1}\\
    \small{129} & \small{\textcite{liu_solving_2012}}\index{Liu, Zhili|pagebf} & - & - & - & - & \textcolor{blue}{\textbf{+}} & - & - & - & - & - & \textbf{1}\\
    \small{130} & \small{\textcite{lu_improving_2018}}\index{Lu, Miaojia|pagebf} & - & - & - & - & \textcolor{blue}{\textbf{+}} & - & - & \textcolor{blue}{\textbf{+}} & - & - & \textbf{2}\\
    \small{131} & \small{\textcite{luan_better_2020}}\index{Luan, Xin|pagebf} & - & - & - & - & - & \textcolor{blue}{\textbf{+}} & \textcolor{blue}{\textbf{+}} & - & - & - & \textbf{2}\\
    \small{132} & \small{\textcite{ma_connecting_2022}}\index{Ma, Qingyu|pagebf} & - & - & - & - & \textcolor{blue}{\textbf{+}} & - & \textcolor{blue}{\textbf{+}} & - & \textcolor{blue}{\textbf{+}} & - & \textbf{3}\\
    \small{133} & \small{\textcite{ma_estimating_2019}}\index{Ma, Ting|pagebf} & - & - & - & \textcolor{blue}{\textbf{+}} & - & - & - & - & \textcolor{blue}{\textbf{+}} & - & \textbf{2}\\
    \small{134} & \small{\textcite{ma_bicycle_2015}}\index{Ma, Ting|pagebf} & \textcolor{blue}{\textbf{+}} & - & \textcolor{blue}{\textbf{+}} & - & - & \textcolor{blue}{\textbf{+}} & \textcolor{blue}{\textbf{+}} & \textcolor{blue}{\textbf{+}} & - & - & \textbf{5}\\
    \small{135} & \small{\textcite{ma_understanding_2018}}\index{Ma, Xinwei|pagebf} & - & - & - & \textcolor{blue}{\textbf{+}} & \textcolor{blue}{\textbf{+}} & \textcolor{blue}{\textbf{+}} & \textcolor{blue}{\textbf{+}} & - & \textcolor{blue}{\textbf{+}} & - & \textbf{5}\\
    \small{136} & \small{\textcite{ma_measuring_2018}}\index{Ma, Xinwei|pagebf} & - & - & - & \textcolor{blue}{\textbf{+}} & \textcolor{blue}{\textbf{+}} & - & \textcolor{blue}{\textbf{+}} & - & \textcolor{blue}{\textbf{+}} & - & \textbf{4}\\
    \small{137} & \small{\textcite{ma_impacts_2019}}\index{Ma, Xiaolei|pagebf} & - & - & - & - & - & - & - & \textcolor{blue}{\textbf{+}} & - & - & \textbf{1}\\
    \small{138} & \small{\textcite{marques_potential_2017}}\index{Marques, R.|pagebf} & - & - & - & \textcolor{blue}{\textbf{+}} & \textcolor{blue}{\textbf{+}} & - & - & - & - & - & \textbf{2}\\
    \small{139} & \small{\textcite{martens_bicycle_2004}}\index{Martens, Karel|pagebf} & - & - & - & \textcolor{blue}{\textbf{+}} & \textcolor{blue}{\textbf{+}} & - & \textcolor{blue}{\textbf{+}} & - & \textcolor{blue}{\textbf{+}} & - & \textbf{4}\\
    \small{141} & \small{\textcite{martin_evaluating_2014}}\index{Martin, Elliot W.|pagebf} & \textcolor{blue}{\textbf{+}} & - & - & \textcolor{blue}{\textbf{+}} & \textcolor{blue}{\textbf{+}} & - & \textcolor{blue}{\textbf{+}} & \textcolor{blue}{\textbf{+}} & - & - & \textbf{5}\\
    \small{142} & \small{\textcite{mcqueen_assessing_2022}}\index{McQueen, Michael|pagebf} & - & - & \textcolor{blue}{\textbf{+}} & - & - & - & \textcolor{blue}{\textbf{+}} & - & \textcolor{blue}{\textbf{+}} & - & \textbf{3}\\
    \small{143} & \small{\textcite{meng_influence_2016}}\index{Meng, Meng|pagebf} & - & - & \textcolor{blue}{\textbf{+}} & - & \textcolor{blue}{\textbf{+}} & - & \textcolor{blue}{\textbf{+}} & - & - & - & \textbf{3}\\
    \small{144} & \small{\textcite{midenet_modal_2018}}\index{Midenet, Sophie|pagebf} & - & - & \textcolor{blue}{\textbf{+}} & - & \textcolor{blue}{\textbf{+}} & \textcolor{blue}{\textbf{+}} & - & - & - & - & \textbf{3}\\
    \small{145} & \small{\textcite{mohammadian_analyzing_2022}}\index{Mohammadian, Abolfazl|pagebf} & - & - & - & - & - & - & \textcolor{blue}{\textbf{+}} & - & - & \textcolor{blue}{\textbf{+}} & \textbf{1}\\
    \small{146} & \small{\textcite{mohanty_effect_2017}}\index{Mohanty, Sudatta|pagebf} & - & - & \textcolor{blue}{\textbf{+}} & - & - & - & \textcolor{blue}{\textbf{+}} & - & - & - & \textbf{2}\\
    \small{147} & \small{\textcite{moinse_intermodal_2022}}\index{Moinse, Dylan|pagebf} & \textcolor{blue}{\textbf{+}} & - & - & - & \textcolor{blue}{\textbf{+}} & \textcolor{blue}{\textbf{+}} & \textcolor{blue}{\textbf{+}} & - & - & - & \textbf{4}\\
    \small{148} & \small{\textcite{molin_bicycle_2015}}\index{Molin, Eric|pagebf} & - & - & \textcolor{blue}{\textbf{+}} & - & - & \textcolor{blue}{\textbf{+}} & - & - & - & - & \textbf{2}\\
    \small{150} & \small{\textcite{montes_shared_2023}}\index{Montes, Alejandro|pagebf} & - & - & - & \textcolor{blue}{\textbf{+}} & - & \textcolor{blue}{\textbf{+}} & \textcolor{blue}{\textbf{+}} & - & \textcolor{blue}{\textbf{+}} & - & \textbf{4}\\
    \small{151} & \small{\textcite{nam_designing_2018}}\index{Nam, Daisik|pagebf}\index{Nam, Daisik|pagebf} & - & - & \textcolor{blue}{\textbf{+}} & - & - & - & - & - & - & \textcolor{blue}{\textbf{+}} & \textbf{1}\\
    \small{152} & \small{\textcite{ni_exploring_2020}}\index{Ni, Ying|pagebf} & \textcolor{blue}{\textbf{+}} & - & \textcolor{blue}{\textbf{+}} & \textcolor{blue}{\textbf{+}} & - & - & \textcolor{blue}{\textbf{+}} & - & \textcolor{blue}{\textbf{+}} & - & \textbf{5}\\
    \small{153} & \small{\textcite{nielsen_bikeability_2018}}\index{Nielsen, Thomas Alexander Sick|pagebf} & \textcolor{blue}{\textbf{+}} & \textcolor{blue}{\textbf{+}} & \textcolor{blue}{\textbf{+}} & \textcolor{blue}{\textbf{+}} & - & \textcolor{blue}{\textbf{+}} & - & - & - & - & \textbf{5}\\
    \small{154} & \small{\textcite{nigro_land_2019}} & - & - & - & - & \textcolor{blue}{\textbf{+}} & - & - & - & - & - & \textbf{1}\\
    \small{155} & \small{\textcite{oostendorp_combining_2018}}\index{Oostendorp, Rebekka|pagebf} & - & - & \textcolor{blue}{\textbf{+}} & - & - & - & \textcolor{blue}{\textbf{+}} & - & \textcolor{blue}{\textbf{+}} & - & \textbf{3}\\
    \small{156} & \small{\textcite{pages_les_2021}}\index{Pages, Thibaud|pagebf} & - & - & - & \textcolor{blue}{\textbf{+}} & - & - & \textcolor{blue}{\textbf{+}} & - & - & - & \textbf{2}\\
    \small{157} & \small{\textcite{pan_intermodal_2010}}\index{Pan, Haixiao|pagebf} & - & - & \textcolor{blue}{\textbf{+}} & - & \textcolor{blue}{\textbf{+}} & \textcolor{blue}{\textbf{+}} & - & - & - & - & \textbf{3}\\
    \small{159} & \small{\textcite{papon_evaluation_2017}}\index{Papon, Francis|pagebf} & - & - & - & - & - & \textcolor{blue}{\textbf{+}} & - & \textcolor{blue}{\textbf{+}} & - & - & \textbf{2}\\
    \small{160} & \small{\textcite{papon_rapport_2015}}\index{Papon, Francis|pagebf} & - & - & \textcolor{blue}{\textbf{+}} & \textcolor{blue}{\textbf{+}} & \textcolor{blue}{\textbf{+}} & \textcolor{blue}{\textbf{+}} & - & - & \textcolor{blue}{\textbf{+}} & - & \textbf{5}\\
    \small{161} & \small{\textcite{park_first-last-mile_2021}}\index{Park, Keunhyun|pagebf} & - & - & \textcolor{blue}{\textbf{+}} & \textcolor{blue}{\textbf{+}} & - & - & - & - & \textcolor{blue}{\textbf{+}} & - & \textbf{3}\\
    \small{162} & \small{\textcite{park_finding_2014}}\index{Park, Sungjin|pagebf} & - & - & \textcolor{blue}{\textbf{+}} & - & \textcolor{blue}{\textbf{+}} & - & \textcolor{blue}{\textbf{+}} & - & - & - & \textbf{3}\\
    \small{164} & \small{\textcite{qiu_interplay_2021}}\index{Qiu, Waishan|pagebf} & - & - & - & \textcolor{blue}{\textbf{+}} & \textcolor{blue}{\textbf{+}} & - & - & - & \textcolor{blue}{\textbf{+}} & - & \textbf{3}\\
    \small{165} & \small{\textcite{quarshie_integrating_2007}}\index{Quarshie, Magnus|pagebf} & - & - & \textcolor{blue}{\textbf{+}} & - & - & - & \textcolor{blue}{\textbf{+}} & - & - & - & \textbf{2}\\
    \small{166} & \small{\textcite{rabaud_quand_2022}}\index{Rabaud, Mathieu|pagebf} & - & - & - & \textcolor{blue}{\textbf{+}} & \textcolor{blue}{\textbf{+}} & - & - & - & \textcolor{blue}{\textbf{+}} & - & \textbf{3}\\
    \small{167} & \small{\textcite{radzimski_exploring_2021}}\index{Radzimski, Adam|pagebf} & - & - & - & - & - & \textcolor{blue}{\textbf{+}} & - & - & \textcolor{blue}{\textbf{+}} & - & \textbf{2}\\
    \small{168} & \small{\textcite{rastogi_willingness_2010}}\index{Rastogi, Rajat|pagebf} & - & - & \textcolor{blue}{\textbf{+}} & - & \textcolor{blue}{\textbf{+}} & - & \textcolor{blue}{\textbf{+}} & - & \textcolor{blue}{\textbf{+}} & - & \textbf{4}\\
    \small{169} & \small{\textcite{rastogi_travel_2003}}\index{Rastogi, Rajat|pagebf} & - & - & - & - & - & - & \textcolor{blue}{\textbf{+}} & - & - & - & \textbf{1}\\
    \small{170} & \small{\textcite{ravensbergen_biking_2018}}\index{Ravensbergen, Léa|pagebf} & - & - & \textcolor{blue}{\textbf{+}} & - & \textcolor{blue}{\textbf{+}} & - & \textcolor{blue}{\textbf{+}} & - & - & - & \textbf{3}\\
    \small{172} & \small{\textcite{rietveld_accessibility_2000}}\index{Rietveld, Piet|pagebf} & - & - & \textcolor{blue}{\textbf{+}} & \textcolor{blue}{\textbf{+}} & \textcolor{blue}{\textbf{+}} & - & - & - & - & - & \textbf{3}\\
    \small{173} & \small{\textcite{rijsman_walking_2019}}\index{Rijsman, Lotte|pagebf} & - & - & - & \textcolor{blue}{\textbf{+}} & \textcolor{blue}{\textbf{+}} & - & \textcolor{blue}{\textbf{+}} & - & - & - & \textbf{3}\\
    \small{174} & \small{\textcite{risimati_spatial_2021}}\index{Risimati, Brightnes|pagebf} & - & - & \textcolor{blue}{\textbf{+}} & - & - & - & - & - & - & - & \textbf{1}\\
    \small{175} & \small{\textcite{romm_differences_2022}}\index{Romm, Daniel|pagebf} & \textcolor{blue}{\textbf{+}} & - & - & \textcolor{blue}{\textbf{+}} & - & - & - & - & - & - & \textbf{2}\\
    \small{176} & \small{\textcite{schlueter_langdon_how_2021}}\index{Schlueter Langdon, Chris|pagebf} & - & - & - & - & - & - & - & \textcolor{blue}{\textbf{+}} & - & - & \textbf{1}\\
    \small{180} & \small{\textcite{shaheen_improving_2005}} & - & - & \textcolor{blue}{\textbf{+}} & - & - & - & - & - & - & - & \textbf{1}\\
    \small{181} & \small{\textcite{shelat_analysing_2018}}\index{Shelat, Sanmay|pagebf} & \textcolor{blue}{\textbf{+}} & - & \textcolor{blue}{\textbf{+}} & - & \textcolor{blue}{\textbf{+}} & - & \textcolor{blue}{\textbf{+}} & - & \textcolor{blue}{\textbf{+}} & - & \textbf{5}\\
    \small{182} & \small{\textcite{sherwin_practices_2011}}\index{Sherwin, Henrietta|pagebf} & - & - & - & \textcolor{blue}{\textbf{+}} & \textcolor{blue}{\textbf{+}} & \textcolor{blue}{\textbf{+}} & \textcolor{blue}{\textbf{+}} & - & - & - & \textbf{4}\\
    \small{183} & \small{\textcite{singleton_exploring_2014}}\index{Singleton, Patrick A.|pagebf} & \textcolor{blue}{\textbf{+}} & \textcolor{blue}{\textbf{+}} & \textcolor{blue}{\textbf{+}} & - & - & \textcolor{blue}{\textbf{+}} & - & \textcolor{blue}{\textbf{+}} & \textcolor{blue}{\textbf{+}} & - & \textbf{6}\\
    \small{185} & \small{\textcite{song_investigating_2020}}\index{Song, Ying|pagebf} & - & - & - & - & - & - & - & \textcolor{blue}{\textbf{+}} & - & - & \textbf{1}\\
    \small{186} & \small{\textcite{staricco_implementing_2020}}\index{Staricco, Luca|pagebf} & - & - & \textcolor{blue}{\textbf{+}} & - & - & \textcolor{blue}{\textbf{+}} & - & - & - & - & \textbf{2}\\
    \small{187} & \small{\textcite{stransky_quartiers_2017}}\index{Stransky, Václav|pagebf} & - & - & \textcolor{blue}{\textbf{+}} & - & - & - & - & - & - & \textcolor{blue}{\textbf{+}} & \textbf{1}\\
    \small{188} & \small{\textcite{stransky_periurbain_2019}}\index{Stransky, Václav|pagebf} & - & - & \textcolor{blue}{\textbf{+}} & \textcolor{blue}{\textbf{+}} & - & - & - & - & - & - & \textbf{2}\\
    \small{189} & \small{\textcite{tamakloe_determinants_2021}}\index{Tamakloe, Reuben|pagebf} & - & - & - & - & - & \textcolor{blue}{\textbf{+}} & - & - & - & - & \textbf{1}\\
    \small{190} & \small{\textcite{tang_uncovering_2021}}\index{Tang, Jinjun|pagebf} & - & - & - & - & - & - & - & - & \textcolor{blue}{\textbf{+}} & - & \textbf{1}\\
    \small{191} & \small{\textcite{tarpin-pitre_typology_2020}}\index{Tarpin-Pitre, Léandre|pagebf} & - & - & - & \textcolor{blue}{\textbf{+}} & \textcolor{blue}{\textbf{+}} & - & - & \textcolor{blue}{\textbf{+}} & \textcolor{blue}{\textbf{+}} & - & \textbf{4}\\
    \small{192} & \small{\textcite{taylor_analysis_1996}}\index{Taylor, Dean|pagebf} & - & - & \textcolor{blue}{\textbf{+}} & - & \textcolor{blue}{\textbf{+}} & - & - & - & \textcolor{blue}{\textbf{+}} & - & \textbf{3}\\
    \small{193} & \small{\textcite{tomita_demand_2017}}\index{Tomita, Yasuo|pagebf}\index{Tomita, Yasuo|pagebf} & - & - & \textcolor{blue}{\textbf{+}} & - & \textcolor{blue}{\textbf{+}} & \textcolor{blue}{\textbf{+}} & \textcolor{blue}{\textbf{+}} & - & - & - & \textbf{4}\\
    \small{194} & \small{\textcite{ton_understanding_2020}}\index{Ton, Danique|pagebf} & - & - & \textcolor{blue}{\textbf{+}} & - & \textcolor{blue}{\textbf{+}} & - & \textcolor{blue}{\textbf{+}} & - & - & - & \textbf{3}\\
    \small{195} & \small{\textcite{tyndall_complementarity_2022}} & - & - & - & - & - & \textcolor{blue}{\textbf{+}} & - & - & \textcolor{blue}{\textbf{+}} & - & \textbf{2}\\
    \small{196} & \small{\textcite{tzouras_describing_2023}}\index{Tzouras, Panagiotis|pagebf} & - & - & \textcolor{blue}{\textbf{+}} & - & \textcolor{blue}{\textbf{+}} & - & - & - & - & - & \textbf{2}\\
    \small{197} & \small{\textcite{van_der_nat_bicycle_2018}}\index{van der Nat, Johanna Debóra|pagebf} & - & - & - & - & - & \textcolor{blue}{\textbf{+}} & \textcolor{blue}{\textbf{+}} & \textcolor{blue}{\textbf{+}} & \textcolor{blue}{\textbf{+}} & - & \textbf{4}\\
    \small{198} & \small{\textcite{waerden_relation_2018}}\index{Waerden, Peter|pagebf} & - & - & - & - & - & \textcolor{blue}{\textbf{+}} & - & - & - & - & \textbf{1}\\
    \small{199} & \small{\textcite{kampen_understanding_2020}}\index{van Kampen, Jullian|pagebf} & - & - & \textcolor{blue}{\textbf{+}} & - & \textcolor{blue}{\textbf{+}} & - & - & - & - & - & \textbf{2}\\
    \small{200} & \small{\textcite{kampen_bicycle_2021}}\index{van Kampen, Jullian|pagebf} & \textcolor{blue}{\textbf{+}} & \textcolor{blue}{\textbf{+}} & - & - & \textcolor{blue}{\textbf{+}} & - & \textcolor{blue}{\textbf{+}} & - & \textcolor{blue}{\textbf{+}} & - & \textbf{5}\\
    \small{201} & \small{\textcite{kampen_understanding_2021}}\index{van Kampen, Jullian|pagebf} & - & - & - & - & \textcolor{blue}{\textbf{+}} & - & - & - & - & - & \textbf{1}\\
    \small{202} & \small{\textcite{kuijk_preferences_2022}}\index{van Kuijk, R.J.|pagebf} & - & - & - & - & \textcolor{blue}{\textbf{+}} & \textcolor{blue}{\textbf{+}} & \textcolor{blue}{\textbf{+}} & - & \textcolor{blue}{\textbf{+}} & - & \textbf{4}\\
    \small{203} & \small{\textcite{van_mil_insights_2020}}\index{van Mil, Joeri F.P.|pagebf} & - & - & - & - & \textcolor{blue}{\textbf{+}} & \textcolor{blue}{\textbf{+}} & - & - & - & - & \textbf{2}\\
    \small{204} & \small{\textcite{vinagre_diaz_blind_2023}}\index{Vinagre Díaz, Juan José|pagebf} & - & - & - & - & - & - & - & \textcolor{blue}{\textbf{+}} & - & - & \textbf{1}\\
    \small{205} & \small{\textcite{wang_bicycle-transit_2013}}\index{Wang, Rui|pagebf} & \textcolor{blue}{\textbf{+}} & \textcolor{blue}{\textbf{+}} & - & \textcolor{blue}{\textbf{+}} & \textcolor{blue}{\textbf{+}} & - & \textcolor{blue}{\textbf{+}} & - & \textcolor{blue}{\textbf{+}} & - & \textbf{6}\\
    \small{206} & \small{\textcite{wang_relationship_2020}}\index{Wang, Ruoyu|pagebf} & \textcolor{blue}{\textbf{+}} & \textcolor{blue}{\textbf{+}} & \textcolor{blue}{\textbf{+}} & - & - & - & - & - & \textcolor{blue}{\textbf{+}} & - & \textbf{4}\\
    \small{207} & \small{\textcite{wang_interchange_2016}}\index{Wang, Zi-jia|pagebf} & - & - & \textcolor{blue}{\textbf{+}} & - & \textcolor{blue}{\textbf{+}} & - & - & - & - & - & \textbf{2}\\
    \small{208} & \small{\textcite{wang_spatiotemporal_2020}}\index{Wang, Zijia|pagebf} & - & - & - & - & - & \textcolor{blue}{\textbf{+}} & - & - & \textcolor{blue}{\textbf{+}} & - & \textbf{2}\\
    \small{209} & \small{\textcite{welch_long-term_2016}}\index{Welch, Timothy F.|pagebf} & - & - & \textcolor{blue}{\textbf{+}} & - & - & - & - & \textcolor{blue}{\textbf{+}} & - & - & \textbf{2}\\
    \small{211} & \small{\textcite{weliwitiya_bicycle_2019}}\index{Weliwitiya, Hesara|pagebf} & \textcolor{blue}{\textbf{+}} & \textcolor{blue}{\textbf{+}} & \textcolor{blue}{\textbf{+}} & \textcolor{blue}{\textbf{+}} & - & \textcolor{blue}{\textbf{+}} & \textcolor{blue}{\textbf{+}} & - & - & - & \textbf{6}\\
    \small{212} & \small{\textcite{wu_identification_2023}}\index{Wu, Hao|pagebf}\index{Wu, Hao|pagebf} & - & - & \textcolor{blue}{\textbf{+}} & \textcolor{blue}{\textbf{+}} & - & - & - & - & \textcolor{blue}{\textbf{+}} & - & \textbf{3}\\
    \small{214} & \small{\textcite{wu_measuring_2019}}\index{Wu, Xueying|pagebf} & - & \textcolor{blue}{\textbf{+}} & \textcolor{blue}{\textbf{+}} & - & \textcolor{blue}{\textbf{+}} & \textcolor{blue}{\textbf{+}} & - & - & - & - & \textbf{4}\\
    \small{215} & \small{\textcite{yan_spatiotemporal_2021}}\index{Yan Xiang|pagebf} & - & - & - & - & \textcolor{blue}{\textbf{+}} & - & - & - & \textcolor{blue}{\textbf{+}} & - & \textbf{2}\\
    \small{216} & \small{\textcite{yan_evaluating_2023}}\index{Yan, Xiang|pagebf} & - & - & - & - & - & \textcolor{blue}{\textbf{+}} & \textcolor{blue}{\textbf{+}} & \textcolor{blue}{\textbf{+}} & - & - & \textbf{3}\\
    \small{217} & \small{\textcite{yang_bike-and-ride_2014}}\index{Yang, Liu|pagebf}\index{Yang, Liu|pagebf} & - & - & \textcolor{blue}{\textbf{+}} & - & \textcolor{blue}{\textbf{+}} & \textcolor{blue}{\textbf{+}} & \textcolor{blue}{\textbf{+}} & - & - & - & \textbf{4}\\
    \small{218} & \small{\textcite{yang_empirical_2016}}\index{Yang, Min|pagebf} & - & - & \textcolor{blue}{\textbf{+}} & - & \textcolor{blue}{\textbf{+}} & \textcolor{blue}{\textbf{+}} & \textcolor{blue}{\textbf{+}} & - & - & - & \textbf{4}\\
    \small{219} & \small{\textcite{yang_metro_2015}}\index{Yang, Min|pagebf} & - & - & \textcolor{blue}{\textbf{+}} & - & \textcolor{blue}{\textbf{+}} & - & \textcolor{blue}{\textbf{+}} & - & - & - & \textbf{3}\\
    \small{220} & \small{\textcite{yang_spatiotemporal_2019}}\index{Yang, Yuanxuan|pagebf} & - & - & - & \textcolor{blue}{\textbf{+}} & - & - & - & \textcolor{blue}{\textbf{+}} & \textcolor{blue}{\textbf{+}} & - & \textbf{3}\\
    \small{221} & \small{\textcite{yen_how_2023}}\index{Yen, Barbara T.H.|pagebf} & - & - & - & \textcolor{blue}{\textbf{+}} & - & - & - & - & \textcolor{blue}{\textbf{+}} & - & \textbf{2}\\
    \small{222} & \small{\textcite{yu_policy_2021}}\index{Yu, Qing|pagebf} & - & \textcolor{blue}{\textbf{+}} & - & - & \textcolor{blue}{\textbf{+}} & - & - & \textcolor{blue}{\textbf{+}} & \textcolor{blue}{\textbf{+}} & - & \textbf{4}\\
    \small{223} & \small{\textcite{yu_understanding_2021}}\index{Yu, Senbin|pagebf} & \textcolor{blue}{\textbf{+}} & - & - & \textcolor{blue}{\textbf{+}} & - & - & - & - & \textcolor{blue}{\textbf{+}} & - & \textbf{3}\\
    \small{224} & \small{\textcite{zhang_make_2023}}\index{Zhang, Mengyuan|pagebf} & - & - & \textcolor{blue}{\textbf{+}} & - & \textcolor{blue}{\textbf{+}} & - & - & - & - & - & \textbf{2}\\
    \small{225} & \small{\textcite{zhang_bicyclemetro_2019}}\index{Zhang, Ze|pagebf} & - & - & - & - & \textcolor{blue}{\textbf{+}} & - & - & - & - & - & \textbf{1}\\
    \small{226} & \small{\textcite{zhao_bicycle-metro_2017}}\index{Zhao, Pengjun|pagebf} & - & \textcolor{blue}{\textbf{+}} & \textcolor{blue}{\textbf{+}} & - & \textcolor{blue}{\textbf{+}} & \textcolor{blue}{\textbf{+}} & \textcolor{blue}{\textbf{+}} & - & \textcolor{blue}{\textbf{+}} & - & \textbf{6}\\
    \small{227} & \small{\textcite{zhao_public_2022}}\index{Zhao, Pengjun|pagebf} & - & - & \textcolor{blue}{\textbf{+}} & - & \textcolor{blue}{\textbf{+}} & - & \textcolor{blue}{\textbf{+}} & - & \textcolor{blue}{\textbf{+}} & - & \textbf{4}\\
    \small{229} & \small{\textcite{zhong_layout_2021}}\index{Zhong, Hongming|pagebf} & - & - & \textcolor{blue}{\textbf{+}} & \textcolor{blue}{\textbf{+}} & - & \textcolor{blue}{\textbf{+}} & - & - & - & - & \textbf{3}\\
    \small{230} & \small{\textcite{zhou_spatially_2023}}\index{Zhou, X.|pagebf} & \textcolor{blue}{\textbf{+}} & \textcolor{blue}{\textbf{+}} & \textcolor{blue}{\textbf{+}} & - & \textcolor{blue}{\textbf{+}} & - & - & - & \textcolor{blue}{\textbf{+}} & - & \textbf{5}\\
    \small{231} & \small{\textcite{zhu_improved_2021}}\index{Zhu, Zhenjun|pagebf} & - & - & - & \textcolor{blue}{\textbf{+}} & - & \textcolor{blue}{\textbf{+}} & - & - & - & - & \textbf{2}\\
    \small{232} & \small{\textcite{ziedan_complement_2021}}\index{Ziedan, Abubakr|pagebf} & - & - & - & - & - & - & - & \textcolor{blue}{\textbf{+}} & - & - & \textbf{1}\\
    \small{233} & \small{\textcite{zuniga-garcia_evaluation_2022}}\index{Zuniga-Garcia, Natalia|pagebf} & - & - & - & \textcolor{blue}{\textbf{+}} & \textcolor{blue}{\textbf{+}} & - & \textcolor{blue}{\textbf{+}} & - & \textcolor{blue}{\textbf{+}} & - & \textbf{4}\\
    \small{234} & \small{\textcite{zuo_bikeway_2019}}\index{Zuo, Ting|pagebf} & - & - & \textcolor{blue}{\textbf{+}} & - & - & \textcolor{blue}{\textbf{+}} & - & - & - & - & \textbf{2}\\
    \small{235} & \small{\textcite{zuo_promote_2020}}\index{Zuo, Ting|pagebf} & - & \textcolor{blue}{\textbf{+}} & \textcolor{blue}{\textbf{+}} & - & - & - & \textcolor{blue}{\textbf{+}} & \textcolor{blue}{\textbf{+}} & - & - & \textbf{4}\\
    \small{236} & \small{\textcite{zuo_incorporating_2021}}\index{Zuo, Ting|pagebf} & - & - & \textcolor{blue}{\textbf{+}} & \textcolor{blue}{\textbf{+}} & - & - & \textcolor{blue}{\textbf{+}} & - & - & - & \textbf{3}\\
    \small{237} & \small{\textcite{zuo_determining_2018}}\index{Zuo, Ting|pagebf} & - & - & \textcolor{blue}{\textbf{+}} & \textcolor{blue}{\textbf{+}} & \textcolor{blue}{\textbf{+}} & - & - & \textcolor{blue}{\textbf{+}} & - & - & \textbf{4}\\
    \small{238} & \small{\textcite{zuo_first-and-last_2020}}\index{Zuo, Ting|pagebf} & - & - & \textcolor{blue}{\textbf{+}} & \textcolor{blue}{\textbf{+}} & - & - & \textcolor{blue}{\textbf{+}} & - & - & - & \textbf{3}\\
\hline
    \textbf{TOTAL} & & \small{\textbf{44}} & \small{\textbf{36}} & \small{\textbf{102}} & \small{\textbf{78}} & \small{\textbf{121}} & \small{\textbf{61}} & \small{\textbf{90}} & \small{\textbf{48}} & \small{\textbf{84}} & \small{\textbf{13}} & \small{\textbf{677}}\\
        \hline
        \caption*{}
        \begin{flushright}
        \scriptsize
    Auteur~: \textcopyright~Moinse 2023
        \end{flushright}
        \end{longtable}

    % Annexe D.1
    \newpage
\subsection{Corpus de la \acrshort{RSL} sur les résultats en lien avec la densité de la population}
    \label{donnees-ouvertes:rsl_resultats_densite}

    % Référence
Le présent tableau synthétise les résultats issus de la revue de littérature portant plus particulièrement sur la \hyperref[Densité de la population]{sous-partie consacrée à la densité de la population} (page \pageref{Densité de la population}).\par
    
    % Tableau résultats RSL (densité)
        \begin{longtable}{p{3cm}p{3cm}p{1.5cm}p{1.8cm}p{3.3cm}}
        \hline
        \textcolor{blue}{\textbf{Références}} & \textcolor{blue}{\textbf{Densité}} & \textcolor{blue}{\textbf{MIL}} & \textcolor{blue}{\textbf{TC}} & \textcolor{blue}{\textbf{Contexte}}
        \hline
        \endhead
\multicolumn{5}{l}{\textbf{Effets positifs}}\\
    \small{\textcite{shelat_analysing_2018}}\index{Shelat, Sanmay|pagebf} & \small{Association positive} & \small{Vélo} & \small{Train} & \small{Pays-Bas}\\ 
    \small{\textcite{la_paix_puello_integration_2016}}\index{La Paix Puello, Lissy|pagebf} & \small{Association positive} & \small{Vélo} & \small{Train} & \small{La Haye, Rotterdam (Pays-Bas)}\\
    \small{\textcite{chan_factors_2020}}\index{Chan, Kevin|pagebf} & \small{Association positive} & \small{Vélo} & \small{Train} & \small{Toronto, Hamilton (Canada)}\\ 
    \small{\textcite{wang_bicycle-transit_2013}}\index{Wang, Rui|pagebf} & \small{Association positive} & \small{Vélo} & \small{Train, Bus} & \small{États-Unis}\\ 
    \small{\textcite{singleton_exploring_2014}}\index{Singleton, Patrick A.|pagebf} & \small{Association positive} & \small{Vélo} & \small{Métro} & \small{Portland (États-Unis)}\\
    \small{\textcite{nielsen_bikeability_2018}}\index{Nielsen, Thomas Alexander Sick|pagebf} & \small{Association positive} & \small{Vélo} & \small{Train, Bus} & \small{Danemark}\\
    \small{\textcite{li_exploring_2017}}\index{Li, Wenxiang|pagebf} & \small{Association positive} & \small{Vélo} & \small{Train, Bus} & \small{Austin (États-Unis)}\\
    \small{\textcite{li_investigating_2022}}\index{Li, Xiaofeng|pagebf} & \small{Association positive} & \small{VLS} & \small{Tramway, Bus} & \small{Tucson (États-Unis)}\\
    \small{\textcite{bocker_bike_2020}}\index{Böcker, Lars|pagebf} & \small{Association positive} & \small{VLS} & \small{Métro} & \small{Chengdu (Chine)}\\
    \small{\textcite{kong_deciphering_2020}}\index{Kong, Hui|pagebf} & \small{Association positive} & \small{VLS} & \small{Train, Métro, Tramway, Bus} & \small{Boston, Chicago, Washington D.C., New York City (États-Unis)}\\
    \small{\textcite{romm_differences_2022}}\index{Romm, Daniel|pagebf} & \small{Association positive en \gls{rabattement}} & \small{VLS} & \small{Métro} & \small{Boston (États-Unis)}\\ 
    \small{\textcite{cheng_expanding_2018}}\index{Cheng, Yung-Hsiang|pagebf} & \small{Association positive en \gls{diffusion}} & \small{VLS} & \small{Métro} & \small{Kaohsiung (Taïwan)}\\
    \small{\textcite{martin_evaluating_2014}}\index{Martin, Elliot W.|pagebf} & \small{Association positive dans les zones résidentielles} & \small{VLS} & \small{Métro, Tramway} & \small{Washington D.C., Minneapolis (États-Unis)}\\
    \small{\textcite{chen_what_2022}}\index{Chen, Wendong|pagebf} & \small{Association positive} & \small{VLS, VFF} & \small{Métro} & \small{Nanjing (Chine)}\\
    \small{\textcite{liu_mode_2022}}\index{Liu, Lumei|pagebf} & \small{Association positive} & \small{VFF} & \small{Métro, Bus} & \small{Beijing (Chine)}\\
    \small{\textcite{yu_understanding_2021}}\index{Yu, Senbin|pagebf} & \small{Association positive} & \small{VFF} & \small{Métro} & \small{Beijing (Chine)}\\
    \small{\textcite{lin_analysis_2019}}\index{Lin, Diao|pagebf} & \small{Association positive} & \small{VFF} & \small{Métro} & \small{Shanghai (Chine)}\\
    \small{\textcite{li_operating_2019}} & \small{Association positive} & \small{VFF} & \small{Métro} & \small{Nanjing (Chine)}\\
    \small{\textcite{li_exploring_2021}}\index{Li, Wei|pagebf} & \small{Association positive} & \small{VFF} & \small{Métro} & \small{Shanghai (Chine)}\\
    \small{\textcite{wang_relationship_2020}}\index{Wang, Ruoyu|pagebf} & \small{Association positive} & \small{VFF} & \small{Métro} & \small{Shenzhen (Chine)}\\
    \small{\textcite{liu_temporal_2022}}\index{Liu, Siyang|pagebf} & \small{Association positive, principalement le matin} & \small{VFF} & \small{Métro} & \small{Beijing (Chine)}\\
    \small{\textcite{cheng_exploring_2022}}\index{Cheng, Long|pagebf} & \small{Association positive le matin} & \small{VFF} & \small{Métro} & \small{Nanjing (Chine)}\\
    \small{\textcite{guo_dockless_2021}}\index{Guo, Yuanyuan|pagebf} & \small{Association positive le matin et le soir} & \small{VFF} & \small{Métro} & \small{Shenzhen (Chine)}\\
    \small{\textcite{lee_forecasting_2021}}\index{Lee, Mina|pagebf} & \small{Association positive} & \small{TEFF} & \small{Métro} & \small{New York City (États-Unis)}\\
    \small{\textcite{kampen_bicycle_2021}}\index{van Kampen, Jullian|pagebf} & \small{Association positive avec le stationnement} & \small{Vélo} & \small{Métro} & \small{Amsterdam (Pays-Bas)}\\
    \hline
\multicolumn{5}{l}{\textbf{Effets négatifs}}\\
    \small{\textcite{gan_associations_2021}}\index{Gan, Zuoxian|pagebf} & \small{Association négative} & \small{Vélo} & \small{Métro} & \small{Nanjing (Chine)}\\
    \small{\textcite{la_paix_puello_modelling_2015}}\index{La Paix Puello, Lissy|pagebf} & \small{Association négative dans les zones résidentielles denses} & \small{Vélo} & \small{Train} & \small{Randstad South (Pays-Bas)}\\
    \small{\textcite{ji_public_2017}}\index{Ji, Yanjie|pagebf} & \small{Association négative} & \small{Vélo, VLS} & \small{Métro} & \small{Nanjing (Chine)}\\
    \small{\textcite{cheng_promoting_2022}} & \small{Association positive en deça de 18~000  habitants/km²} & \small{VLS} & \small{Métro} & \small{Nanjing (Chine)}\\     
    \small{\textcite{cheng_comparison_2023}}\index{Cheng, Long|pagebf} & \small{Association positive en deça de 18~000  habitants/km²} & \small{VLS, VFF} & \small{Métro} & \small{Nanjing (Chine)}\\     
    \small{\textcite{glass_role_2020}}\index{Glass, Caroline|pagebf} & \small{Association négative} & \small{VLS} & \small{Bus} & \small{Birmingham (États-Unis)}\\
    \small{\textcite{hu_examining_2022}}\index{Hu, Songhua|pagebf} & \small{Association négative} & \small{VFF} & \small{Métro} & \small{Shanghai (Chine)}\\ 
    \hline
\multicolumn{5}{l}{\textbf{Effets ambivalents}}\\
    \small{\textcite{cooke_relationship_2018}}\index{Cooke, Sean|pagebf} & \small{Association positive accompagnée de risques de congestion} & \small{Vélo} & \small{Bus} & \small{Cape Town, Tshwane, Joburg, Nelson Mandela Bay, eThekwini (Afrique du Sud)}\\
    \small{\textcite{kager_characterisation_2016}}\index{Kager, Roland|pagebf} & \small{Interface entre les zones denses et peu denses} & \small{Vélo} & \small{Train} & \small{Amsterdam, Amstelveen, Diemen, Ouder-Amstel (Pays-Bas)}\\
    \small{\textcite{weliwitiya_bicycle_2019}}\index{Weliwitiya, Hesara|pagebf} & \small{Facteur non significatif} & \small{Vélo} & \small{Train} & \small{Melbourne (Australie)}\\
    \small{\textcite{krygsman_multimodal_2004}}\index{Krygsman, Stephan|pagebf} & \small{Facteur non significatif} & \small{Vélo} & \small{Train, Métro, Tramway, Bus} & \small{Utrecht (Pays-Bas)}\\
    \small{\textcite{moinse_intermodal_2022}}\index{Moinse, Dylan|pagebf} & \small{Association positive pour la TEP, mais négative pour le vélo} & \small{Vélo, TEP} & \small{Train} & \small{Provence-Alpes-Côte d'Azur (France)}\\
    \small{\textcite{lin_built_2018}}\index{Lin, Jen-Jia|pagebf} & \small{Association positive à Beijing, mais non significative à Taipei et Tokyo} & \small{VLS} & \small{Métro} & \small{Beijing (Chine), Taipei (Taïwan), Tokyo (Japon)}\\
    \small{\textcite{griffin_planning_2016}}\index{Griffin, Greg|pagebf} & \small{Facteur non significatif} & \small{VLS} & \small{Train, Bus} & \small{Austin, Chicago (États-Unis)}\\
    \small{\textcite{guo_built_2020}}\index{Guo, Yuanyuan|pagebf} & \small{Facteur non significatif} & \small{VFF} & \small{Métro} & \small{Shenzhen (Chine)}\\
    \small{\textcite{guo_role_2021}}\index{Guo, Yuanyuan|pagebf} & \small{Facteur non significatif} & \small{VFF} & \small{Métro} & \small{Shenzhen (Chine)}\\
    \small{\textcite{ni_exploring_2020}}\index{Ni, Ying|pagebf} & \small{Association non significative, mais positive pour le taxi} & \small{VFF} & \small{Métro} & \small{Beijing (Chine)}\\
    \small{\textcite{zhou_spatially_2023}}\index{Zhou, X.|pagebf} & \small{Facteur non significatif, voire désincitatif} & \small{VFF} & \small{Métro, Bus} & \small{Beijing (Chine)}\\
        \hline
        \caption*{Corpus scientifique se rapportant à la relation entre la densité de la population et l'intégration de la mobilité individuelle légère avec les systèmes de transport en commun, dans le cadre de la \acrshort{RSL}}
        \label{Corpus scientifique se rapportant à la relation entre la densité de la population et l'intégration de la mobilité individuelle légère avec les systèmes de transport en commun, dans le cadre de la RSL}
        \begin{flushright}
        \scriptsize
    Auteur~: \textcopyright~Moinse 2023
        \end{flushright}
        \end{longtable}

    % Annexe D.2
    \newpage
\subsection{Corpus de la \acrshort{RSL} sur les résultats en lien avec la diversité fonctionnelle}
    \label{donnees-ouvertes:rsl_resultats_diversite}

    % Référence
Le présent tableau synthétise les résultats issus de la revue de littérature portant plus particulièrement sur la \hyperref[Diversité fonctionnelle]{sous-partie consacrée à la diversité fonctionnelle} (page \pageref{Diversité fonctionnelle}).\par

    % Tableau résultats RSL (diversité)
        \begin{longtable}{p{3cm}p{4cm}p{1.5cm}p{1.8cm}p{2.3cm}}
        \hline
        \textcolor{blue}{\textbf{Références}} & \textcolor{blue}{\textbf{Diversité}} & \textcolor{blue}{\textbf{MIL}} & \textcolor{blue}{\textbf{TC}} & \textcolor{blue}{\textbf{Contexte}}
        \hline
        \endhead
\multicolumn{5}{l}{\textbf{Effets positifs de la densité d'emplois}}\\
    \small{\textcite{zuo_promote_2020}}\index{Zuo, Ting|pagebf} & \small{Association positive} & \small{Vélo} & \small{Bus} & \small{Hamilton (États-Unis)}\\
    \small{\textcite{nielsen_bikeability_2018}}\index{Nielsen, Thomas Alexander Sick|pagebf} & \small{Association positive avec le secteur du détail} & \small{Vélo} & \small{Train} & \small{Danemark}\\
    \small{\textcite{ma_bicycle_2015}}\index{Ma, Ting|pagebf} & \small{Association positive} & \small{VLS} & \small{Métro} & \small{Washington D.C. (États-Unis)}\\
    \small{\textcite{cheng_comparison_2023}}\index{Cheng, Long|pagebf} & \small{Association positive} & \small{VLS} & \small{Métro} & \small{Nanjing (Chine)}\\
    \small{\textcite{bocker_bike_2020}}\index{Böcker, Lars|pagebf} & \small{Association positive en \gls{rabattement}} & \small{VLS} & \small{Métro} & \small{Oslo (Norvège)}\\
    \small{\textcite{cheng_promoting_2022}} & \small{Association positive jusqu'à 13~000 habitants/km²} & \small{VLS} & \small{Métro} & \small{Nanjing (Chine)}\\
    \small{\textcite{cheng_exploring_2022}}\index{Cheng, Long|pagebf} & \small{Association positive en soirée} & \small{VFF} & \small{Métro} & \small{Nanjing (Chine)}\\
    \hline
\multicolumn{5}{l}{\textbf{Effets négatifs de la densité d'emplois}}\\
    \small{\textcite{guo_built_2020, guo_role_2021}} & \small{Association négative la matinée} & \small{VFF} & \small{Métro} & \small{Shenzhen (Chine)}\\
    \small{\textcite{lin_built_2018}}\index{Lin, Jen-Jia|pagebf} & \small{Association négative à Taipei et Tokyo, mais positive à Beijing} & \small{VLS} & \small{Métro} & \small{Beijing (Chine), Taipei (Taïwan), Tokyo (Japon)}\\
    \hline
\multicolumn{5}{l}{\textbf{Effets positifs de la diversité fonctionnelle}}\\
    \small{\textcite{gan_associations_2021}}\index{Gan, Zuoxian|pagebf} & \small{Association positive} & \small{Vélo} & \small{Métro} & \small{Nanjing (Chine)}\\
    \small{\textcite{weliwitiya_bicycle_2019}}\index{Weliwitiya, Hesara|pagebf} & \small{Association positive} & \small{Vélo} & \small{Train} & \small{Melbourne (Australie)}\\
    \small{\textcite{kampen_bicycle_2021}}\index{van Kampen, Jullian|pagebf} & \small{Rôle des services de restauration et des activités commerciales dans le choix du lieu de stationnement} & \small{Vélo} & \small{Métro} & \small{Amsterdam (Pays-Bas)}\\
    \small{\textcite{bocker_bike_2020}}\index{Böcker, Lars|pagebf} & \small{Association positive} & \small{VLS} & \small{Métro} & \small{Oslo (Norvège)}\\
    \small{\textcite{song_investigating_2020}}\index{Song, Ying|pagebf} & \small{Association positive} & \small{VLS} & \small{Métro} & \small{Minneapolis, Saint-Paul (États-Unis)}\\
    \small{\textcite{yu_policy_2021}}\index{Yu, Qing|pagebf} & \small{Association positive} & \small{VLS} & \small{Métro} & \small{Shanghai (Chine)}\\
    \small{\textcite{kong_deciphering_2020}}\index{Kong, Hui|pagebf} & \small{Association positive} & \small{VLS} & \small{Train, Métro, Tramway, Bus} & \small{Boston, Chicago, Washington D.C., New York City (États-Unis)}\\
    \small{\textcite{ji_exploring_2018}}\index{Ji, Yanjie|pagebf} & \small{Association positive avec les activités liées aux loisirs} & \small{VLS} & \small{Métro} & \small{Nanjing (Chine)}\\
    \small{\textcite{bi_analysis_2021}}\index{Bi, Hui|pagebf} & \small{Association positive avec les activités liées aux commerces} & \small{VLS} & \small{Métro} & \small{Chengdu (Chine)}\\
    \small{\textcite{chen_what_2022}}\index{Chen, Wendong|pagebf} & \small{Association positive avec les services liés à la restauration en dehors des heures de pointe et les week-ends} & \small{VLS} & \small{Métro} & \small{Nanjing (Chine)}\\
    \small{\textcite{liu_use_2020}}\index{Liu, Yang|pagebf} & \small{Association positive} & \small{VFF} & \small{Métro} & \small{Nanjing (Chine)}\\
    \small{\textcite{wang_relationship_2020}}\index{Wang, Ruoyu|pagebf} & \small{Association positive} & \small{VFF} & \small{Métro} & \small{Shenzhen (Chine)}\\
    \small{\textcite{guo_built_2020, guo_role_2021}} & \small{Association positive avec les zones résidentielles et industrielles en périphérie urbaine} & \small{VFF} & \small{Métro} & \small{Shenzhen (Chine)}\\
    \hline
\multicolumn{5}{l}{\textbf{Effets ambivalents de la diversité fonctionnelle}}\\
    \small{\textcite{zhao_bicycle-metro_2017}}\index{Zhao, Pengjun|pagebf} & \small{Association positive avec les parcs, mais négative avec les centres commerciaux} & \small{Vélo, VLS} & \small{Métro} & \small{Beijing (Chine)}\\
    \small{\textcite{kim_analysis_2021}}\index{Kim, Minjun|pagebf} & \small{Association positive avec les zones résidentielles, mais négative avec les zones commerciales, les centres d'affaires et les campus} & \small{Vélo, VLS} & \small{Métro, Bus} & \small{Séoul (Corée du Sud)}\\
    \small{\textcite{lin_built_2018}}\index{Lin, Jen-Jia|pagebf} & \small{Facteur non significatif à Beijing et Tokyo} & \small{VLS} & \small{Métro} & \small{Beijing (Chine), Taipei (Taïwan), Tokyo (Japon)}\\
    \small{\textcite{zhou_spatially_2023}}\index{Zhou, X.|pagebf} & \small{Association positive avec les lieux d'enseignement et de culture, mais négative avec les centres commerciaux, les activités liées aux loisirs et les services d'hébergement} & \small{VFF} & \small{Métro} & \small{Beijing (Chine)}\\
    \small{\textcite{guo_dockless_2021}}\index{Guo, Yuanyuan|pagebf} & \small{Association positive avec les zones commerciales et les activités liées aux loisirs, mais négative avec les bureaux} & \small{VFF} & \small{Métro} & \small{Shenzhen (Chine)}\\
    \small{\textcite{hu_examining_2022}}\index{Hu, Songhua|pagebf} & \small{Association positive avec les zones résidentielles et commerciales et les campus, mais négative avec les zones agricoles} & \small{VFF} & \small{Métro} & \small{Shanghai (Chine)}\\
    \small{\textcite{liu_temporal_2022}}\index{Liu, Siyang|pagebf} & \small{Association positive, mais négative avec les espaces verts et les zones industrielles} & \small{VFF} & \small{Métro} & \small{Beijing (Chine)}\\
    \hline
\multicolumn{5}{l}{\textbf{Effets négatifs de la diversité fonctionnelle}}\\
    \small{\textcite{cheng_promoting_2022}} & \small{Association négative avec les zones commerciales} & \small{VLS} & \small{Métro} & \small{Nanjing (Chine)}\\
    \small{\textcite{cheng_comparison_2023}}\index{Cheng, Long|pagebf} & \small{Association négative} & \small{VFF} & \small{Métro} & \small{Nanjing (Chine)}\\
    \hline
\multicolumn{5}{l}{\textbf{Rôle de la mixité de l'habitat}}\\
    \small{\textcite{wang_bicycle-transit_2013}}\index{Wang, Rui|pagebf} & \small{Association positive avec la proportion de logements locatifs} & \small{Vélo} & \small{Train, Bus} & \small{États-Unis}\\
        \hline
        \caption*{Corpus scientifique se rapportant à la relation entre la diversité des \gls{fonctions urbaines} et l'intégration de la \gls{mobilité individuelle légère} avec les systèmes de transport en commun, dans le cadre de la \acrshort{RSL}}
        \label{Corpus scientifique se rapportant à la relation entre la diversité des fonctions urbaines et l'intégration de la mobilité individuelle légère avec les systèmes de transport en commun, dans le cadre de la RSL}
        \begin{flushright}
        \scriptsize
    Auteur~: \textcopyright~Moinse 2023
        \end{flushright}
        \end{longtable}

    % Annexe D.3
    \newpage
\subsection{Corpus de la \acrshort{RSL} sur les résultats en lien avec le design du stationnement vélo}
    \label{donnees-ouvertes:rsl_resultats_stationnement_velo}

    % Référence
Le présent tableau synthétise les résultats issus de la revue de littérature portant plus particulièrement sur la \hyperref[Traitement des espaces publics]{sous-partie consacrée au traitement des espaces publics} (page \pageref{Traitement des espaces publics}).\par
    
    % Tableau résultats RSL (stationnement vélo)
        \begin{longtable}{p{3cm}p{4cm}p{1.5cm}p{1.8cm}p{2.3cm}}
        \hline
        \textcolor{blue}{\textbf{Références}} & \textcolor{blue}{\textbf{Design (parkings)}} & \textcolor{blue}{\textbf{MIL}} & \textcolor{blue}{\textbf{TC}} & \textcolor{blue}{\textbf{Contexte}}
        \hline
        \endhead
\multicolumn{5}{l}{\textbf{Emplacements dédiés au vélo dans les transports en commun et les stations}}\\
    \small{\textcite{advani_bicycle_2006}}\index{Advani, Mukti|pagebf} & \small{Stationnement vélo dans et autour des Bus} & \small{Vélo} & \small{Bus} & \small{New Delhi (Inde)}\\
    \small{\textcite{ravensbergen_biking_2018}}\index{Ravensbergen, Léa|pagebf} & \small{Aménagement adapté aux cyclistes en gare} & \small{Vélo} & \small{Train} & \small{Toronto, Hamilton (Canada)}\\
    \hline
\multicolumn{5}{l}{\textbf{Offre de stationnement dédiée au vélo à proximité directe des stations}}\\
    \small{\textcite{rietveld_accessibility_2000}}\index{Rietveld, Piet|pagebf} & \small{Augmentation de l'offre} & \small{Vélo} & \small{Train} & \small{Pays-Bas}\\
    \small{\textcite{geurs_multi-modal_2016}}\index{Geurs, Karst T.|pagebf} & \small{Augmentation de l'offre} & \small{Vélo} & \small{Train} & \small{Randstad South (Pays-Bas)}\\
    \small{\textcite{bechstein_cycling_2010}}\index{Bechstein, Eva|pagebf} & \small{Augmentation de l'offre} & \small{Vélo} & \small{Train} & \small{Mamelodi, Nellmapius (Afrique du Sud)}\\
    \small{\textcite{lee_strategies_2010, lee_bicycle-based_2016}} & \small{Augmentation de l'offre} & \small{Vélo} & \small{Métro} & \small{Séoul, Daejeon (Corée du Sud)}\\
    \small{\textcite{ton_understanding_2020}}\index{Ton, Danique|pagebf} & \small{Augmentation de l'offre} & \small{Vélo} & \small{Tramway} & \small{La Haye (Pays-Bas)}\\
    \small{\textcite{molin_bicycle_2015}}\index{Molin, Eric|pagebf} & \small{Saturation dans les gares urbaines} & \small{Vélo} & \small{Train} & \small{Delft (Pays-Bas)}\\
    \small{\textcite{la_paix_puello_integration_2016}}\index{La Paix Puello, Lissy|pagebf} & \small{Offre insuffisante dans les gares urbaines} & \small{Vélo} & \small{Train} & \small{La Haye, Rotterdam (Pays-Bas)}\\
    \small{\textcite{weliwitiya_bicycle_2019}}\index{Weliwitiya, Hesara|pagebf} & \small{Association positive entre l'offre et la fréquentation à vélo} & \small{Vélo} & \small{Train} & \small{Melbourne (Australie)}\\
    \small{\textcite{halldorsdottir_home-end_2017}}\index{Halldórsdóttir, Katrín|pagebf} & \small{Association positive entre l'offre et la fréquentation à vélo} & \small{Vélo} & \small{Train} & \small{Copenhague (Danemark)}\\
    \small{\textcite{arbis_analysis_2016}}\index{Arbis, David|pagebf} & \small{Association négative entre la distance de stationnement et l'occupation} & \small{Vélo} & \small{Train} & \small{Nouvelle-Galles du Sud (Australie)}\\
    \small{\textcite{jonkeren_bicycle_2021}}\index{Jonkeren, Olaf|pagebf} & \small{Solution du VLS et du vélo pliant pour contrer l'offre insuffisante} & \small{Vélo} & \small{Train} & \small{Pays-Bas}\\
        \hline
\multicolumn{5}{l}{\textbf{Stationnement vélo sécurisé autour des stations}}\\
    \small{\textcite{la_paix_puello_role_2021}}\index{La Paix Puello, Lissy|pagebf} & \small{Facteur significatif lié à la sûreté des parkings à vélo} & \small{Vélo} & \small{Train} & \small{La Haye, Rotterdam (Pays-Bas)}\\
    \small{\textcite{la_paix_puello_modelling_2015}}\index{La Paix Puello, Lissy|pagebf} & \small{Facteur significatif lié à la sûreté des parkings à vélo} & \small{Vélo} & \small{Train} & \small{Randstad South (Pays-Bas)}\\
    \small{\textcite{flamm_public_2014}}\index{Flamm, Bradley J.|pagebf} & \small{Facteur significatif lié à la sûreté des parkings à vélo} & \small{Vélo} & \small{Train} & \small{Philadelphie, San Francisco (États-Unis)}\\
    \small{\textcite{yang_metro_2015}}\index{Yang, Min|pagebf} & \small{Facteur significatif lié à la sûreté des parkings à vélo} & \small{Vélo} & \small{Métro} & \small{Nanjing (Chine)}\\
    \small{\textcite{quarshie_integrating_2007}}\index{Quarshie, Magnus|pagebf} & \small{Facteur significatif lié à la sûreté des parkings à vélo} & \small{Vélo} & \small{Bus} & \small{Accra (Ghana)}\\
    \small{\textcite{taylor_analysis_1996}}\index{Taylor, Dean|pagebf} & \small{Facteur significatif lié à la sûreté des parkings à vélo} & \small{Vélo} & \small{Bus} & \small{Texas (États-Unis)}\\
    \small{\textcite{singleton_exploring_2014}}\index{Singleton, Patrick A.|pagebf} & \small{Facteur significatif lié à la sûreté des parkings à vélo} & \small{Vélo} & \small{Métro} & \small{Portland (États-Unis)}\\
    \small{\textcite{krizek_assessing_2011}}\index{Krizek, Kevin|pagebf} & \small{Facteur significatif lié à la sûreté des parkings à vélo} & \small{Vélo} & \small{Métro} & \small{Chicago, Portland (États-Unis)}\\
    \small{\textcite{giansoldati_train-feeder_2021}}\index{Giansoldati, Marco|pagebf} & \small{Facteur significatif lié à la sûreté des parkings à vélo} & \small{Vélo} & \small{Train} & \small{Frioul-Vénétie julienne, Émilie-Romagne, Piémont, Toscane, Lombardie, Latium, Campanie (Italie)}\\
    \small{\textcite{pan_intermodal_2010}}\index{Pan, Haixiao|pagebf} & \small{Impact négatif en raison des craintes liées au vol des vélos} & \small{Vélo, VAE} & \small{Métro} & \small{Shanghai (Chine)}\\
        \hline
        \caption*{Corpus scientifique se rapportant au stationnement de la \gls{mobilité individuelle légère} autour des stations de transport en commun, dans le cadre de la \acrshort{RSL}}
        \label{Corpus scientifique se rapportant au stationnement de la mobilité individuelle légère autour des stations de transport en commun, dans le cadre de la RSL}
        \begin{flushright}
        \scriptsize
    Auteur~: \textcopyright~Moinse 2023
        \end{flushright}
        \end{longtable}

    % Annexe D.4
    \newpage
\subsection{Corpus de la \acrshort{RSL} sur les résultats en lien avec le design des services de vélo partagé}
    \label{donnees-ouvertes:rsl_resultats_services_velo}

    % Référence
Le présent tableau synthétise les résultats issus de la revue de littérature portant plus particulièrement sur la \hyperref[Traitement des espaces publics]{sous-partie consacrée au traitement des espaces publics} (page \pageref{Traitement des espaces publics}).\par

    % Tableau résultats RSL (vélopartage)
        \begin{longtable}{p{3cm}p{4cm}p{1.5cm}p{1.8cm}p{2.3cm}}
        \hline
        \textcolor{blue}{\textbf{Références}} & \textcolor{blue}{\textbf{Design (vélopartage)}} & \textcolor{blue}{\textbf{MIL}} & \textcolor{blue}{\textbf{TC}} & \textcolor{blue}{\textbf{Contexte}}
        \hline
        \endhead
    \small{\textcite{van_goeverden_potential_2018}}\index{van Goeverden, Kees|pagebf} & \small{Soulagement de la demande de stationnement de vélo} & \small{Vélopartage entre pairs} & \small{Train} & \small{Pays-Bas}\\
    \small{\textcite{nam_designing_2018}}\index{Nam, Daisik|pagebf}\index{Nam, Daisik|pagebf} & \small{Développement de plateformes multimodales intégrées} & \small{VLS} & \small{Métro} & \small{Los Angeles (États-Unis)}\\
    \small{\textcite{yang_bike-and-ride_2014}}\index{Yang, Liu|pagebf}\index{Yang, Liu|pagebf} & \small{Minimiser la distance entre la station de VLS et la gare} & \small{Vélo, VLS} & \small{Train, Métro, Tramway, Bus} & \small{Xi'an (Chine)}\\
    \small{\textcite{tomita_demand_2017}}\index{Tomita, Yasuo|pagebf}\index{Tomita, Yasuo|pagebf} & \small{Minimiser la distance entre la station de VLS et la gare} & \small{VLS} & \small{Train} & \small{Osaka (Japon)}\\
    \small{\textcite{wu_identification_2023}}\index{Wu, Hao|pagebf}\index{Wu, Hao|pagebf} & \small{Distance acceptable de 50 mètres entre la station de VLS et l'arrêt de Métro} & \small{VFF} & \small{Métro} & \small{Shenzhen (Chine)}\\
    \small{\textcite{cheng_exploring_2022}}\index{Cheng, Long|pagebf} & \small{Minimiser la distance entre l'emplacement des VFF et l'arrêt de Métro} & \small{VFF} & \small{Métro} & \small{Nanjing (Chine)}\\
    \small{\textcite{cho_estimation_2022}}\index{Cho, Shin-Hyung|pagebf} & \small{Association positive entre la présence d'une station de VLS et son intégration avec le Métro} & \small{VLS} & \small{Métro} & \small{Séoul (Corée du Sud)}\\
    \small{\textcite{ashraf_impacts_2021}}\index{Ashraf, Md Tanvir|pagebf} & \small{Association positive entre la présence d'une station de VLS et son intégration avec le Métro} & \small{VLS} & \small{Métro} & \small{New York City (États-Unis)}\\
    \small{\textcite{li_investigating_2022}}\index{Li, Xiaofeng|pagebf} & \small{Association positive entre la présence d'une station de VLS et son intégration avec le Métro} & \small{VLS} & \small{Tramway, Bus} & \small{Tucson (États-Unis)}\\
    \small{\textcite{ji_exploring_2018}}\index{Ji, Yanjie|pagebf} & \small{Association positive entre la présence d'une station de VLS et son intégration avec le Métro en périphérie urbaine} & \small{VLS} & \small{Métro} & \small{Nanjing (Chine)}\\
    \small{\textcite{liu_use_2020}}\index{Liu, Yang|pagebf} & \small{Association positive entre la présence de VFF et leur intégration avec le Métro} & \small{VFF} & \small{Métro} & \small{Nanjing (Chine)}\\
    \small{\textcite{chu_last_2021}}\index{Chu, Junhong|pagebf} & \small{Risques d'encombrement des rues aux alentours des stations} & \small{VFF} & \small{Métro} & \small{Beijing, Chengdu, Chongqing, Dalian, Hangzhou, Nanjing, Shanghai, Shenzhen, Tianjin, Wuhan (Chine)}\\
        \hline
        \caption*{Corpus scientifique se rapportant au vélo et à la \gls{micro-mobilité} partagés autour des stations de transport en commun, dans le cadre de la \acrshort{RSL}}
        \label{Corpus scientifique se rapportant au vélo et à la micro-mobilité partagés autour des stations de transport en commun, dans le cadre de la RSL}
        \begin{flushright}
        \scriptsize
    Auteur~: \textcopyright~Moinse 2023
        \end{flushright}
        \end{longtable}

    % Annexe D.5
    \newpage
\subsection{Corpus de la \acrshort{RSL} sur les résultats en lien avec le design des itinéraires cyclables}
    \label{donnees-ouvertes:rsl_resultats_design_itineraires_velo}

    % Référence
Le présent tableau synthétise les résultats issus de la revue de littérature portant plus particulièrement sur la \hyperref[Traitement des espaces publics]{sous-partie consacrée au traitement des espaces publics} (page \pageref{Traitement des espaces publics}).\par

    % Tableau résultats RSL (pistes cyclables)
        \begin{longtable}{p{3cm}p{4cm}p{1.5cm}p{1.8cm}p{2.3cm}}
        \hline
        \textcolor{blue}{\textbf{Références}} & \textcolor{blue}{\textbf{Design (réseau cyclable)}} & \textcolor{blue}{\textbf{MIL}} & \textcolor{blue}{\textbf{TC}} & \textcolor{blue}{\textbf{Contexte}}
        \hline
        \endhead
\multicolumn{5}{l}{\textbf{Réseau cyclable peu développé}}\\
    \small{\textcite{zhang_make_2023}}\index{Zhang, Mengyuan|pagebf} & \small{Insuffisante d'itinéraires cyclables} & \small{Vélo} & \small{Train} & \small{Sydney (Australie)}\\
    \small{\textcite{risimati_spatial_2021}}\index{Risimati, Brightnes|pagebf} & \small{Insuffisance d'itinéraires cyclables} & \small{Vélo} & \small{Train, Métro, Tramway, Bus} & \small{Johannesburg (Afrique du Sud)}\\
    \small{\textcite{bechstein_cycling_2010}}\index{Bechstein, Eva|pagebf} & \small{Insuffisance d'itinéraires cyclables} & \small{Vélo} & \small{Train} & \small{Mamelodi, Nellmapius (Afrique du Sud)}\\
    \small{\textcite{lee_strategies_2010}}\index{Lee, Jaeyeong|pagebf} & \small{Insuffisance d'itinéraires cyclables} & \small{Vélo} & \small{Métro} & \small{Séoul, Daejeon (Afrique du Sud)}\\
    \small{\textcite{zuo_first-and-last_2020}}\index{Zuo, Ting|pagebf} & \small{La moitié du réseau viaire peu accessible aux cyclistes} & \small{Vélo} & \small{Bus} & \small{Hamilton (États-Unis)}\\
    \small{\textcite{zuo_promote_2020}}\index{Zuo, Ting|pagebf} & \small{Communautés défavorisées exclues des rues praticables à vélo} & \small{Vélo} & \small{Bus} & \small{Hamilton (États-Unis)}\\
    \small{\textcite{meng_influence_2016}}\index{Meng, Meng|pagebf} & \small{Réseau cyclable peu développé entre les stations de Métro et les zones résidentielles} & \small{Vélo} & \small{Métro} & \small{Singapour}\\
    \hline
\multicolumn{5}{l}{\textbf{Sentiment d'insécurité}}\\
    \small{\textcite{yang_empirical_2016}}\index{Yang, Min|pagebf} & \small{Sentiment de sécurité en circulant comme préoccupation majeure} & \small{VLS} & \small{Métro} & \small{Nanjing (Chine)}\\
    \small{\textcite{quarshie_integrating_2007}}\index{Quarshie, Magnus|pagebf} & \small{Craintes liées à l'insécurité routière et sociale} & \small{Vélo} & \small{Bus} & \small{Accra (Ghana)}\\
    \small{\textcite{mcqueen_assessing_2022}}\index{McQueen, Michael|pagebf} & \small{Populations noires préoccupées par rapport à la sécurité routière} & \small{TEFF} & \small{Tramway} & \small{Portland (États-Unis)}\\
    \small{\textcite{bopp_examining_2015}}\index{Bopp, Melissa|pagebf} & \small{Les non-usager·ère·s plus sensibles à l'absence d'itinéraires cyclables} & \small{Vélo} & \small{Train, Métro, Tramway} & \small{Delaware, New Jersey, Maryland, Virginie-Occidentale, Pennsylvanie, Ohio (États-Unis)}\\
    \small{\textcite{fan_how_2019}}\index{Fan, Aihua|pagebf} & \small{45,5\% des cyclistes désirent une amélioration du réseau cyclable} & \small{VFF} & \small{Train, Métro, Tramway, Bus} & \small{Beijing (Chine)}\\
    \small{\textcite{stransky_quartiers_2017}}\index{Stransky, Václav|pagebf} & \small{Impacts du réseau cyclable sur l'appréciation de la sécurité routière} & \small{Vélo} & \small{Train} & \small{Île-de-France (France)}\\
    \hline
\multicolumn{5}{l}{\textbf{Association positive avec la présence d'un réseau cyclable}}\\
    \small{\textcite{midenet_modal_2018}}\index{Midenet, Sophie|pagebf} & \small{Association positive} & \small{Vélo} & \small{Train} & \small{Amboise (France)}\\
    \small{\textcite{cervero_bike-and-ride_2013}}\index{Cervero, Robert|pagebf} & \small{Association positive} & \small{Vélo} & \small{Métro} & \small{San Francisco (États-Unis)}\\
    \small{\textcite{nielsen_bikeability_2018}}\index{Nielsen, Thomas Alexander Sick|pagebf} & \small{Association positive} & \small{Vélo} & \small{Train, Bus} & \small{Danemark}\\
    \small{\textcite{mohanty_effect_2017}}\index{Mohanty, Sudatta|pagebf} & \small{Association positive} & \small{Vélo} & \small{Train, Bus} & \small{New Delhi (Inde)}\\
    \small{\textcite{fillone_i_2018}}\index{Fillone, Alexis|pagebf} & \small{Association positive} & \small{Vélo} & \small{Tramway, Bus} & \small{Manille (Philippines)}\\
    \small{\textcite{taylor_analysis_1996}}\index{Taylor, Dean|pagebf} & \small{Association positive} & \small{Vélo} & \small{Bus} & \small{Texas (États-Unis)}\\
    \small{\textcite{zuo_determining_2018}}\index{Zuo, Ting|pagebf} & \small{Association positive} & \small{Vélo} & \small{Bus} & \small{Cincinnati (États-Unis)}\\
    \small{\textcite{zuo_bikeway_2019}}\index{Zuo, Ting|pagebf} & \small{Association positive, principalement avec les pistes cyclables protégées, moins stressantes} & \small{Vélo} & \small{Bus} & \small{Cincinnati (États-Unis)}\\
    \small{\textcite{zuo_incorporating_2021}}\index{Zuo, Ting|pagebf} & \small{Association positive avec les pistes cyclables protégées} & \small{Vélo} & \small{Bus} & \small{Hamilton (États-Unis)}\\
    \small{\textcite{halldorsdottir_home-end_2017}}\index{Halldórsdóttir, Katrín|pagebf} & \small{Association positive en heures de pointe le soir} & \small{Vélo} & \small{Train} & \small{Copenhague (Danemark)}\\
    \small{\textcite{krizek_detailed_2007}}\index{Krizek, Kevin J.|pagebf} & \small{Disposition à parcourir plus de 2,5 kilomètres pour atteindre une piste cyclable} & \small{Vélo} & \small{Tramway} & \small{Minneapolis (États-Unis)}\\
    \small{\textcite{kostrzewska_towards_2017}} & \small{Association positive avec les voies partagées} & \small{Trottinette mécanique} & \small{Métro, Tramway, Bus} & \small{Berlin (Allemagne), Szczecin (Pologne)}\\
    \small{\textcite{ashraf_impacts_2021}}\index{Ashraf, Md Tanvir|pagebf} & \small{Association positive} & \small{VLS} & \small{Métro} & \small{New York City (États-Unis)}\\
    \small{\textcite{cho_estimation_2022}}\index{Cho, Shin-Hyung|pagebf} & \small{Association positive} & \small{VLS} & \small{Métro} & \small{Séoul (Corée du Sud)}\\
    \small{\textcite{kim_analysis_2021}}\index{Kim, Minjun|pagebf} & \small{Association positive avec les pistes cyclables protégées et l'absence d'obstacles} & \small{VLS} & \small{Métro, Bus} & \small{Séoul (Corée du Sud)}\\
    \small{\textcite{ji_exploring_2018}}\index{Ji, Yanjie|pagebf} & \small{Association positive en périphérie urbaine} & \small{VLS} & \small{Métro} & \small{Nanjing (Chine)}\\
    \small{\textcite{liu_use_2020}}\index{Liu, Yang|pagebf} & \small{Association positive} & \small{VFF} & \small{Métro} & \small{Nanjing (Chine)}\\
    \small{\textcite{lin_analysis_2019}}\index{Lin, Diao|pagebf} & \small{Association positive} & \small{VFF} & \small{Métro} & \small{Shanghai (Chine)}\\
    \small{\textcite{wu_measuring_2019}}\index{Wu, Xueying|pagebf} & \small{Association positive} & \small{VFF} & \small{Métro} & \small{Shenzhen (Chine)}\\
    \small{\textcite{liu_mode_2022}}\index{Liu, Lumei|pagebf} & \small{Association positive avec les pistes cyclables protégées} & \small{VFF} & \small{Métro, Bus} & \small{Beijing (Chine)}\\
    \small{\textcite{guo_built_2020, guo_role_2021, guo_dockless_2021}} & \small{Association positive en heures de pointe le soir} & \small{VFF} & \small{Métro} & \small{Shenzhen (Chine)}\\
    \small{\textcite{zhou_spatially_2023}}\index{Zhou, X.|pagebf} & \small{Association positive en semaine} & \small{VFF} & \small{Métro, Bus} & \small{Beijing (Chine)}\\
    \hline
\multicolumn{5}{l}{\textbf{Absence de liens avec la présence d'aménagements cyclables}}\\
    \small{\textcite{weliwitiya_bicycle_2019}}\index{Weliwitiya, Hesara|pagebf} & \small{Facteur non significatif} & \small{Vélo} & \small{Train} & \small{Melbourne (Australie)}\\
    \small{\textcite{balya_integration_2016}}\index{Balya, Manjurali|pagebf} & \small{Facteur non significatif} & \small{Vélo} & \small{Bus} & \small{Ahmedabad (Inde)}\\
    \small{\textcite{zhao_bicycle-metro_2017}}\index{Zhao, Pengjun|pagebf} & \small{Longueur des pistes cyclables non significative} & \small{Vélo, VLS} & \small{Métro} & \small{Beijing (Chine)}\\
        \hline
        \caption*{Corpus scientifique se rapportant à l'aménagement d'itinéraires cyclables autour des stations de transport en commun, dans le cadre de la \acrshort{RSL}}
        \label{Corpus scientifique se rapportant à l'aménagement d'itinéraires cyclables autour des stations de transport en commun, dans le cadre de la RSL}
        \begin{flushright}
        \scriptsize
    Auteur~: \textcopyright~Moinse 2023
        \end{flushright}
        \end{longtable}

    % Annexe D.6
    \newpage
\subsection{Corpus de la \acrshort{RSL} sur les résultats en lien avec le design du réseau viaire}
    \label{donnees-ouvertes:rsl_resultats_design_reseau_viaire}
    
    % Référence
Le présent tableau synthétise les résultats issus de la revue de littérature portant plus particulièrement sur la \hyperref[Traitement des espaces publics]{sous-partie consacrée au traitement des espaces publics} (page \pageref{Traitement des espaces publics}).\par

    
    % Tableau résultats RSL (design réseau viaire)
        \begin{longtable}{p{3cm}p{4cm}p{1.5cm}p{1.8cm}p{2.3cm}}
        \hline
        \textcolor{blue}{\textbf{Références}} & \textcolor{blue}{\textbf{Design (réseau cyclable)}} & \textcolor{blue}{\textbf{MIL}} & \textcolor{blue}{\textbf{TC}} & \textcolor{blue}{\textbf{Contexte}}
        \hline
        \endhead
\multicolumn{5}{l}{\textbf{Effets positifs de la connectivité}}\\
    \small{\textcite{giansoldati_train-feeder_2021}}\index{Giansoldati, Marco|pagebf} & \small{Association positive} & \small{Vélo} & \small{Train} & \small{Frioul-Vénétie julienne, Émilie-Romagne, Piémont, Toscane, Lombardie, Latium, Campanie (Italie)}\\
    \small{\textcite{geurs_multi-modal_2016}}\index{Geurs, Karst T.|pagebf} & \small{Association positive} & \small{Vélo} & \small{Train} & \small{Randstad South (Pays-Bas)}\\
    \small{\textcite{la_paix_puello_modelling_2015}}\index{La Paix Puello, Lissy|pagebf} & \small{Association positive avec la connectivité perçue} & \small{Vélo} & \small{Train} & \small{Randstad South (Pays-Bas)}\\
    \small{\textcite{hochmair_assessment_2015}}\index{Hochmair, Hartwig H.|pagebf} & \small{Association positive} & \small{Vélo} & \small{Train, Métro, Tramway, Bus} & \small{Los Angeles, Atlanta, Minneapolis, Saint-Paul (États-Unis)}\\
    \small{\textcite{wang_interchange_2016}}\index{Wang, Zi-jia|pagebf} & \small{Association positive} & \small{Vélo} & \small{Métro} & \small{Beijing (Chine)}\\
    \small{\textcite{wang_relationship_2020}}\index{Wang, Ruoyu|pagebf} & \small{Association positive avec la connectivité perçue} & \small{VFF} & \small{Métro} & \small{Shenzhen (Chine)}\\
    \small{\textcite{cheng_exploring_2022}}\index{Cheng, Long|pagebf} & \small{Association positive avec la connectivité perçue} & \small{VFF} & \small{Métro} & \small{Nanjing (Chine)}\\
    \small{\textcite{lin_analysis_2019}}\index{Lin, Diao|pagebf} & \small{Association positive avec la connectivité perçue} & \small{VFF} & \small{Métro} & \small{Shanghai (Chine)}\\
        \hline
\multicolumn{5}{l}{\textbf{Effets négatifs de la connectivité}}\\
    \small{\textcite{weliwitiya_bicycle_2019}}\index{Weliwitiya, Hesara|pagebf} & \small{Facteur non significatif} & \small{Vélo} & \small{Train} & \small{Melbourne (Australie)}\\
    \small{\textcite{ma_bicycle_2015}}\index{Ma, Ting|pagebf} & \small{Facteur non significatif} & \small{VLS} & \small{Train} & \small{Washington D.C. (États-Unis)}\\
    \small{\textcite{la_paix_puello_role_2021}}\index{La Paix Puello, Lissy|pagebf} & \small{Association négative en raison de la présence de feux de circulation} & \small{Vélo} & \small{Train} & \small{La Haye, Rotterdam (Pays-Bas)}\\
    \small{\textcite{cheng_comparison_2023}}\index{Cheng, Long|pagebf} & \small{Association négative} & \small{VLS, VFF} & \small{Métro} & \small{Nanjing (Chine)}\\
    \small{\textcite{ni_exploring_2020}}\index{Ni, Ying|pagebf} & \small{Association négative} & \small{VFF} & \small{Métro} & \small{Beijing (Chine)}\\
    \small{\textcite{guo_built_2020}}\index{Guo, Yuanyuan|pagebf} & \small{Association négative} & \small{VFF} & \small{Métro} & \small{Shenzhen (Chine)}\\
        \hline
\multicolumn{5}{l}{\textbf{Effets de la densité du réseau viaire}}\\
    \small{\textcite{cervero_influences_2009}}\index{Cervero, Robert|pagebf} & \small{Densité des routes supérieure à 0,20 km/km² double l'\gls{usage} pour motif utilitaire} & \small{Vélo} & \small{Bus} & \small{Bogotá (Colombie)}\\
    \small{\textcite{wu_measuring_2019}}\index{Wu, Xueying|pagebf} & \small{Association positive} & \small{VFF} & \small{Métro} & \small{Shenzhen (Chine)}\\
    \small{\textcite{hu_examining_2022}}\index{Hu, Songhua|pagebf} & \small{Association positive avec la densité de voies principales et secondaires} & \small{VFF} & \small{Métro} & \small{Shanghai (Chine)}\\
    \small{\textcite{chen_what_2022}}\index{Chen, Wendong|pagebf} & \small{Association positive avec la densité et la longueur des routes} & \small{VLS, VFF} & \small{Métro} & \small{Nanjing (Chine)}\\
    \small{\textcite{cheng_comparison_2023}}\index{Cheng, Long|pagebf} & \small{Association positive jusqu'à un certain seuil de densité} & \small{VLS, VFF} & \small{Métro} & \small{Nanjing (Chine)}\\
    \small{\textcite{chan_factors_2020}}\index{Chan, Kevin|pagebf} & \small{Association négative} & \small{Vélo} & \small{Train} & \small{Toronto, Hamilton (Canada)}\\
        \hline
\multicolumn{5}{l}{\textbf{Volume et vitesse du trafic automobile}}\\
    \small{\textcite{weliwitiya_bicycle_2019}}\index{Weliwitiya, Hesara|pagebf} & \small{Importance des rues à faible vitesse motorisée} & \small{Vélo} & \small{Train} & \small{Melbourne (Australie)}\\
    \small{\textcite{park_finding_2014}}\index{Park, Sungjin|pagebf} & \small{Importance des rues à faible volume motorisé} & \small{Vélo} & \small{Train} & \small{Mountain View (États-Unis)}\\
    \small{\textcite{tzouras_describing_2023}}\index{Tzouras, Panagiotis|pagebf} & \small{Importance des rues à faible volume motorisé} & \small{TEP} & \small{Métro, Tramway} & \small{Athènes (Grèce)}\\
        \hline
\multicolumn{5}{l}{\textbf{Formes du réseau routier}}\\
    \small{\textcite{weliwitiya_bicycle_2019}}\index{Weliwitiya, Hesara|pagebf} & \small{Effets négatifs d'une pente de 2°} & \small{Vélo} & \small{Train} & \small{Melbourne (Australie)}\\
    \small{\textcite{bocker_bike_2020}}\index{Böcker, Lars|pagebf} & \small{Effets négatifs des pentes et importance des itinéraires directs} & \small{VLS} & \small{Métro} & \small{Oslo (Norvège)}\\
    \small{\textcite{zhao_public_2022}}\index{Zhao, Pengjun|pagebf} & \small{Importance des itinéraires directs} & \small{VLS} & \small{Métro} & \small{Beijing (Chine)}\\
    \small{\textcite{papon_rapport_2015}}\index{Papon, Francis|pagebf} & \small{Effets négatifs des coupures urbaines} & \small{Vélo} & \small{Train} & \small{Amboise (France)}\\
        \hline
\multicolumn{5}{l}{\textbf{Qualité de l'aménagement des espaces publics}}\\
    \small{\textcite{guo_dockless_2021}}\index{Guo, Yuanyuan|pagebf} & \small{Importance des parcs et places publiques} & \small{VFF} & \small{Métro} & \small{Shenzhen (Chine)}\\
    \small{\textcite{wang_relationship_2020}}\index{Wang, Ruoyu|pagebf} & \small{Importance de l'exposition visuelle à la végétation} & \small{VFF} & \small{Métro} & \small{Shenzhen (Chine)}\\
    \small{\textcite{la_paix_puello_modelling_2015}}\index{La Paix Puello, Lissy|pagebf} & \small{Importance de l'éclairage autour des gares} & \small{Vélo} & \small{Train} & \small{Randstad South (Pays-Bas)}\\ 
        \hline
        \caption*{Corpus scientifique se rapportant aux formes du réseau viaire autour des stations de transport en commun, dans le cadre de la \acrshort{RSL}}
        \label{Corpus scientifique se rapportant aux formes du réseau viaire autour des stations de transport en commun, dans le cadre de la RSL}
        \begin{flushright}
        \scriptsize
    Auteur~: \textcopyright~Moinse 2023
        \end{flushright}
        \end{longtable}

    % Annexe D.7
    \newpage
\subsection{Corpus de la \acrshort{RSL} sur les résultats en lien avec l'\gls{accessibilité} vers les destinations}
    \label{donnees-ouvertes:rsl_resultats_accessibilite_destinations}

    % Référence
Le présent tableau synthétise les résultats issus de la revue de littérature portant plus particulièrement sur la \hyperref[Accessibilité intermodale]{sous-partie consacrée à l'accessibilité intermodale} (page \pageref{Accessibilité intermodale}).\par

    % Tableau résultats RSL (accessibilité vers les destinations)
        \begin{longtable}{p{3cm}p{4cm}p{1.5cm}p{1.8cm}p{2.3cm}}
        \hline
        \textcolor{blue}{\textbf{Références}} & \textcolor{blue}{\textbf{Accessibilité (couverture)}} & \textcolor{blue}{\textbf{MIL}} & \textcolor{blue}{\textbf{TC}} & \textcolor{blue}{\textbf{Contexte}}
        \hline
        \endhead
    \small{\textcite{krygsman_multimodal_2004}}\index{Krygsman, Stephan|pagebf} & \small{Amélioration de l'interconnectivité des transports en commun} & \small{Vélo} & \small{Train, Métro, Tramway, Bus} & \small{Utrecht (Pays-Bas)}\\
    \small{\textcite{cheng_comparison_2023}}\index{Cheng, Long|pagebf} & \small{Gains d'accessibilité intermodale, notamment au-delà de 20 km du \acrshort{CBD}} & \small{VLS, VFF} & \small{Métro} & \small{Nanjing (Chine)}\\
    \small{\textcite{nielsen_bikeability_2018}}\index{Nielsen, Thomas Alexander Sick|pagebf} & \small{Gains d'accessibilité intermodale en fonction de la position régionale des zones urbaines} & \small{Vélo} & \small{Train, Bus} & \small{Danemark}\\
    \small{\textcite{djurhuus_building_2016}}\index{Djurhuus, Sune|pagebf} & \small{Gains d'accessibilité intermodale, aussi bien en milieu urbain, périurbain que rural} & \small{Vélo} & \small{Train, Métro, Tramway, Bus, Ferry} & \small{Copenhague (Danemark)}\\
    \small{\textcite{marques_potential_2017}}\index{Marques, R.|pagebf} & \small{Capture cinq fois plus de résident·e·s} & \small{Vélo} & \small{Train, Métro, Tramway} & \small{Séville (Espagne)}\\
    \small{\textcite{kager_characterisation_2016}}\index{Kager, Roland|pagebf} & \small{Capture trois à quatre fois plus de résident·e·s} & \small{Vélo} & \small{Train} & \small{Pays-Bas}\\
    \small{\textcite{lee_bicycle-based_2016}}\index{Lee, Jaeyeong|pagebf} & \small{Capture trois fois plus de résident·e·s} & \small{Vélo} & \small{Métro} & \small{Séoul (Corée du Sud)}\\
    \small{\textcite{geurs_multi-modal_2016}}\index{Geurs, Karst T.|pagebf} & \small{Amélioration de l'accessibilité aux emplois} & \small{Vélo} & \small{Train} & \small{Randstad South (Pays-Bas)}\\
    \small{\textcite{zuo_first-and-last_2020}}\index{Zuo, Ting|pagebf} & \small{Amélioration de l'accessibilité aux emplois de 44\%, notamment à destination des populations les plus défavorisées} & \small{Vélo} & \small{Bus} & \small{Hamilton (États-Unis)}\\
    \small{\textcite{yu_understanding_2021}}\index{Yu, Senbin|pagebf} & \small{Influence de la position des territoires dans les gains d'accessibilité} & \small{VFF} & \small{Métro} & \small{Beijing (Chine)}\\
        \hline
        \caption*{Corpus scientifique se rapportant à l'accessibilité des destinations autour des stations de transport en commun, dans le cadre de la \acrshort{RSL}}
        \label{Corpus scientifique se rapportant à l'accessibilité des destinations autour des stations de transport en commun, dans le cadre de la RSL}
        \begin{flushright}
        \scriptsize
    Auteur~: \textcopyright~Moinse 2023
        \end{flushright}
        \end{longtable}

    % Annexe D.8
    \newpage
\subsection{Corpus de la \acrshort{RSL} sur les résultats en lien avec l'accessibilité en milieu urbain}
    \label{donnees-ouvertes:rsl_resultats_accessibilite_urbain}

    % Référence
Le présent tableau synthétise les résultats issus de la revue de littérature portant plus particulièrement sur la \hyperref[Accessibilité intermodale]{sous-partie consacrée à l'accessibilité intermodale} (page \pageref{Accessibilité intermodale}).\par

    % Tableau résultats RSL (accessibilité dans les centres urbains)
        \begin{longtable}{p{3cm}p{4cm}p{1.5cm}p{1.8cm}p{2.3cm}}
        \hline
        \textcolor{blue}{\textbf{Références}} & \textcolor{blue}{\textbf{Accessibilité (urbain)}} & \textcolor{blue}{\textbf{MIL}} & \textcolor{blue}{\textbf{TC}} & \textcolor{blue}{\textbf{Contexte}}
        \hline
        \endhead
    \small{\textcite{zhu_improved_2021}}\index{Zhu, Zhenjun|pagebf} & \small{Association positive} & \small{Vélo} & \small{Métro} & \small{Xi'an (Chine)}\\
    \small{\textcite{flamm_determinants_2013}}\index{Flamm, Bradley J.|pagebf} & \small{Association positive} & \small{Vélo} & \small{Bus} & \small{Cleveland (États-Unis)}\\
    \small{\textcite{jappinen_modelling_2013}}\index{Jäppinen, Sakari|pagebf} & \small{Association positive} & \small{VLS} & \small{Train, Métro, Tramway, Bus, Ferry} & \small{Helsinki (Finlande)}\\
    \small{\textcite{kong_deciphering_2020}}\index{Kong, Hui|pagebf} & \small{Association positive} & \small{VLS} & \small{Train, Métro, Tramway, Bus} & \small{Boston, Chicago, Washington D.C., New York City (États-Unis)}\\
    \small{\textcite{chen_what_2022}}\index{Chen, Wendong|pagebf} & \small{Part modale plus faible en s'éloignant du centre-ville} & \small{VLS, VFF} & \small{Métro} & \small{Nanjing (Chine)}\\
    \small{\textcite{cheng_promoting_2022}} & \small{Association positive} & \small{VLS} & \small{Métro} & \small{Nanjing (Chine)}\\
    \small{\textcite{bi_analysis_2021}}\index{Bi, Hui|pagebf} & \small{Association positive dans les quartiers commerciaux et anciens} & \small{VLS} & \small{Métro} & \small{Chengdu (Chine)}\\
    \small{\textcite{bocker_bike_2020}}\index{Böcker, Lars|pagebf} & \small{Association positive avec les usagers masculins} & \small{VLS} & \small{Métro} & \small{Oslo (Norvège)}\\
    \small{\textcite{jin_competition_2019}}\index{Jin, Haitao|pagebf} & \small{Association positive} & \small{VFF} & \small{Métro} & \small{Beijing (Chine)}\\
    \small{\textcite{ni_exploring_2020}}\index{Ni, Ying|pagebf} & \small{Association positive} & \small{VFF} & \small{Métro} & \small{Beijing (Chine)}\\
    \small{\textcite{yu_understanding_2021}}\index{Yu, Senbin|pagebf} & \small{Association positive} & \small{VFF} & \small{Métro} & \small{Beijing (Chine)}\\
    \small{\textcite{li_factors_2020}}\index{Li, Xuefeng|pagebf} & \small{Association positive} & \small{VFF} & \small{Métro} & \small{Shenzhen (Chine)}\\
    \small{\textcite{yang_spatiotemporal_2019}}\index{Yang, Yuanxuan|pagebf} & \small{Association positive} & \small{VFF} & \small{Métro} & \small{Nanjing (Chine)}\\
    \small{\textcite{cheng_exploring_2022}}\index{Cheng, Long|pagebf} & \small{Association positive} & \small{VFF} & \small{Métro} & \small{Nanjing (Chine)}\\
    \small{\textcite{lin_analysis_2019}}\index{Lin, Diao|pagebf} & \small{Association positive} & \small{VFF} & \small{Métro} & \small{Shanghai (Chine)}\\
    \small{\textcite{hu_examining_2022}}\index{Hu, Songhua|pagebf} & \small{Association positive} & \small{VFF} & \small{Métro} & \small{Shanghai (Chine)}\\
    \small{\textcite{liu_measuring_2022}}\index{Liu, Lumei|pagebf} & \small{Association positive} & \small{TEFF} & \small{Bus} & \small{Columbus (États-Unis)}\\
    \small{\textcite{zuniga-garcia_evaluation_2022}}\index{Zuniga-Garcia, Natalia|pagebf} & \small{Association positive dans le centre-ville et dans les campus} & \small{TEFF} & \small{Bus} & \small{Austin (États-Unis)}\\
    \small{\textcite{fearnley_patterns_2020}}\index{Fearnley, Nils|pagebf} & \small{Association positive} & \small{TEFF} & \small{Train, Métro, Tramway, Bus} & \small{Oslo (Norvège)}\\
    \small{\textcite{van_goeverden_potential_2018}}\index{van Goeverden, Kees|pagebf} & \small{Association positive} & \small{Vélopartage entre pairs} & \small{Train} & \small{Pays-Bas}\\
    \small{\textcite{liu_use_2020}}\index{Liu, Yang|pagebf} & \small{Association positive en raison d'une densité élevée de stations, de VFF et d'activités autour} & \small{VFF} & \small{Métro} & \small{Nanjing (Chine)}\\
    \small{\textcite{wang_bicycle-transit_2013}}\index{Wang, Rui|pagebf} & \small{Association positive en raison d'une plus grande fréquentation du transport public} & \small{Vélo} & \small{Train, Bus} & \small{États-Unis}\\
        \hline
        \caption*{Corpus scientifique se rapportant à l'accessibilité dans les centres urbains autour des stations de transport en commun, dans le cadre de la \acrshort{RSL}}
        \label{Corpus scientifique se rapportant à l'accessibilité dans les centres urbains autour des stations de transport en commun, dans le cadre de la RSL}
        \begin{flushright}
        \scriptsize
    Auteur~: \textcopyright~Moinse 2023
        \end{flushright}
        \end{longtable}

    % Annexe D.9
    \newpage
\subsection{Corpus de la \acrshort{RSL} sur les résultats en lien avec l'accessibilité en milieu \gls{périurbain}}
    \label{donnees-ouvertes:rsl_resultats_accessibilite_periurbain}

    % Référence
Le présent tableau synthétise les résultats issus de la revue de littérature portant plus particulièrement sur la \hyperref[Accessibilité intermodale]{sous-partie consacrée à l'accessibilité intermodale} (page \pageref{Accessibilité intermodale}).\par

    % Tableau résultats RSL (accessibilité dans les territoires périurbains)
        \begin{longtable}{p{3cm}p{4cm}p{1.5cm}p{1.8cm}p{2.3cm}}
        \hline
        \textcolor{blue}{\textbf{Références}} & \textcolor{blue}{\textbf{Accessibilité (périurbain)}} & \textcolor{blue}{\textbf{MIL}} & \textcolor{blue}{\textbf{TC}} & \textcolor{blue}{\textbf{Contexte}}
        \hline
        \endhead
    \small{\textcite{martens_bicycle_2004}}\index{Martens, Karel|pagebf} & \small{Part modale plus élevée} & \small{Vélo} & \small{Train, Métro, Tramway, Bus} & \small{Pays-Bas, Allemagne, Grande-Bretagne}\\
    \small{\textcite{halldorsdottir_home-end_2017}}\index{Halldórsdóttir, Katrín|pagebf} & \small{Part modale plus élevée} & \small{Vélo} & \small{Train} & \small{Copenhague (Danemark)}\\
    \small{\textcite{stransky_periurbain_2019}}\index{Stransky, Václav|pagebf} & \small{Potentiel \acrshort{B-TOD} dans les territoires ruraux et périubains} & \small{Vélo} & \small{Train, Bus} & \small{Brie Boisée, Carnelle Pays de France, Haute Vallée de Chevreuse (France)}\\
    \small{\textcite{zuo_incorporating_2021}}\index{Zuo, Ting|pagebf} & \small{Réseau cyclable insuffisamment développé dans les territoires périurbains pour capter les gains d'accessibilité} & \small{Vélo} & \small{Bus} & \small{Hamilton (États-Unis)}\\
    \small{\textcite{zuo_determining_2018}}\index{Zuo, Ting|pagebf} & \small{Nécessité de développer le réseau cyclable dans les territoires périurbains en faveur des populations défavorisées} & \small{Vélo} & \small{Bus} & \small{Cincinnati (États-Unis)}\\
    \small{\textcite{bachand-marleau_much-anticipated_2011}}\index{Bachand-Marleau, Julie|pagebf} & \small{Potentiel \acrshort{B-TOD} dans les banlieues} & \small{VLS} & \small{Train, Métro, Tramway, Bus} & \small{Montréal (Canada)}\\
    \small{\textcite{ma_measuring_2018}}\index{Ma, Xinwei|pagebf} & \small{Part modale plus élevée, malgré une couverture plus faible} & \small{VLS} & \small{Métro} & \small{Nanjing (Chine)}\\
    \small{\textcite{ji_exploring_2018}}\index{Ji, Yanjie|pagebf} & \small{Nécessité d'améliorer la connectivité du réseau viaire pour stimuler le VLS en \gls{rabattement} dans les territoires périurbians} & \small{VLS} & \small{Métro} & \small{Nanjing (Chine)}\\
    \small{\textcite{ma_understanding_2018}}\index{Ma, Xinwei|pagebf} & \small{Part modale plus élevée, en raison d'un temps de marche et un temps d'attente plus long pour accéder au Métro} & \small{VLS} & \small{Métro} & \small{Nanjing (Chine)}\\
    \small{\textcite{yen_how_2023}}\index{Yen, Barbara T.H.|pagebf} & \small{Gains d'accessibilité pour les communes périurbaines proches dU \acrshort{CBD}} & \small{VLS} & \small{Métro} & \small{Taipei (Taïwan)}\\
    \small{\textcite{tarpin-pitre_typology_2020}}\index{Tarpin-Pitre, Léandre|pagebf} & \small{Complémentaire dans les territoires périurbains, compétitif dans les centres urbains} & \small{VLS} & \small{Métro} & \small{Montréal (Canada)}\\
    \small{\textcite{martin_evaluating_2014}}\index{Martin, Elliot W.|pagebf} & \small{Complémentaire dans les territoires périurbains, compétitif dans les centres urbains} & \small{VLS} & \small{Métro, Tramway} & \small{Washington D.C., Minneapolis (États-Unis)}\\
    \small{\textcite{ma_estimating_2019}}\index{Ma, Ting|pagebf} & \small{Complémentaire dans les territoires périurbains, compétitif dans les centres urbains} & \small{VLS} & \small{Métro} & \small{Washington D.C. (États-Unis)}\\
    \small{\textcite{fan_dockless_2020}}\index{Fan, Yichun|pagebf} & \small{Part modale plus élevée, en raison d'une densité de stations plus faible} & \small{VFF} & \small{Métro} & \small{Beijing (Chine)}\\
    \small{\textcite{guo_dockless_2021}}\index{Guo, Yuanyuan|pagebf} & \small{Part modale plus élevée, principalement en fin de journée} & \small{VFF} & \small{Métro} & \small{Shenzhen (Chine)}\\
    \small{\textcite{hu_examining_2022}}\index{Hu, Songhua|pagebf} & \small{Part modale et distances plus élevées} & \small{VFF} & \small{Métro} & \small{Shanghai (Chine)}\\
    \small{\textcite{guo_role_2021}}\index{Guo, Yuanyuan|pagebf} & \small{Part modale plus élevée, notamment dans les zones industrielles en périphérie urbaine} & \small{VFF} & \small{Métro} & \small{Shenzhen (Chine)}\\
    \small{\textcite{flamm_changes_2014}}\index{Flamm, Bradley J.|pagebf} & \small{Parts modales similaires en zone urbaine et périurbaine} & \small{Vélo} & \small{Train, Métro, Tramway, Bus} & \small{Nord-ouest de l'Ohio (États-Unis)}\\
    \small{\textcite{hamidi_inequalities_2019}}\index{Hamidi, Zahra|pagebf} & \small{Influence des divers territoires de l'agglomération dans l'accès cyclable aux gares} & \small{Vélo} & \small{Train, Bus} & \small{Malmö (Suède)}\\
        \hline
        \caption*{Corpus scientifique se rapportant à l'accessibilité dans les territoires périurbains autour des stations de transport en commun, dans le cadre de la \acrshort{RSL}}
        \label{Corpus scientifique se rapportant à l'accessibilité dans les territoires périurbains autour des stations de transport en commun, dans le cadre de la RSL}
        \begin{flushright}
        \scriptsize
    Auteur~: \textcopyright~Moinse 2023
        \end{flushright}
        \end{longtable}

    % Annexe D.10
    \newpage
\subsection{Corpus de la \acrshort{RSL} sur les résultats en lien avec l'accessibilité vers les types de stations}
    \label{donnees-ouvertes:rsl_resultats_accessibilite_types_stations}

    % Référence
Le présent tableau synthétise les résultats issus de la revue de littérature portant plus particulièrement sur la \hyperref[Accessibilité intermodale]{sous-partie consacrée à l'accessibilité intermodale} (page \pageref{Accessibilité intermodale}).\par

    % Tableau résultats RSL (accessibilité vers les types de stations)
        \begin{longtable}{p{3cm}p{4cm}p{1.5cm}p{1.8cm}p{2.3cm}}
        \hline
        \textcolor{blue}{\textbf{Références}} & \textcolor{blue}{\textbf{Accessibilité (nodalité)}} & \textcolor{blue}{\textbf{MIL}} & \textcolor{blue}{\textbf{TC}} & \textcolor{blue}{\textbf{Contexte}}
        \hline
        \endhead
    \small{\textcite{chen_study_2013}}\index{Chen, Wan|pagebf} & \small{Importance de la fonctionnalité des stations concernant le stationnement vélo} & \small{Vélo} & \small{Métro} & \small{Xi'an (Chine)}\\
    \small{\textcite{la_paix_puello_role_2021}}\index{La Paix Puello, Lissy|pagebf} & \small{Association positive avec les gares centrales} & \small{Vélo} & \small{Train} & \small{La Haye, Rotterdam (Pays-Bas)}\\
    \small{\textcite{weliwitiya_bicycle_2019}}\index{Weliwitiya, Hesara|pagebf} & \small{Influence positive de la fréquentation des gares} & \small{Vélo} & \small{Train} & \small{Melbourne (Australie)}\\
    \small{\textcite{guo_built_2020, guo_role_2021}} & \small{Influence positive de la fréquentation des stations} & \small{VFF} & \small{Métro} & \small{Shenzhen (Chine)}\\
    \small{\textcite{la_paix_puello_integration_2016}}\index{La Paix Puello, Lissy|pagebf} & \small{Association positive avec les gares périurbaines} & \small{Vélo} & \small{Train} & \small{La Haye, Rotterdam (Pays-Bas)}\\
    \small{\textcite{cheng_promoting_2022}} & \small{Association positive avec les stations de transfert} & \small{VLS} & \small{Métro} & \small{Nanjing (Chine)}\\
        \hline
        \caption*{Corpus scientifique se rapportant à l'accessibilité en fonction des types de stations autour des stations de transport en commun, dans le cadre de la \acrshort{RSL}}
        \label{Corpus scientifique se rapportant à l'accessibilité en fonction des types de stations autour des stations de transport en commun, dans le cadre de la RSL}
        \begin{flushright}
        \scriptsize
    Auteur~: \textcopyright~Moinse 2023
        \end{flushright}
        \end{longtable}

    % Annexe D.11
    \newpage
\subsection{Corpus de la \acrshort{RSL} sur les résultats en lien avec l'accessibilité locale}
    \label{donnees-ouvertes:rsl_resultats_accessibilite_locale}

    % Référence
Le présent tableau synthétise les résultats issus de la revue de littérature portant plus particulièrement sur la \hyperref[Accessibilité intermodale]{sous-partie consacrée à l'accessibilité intermodale} (page \pageref{Accessibilité intermodale}).\par

    % Tableau résultats RSL (accessibilité vers et depuis les noeuds de transport en commun)
        \begin{longtable}{p{3cm}p{4cm}p{1.5cm}p{1.8cm}p{2.3cm}}
        \hline
        \textcolor{blue}{\textbf{Références}} & \textcolor{blue}{\textbf{Accessibilité locale}} & \textcolor{blue}{\textbf{MIL}} & \textcolor{blue}{\textbf{TC}} & \textcolor{blue}{\textbf{Contexte}}
        \hline
        \endhead
    \small{\textcite{rabaud_quand_2022}}\index{Rabaud, Mathieu|pagebf} & \small{Potentiel intermodal important de la \gls{micro-mobilité}} & \small{VLS, TEP} & \small{Train, Métro, Tramway, Bus} & \small{France}\\
    \small{\textcite{rietveld_accessibility_2000}}\index{Rietveld, Piet|pagebf} & \small{Principalement un mode de \gls{rabattement}} & \small{Vélo} & \small{Train} & \small{Pays-Bas}\\
    \small{\textcite{givoni_access_2007}}\index{Givoni, Moshe|pagebf} & \small{Principalement un mode de \gls{rabattement}} & \small{Vélo} & \small{Train} & \small{Pays-Bas}\\
    \small{\textcite{romm_differences_2022}}\index{Romm, Daniel|pagebf} & \small{Principalement un mode de \gls{rabattement} dans les centres urbains, et de \gls{diffusion} en périphérie urbaine} & \small{VLS} & \small{Métro} & \small{Boston (États-Unis)}\\
    \small{\textcite{qiu_interplay_2021}}\index{Qiu, Waishan|pagebf} & \small{Principalement un mode de \gls{diffusion}} & \small{VFF} & \small{Bus} & \small{Ithaca (États-Unis)}\\
    \small{\textcite{moinse_intermodal_2022}}\index{Moinse, Dylan|pagebf} & \small{TEP et trottinette mécanique utilisée à la fois en \gls{rabattement} et en \gls{diffusion}, pour une distance totale de 41 kilomètres} & \small{TEP, trottinette mécanique} & \small{Train} & \small{Provence-Alpes-Côte d'Azur (France)}\\
    \small{\textcite{shelat_analysing_2018}}\index{Shelat, Sanmay|pagebf} & \small{Distance totale de 41 kilomètres} & \small{Vélo} & \small{Train} & \small{Pays-Bas}\\
    \small{\textcite{keijer_how_2000}}\index{Keijer, Majanka|pagebf} & \small{Distance totale de 53 kilomètres} & \small{Vélo} & \small{Train} & \small{Pays-Bas}\\
    \small{\textcite{lee_strategies_2010}}\index{Lee, Jaeyeong|pagebf} & \small{Distances totales de 12,5 kilomètres et de 29 minutes} & \small{Vélo} & \small{Métro} & \small{Séoul, Daejeon (Corée du Sud)}\\
    \small{\textcite{liu_use_2020}}\index{Liu, Yang|pagebf} & \small{Les \glspl{trajet} plus longs en métro augmentent la probabilité d'avoir recours au VFF} & \small{VFF} & \small{Métro} & \small{Nanjing (Chine)}\\
        \hline
        \caption*{Corpus scientifique se rapportant à l'accessibilité locale autour des stations de transport en commun, dans le cadre de la \acrshort{RSL}}
        \label{Corpus scientifique se rapportant à l'accessibilité locale autour des stations de transport en commun, dans le cadre de la RSL}
        \begin{flushright}
        \scriptsize
    Auteur~: \textcopyright~Moinse 2023
        \end{flushright}
        \end{longtable}

    % Annexe D.12
    \newpage
\subsection{Corpus de la \acrshort{RSL} sur les résultats en lien avec le rôle de la distance}
    \label{donnees-ouvertes:rsl_resultats_role_distance}
    
    % Référence
Le présent tableau synthétise les résultats issus de la revue de littérature portant plus particulièrement sur la \hyperref[Distances vers et depuis les nœuds de transport en commun]{sous-partie consacrée aux distances vers et depuis les nœuds de transport en commun} (page \pageref{Distances vers et depuis les nœuds de transport en commun}).\par

    % Tableau résultats RSL (rôle de la distance)
        \begin{longtable}{p{3cm}p{4cm}p{1.5cm}p{1.8cm}p{2.3cm}}
        \hline
        \textcolor{blue}{\textbf{Références}} & \textcolor{blue}{\textbf{Rôle de la distance}} & \textcolor{blue}{\textbf{MIL}} & \textcolor{blue}{\textbf{TC}} & \textcolor{blue}{\textbf{Contexte}}
        \hline
        \endhead
\multicolumn{5}{l}{\textbf{Effets négatifs de la distance}}\\
    \small{\textcite{krizek_assessing_2011}}\index{Krizek, Kevin|pagebf} & \small{Probabilité décroissante} & \small{Vélo} & \small{Tramway, Bus} & \small{Boulder, Denver, Chicago, Ithaca, New York City, Portland, Santa Clara (États-Unis)}\\
    \small{\textcite{taylor_analysis_1996}}\index{Taylor, Dean|pagebf} & \small{Probabilité décroissante} & \small{Vélo} & \small{Bus} & \small{Texas (États-Unis)}\\
    \small{\textcite{halldorsdottir_home-end_2017}}\index{Halldórsdóttir, Katrín|pagebf} & \small{Probabilité décroissante} & \small{Vélo} & \small{Train} & \small{Copenhague (Danemark)}\\
    \small{\textcite{park_finding_2014}}\index{Park, Sungjin|pagebf} & \small{Probabilité décroissante} & \small{Vélo} & \small{Train} & \small{Mountain View (États-Unis)}\\
    \small{\textcite{meng_influence_2016}}\index{Meng, Meng|pagebf} & \small{Probabilité décroissante} & \small{Vélo} & \small{Métro} & \small{Singapour}\\
    \small{\textcite{guo_built_2020, guo_role_2021, guo_dockless_2021}} & \small{Probabilité décroissante} & \small{VFF} & \small{Métro} & \small{Shenzhen (Chine)}\\
    \small{\textcite{zhou_spatially_2023}}\index{Zhou, X.|pagebf} & \small{Probabilité décroissante} & \small{VFF} & \small{Métro, Bus} & \small{Beijing (Chine)}\\
    \small{\textcite{guo_exploring_2023}}\index{Guo, Dongbo|pagebf} & \small{Probabilité décroissante} & \small{VFF} & \small{Métro, Bus} & \small{Beijing (Chine)}\\
    \small{\textcite{gan_associations_2021}}\index{Gan, Zuoxian|pagebf} & \small{Perception plus négative de l'environnement urbain} & \small{Vélo} & \small{Métro} & \small{Nanjing (Chine)}\\
    \small{\textcite{yang_metro_2015}}\index{Yang, Min|pagebf} & \small{Réduction de la satisfaction lorsque la distance de transfert est disproportionnée} & \small{Vélo, VAE, VLS} & \small{Métro} & \small{Nanjing (Chine)}\\
    \small{\textcite{chan_factors_2020}}\index{Chan, Kevin|pagebf} & \small{Une augmentation de dix minutes de trajet revient à réduire de 15\% la probabilité} & \small{Vélo} & \small{Train} & \small{Toronto, Hamilton (Canada)}\\
    \small{\textcite{van_mil_insights_2020}}\index{van Mil, Joeri F.P.|pagebf} & \small{Disposition à payer 0,11\€ pour réduire d'une minute la distance temps à vélo} & \small{Vélo} & \small{Train} & \small{Pays-Bas}\\
    \small{\textcite{van_goeverden_potential_2018}}\index{van Goeverden, Kees|pagebf} & \small{Impact négatif des temps de marge supplémentaires} & \small{Vélopartage entre pairs} & \small{Train} & \small{Pays-Bas}\\
    \hline
\multicolumn{5}{l}{\textbf{Effets positifs de la distance}}\\
    \small{\textcite{ji_public_2017}}\index{Ji, Yanjie|pagebf} & \small{Probabilité croissante} & \small{Vélo, VLS} & \small{Métro} & \small{Nanjing (Chine)}\\
    \small{\textcite{lin_built_2018}}\index{Lin, Jen-Jia|pagebf} & \small{Probabilité croissante} & \small{VLS} & \small{Métro} & \small{Beijing (Chine), Taipei (Taïwan), Tokyo (Japon)}\\
    \small{\textcite{zhao_public_2022}}\index{Zhao, Pengjun|pagebf} & \small{Probabilité croissante} & \small{VLS} & \small{Métro} & \small{Beijing (Chine)}\\
    \hline
\multicolumn{5}{l}{\textbf{Association nulle avec la distance}}\\
    \small{\textcite{ji_exploring_2018}}\index{Ji, Yanjie|pagebf} & \small{Risque de substitution modale} & \small{VLS} & \small{Métro} & \small{Nanjing (Chine)}\\
    \small{\textcite{cervero_influences_2009}}\index{Cervero, Robert|pagebf} & \small{Facteur non significatif} & \small{Vélo} & \small{Bus} & \small{Bogotá (Colombie)}\\
    \small{\textcite{heinen_multimodal_2014}}\index{Heinen, Eva|pagebf} & \small{Facteur non significatif} & \small{Vélo} & \small{Train, Métro, Tramway, Bus} & \small{Delft, Zwolle, Midden-Delfland, Pijnacker-Nootdorp (Pays-Bas)}\\
    \small{\textcite{martin_evaluating_2014}}\index{Martin, Elliot W.|pagebf} & \small{Facteur non significatif} & \small{VLS} & \small{Métro, Tramway} & \small{Washington D.C., Minneapolis (États-Unis)}\\
    \hline
\multicolumn{5}{l}{\textbf{Délimitations géographiques}}\\
    \small{\textcite{zhang_make_2023}}\index{Zhang, Mengyuan|pagebf} & \small{Rayon de pertinence de 2 kilomètres} & \small{Vélo} & \small{Train} & \small{Sydney (Australie)}\\
    \small{\textcite{papon_rapport_2015}}\index{Papon, Francis|pagebf} & \small{Rayon de pertinence de 3 kilomètres} & \small{Vélo} & \small{Train} & \small{Amboise (France)}\\
    \small{\textcite{marques_potential_2017}}\index{Marques, R.|pagebf} & \small{Rayon de pertinence de 3 kilomètres} & \small{Vélo} & \small{Train} & \small{Séville (Espagne)}\\
    \small{\textcite{garcia-bello_methodological_2019}}\index{Marques, R.|pagebf} & \small{Rayon de pertinence de 3 kilomètres} & \small{Vélo} & \small{Train} & \small{Andalousie (Espagne)}\\
    \small{\textcite{krizek_bicycling_2010}}\index{Krizek, Kevin J.|pagebf} & \small{Rayon de pertinence de 3,2 kilomètres (2 miles)} & \small{Vélo} & \small{Bus} & \small{Longmont, Boulder (États-Unis)}\\
    \small{\textcite{liu_solving_2012}}\index{Liu, Zhili|pagebf} & \small{Rayon de pertinence de 3 kilomètres} & \small{VLS} & \small{Métro} & \small{Beijing (Chine)}\\
        \hline
        \caption*{Corpus scientifique se rapportant au rôle de la distance, dans le cadre de la \acrshort{RSL}}
        \label{Corpus scientifique se rapportant au rôle de la distance, dans le cadre de la RSL}
        \begin{flushright}
        \scriptsize
    Auteur~: \textcopyright~Moinse 2023
        \end{flushright}
        \end{longtable}

    % Annexe D.13
    \newpage
\subsection{Corpus de la \acrshort{RSL} sur les résultats en lien avec la portée comparée}
    \label{donnees-ouvertes:rsl_resultats_portee_comparee}

    % Référence
Le présent tableau synthétise les résultats issus de la revue de littérature portant plus particulièrement sur la \hyperref[Distances vers et depuis les nœuds de transport en commun]{sous-partie consacrée aux distances vers et depuis les nœuds de transport en commun} (page \pageref{Distances vers et depuis les nœuds de transport en commun}).\par

    % Tableau résultats RSL (portée comparée)
        \begin{longtable}{p{3cm}p{4cm}p{1.5cm}p{1.8cm}p{2.3cm}}
        \hline
        \textcolor{blue}{\textbf{Références}} & \textcolor{blue}{\textbf{Avantages comparatifs}} & \textcolor{blue}{\textbf{MIL}} & \textcolor{blue}{\textbf{TC}} & \textcolor{blue}{\textbf{Contexte}}
        \hline
        \endhead
\multicolumn{5}{l}{\textbf{Avantages comparatifs par rapport à la \gls{marche combinée}}}\\
    \small{\textcite{advani_bicycle_2006}}\index{Advani, Mukti|pagebf} & \small{Distance supérieure à la marche} & \small{Vélo} & \small{Bus} & \small{New Delhi (Inde)}\\
    \small{\textcite{lee_strategies_2010}}\index{Lee, Jaeyeong|pagebf} & \small{Vitesse supérieure à la marche} & \small{Vélo} & \small{Métro} & \small{Séoul, Daejeon (Corée du Sud)}\\
    \small{\textcite{bachand-marleau_much-anticipated_2011}}\index{Bachand-Marleau, Julie|pagebf} & \small{Acceptation sociale de distances temps plus longues par rapport à la marche} & \small{VLS} & \small{Train, Métro, Tramway, Bus} & \small{Montréal (Canada)}\\
    \small{\textcite{flamm_determinants_2013}}\index{Flamm, Bradley J.|pagebf} & \small{Distance à vélo et en micro-mobilité supérieure à la marche} & \small{Vélo} & \small{Bus} & \small{Cleveland (États-Unis)}\\
    \small{\textcite{ton_understanding_2020}}\index{Ton, Danique|pagebf} & \small{Distance à vélo et en micro-mobilité supérieure à la marche} & \small{Vélo} & \small{Tramway} & \small{La Haye (Pays-Bas)}\\
    \small{\textcite{kostrzewska_towards_2017}} & \small{\acrshort{EDP} trois fois plus rapide que la marche et complète le vélo} & \small{Trottinette mécanique} & \small{Métro, Tramway, Bus} & \small{Berlin (Allemagne), Szczecin (Pologne)}\\
    \small{\textcite{rastogi_willingness_2010}}\index{Rastogi, Rajat|pagebf} & \small{Compétitif à partir de 1,25 kilomètre à pied} & \small{Vélo} & \small{Train} & \small{Mumbai (Inde)}\\
    \small{\textcite{chen_determinants_2012}}\index{Chen, Lijun|pagebf} & \small{Compétitif à partir de quinze minutes à pied pour 65\% des répondant·e·s} & \small{Vélo} & \small{Métro} & \small{Nanjing (Chine)}\\
    \hline
\multicolumn{5}{l}{\textbf{Avantages comparatifs par rapport aux modes motorisés}}\\
    \small{\textcite{chen_study_2013}}\index{Chen, Wan|pagebf} & \small{Interstice entre la marche et les transports publics et l'automobile} & \small{Vélo} & \small{Métro} & \small{Xi'an (Chine)}\\
    \small{\textcite{chen_demand_2013}}\index{Chen, Jingxu|pagebf} & \small{Interstice entre la marche et les transports publics et l'automobile} & \small{Vélo} & \small{Métro} & \small{Nanjing (Chine)}\\
    \small{\textcite{baek_electric_2021}}\index{Baek, Kwangho|pagebf} & \small{Valeur du temps de \gls{déplacement} moins contraignante pour la TEFF} & \small{TEFF} & \small{Métro} & \small{Séoul (Corée du Sud)}\\
    \small{\textcite{lee_forecasting_2021}}\index{Lee, Mina|pagebf} & \small{Moins compétitif à mesure que la distance augmente} & \small{TEFF} & \small{Métro} & \small{New York City (États-Unis)}\\
    \small{\textcite{givoni_access_2007}}\index{Givoni, Moshe|pagebf} & \small{Moins compétitif au-delà de trois kilomètres} & \small{Vélo} & \small{Train} & \small{Pays-Bas}\\
    \small{\textcite{bi_analysis_2021}}\index{Bi, Hui|pagebf} & \small{Compétitif entre un et trois kilomètres} & \small{VLS} & \small{Métro} & \small{Chengdu (Chine)}\\
    \small{\textcite{wang_interchange_2016}}\index{Wang, Zi-jia|pagebf} & \small{Compétitif entre 0,40 et 1,40 kilomètre} & \small{Vélo} & \small{Métro} & \small{Beijing (Chine)}\\
    \small{\textcite{liu_mode_2022}}\index{Liu, Lumei|pagebf} & \small{Une augmentation d'un mètre diminue de 0,3\% le choix du Bus de transfert entre 0,50 et 3 kilomètres} & \small{VFF} & \small{Métro, Bus} & \small{Beijing (Chine)}\\
        \hline
        \caption*{Corpus scientifique se rapportant à la portée comparée de la \gls{mobilité individuelle légère}, dans le cadre de la \acrshort{RSL}}
        \label{Corpus scientifique se rapportant à la portée comparée de la mobilité individuelle légère, dans le cadre de la RSL}
        \begin{flushright}
        \scriptsize
    Auteur~: \textcopyright~Moinse 2023
        \end{flushright}
        \end{longtable}

    % Annexe D.14
    \newpage
\subsection{Corpus de la \acrshort{RSL} sur les résultats en lien avec les distances effectives parcourues}
    \label{donnees-ouvertes:rsl_resultats_distances_parcourues}
    
    % Référence
Le présent tableau synthétise les résultats issus de la revue de littérature portant plus particulièrement sur la \hyperref[Distances vers et depuis les nœuds de transport en commun]{sous-partie consacrée aux distances vers et depuis les nœuds de transport en commun} (page \pageref{Distances vers et depuis les nœuds de transport en commun}).\par

    % Tableau résultats RSL (distances mesurées)
        \begin{longtable}{p{3cm}p{4cm}p{1.5cm}p{1.8cm}p{2.3cm}}
        \hline
        \textcolor{blue}{\textbf{Références}} & \textcolor{blue}{\textbf{Périmètres d'étude}} & \textcolor{blue}{\textbf{MIL}} & \textcolor{blue}{\textbf{TC}} & \textcolor{blue}{\textbf{Contexte}}
        \hline
        \endhead
    \small{\textcite{zhang_bicyclemetro_2019}}\index{Zhang, Ze|pagebf} & \small{Médiane de 0,50 kilomètre} & \small{VFF} & \small{Métro} & \small{Shanghai (Chine)}\\
    \small{\textcite{fearnley_patterns_2020}}\index{Fearnley, Nils|pagebf} & \small{Moyenne de 1 kilomètre et médiane de 0,66 kilomètre} & \small{TEFF} & \small{Train, Métro, Tramway, Bus} & \small{Oslo (Norvège)}\\
    \small{\textcite{rabaud_quand_2022}}\index{Rabaud, Mathieu|pagebf} & \small{Médiane de 1 kilomètre pour la TEP et de 1,40 kilomètre pour le vélo et le VLS, 85\textsuperscript{e} centile de 2,40 kilomètres pour la TEP, de 2,70 kilomètres de le VLS et de 3,6 kilomètres pour le vélo} & \small{Vélo, VLS, TEP} & \small{Train, Métro, Tramway, Bus} & \small{France}\\
    \small{\textcite{fillone_i_2018}}\index{Fillone, Alexis|pagebf} & \small{Moyenne de 1 kilomètre} & \small{Vélo} & \small{Tramway, Bus} & \small{Manille (Philippines)}\\
    \small{\textcite{rijsman_walking_2019}}\index{Rijsman, Lotte|pagebf} & \small{Médiane de 1 kilomètre} & \small{Vélo} & \small{Tramway} & \small{La Haye (Pays-Bas)}\\
    \small{\textcite{tarpin-pitre_typology_2020}}\index{Tarpin-Pitre, Léandre|pagebf} & \small{Moyenne de 1,18 kilomètre} & \small{VLS} & \small{Métro} & \small{Montréal (Canada)}\\
    \small{\textcite{ma_measuring_2018}}\index{Ma, Xinwei|pagebf} & \small{Moyenne de 1,28 à 1,5 kilomètre en fonction des stations (surface allant de 5,15 à 7,10 km²)} & \small{VLS} & \small{Métro} & \small{Suzhou (Chine)}\\
    \small{\textcite{hasiak_access_2019}}\index{Hasiak, Sophie|pagebf} & \small{Médiane de 1,50 kilomètre, 80\textsuperscript{e} centile de 4,90 kilomètres et distance maximale de 7,90 kilomètres} & \small{Vélo} & \small{Train} & \small{Gares locales (France)}\\
    \small{\textcite{jappinen_modelling_2013}}\index{Jäppinen, Sakari|pagebf} & \small{Moyenne de 1,50 kilomètre} & \small{VLS} & \small{Train, Métro, Tramway, Bus, Ferry} & \small{Helsinki (Finlande)}\\
    \small{\textcite{krygsman_multimodal_2004}}\index{Krygsman, Stephan|pagebf} & \small{Médianes de 1,80 kilomètre ou de 10,10 minutes en \gls{rabattement} et de 2,4 kilomètres ou de 12,50 minutes} & \small{Vélo} & \small{Train, Métro, Tramway, Bus} & \small{Utrecht (Pays-Bas)}\\
    \small{\textcite{lee_bicycle-based_2016}}\index{Lee, Jaeyeong|pagebf} & \small{85\textsuperscript{e} centile de 1,96 kilomètre en \gls{rabattement} de 2,13 kilomètres en \gls{diffusion}} & \small{Vélo} & \small{Métro} & \small{Séoul, Daejeon (Corée du Sud)}\\
    \small{\textcite{lin_analysis_2019}}\index{Lin, Diao|pagebf} & \small{Moyenne de 2 kilomètres ou de 8,20 minutes} & \small{VFF} & \small{Métro} & \small{Shanghai (Chine)}\\
    \small{\textcite{ma_understanding_2018}}\index{Ma, Xinwei|pagebf} & \small{90\textsuperscript{e} centile de 2 kilomètres ou de 30 minutes} & \small{VLS} & \small{Métro} & \small{Nanjing (Chine)}\\
    \small{\textcite{moinse_intermodal_2022}}\index{Moinse, Dylan|pagebf} & \small{Moyenne de 2,4 kilomètres ou de 10,60 minutes} & \small{TEP} & \small{Train} & \small{Provence-Alpes-Côte d'Azur (France)}\\
    \small{\textcite{cervero_bike-and-ride_2013}}\index{Cervero, Robert|pagebf} & \small{Moyenne de 2,80 kilomètres (1,75 mile) en 2008, contre 1,90 kilomètre (1,17 mile) en 1998} & \small{Vélo} & \small{Métro} & \small{San Francisco (États-Unis)}\\
    \small{\textcite{wang_bicycle-transit_2013}}\index{Wang, Rui|pagebf} & \small{Augmentation de la distance moyenne entre 2001 et 2009} & \small{Vélo} & \small{Train, Bus} & \small{États-Unis}\\
    \small{\textcite{ann_examination_2019}}\index{Ann, Sangeetha|pagebf} & \small{Moyenne de 2,90 kilomètres} & \small{Vélo} & \small{Métro} & \small{New Delhi (Inde)}\\
    \small{\textcite{la_paix_puello_modelling_2015}}\index{La Paix Puello, Lissy|pagebf} & \small{85\textsuperscript{e} centile de 3,6 kilomètres} & \small{Vélo} & \small{Train} & \small{Randstad South (Pays-Bas)}\\
    \small{\textcite{sherwin_practices_2011}}\index{Sherwin, Henrietta|pagebf} & \small{Moyenne de 3,7 kilomètres} & \small{Vélo} & \small{Train} & \small{Bristol (Angleterre)}\\
    \small{\textcite{shelat_analysing_2018}}\index{Shelat, Sanmay|pagebf} & \small{Moyenne de 3,8 kilomètres, contre 1,5 kilomètre pour les autres modes de transfert} & \small{Vélo} & \small{Train} & \small{Pays-Bas}\\
    \small{\textcite{keijer_how_2000}}\index{Keijer, Majanka|pagebf} & \small{Moyenne de 3,9 kilomètres en \gls{rabattement} et de 4,1 kilomètres en \gls{diffusion}, soit 7\% et 8\% de la distance spatiale totale} & \small{Vélo} & \small{Train} & \small{Pays-Bas}\\
    \small{\textcite{lu_improving_2018}}\index{Lu, Miaojia|pagebf} & \small{Moyenne de 16 minutes} & \small{VLS} & \small{Métro, Bus} & \small{Taipei (Taïwan)}\\
        \hline
        \caption*{Corpus scientifique se rapportant aux distances mesurées en lien avec la \gls{mobilité individuelle légère}, dans le cadre de la \acrshort{RSL}}
        \label{Corpus scientifique se rapportant aux distances mesurées en lien avec la mobilité individuelle légère, dans le cadre de la RSL}
        \begin{flushright}
        \scriptsize
    Auteur~: \textcopyright~Moinse 2023
        \end{flushright}
        \end{longtable}

    % Annexe D.15
    \newpage
\subsection{Corpus de la \acrshort{RSL} sur les résultats en lien avec les périmètres d'étude}
    \label{donnees-ouvertes:rsl_resultats_rayons_distances}

    % Référence
Le présent tableau synthétise les résultats issus de la revue de littérature portant plus particulièrement sur la \hyperref[Distances vers et depuis les nœuds de transport en commun]{sous-partie consacrée aux distances vers et depuis les nœuds de transport en commun} (page \pageref{Distances vers et depuis les nœuds de transport en commun}).\par

    % Tableau résultats RSL (aire d'influence)
        \begin{longtable}{p{3cm}p{4cm}p{1.5cm}p{1.8cm}p{2.3cm}}
        \hline
        \textcolor{blue}{\textbf{Références}} & \textcolor{blue}{\textbf{Périmètres d'étude}} & \textcolor{blue}{\textbf{MIL}} & \textcolor{blue}{\textbf{TC}} & \textcolor{blue}{\textbf{Contexte}}
        \hline
        \endhead
\multicolumn{5}{l}{\textbf{Distances spatiales}}\\
    \small{\textcite{wang_interchange_2016}}\index{Wang, Zi-jia|pagebf} & \small{Rayon de pertinence de 0,40 à 1,40 kilomètre} & \small{Vélo} & \small{Métro} & \small{Beijing (Chine)}\\
    \small{\textcite{hu_examining_2022}}\index{Hu, Songhua|pagebf} & \small{Rayon de pertinence de 1 à 1,50 kilomètre} & \small{VFF} & \small{Métro} & \small{Shanghai (Chine)}\\
    \small{\textcite{jin_competition_2019}}\index{Jin, Haitao|pagebf} & \small{Rayon de pertinence jusqu'à 2 kilomètres} & \small{VFF} & \small{Métro, Bus} & \small{Beijing (Chine)}\\
    \small{\textcite{fan_how_2019}}\index{Fan, Aihua|pagebf} & \small{Rayon de pertinence de 0,50 à 2 kilomètres} & \small{VFF} & \small{Train, Métro, Tramway, Bus} & \small{Beijing (Chine)}\\
    \small{\textcite{pan_intermodal_2010}}\index{Pan, Haixiao|pagebf} & \small{Rayon de pertinence de 0,80 à 2,50 kilomètres} & \small{Vélo, VAE} & \small{Métro} & \small{Shanghai (Chine)}\\
    \small{\textcite{ma_connecting_2022}}\index{Ma, Qingyu|pagebf} & \small{Rayon de pertinence de 2 à 3 kilomètres} & \small{TEFF} & \small{Métro} & \small{Washington D.C. (États-Unis)}\\
    \small{\textcite{li_unbalanced_2022}}\index{Li, Lili|pagebf} & \small{Rayon de pertinence jusqu'à 3 kilomètres} & \small{VFF} & \small{Métro} & \small{Beijing (Chine)}\\
    \small{\textcite{bearn_adaption_2018}}\index{Bearn, Cary|pagebf} & \small{Rayon de pertinence jusqu'à 3,20 kilomètres (2 miles)} & \small{Vélo} & \small{Métro} & \small{Atlanta (États-Unis)}\\
    \small{\textcite{keijer_how_2000}}\index{Keijer, Majanka|pagebf} & \small{Rayon de pertinence de 1 à 3,5 kilomètres} & \small{Vélo} & \small{Train} & \small{Pays-Bas}\\
    \small{\textcite{rietveld_accessibility_2000}}\index{Rietveld, Piet|pagebf} & \small{Rayon de pertinence de 1,20 à 3,70 kilomètres} & \small{Vélo} & \small{Train} & \small{Pays-Bas}\\
    \small{\textcite{debrezion_modelling_2009}}\index{Debrezion, Ghebreegziabiher|pagebf} & \small{Rayon de pertinence de 1,10 à 4,20 kilomètres} & \small{Vélo} & \small{Train} & \small{Pays-Bas}\\
    \small{\textcite{wu_measuring_2019}}\index{Wu, Xueying|pagebf} & \small{Rayon de pertinence de 0,50 à 5 kilomètres} & \small{VFF} & \small{Métro} & \small{Shenzhen (Chine)}\\
    \small{\textcite{zhao_bicycle-metro_2017}}\index{Zhao, Pengjun|pagebf} & \small{Rayon de pertinence de 1 à 5 kilomètres} & \small{Vélo, VLS} & \small{Métro} & \small{Beijing (Chine)}\\
    \small{\textcite{midenet_modal_2018}}\index{Midenet, Sophie|pagebf} & \small{Rayon de pertinence de 2 à 5 kilomètres} & \small{Vélo} & \small{Train} & \small{Amboise (France)}\\
    \small{\textcite{hamidi_shaping_2020}}\index{Hamidi, Zahra|pagebf} & \small{Rayon de pertinence de 2 à 5 kilomètres} & \small{Vélo} & \small{Train, Métro, Tramway, Bus} & \small{Göteborg, Malmö (Suède), Beijing (Chine)}\\
    \small{\textcite{cottrell_transforming_2007}}\index{Cottrell, Wayne D.|pagebf} & \small{Rayon de pertinence de 2 à 5 kilomètres} & \small{Vélo} & \small{Bus} & \small{El Monte (États-Unis)}\\
    \small{\textcite{yang_bike-and-ride_2014}}\index{Yang, Liu|pagebf}\index{Yang, Liu|pagebf} & \small{Rayon de pertinence jusqu'à 5 kilomètres} & \small{Vélo, VLS} & \small{Train, Métro, Tramway, Bus} & \small{Xi'an (Chine)}\\
    \small{\textcite{kim_analysis_2021}}\index{Kim, Minjun|pagebf} & \small{Rayon de pertinence jusqu'à 10 kilomètres} & \small{VLS} & \small{Métro, Bus} & \small{Séoul (Corée du Sud)}\\
    \hline
\multicolumn{5}{l}{\textbf{Distances temps}}\\
    \small{\textcite{li_factors_2020}}\index{Li, Xuefeng|pagebf} & \small{Rayon de pertinence de 1 à 7 minutes} & \small{VFF} & \small{Métro} & \small{Shenzhen (Chine)}\\
    \small{\textcite{liu_understanding_2020}}\index{Liu, Yang|pagebf} & \small{Rayon de pertinence jusqu'à 10 minutes} & \small{VLS} & \small{Métro} & \small{Nanjing (Chine)}\\
    \small{\textcite{kong_deciphering_2020}}\index{Kong, Hui|pagebf} & \small{Rayon de pertinence jusqu'à 12 minutes} & \small{VLS} & \small{Train, Métro, Tramway, Bus} & \small{Boston, Chicago, Washington D.C., New York City (États-Unis)}\\
    \small{\textcite{tomita_demand_2017}}\index{Tomita, Yasuo|pagebf}\index{Tomita, Yasuo|pagebf} & \small{Rayon de pertinence jusqu'à 15 minutes} & \small{VLS} & \small{Train} & \small{Osaka (Japon)}\\
    \small{\textcite{yang_empirical_2016}}\index{Yang, Min|pagebf} & \small{Rayon de pertinence jusqu'à 20 minutes} & \small{VLS} & \small{Métro} & \small{Nanjing (Chine)}\\
    \small{\textcite{cheng_evaluating_2012}}\index{Cheng, Yung-Hsiang|pagebf} & \small{Rayon de pertinence} & \small{Vélo} & \small{Métro} & \small{Kaohsiung (Taïwan)}\\
        \hline
        \caption*{Corpus scientifique se rapportant à l'aire d'influence couverte par la \gls{mobilité individuelle légère}, dans le cadre de la \acrshort{RSL}}
        \label{Corpus scientifique se rapportant à l'aire d'influence couverte par la mobilité individuelle légère, dans le cadre de la RSL}
        \begin{flushright}
        \scriptsize
    Auteur~: \textcopyright~Moinse 2023
        \end{flushright}
        \end{longtable}

    % Annexe D.16
    \newpage
\subsection{Corpus de la \acrshort{RSL} sur les résultats en lien avec les facteurs influençant le choix des itinéraires}
    \label{donnees-ouvertes:rsl_resultats_variabilite_distances}

    % Référence
Le présent tableau synthétise les résultats issus de la revue de littérature portant plus particulièrement sur la \hyperref[Distances vers et depuis les nœuds de transport en commun]{sous-partie consacrée aux distances vers et depuis les nœuds de transport en commun} (page \pageref{Distances vers et depuis les nœuds de transport en commun}).\par

    % Tableau résultats RSL (effets de la distance)
        \begin{longtable}{p{3cm}p{4cm}p{1.5cm}p{1.8cm}p{2.3cm}}
        \hline
        \textcolor{blue}{\textbf{Références}} & \textcolor{blue}{\textbf{Effets de la distance}} & \textcolor{blue}{\textbf{MIL}} & \textcolor{blue}{\textbf{TC}} & \textcolor{blue}{\textbf{Contexte}}
        \hline
        \endhead
\multicolumn{5}{l}{\textbf{Influence de l'environnement urbain et du contexte temporel}}\\
    \small{\textcite{tzouras_describing_2023}}\index{Tzouras, Panagiotis|pagebf} & \small{Influence de la sécurité perçue} & \small{TEP} & \small{Métro, Tramway} & \small{Athènes (Grèce)}\\
    \small{\textcite{krizek_detailed_2007}}\index{Krizek, Kevin J.|pagebf} & \small{Influence du motif de déplacement} & \small{Vélo} & \small{Tramway} & \small{Minneapolis (États-Unis)}\\
    \small{\textcite{adnan_last-mile_2019}}\index{Adnan, Muhammad|pagebf} & \small{Influence de la température et du taux de précipitation} & \small{VLS} & \small{Train, Métro, Tramway, Bus} & \small{Villes entre 30~000 et 200~000 habitants (Belgique)}\\
    \small{\textcite{cho_estimation_2022}}\index{Cho, Shin-Hyung|pagebf} & \small{Influence de la nuit} & \small{VLS} & \small{Métro} & \small{Séoul (Corée du Sud)}\\
    \small{\textcite{li_operating_2019}} & \small{Influence du jour de la semaine et du type de station} & \small{VFF} & \small{Métro} & \small{Nanjing (Chine)}\\
    \small{\textcite{flamm_public_2014}}\index{Flamm, Bradley J.|pagebf} & \small{Influence de la topographie et du type de mode collectif} & \small{Vélo} & \small{Train, Métro, Tramway, Bus} & \small{Philadelphie, San Francisco (États-Unis)}\\
    \small{\textcite{hochmair_assessment_2015}}\index{Hochmair, Hartwig H.|pagebf} & \small{Influence du type de mode collectif plus prononcée que les variables socio-démographiques} & \small{Vélo} & \small{Train, Métro, Tramway, Bus} & \small{Los Angeles, Atlanta, Minneapolis, Saint-Paul (États-Unis)}\\
    \small{\textcite{yu_policy_2021}}\index{Yu, Qing|pagebf} & \small{Influence de la localisation en centre urbain ou dans le périurbain} & \small{VLS} & \small{Métro} & \small{Shanghai (Chine)}\\
    \hline
\multicolumn{5}{l}{\textbf{Détours}}\\
    \small{\textcite{li_exploring_2021}}\index{Li, Wei|pagebf} & \small{Attractivité des gares fréquentées} & \small{VFF} & \small{Métro} & \small{Shanghai (Chine)}\\
    \small{\textcite{kampen_understanding_2020}}\index{van Kampen, Jullian|pagebf} & \small{Influence du type de gare et de son niveau de service} & \small{Vélo} & \small{Train} & \small{North-Holland, South-Holland, Flevoland, Utrecht (Pays-Bas)}\\
    \small{\textcite{jonkeren_bicycle-train_2021}}\index{Jonkeren, Olaf|pagebf} & \small{Stratégies pour éviter des ruptures de charge} & \small{Vélo} & \small{Train} & \small{Utrecht, Rotterdam, Eindhoven (Pays-Bas)}\\
    \small{\textcite{kampen_bicycle_2021}}\index{van Kampen, Jullian|pagebf} & \small{Influence des places de stationnement vélo aménagées à proximité des stations} & \small{Vélo} & \small{Métro} & \small{Amsterdam (Pays-Bas)}\\
    \small{\textcite{kampen_understanding_2021}}\index{van Kampen, Jullian|pagebf} & \small{} & \small{} & \small{} & \small{}\\
        \hline
        \caption*{Corpus scientifique se rapportant aux effets de la distance sur l'usage intermodal de la \gls{mobilité individuelle légère}, dans le cadre de la \acrshort{RSL}}
        \label{Corpus scientifique se rapportant aux effets de la distance sur l'usage intermodal de la mobilité individuelle légère, dans le cadre de la RSL}
        \begin{flushright}
        \scriptsize
    Auteur~: \textcopyright~Moinse 2023
        \end{flushright}
        \end{longtable}

    % Annexe D.17
    \newpage
\subsection{Corpus de la \acrshort{RSL} sur les résultats en lien avec le niveau de service du transport public}
    \label{donnees-ouvertes:rsl_resultats_niveau_service}

    % Référence
Le présent tableau synthétise les résultats issus de la revue de littérature portant plus particulièrement sur la \hyperref[Gestion de la demande de mobilité]{sous-partie consacrée à la gestion de la demande de mobilité} (page \pageref{Gestion de la demande de mobilité}).\par

    % Tableau résultats RSL (LoS)
        \begin{longtable}{p{3cm}p{4cm}p{1.5cm}p{1.8cm}p{2.3cm}}
        \hline
        \textcolor{blue}{\textbf{Références}} & \textcolor{blue}{\textbf{LoS}} & \textcolor{blue}{\textbf{MIL}} & \textcolor{blue}{\textbf{TC}} & \textcolor{blue}{\textbf{Contexte}}
        \hline
        \endhead
\multicolumn{5}{l}{\textbf{Rôle de la fréquence des services ferroviaires}}\\
    \small{\textcite{la_paix_puello_modelling_2015}}\index{La Paix Puello, Lissy|pagebf} & \small{Association positive} & \small{Vélo} & \small{Train} & \small{Randstad South (Pays-Bas)}\\
    \small{\textcite{la_paix_puello_modelling_2015}}\index{La Paix Puello, Lissy|pagebf} & \small{Association positive} & \small{Vélo, VLS} & \small{Métro, Tramway, Bus} & \small{Rotterdam (Pays-Bas)}\\
    \small{\textcite{staricco_implementing_2020}}\index{Staricco, Luca|pagebf} & \small{Association positive} & \small{Vélo} & \small{Train} & \small{Turin (Italie)}\\ 
    \small{\textcite{lin_analysis_2019}}\index{Lin, Diao|pagebf} & \small{Association positive} & \small{VFF} & \small{Métro} & \small{Shanghai (Chine)}\\
    \small{\textcite{weliwitiya_bicycle_2019}}\index{Weliwitiya, Hesara|pagebf} & \small{Une augmentation d'une unité de fréquence entraîne une augmentation de 1,027 de cyclistes, notamment durant les heures de pointe matinales} & \small{Vélo} & \small{Train} & \small{Melbourne (Australie)}\\
    \small{\textcite{radzimski_exploring_2021}}\index{Radzimski, Adam|pagebf} & \small{Association positive pour les trajets en VLS, longs de moins de trois kilomètres} & \small{VLS} & \small{Tramway, Bus} & \small{Poznań (Pologne)}\\
    \small{\textcite{kuijk_preferences_2022}}\index{van Kuijk, R.J.|pagebf} & \small{Facteur non significatif} & \small{VLS} & \small{Tramway, Bus} & \small{Utrecht (Pays-Bas)}\\
    \small{\textcite{nielsen_bikeability_2018}}\index{Nielsen, Thomas Alexander Sick|pagebf} & \small{Association négative} & \small{Vélo} & \small{Train, Bus} & \small{Danemark}\\
    \hline
\multicolumn{5}{l}{\textbf{Rôle de la densité des stations de transport en commun}}\\
    \small{\textcite{waerden_relation_2018}}\index{Waerden, Peter|pagebf} & \small{Association positive avec le temps de voyage et les coûts de déplacement} & \small{Vélo} & \small{Train} & \small{Eindhoven (Pays-Bas)}\\
    \small{\textcite{nielsen_bikeability_2018}}\index{Nielsen, Thomas Alexander Sick|pagebf} & \small{Association négative} & \small{Vélo} & \small{Train, Bus} & \small{Danemark}\\
    \small{\textcite{ji_exploring_2018}}\index{Ji, Yanjie|pagebf} & \small{Association positive} & \small{VLS} & \small{Métro} & \small{Nanjing (Chine)}\\
        \hline
        \caption*{Corpus scientifique se rapportant au niveau de service des transports en commun, dans le cadre de la \acrshort{RSL}}
        \label{Corpus scientifique se rapportant aux effets de genre, dans le cadre de la RSL}
        \begin{flushright}
        \scriptsize
    Auteur~: \textcopyright~Moinse 2023
        \end{flushright}
        \end{longtable}

    % Annexe D.18
    \newpage
\subsection{Corpus de la \acrshort{RSL} sur les résultats en lien avec la tarification intégrée}
    \label{donnees-ouvertes:rsl_resultats_tarification}

    % Référence
Le présent tableau synthétise les résultats issus de la revue de littérature portant plus particulièrement sur la \hyperref[Gestion de la demande de mobilité]{sous-partie consacrée à la gestion de la demande de mobilité} (page \pageref{Gestion de la demande de mobilité}).\par

    % Tableau résultats RSL (tarification)
        \begin{longtable}{p{3cm}p{4cm}p{1.5cm}p{1.8cm}p{2.3cm}}
        \hline
        \textcolor{blue}{\textbf{Références}} & \textcolor{blue}{\textbf{Tarification}} & \textcolor{blue}{\textbf{MIL}} & \textcolor{blue}{\textbf{TC}} & \textcolor{blue}{\textbf{Contexte}}
        \hline
        \endhead
\multicolumn{5}{l}{\textbf{Plate-forme multimodale}}\\
    \small{\textcite{guo_exploring_2023}}\index{Guo, Dongbo|pagebf} & \small{Perception négative du temps d'attente} & \small{VFF} & \small{Métro, Bus} & \small{Beijing (Chine)}\\
    \small{\textcite{yang_bike-and-ride_2014}}\index{Yang, Liu|pagebf}\index{Yang, Liu|pagebf} & \small{Développement des cartes de transport intégrées} & \small{Vélo, VLS} & \small{Train, Métro, Tramway, Bus} & \small{Xi'an (Chine)}\\
    \small{\textcite{yang_empirical_2016}}\index{Yang, Min|pagebf} & \small{Développement des cartes de transport intégrées} & \small{VLS} & \small{Métro} & \small{Nanjing (Chine)}\\
    \small{\textcite{fearnley_patterns_2020}}\index{Fearnley, Nils|pagebf} & \small{Amélioration de la plate-forme multimodale} & \small{TEFF} & \small{Train, Métro, Tramway, Bus} & \small{Oslo (Norvège)}\\
    \small{\textcite{chen_determinants_2012}}\index{Chen, Lijun|pagebf} & \small{Amélioration de la plate-forme multimodale} & \small{Vélo} & \small{Métro} & \small{Nanjing (Chine)}\\
    \small{\textcite{ma_understanding_2018}}\index{Ma, Xinwei|pagebf} & \small{Segmentation des types d'usager·ère·s pour optimiser les déplacements intermodaux} & \small{VLS} & \small{Métro} & \small{Nanjing (Chine)}\\
    \hline
\multicolumn{5}{l}{\textbf{Stratégies de tarification}}\\
    \small{\textcite{fan_how_2019}}\index{Fan, Aihua|pagebf} & \small{Coût monétaire égal entre le trajet en VFF et en transport en commun, la tarification étant un facteur pesant dans le choix modal pour 31,8\% des répondant·e·s} & \small{VFF} & \small{Train, Métro, Tramway, Bus} & \small{Beijing (Chine)}\\
    \small{\textcite{montes_shared_2023}}\index{Montes, Alejandro|pagebf} & \small{Repenser la tarification actuellement désincitative concernant le VLS} & \small{Vélo, VLS} & \small{Métro, Tramway, Bus} & \small{Rotterdam (Pays-Bas)}\\
    \small{\textcite{liu_understanding_2020}}\index{Liu, Yang|pagebf} & \small{Les jeunes actif·ve·s choisissent le VLS en raison de contraintes budgétaires et non par choix} & \small{VLS} & \small{Métro} & \small{Nanjing (Chine)}\\
    \small{\textcite{zhong_layout_2021}}\index{Zhong, Hongming|pagebf} & \small{Sensibilité significative vis-à-vis du coût monétaire d'un trajet en VFF, se répercutant sur le choix modal} & \small{VFF} & \small{Métro} & \small{Nanjing (Chine)}\\
    \small{\textcite{chen_determinants_2012}}\index{Chen, Lijun|pagebf} & \small{Préférence pour un temps gratuit de location de vélo de deux heures} & \small{Vélo} & \small{Métro} & \small{Nanjing (Chine)}\\
    \small{\textcite{yan_evaluating_2023}}\index{Yan, Xiang|pagebf} & \small{Effets positifs de la mise en place d'un crédit de 3\$ sur l'usage de la TEFF par l'opérateur} & \small{TEFF} & \small{Métro, Bus} & \small{Washington D.C., Los Angeles (États-Unis)}\\
    \small{\textcite{fournier_continuous_2021}}\index{Fournier, Nicholas|pagebf} & \small{Atténuation de l'impact du coût d'un déplacement par la mise en place d'une tarification basée sur la distance parcourue} & \small{Vélo} & \small{Train} & \small{Boston, Worcester (États-Unis)}\\
    \small{\textcite{fearnley_patterns_2020}}\index{Fearnley, Nils|pagebf} & \small{Efficacité de la tarification basée à la minute} & \small{TEFF} & \small{Train, Métro, Tramway, Bus} & \small{Oslo (Norvège)}\\
    \small{\textcite{beale_integrating_2023}}\index{Beale, Kirsten|pagebf} & \small{Effets positifs d'une tarification sociale basée sur les revenus} & \small{VFF, TEFF} & \small{Métro, Tramway} & \small{Seattle (États-Unis)}\\
    \small{\textcite{givoni_access_2007}}\index{Givoni, Moshe|pagebf} & \small{Effets positifs de la gratuité des transports en commun adressée aux étudiant·e·s} & \small{Vélo} & \small{Train} & \small{Pays-Bas}\\
    \small{\textcite{van_mil_insights_2020}}\index{van Mil, Joeri F.P.|pagebf} & \small{Aversion significative pour les coûts de stationnement vélo par les étudiant·e·s} & \small{Vélo} & \small{Train} & \small{Pays-Bas}\\
    \hline
\multicolumn{5}{l}{\textbf{Facteur lié au prix non significatif}}\\
    \small{\textcite{molin_bicycle_2015}}\index{Molin, Eric|pagebf} & \small{Demande inélastique du prix du stationnement vélo autour des gares par rapport à son taux d'occupation} & \small{Vélo} & \small{Train} & \small{Delft (Pays-Bas)}\\
    \small{\textcite{liu_mode_2022}}\index{Liu, Lumei|pagebf} & \small{Demande inélastique du prix du bus et du VFF par rapport au choix modal du mode de transfert} & \small{VFF} & \small{Métro, Bus} & \small{Beijing (Chine)}\\
        \hline
        \caption*{Corpus scientifique se rapportant à la tarification intégrée, dans le cadre de la \acrshort{RSL}}
        \label{Corpus scientifique se rapportant à la tarification intégrée, dans le cadre de la RSL}
        \begin{flushright}
        \scriptsize
    Auteur~: \textcopyright~Moinse 2023
        \end{flushright}
        \end{longtable}

    % Annexe D.19
    \newpage
\subsection{Corpus de la \acrshort{RSL} sur les résultats en lien avec les services de transfert}
    \label{donnees-ouvertes:rsl_resultats_services_transfert}

    % Référence
Le présent tableau synthétise les résultats issus de la revue de littérature portant plus particulièrement sur la \hyperref[Gestion de la demande de mobilité]{sous-partie consacrée à la gestion de la demande de mobilité} (page \pageref{Gestion de la demande de mobilité}).\par

    % Tableau résultats RSL (vélopartage et embarquement)
        \begin{longtable}{p{3cm}p{4cm}p{1.5cm}p{1.8cm}p{2.3cm}}
        \hline
        \textcolor{blue}{\textbf{Références}} & \textcolor{blue}{\textbf{Transfert}} & \textcolor{blue}{\textbf{MIL}} & \textcolor{blue}{\textbf{TC}} & \textcolor{blue}{\textbf{Contexte}}
        \hline
        \endhead
\multicolumn{5}{l}{\textbf{Système de vélo et de \gls{micro-mobilité} partagés}}\\
    \small{\textcite{wu_measuring_2019}}\index{Wu, Xueying|pagebf} & \small{Présence stratégique du VFF} & \small{VFF} & \small{Métro} & \small{Shenzhen (Chine)}\\
    \small{\textcite{tamakloe_determinants_2021}}\index{Tamakloe, Reuben|pagebf} & \small{Présence stratégique du VFF} & \small{VLS} & \small{Métro, Bus} & \small{Séoul (Corée du Sud)}\\
    \small{\textcite{guo_dockless_2021}}\index{Guo, Yuanyuan|pagebf} & \small{La disponibilité du VFF augmente la probabilité de leur utilisation en tant que mode transfert} & \small{VFF} & \small{Métro} & \small{Shenzhen (Chine)}\\
    \small{\textcite{jonkeren_bicycle_2021}}\index{Jonkeren, Olaf|pagebf} & \small{Limite la pratique du second vélo en gare} & \small{Vélo} & \small{Train} & \small{Pays-Bas}\\
    \small{\textcite{pan_intermodal_2010}}\index{Pan, Haixiao|pagebf} & \small{Volonté d'utiliser un système de location de vélos en complément avec le métro} & \small{Vélo, VAE} & \small{Métro} & \small{Shanghai (Chine)}\\
    \small{\textcite{sherwin_practices_2011}}\index{Sherwin, Henrietta|pagebf} & \small{Mise en place d'un système de location de vélos organisé nationalement} & \small{Vélo} & \small{Train} & \small{Bristol (Angleterre)}\\
    \hline
\multicolumn{5}{l}{\textbf{Gestion optimisée du système de mobilité partagée}}\\
    \small{\textcite{radzimski_exploring_2021}}\index{Radzimski, Adam|pagebf} & \small{Le système de VLS est plus attaractif que le VFF en \gls{intermodalité}} & \small{VLS} & \small{Tramway, Bus} & \small{Poznań (Pologne)}\\
    \small{\textcite{liu_temporal_2022}}\index{Liu, Siyang|pagebf} & \small{Nécessité d'une allocation efficace du VFF dans les espaces verts, les zones commerciales et industrielles et dans les quartiers résidentiels} & \small{VFF} & \small{Métro} & \small{Beijing (Chine)}\\
    \small{\textcite{liu_measuring_2022}}\index{Liu, Lumei|pagebf} & \small{Contraintes de saturation de la capacité de la TEFF lorsqu'il y a de nombreux·ses utilisateur·rice·s simultané·e·s} & \small{TEFF} & \small{Bus} & \small{Columbus (États-Unis)}\\
    \small{\textcite{van_der_nat_bicycle_2018}}\index{van der Nat, Johanna Debóra|pagebf} & \small{Suggestion d'un modèle hybride de location de vélos} & \small{VLS, VFF} & \small{Train} & \small{Amsterdam (Pays-Bas)}\\
    \hline
\multicolumn{5}{l}{\textbf{Emport de la mobilité individuelle légère}}\\
    \small{\textcite{singleton_exploring_2014}}\index{Singleton, Patrick A.|pagebf} & \small{Importance de la possibilité d'embarquer le vélo à bord du métro} & \small{Vélo} & \small{Métro} & \small{Portland (États-Unis)}\\
    \small{\textcite{halldorsdottir_home-end_2017}}\index{Halldórsdóttir, Katrín|pagebf} & \small{La gratuité de l'emport du vélo dans le train augmente son utilisation} & \small{Vélo} & \small{Train} & \small{Copenhague (Danemark)}\\
        \hline
        \caption*{Corpus scientifique se rapportant à la tarification intégrée, dans le cadre de la \acrshort{RSL}}
        \label{Corpus scientifique se rapportant à la tarification intégrée, dans le cadre de la RSL}
        \begin{flushright}
        \scriptsize
    Auteur~: \textcopyright~Moinse 2023
        \end{flushright}
        \end{longtable}

    % Annexe D.20
    \newpage
\subsection{Corpus de la \acrshort{RSL} sur les résultats en lien avec la desserte en bus}
    \label{donnees-ouvertes:rsl_resultats_desserte_bus}

    % Référence
Le présent tableau synthétise les résultats issus de la revue de littérature portant plus particulièrement sur la \hyperref[Gestion de la demande de mobilité]{sous-partie consacrée à la gestion de la demande de mobilité} (page \pageref{Gestion de la demande de mobilité}).\par

    % Tableau résultats RSL (bus)
        \begin{longtable}{p{3cm}p{4cm}p{1.5cm}p{1.8cm}p{2.3cm}}
        \hline
        \textcolor{blue}{\textbf{Références}} & \textcolor{blue}{\textbf{Services de bus}} & \textcolor{blue}{\textbf{MIL}} & \textcolor{blue}{\textbf{TC}} & \textcolor{blue}{\textbf{Contexte}}
        \hline
        \endhead
    \small{\textcite{chen_what_2022}}\index{Chen, Wendong|pagebf} & \small{Effet de substitution modale en faveur du bus, nécessitant une meilleure gestion des systèmes de vélopartage} & \small{VLS, VFF} & \small{Métro} & \small{Nanjing (Chine)}\\
    \small{\textcite{zhao_bicycle-metro_2017}}\index{Zhao, Pengjun|pagebf} & \small{Effet de substitution modale en faveur du bus} & \small{Vélo, VLS} & \small{Métro} & \small{Beijing (Chine)}\\
    \small{\textcite{wang_spatiotemporal_2020}}\index{Wang, Zijia|pagebf} & \small{Effet de substitution modale en faveur du bus} & \small{VFF} & \small{Métro} & \small{Beijing (Chine)}\\
    \small{\textcite{li_exploring_2021}}\index{Li, Wei|pagebf} & \small{Association négative entre la densité des arrêts de bus et la distance de transfert} & \small{VFF} & \small{Métro} & \small{Shanghai (Chine)}\\
    \small{\textcite{luan_better_2020}}\index{Luan, Xin|pagebf} & \small{Préférence du vélo par rapport au bus} & \small{Vélo} & \small{Métro} & \small{Nanjing (Chine)}\\
    \small{\textcite{ma_bicycle_2015}}\index{Ma, Ting|pagebf} & \small{Effets positifs de la desserte en bus} & \small{VLS} & \small{Train} & \small{Washington D.C. (États-Unis)}\\
    \small{\textcite{guo_built_2020}}\index{Guo, Yuanyuan|pagebf} & \small{Effets positifs de la desserte en bus} & \small{VFF} & \small{Métro} & \small{Shenzhen (Chine)}\\
    \small{\textcite{arbis_analysis_2016}}\index{Arbis, David|pagebf} & \small{Effets positifs de la desserte en bus sur l'occupation des stationnements vélo} & \small{Vélo} & \small{Train} & \small{Nouvelle-Galles du Sud (Australie)}\\
        \hline
        \caption*{Corpus scientifique se rapportant à la tarification intégrée, dans le cadre de la \acrshort{RSL}}
        \label{Corpus scientifique se rapportant à la tarification intégrée, dans le cadre de la RSL}
        \begin{flushright}
        \scriptsize
    Auteur~: \textcopyright~Moinse 2023
        \end{flushright}
        \end{longtable}

    % Annexe D.21
    \newpage
\subsection{Corpus de la \acrshort{RSL} sur les résultats en lien avec la gestion du stationnement automobile}
    \label{donnees-ouvertes:rsl_resultats_stationnement_automobile}

    % Référence
Le présent tableau synthétise les résultats issus de la revue de littérature portant plus particulièrement sur la \hyperref[Gestion de la demande de mobilité]{sous-partie consacrée à la gestion de la demande de mobilité} (page \pageref{Gestion de la demande de mobilité}).\par
  
    % Tableau résultats RSL (voiture)
        \begin{longtable}{p{3cm}p{4cm}p{1.5cm}p{1.8cm}p{2.3cm}}
        \hline
        \textcolor{blue}{\textbf{Références}} & \textcolor{blue}{\textbf{Stationnement automobile}} & \textcolor{blue}{\textbf{MIL}} & \textcolor{blue}{\textbf{TC}} & \textcolor{blue}{\textbf{Contexte}}
        \hline
        \endhead
\multicolumn{5}{l}{\textbf{}}\\
    \small{\textcite{debrezion_modelling_2009}}\index{Debrezion, Ghebreegziabiher|pagebf} & \small{Un taux de motorisation supérieur à 0,60 par individu rend la part de l'automobile dominante pour les trajets en \gls{rabattement} de plus de dix kilomètres} & \small{Vélo} & \small{Train} & \small{Pays-Bas}\\
    \small{\textcite{moinse_intermodal_2022}}\index{Moinse, Dylan|pagebf} & \small{Pratique intermodale moins compétitive que l'usage monomodal de l'automobile hors contraintes liées à la circulation et au stationnement} & \small{Vélo, TEP} & \small{Train} & \small{Provence-Alpes-Côte d'Azur (France)}\\
    \small{\textcite{halldorsdottir_home-end_2017}}\index{Halldórsdóttir, Katrín|pagebf} & \small{Effets négatifs de la disponibilité de places de stationnement automobile} & \small{Vélo} & \small{Train} & \small{Copenhague (Danemark)}\\
    \small{\textcite{chan_factors_2020}}\index{Chan, Kevin|pagebf} & \small{Effets négatifs de la disponibilité de places de stationnement automobile} & \small{Vélo} & \small{Train} & \small{Toronto, Hamilton (Canada)}\\ 
    \small{\textcite{bopp_examining_2015}}\index{Bopp, Melissa|pagebf} & \small{Effets négatifs de la disponibilité de places de stationnement automobile} & \small{Vélo} & \small{Train, Métro, Tramway} & \small{Delaware, New Jersey, Maryland, Virginie-Occidentale, Pennsylvanie, Ohio (États-Unis)}\\
    \small{\textcite{cheng_expanding_2018}}\index{Cheng, Yung-Hsiang|pagebf} & \small{Effets négatifs de la disponibilité de places de stationnement dédié à la moto en \gls{diffusion}} & \small{VLS} & \small{Métro} & \small{Kaohsiung (Taïwan)}\\
    \small{\textcite{zhu_improved_2021}}\index{Zhu, Zhenjun|pagebf} & \small{Nécessité de réduire les normes de stationnement automobile autour des stations attractives} & \small{Vélo} & \small{Métro} & \small{Xi'an (Chine)}\\
    \small{\textcite{weliwitiya_bicycle_2019}}\index{Weliwitiya, Hesara|pagebf} & \small{Nécessité de prendre en compte le stationnement automobile hors P+R qui représente 72\% de la demande} & \small{Vélo} & \small{Train} & \small{Melbourne (Australie)}\\
    \small{\textcite{cervero_bike-and-ride_2013}}\index{Cervero, Robert|pagebf} & \small{Impact positif des frais de stationnement automobile après la mise en service du BART} & \small{Vélo} & \small{Métro} & \small{San Francisco (États-Unis)}\\
    \small{\textcite{midenet_modal_2018}}\index{Midenet, Sophie|pagebf} & \small{Nécessité de la mise en place d'une politique de tarification du stationnement automobile autour des gares} & \small{Vélo} & \small{Train} & \small{Amboise (France)}\\
    \small{\textcite{papon_evaluation_2017}}\index{Papon, Francis|pagebf} & \small{Effets positifs d'une augmentation des prix du carburant} & \small{Vélo} & \small{Train} & \small{Amboise (France)}\\
        \hline
        \caption*{Corpus scientifique se rapportant à la gestion du stationnement automobile, dans le cadre de la \acrshort{RSL}}
        \label{Corpus scientifique se rapportant à la gestion du stationnement automobile, dans le cadre de la RSL}
        \begin{flushright}
        \scriptsize
    Auteur~: \textcopyright~Moinse 2023
        \end{flushright}
        \end{longtable}

    % Annexe D.22
    \newpage
\subsection{Corpus de la \acrshort{RSL} sur les résultats en lien avec l'écart de genre}
    \label{donnees-ouvertes:rsl_resultats_genre}

    % Référence
Le présent tableau synthétise les résultats issus de la revue de littérature portant plus particulièrement sur la \hyperref[Caractéristiques socio-démographiques des usagers]{sous-partie consacrée aux caractéristiques socio-démographiques des usager·ère·s} (page \pageref{Caractéristiques socio-démographiques des usagers}).\par

    % Tableau résultats RSL (genre)
        \begin{longtable}{p{3cm}p{4cm}p{1.5cm}p{1.8cm}p{2.3cm}}
        \hline
        \textcolor{blue}{\textbf{Références}} & \textcolor{blue}{\textbf{Genre}} & \textcolor{blue}{\textbf{MIL}} & \textcolor{blue}{\textbf{TC}} & \textcolor{blue}{\textbf{Contexte}}
        \hline
        \endhead
\multicolumn{5}{l}{\textbf{Effets de genre en rapport avec le vélo}}\\
    \small{\textcite{debrezion_modelling_2009}}\index{Debrezion, Ghebreegziabiher|pagebf} & \small{Les femmes sont moins susceptibles d'opter pour ces pratiques intermodales} & \small{Vélo} & \small{Train} & \small{Pays-Bas}\\
    \small{\textcite{mohanty_effect_2017}}\index{Mohanty, Sudatta|pagebf} & \small{Les femmes sont moins susceptibles d'opter pour ces pratiques intermodales} & \small{Vélo} & \small{Train, Bus} & \small{New Delhi (Inde)}\\
    \small{\textcite{park_finding_2014}}\index{Park, Sungjin|pagebf} & \small{Les femmes sont moins susceptibles d'opter pour ces pratiques intermodales} & \small{Vélo} & \small{Train} & \small{Mountain View (États-Unis)}\\
    \small{\textcite{meng_influence_2016}}\index{Meng, Meng|pagebf} & \small{Les femmes sont quasiment autant susceptibles d'opter pour ces pratiques intermodales} & \small{Vélo} & \small{Métro} & \small{Singapour}\\
    \small{\textcite{souza_modelling_2017}}\index{Souza, Flavia de|pagebf} & \small{Les femmes sont quasiment autant susceptibles d'opter pour ces pratiques intermodales} & \small{Vélo} & \small{Train, Métro, Tramway, Bus} & \small{Rio de Janeiro (Brésil)}\\
    \small{\textcite{sherwin_practices_2011}}\index{Sherwin, Henrietta|pagebf} & \small{Surreprésentation des hommes parmi les usager·ère·s} & \small{Vélo} & \small{Train} & \small{Bristol (Angleterre)}\\
    \small{\textcite{flamm_public_2014}}\index{Flamm, Bradley J.|pagebf} & \small{Surreprésentation des hommes parmi les usager·ère·s} & \small{Vélo} & \small{Train, Métro, Tramway, Bus} & \small{Philadelphie, San Francisco (États-Unis)}\\
    \small{\textcite{cheng_evaluating_2012}}\index{Cheng, Yung-Hsiang|pagebf} & \small{58,3\% des usager·ère·s sont des hommes} & \small{Vélo} & \small{Métro} & \small{Kaohsiung (Taïwan)}\\
    \small{\textcite{ravensbergen_biking_2018}}\index{Ravensbergen, Léa|pagebf} & \small{Les deux tiers des usager·ère·s sont des hommes} & \small{Vélo} & \small{Train} & \small{Toronto, Hamilton (Canada)}\\
    \small{\textcite{heinen_multimodal_2014}}\index{Heinen, Eva|pagebf} & \small{Surreprésentation des hommes parmi les usager·ère·s, contrairement aux autres formes de mobilité} & \small{Vélo} & \small{Train, Métro, Tramway, Bus} & \small{Delft, Zwolle, Midden-Delfland, Pijnacker-Nootdorp (Pays-Bas)}\\
    \small{\textcite{krygsman_multimodal_2004}}\index{Krygsman, Stephan|pagebf} & \small{Les hommes parcourent des distances plus longues} & \small{Vélo} & \small{Train, Métro, Tramway, Bus} & \small{Utrecht (Pays-Bas)}\\
    \small{\textcite{wang_bicycle-transit_2013}}\index{Wang, Rui|pagebf} & \small{Surreprésentation des hommes parmi les usager·ère·s, avec un déséquilibre qui s'accroît entre 2001 et 2009} & \small{Vélo} & \small{Train, Bus} & \small{États-Unis}\\
    \small{\textcite{bearn_adaption_2018}}\index{Bearn, Cary|pagebf} & \small{Développement d'un réseau cyclable vers les quartiers isolés promouvant la participation féminine au vélo} & \small{Vélo} & \small{Métro} & \small{Atlanta (États-Unis)}\\
    \hline
\multicolumn{5}{l}{\textbf{Effets de genre en rapport avec le vélopartage et la trottinette}}\\
    \small{\textcite{adnan_last-mile_2019}}\index{Adnan, Muhammad|pagebf} & \small{Les femmes sont moins susceptibles d'opter pour ces pratiques intermodales} & \small{VLS} & \small{Train, Métro, Tramway, Bus} & \small{Villes entre 30~000 et 200~000 habitants (Belgique)}\\
    \small{\textcite{ma_measuring_2018}}\index{Ma, Xinwei|pagebf} & \small{Surreprésentation des hommes parmi les usager·ère·s} & \small{vls} & \small{Métro} & \small{Suzhou (Chine)}\\
    \small{\textcite{martin_evaluating_2014}}\index{Martin, Elliot W.|pagebf} & \small{Surreprésentation des hommes parmi les usager·ère·s} & \small{VLS} & \small{Métro, Tramway} & \small{Washington D.C., Minneapolis (États-Unis)}\\
    \small{\textcite{bachand-marleau_much-anticipated_2011}}\index{Bachand-Marleau, Julie|pagebf} & \small{58\% des usager·ère·s sont des hommes} & \small{VLS} & \small{Train, Métro, Tramway, Bus} & \small{Montréal (Canada)}\\
    \small{\textcite{bocker_bike_2020}}\index{Böcker, Lars|pagebf} & \small{58\% des usager·ère·s sont des hommes qui réalisent 68\% des trajets, notamment en centre urbain} & \small{VLS} & \small{Métro} & \small{Oslo (Norvège)}\\
    \small{\textcite{ma_understanding_2018}}\index{Ma, Xinwei|pagebf} & \small{Les femmes sont plus nombreuses entre 6h et 7h ainsi qu'entre 16h et 17h} & \small{VLS} & \small{Métro} & \small{Nanjing (Chine)}\\
    \small{\textcite{fan_how_2019}}\index{Fan, Aihua|pagebf} & \small{Les femmes sont moins susceptibles d'opter pour ces pratiques intermodales} & \small{VFF} & \small{Train, Métro, Tramway, Bus} & \small{Beijing (Chine)}\\
    \small{\textcite{pages_les_2021}}\index{Pages, Thibaud|pagebf} & \small{Surreprésentation des hommes parmi les usager·ère·s} & \small{TEP} & \small{Train, Métro, Tramway, Bus} & \small{Marseille, Montpellier (France)}\\
    \small{\textcite{moinse_intermodal_2022}}\index{Moinse, Dylan|pagebf} & \small{83\% des usager·ère·s sont des hommes, contre 51\% tou·te·s voyageur·se·s confondu·e·s} & \small{TEP} & \small{Train} & \small{Provence-Alpes-Côte d'Azur (France)}\\
    \small{\textcite{fearnley_patterns_2020}}\index{Fearnley, Nils|pagebf} & \small{Surreprésentation des hommes parmi les usager·ère·s} & \small{TEFF} & \small{Train, Métro, Tramway, Bus} & \small{Oslo (Norvège)}\\
    \small{\textcite{yan_evaluating_2023}}\index{Yan, Xiang|pagebf} & \small{Surreprésentation des hommes parmi les usager·ère·s} & \small{TEFF} & \small{Métro, Bus} & \small{Washington D.C., Los Angeles (États-Unis)}\\
    \hline
\multicolumn{5}{l}{\textbf{Association ambivalente}}\\
    \small{\textcite{guo_exploring_2023}}\index{Guo, Dongbo|pagebf} & \small{Préférence de ces pratiques intermodales par les femmes} & \small{VFF} & \small{Métro, Bus} & \small{Beijing (Chine)}\\
    \small{\textcite{cao_e-scooter_2021}}\index{Cao, Zhejing|pagebf} & \small{Préférence de ces pratiques intermodales par les femmes} & \small{TEFF} & \small{Métro} & \small{Singapour}\\
    \small{\textcite{ji_public_2017}}\index{Ji, Yanjie|pagebf} & \small{Préférence du vélo à usage personnel plutôt que le vélopartage par les femmes} & \small{Vélo, VLS} & \small{Métro} & \small{Nanjing (Chine)}\\
    \small{\textcite{oostendorp_combining_2018}}\index{Oostendorp, Rebekka|pagebf} & \small{Surreprésentation des femmes parmi les usager·ère·s intermoda·le·s, contrairement aux usager·ère·s monomoda·le·s} & \small{Vélo} & \small{Train, Métro, Tramway, Bus} & \small{Berlin (Allemagne)}\\
    \small{\textcite{fillone_i_2018}}\index{Fillone, Alexis|pagebf} & \small{62\% des usager·ère·s sont des femmes} & \small{Vélo} & \small{Tramway, Bus} & \small{Manille (Philippines)}\\
    \hline
\multicolumn{5}{l}{\textbf{Influence non vérifiée}}\\
    \small{\textcite{hasiak_access_2019}}\index{Hasiak, Sophie|pagebf} & \small{Facteur non significatif} & \small{Vélo} & \small{Train} & \small{Gares locales (France)}\\
    \small{\textcite{liu_understanding_2020}}\index{Liu, Yang|pagebf} & \small{Facteur non significatif} & \small{VLS} & \small{Métro} & \small{Nanjing (Chine)}\\
    \small{\textcite{liu_use_2020}}\index{Liu, Yang|pagebf} & \small{Facteur non significatif} & \small{VFF} & \small{Métro} & \small{Nanjing (Chine)}\\
    \small{\textcite{ni_exploring_2020}}\index{Ni, Ying|pagebf} & \small{Facteur non significatif, contrairement au taxi privilégié par les femmes} & \small{VFF} & \small{Métro} & \small{Beijing (Chine)}\\
        \hline
        \caption*{Corpus scientifique se rapportant aux effets de genre, dans le cadre de la \acrshort{RSL}}
        \label{Corpus scientifique se rapportant aux effets de genre, dans le cadre de la RSL}
        \begin{flushright}
        \scriptsize
    Auteur~: \textcopyright~Moinse 2023
        \end{flushright}
        \end{longtable}

    % Annexe D.23
    \newpage
\subsection{Corpus de la \acrshort{RSL} sur les résultats en lien avec l'usage différencié selon l'âge}
    \label{donnees-ouvertes:rsl_resultats_age}

    % Référence
Le présent tableau synthétise les résultats issus de la revue de littérature portant plus particulièrement sur la \hyperref[Caractéristiques socio-démographiques des usagers]{sous-partie consacrée aux caractéristiques socio-démographiques des usager·ère·s} (page \pageref{Caractéristiques socio-démographiques des usagers}).\par

    % Tableau résultats RSL (âge)
        \begin{longtable}{p{3cm}p{4cm}p{1.5cm}p{1.8cm}p{2.3cm}}
        \hline
        \textcolor{blue}{\textbf{Références}} & \textcolor{blue}{\textbf{Tranches d'âge}} & \textcolor{blue}{\textbf{MIL}} & \textcolor{blue}{\textbf{TC}} & \textcolor{blue}{\textbf{Contexte}}
        \hline
        \endhead
\multicolumn{5}{l}{\textbf{Profil social axé vers les jeunes populations}}\\
    \small{\textcite{oostendorp_combining_2018}}\index{Oostendorp, Rebekka|pagebf} & \small{Surreprésentation de jeunes usager·ère·s} & \small{Vélo} & \small{Train, Métro, Tramway, Bus} & \small{Berlin (Allemagne)}\\
    \small{\textcite{jonkeren_bicycle-train_2021}}\index{Jonkeren, Olaf|pagebf} & \small{Surreprésentation de jeunes usager·ère·s} & \small{Vélo} & \small{Train} & \small{Utrecht, Rotterdam, Eindhoven (Pays-Bas)}\\
    \small{\textcite{flamm_determinants_2013}}\index{Flamm, Bradley J.|pagebf} & \small{Surreprésentation de jeunes usager·ère·s} & \small{Vélo} & \small{Bus} & \small{Cleveland (États-Unis)}\\
    \small{\textcite{van_der_nat_bicycle_2018}}\index{van der Nat, Johanna Debóra|pagebf} & \small{Surreprésentation de jeunes usager·ère·s} & \small{VLS, VFF} & \small{Train} & \small{Amsterdam (Pays-Bas)}\\
    \small{\textcite{ma_connecting_2022}}\index{Ma, Qingyu|pagebf} & \small{Surreprésentation de jeunes usager·ère·s} & \small{VLS, TEFF} & \small{Métro} & \small{Washington D.C. (États-Unis)}\\
    \small{\textcite{fan_how_2019}}\index{Fan, Aihua|pagebf} & \small{Surreprésentation de jeunes usager·ère·s} & \small{VFF} & \small{Train, Métro, Tramway, Bus} & \small{Beijing (Chine)}\\
    \small{\textcite{guo_exploring_2023}}\index{Guo, Dongbo|pagebf} & \small{Surreprésentation de jeunes usager·ère·s} & \small{VFF} & \small{Métro, Bus} & \small{Beijing (Chine)}\\
    \small{\textcite{cheng_comparison_2023}}\index{Cheng, Long|pagebf} & \small{Surreprésentation de jeunes usager·ère·s} & \small{VLS, VFF} & \small{Métro} & \small{Nanjing (Chine)}\\
    \small{\textcite{yang_empirical_2016}}\index{Yang, Min|pagebf} & \small{Surreprésentation de jeunes usager·ère·s} & \small{VLS} & \small{Métro} & \small{Nanjing (Chine)}\\
    \small{\textcite{montes_shared_2023}}\index{Montes, Alejandro|pagebf} & \small{Perception plus positive de la mobilité partagée par les jeunes usager·ère·s} & \small{Vélo, VLS} & \small{Métro, Tramway, Bus} & \small{Rotterdam (Pays-Bas)}\\
    \small{\textcite{li_exploring_2017}}\index{Li, Wenxiang|pagebf} & \small{Perception plus positive de la mobilité partagée par les jeunes usager·ère·s} & \small{Vélo} & \small{Train, Bus} & \small{Austin (États-Unis)}\\
    \hline
\multicolumn{5}{l}{\textbf{Tranches d'âge représentées}}\\
    \small{\textcite{yang_bike-and-ride_2014}}\index{Yang, Liu|pagebf}\index{Yang, Liu|pagebf} & \small{Surreprésentation des individus de moins de 18 ans} & \small{Vélo, VLS} & \small{Train, Métro, Tramway, Bus} & \small{Xi'an (Chine)}\\
    \small{\textcite{fillone_i_2018}}\index{Fillone, Alexis|pagebf} & \small{Surreprésentation des individus entre 21 et 23 ans} & \small{Vélo} & \small{Tramway, Bus} & \small{Manille (Philippines)}\\
    \small{\textcite{cheng_evaluating_2012}}\index{Cheng, Yung-Hsiang|pagebf} & \small{Surreprésentation des individus entre 21 et 30 ans} & \small{Vélo} & \small{Métro} & \small{Kaohsiung (Taïwan)}\\
    \small{\textcite{quarshie_integrating_2007}}\index{Quarshie, Magnus|pagebf} & \small{Surreprésentation des individus de moins de 30 ans} & \small{Vélo} & \small{Bus} & \small{Accra (Ghana)}\\
    \small{\textcite{ravensbergen_biking_2018}}\index{Ravensbergen, Léa|pagebf} & \small{Intérêt décroissant des individus entre 35 à 64 ans} & \small{Vélo} & \small{Train} & \small{Toronto, Hamilton (Canada)}\\
    \small{\textcite{liu_understanding_2020}}\index{Liu, Yang|pagebf} & \small{Surreprésentation des individus entre 18 à 35 ans} & \small{VLS} & \small{Métro} & \small{Nanjing (Chine)}\\
    \small{\textcite{ma_measuring_2018}}\index{Ma, Xinwei|pagebf} & \small{Surreprésentation des individus entre 19 à 35 ans} & \small{VLS} & \small{Métro} & \small{Suzhou (Chine)}\\
    \small{\textcite{bachand-marleau_much-anticipated_2011}}\index{Bachand-Marleau, Julie|pagebf} & \small{Surreprésentation des individus entre 25 et 35 ans} & \small{VLS} & \small{Train, Métro, Tramway, Bus} & \small{Montréal (Canada)}\\
    \small{\textcite{bocker_bike_2020}}\index{Böcker, Lars|pagebf} & \small{Âge moyen de 30 ans} & \small{VLS} & \small{Métro} & \small{Oslo (Norvège)}\\
    \small{\textcite{guo_built_2020, guo_dockless_2021, guo_role_2021}} & \small{Surreprésentation des individus de moins de 30 ans} & \small{VFF} & \small{Métro} & \small{Shenzhen (Chine)}\\
    \small{\textcite{moinse_intermodal_2022}}\index{Moinse, Dylan|pagebf} & \small{Surreprésentation des individus entre 18 et 34 ans} & \small{TEP} & \small{Train} & \small{Provence-Alpes-Côte d'Azur (France)}\\
    \small{\textcite{kuijk_preferences_2022}}\index{van Kuijk, R.J.|pagebf} & \small{Surreprésentation des individus de moins de 26 ans et sous-représentation des individus de plus de 45 ans} & \small{VLS} & \small{Tramway, Bus} & \small{Utrecht (Pays-Bas)}\\
    \small{\textcite{cheng_expanding_2018}}\index{Cheng, Yung-Hsiang|pagebf} & \small{Faible perception des gains d'accessibilité générés par l'intégration par les individus entre 31 et 64 ans} & \small{VLS} & \small{Métro} & \small{Kaohsiung (Taïwan)}\\
    \hline
\multicolumn{5}{l}{\textbf{Un marché conquis par les adultes}}\\
    \small{\textcite{weliwitiya_bicycle_2019}}\index{Weliwitiya, Hesara|pagebf} & \small{Préférence croissante pour cette combinaison modale avec l'âge, pour un motif utilitaire} & \small{Vélo} & \small{Train} & \small{Melbourne (Australie)}\\
    \small{\textcite{martin_evaluating_2014}}\index{Martin, Elliot W.|pagebf} & \small{Préférence croissante pour cette combinaison modale avec l'âge} & \small{VLS} & \small{Métro, Tramway} & \small{Washington D.C., Minneapolis (États-Unis)}\\
    \small{\textcite{lin_built_2018}}\index{Lin, Jen-Jia|pagebf} & \small{Préférence croissante pour cette combinaison modale avec l'âge} & \small{VLS} & \small{Métro} & \small{Beijing (Chine), Taipei (Taïwan), Tokyo (Japon)}\\
    \small{\textcite{zhao_public_2022}}\index{Zhao, Pengjun|pagebf} & \small{Préférence croissante pour cette combinaison modale avec l'âge} & \small{VLS} & \small{Métro} & \small{Beijing (Chine)}\\
    \small{\textcite{sherwin_practices_2011}}\index{Sherwin, Henrietta|pagebf} & \small{Majoritairement des individus d'une trentaine d'années} & \small{Vélo} & \small{Train} & \small{Bristol (Angleterre)}\\
    \small{\textcite{rastogi_willingness_2010}}\index{Rastogi, Rajat|pagebf} & \small{Potentiel de report modal tiré par la tranche d'âge incluant les 23 à 45 ans} & \small{Vélo} & \small{Train} & \small{Mumbai (Inde)}\\
    \small{\textcite{zhao_bicycle-metro_2017}}\index{Zhao, Pengjun|pagebf} & \small{Les individus de moins de 30 ans sont plus enclins à favoriser l'automobile} & \small{Vélo, VLS} & \small{Métro} & \small{Beijing (Chine)}\\
    \small{\textcite{cao_e-scooter_2021}}\index{Cao, Zhejing|pagebf} & \small{Faible préférence de la TEFF par les individus ayant entre 23 et 34 ans} & \small{TEFF} & \small{Métro} & \small{Singapour}\\
    \small{\textcite{yang_metro_2015}}\index{Yang, Min|pagebf} & \small{Les usager·ère·s plus âgé·e·s ont tendance à être plus satisfait·e·s de leur voyage} & \small{Vélo, VAE, VLS} & \small{Métro} & \small{Nanjing (Chine)}\\
    \small{\textcite{arbis_analysis_2016}}\index{Arbis, David|pagebf} & \small{Occupation des casiers à vélo par les individus entre 40 et 59 ans principalement} & \small{Vélo} & \small{Train} & \small{Nouvelle-Galles du Sud (Australie)}\\
    \hline
\multicolumn{5}{l}{\textbf{Facteur non significatif}}\\
    \small{\textcite{wang_bicycle-transit_2013}}\index{Wang, Rui|pagebf} & \small{Répartition équilibrée des usager·ère·s de 19 à 65 ans} & \small{Vélo} & \small{Train, Bus} & \small{États-Unis}\\
    \small{\textcite{chan_factors_2020}}\index{Chan, Kevin|pagebf} & \small{Répartition équilibrée des usager·ère·s de 25 à 54 ans} & \small{Vélo} & \small{Train} & \small{Toronto, Hamilton (Canada)}\\
    \small{\textcite{kostrzewska_towards_2017}} & \small{Usager·ère·s toutes tranches d'âge confondues} & \small{Trottinette mécanique} & \small{Métro, Tramway, Bus} & \small{Berlin (Allemagne), Szczecin (Pologne)}\\
    \small{\textcite{bearn_adaption_2018}}\index{Bearn, Cary|pagebf} & \small{Amélioration de l'accessibilité de 273\% pour les populations entre 18 et 24 ans et de plus de 45 ans par l'aménagement d'un réseau cyclable à «~faible stress~»} & \small{Vélo} & \small{Métro} & \small{Atlanta (États-Unis)}\\
        \hline
        \caption*{Corpus scientifique se rapportant aux effets de l'âge, dans le cadre de la \acrshort{RSL}}
        \label{Corpus scientifique se rapportant aux effets de l'âge, dans le cadre de la RSL}
        \begin{flushright}
        \scriptsize
    Auteur~: \textcopyright~Moinse 2023
        \end{flushright}
        \end{longtable}

    % Annexe D.24
    \newpage
\subsection{Corpus de la \acrshort{RSL} sur les résultats en lien avec l'influence de la taille des ménages}
    \label{donnees-ouvertes:rsl_resultats_taille_menages}

    % Référence
Le présent tableau synthétise les résultats issus de la revue de littérature portant plus particulièrement sur la \hyperref[Caractéristiques socio-démographiques des usagers]{sous-partie consacrée aux caractéristiques socio-démographiques des usager·ère·s} (page \pageref{Caractéristiques socio-démographiques des usagers}).\par

    % Tableau résultats RSL (taille des ménages)
        \begin{longtable}{p{3cm}p{4cm}p{1.5cm}p{1.8cm}p{2.3cm}}
        \hline
        \textcolor{blue}{\textbf{Références}} & \textcolor{blue}{\textbf{Taille des ménages}} & \textcolor{blue}{\textbf{MIL}} & \textcolor{blue}{\textbf{TC}} & \textcolor{blue}{\textbf{Contexte}}
        \hline
        \endhead
    \small{\textcite{fillone_i_2018}}\index{Fillone, Alexis|pagebf} & \small{Surreprésentation des personnes célibataires parmi les usager·ère·s} & \small{Vélo} & \small{Tramway, Bus} & \small{Manille (Philippines)}\\
    \small{\textcite{krygsman_multimodal_2004}}\index{Krygsman, Stephan|pagebf} & \small{Sous-représentation des personnes avec de jeunes enfants parmi les usager·ère·s} & \small{Vélo} & \small{Train, Métro, Tramway, Bus} & \small{Utrecht (Pays-Bas)}\\
    \small{\textcite{bachand-marleau_much-anticipated_2011}}\index{Bachand-Marleau, Julie|pagebf} & \small{Surreprésentation des personnes vivant dans des ménages, d'un ou de deux individus et sans enfants, parmi les usager·ère·s} & \small{VLS} & \small{Train, Métro, Tramway, Bus} & \small{Montréal (Canada)}\\
    \small{\textcite{hu_examining_2022}}\index{Hu, Songhua|pagebf} & \small{Effets négatifs des territoires abritant une proportion élevée de jeunes de moins de seize ans} & \small{VFF} & \small{Métro} & \small{Shanghai (Chine)}\\
    \small{\textcite{oostendorp_combining_2018}}\index{Oostendorp, Rebekka|pagebf} & \small{Surreprésentation des personnes vivant dans des ménages familiaux parmi les usager·ère·s} & \small{Vélo} & \small{Train, Métro, Tramway, Bus} & \small{Berlin (Allemagne)}\\
    \small{\textcite{kuijk_preferences_2022}}\index{van Kuijk, R.J.|pagebf} & \small{Surreprésentation des personnes vivant dans des ménages avec des enfants parmi les usager·ère·s dans le centre urbain, mais pas en périphérie urbaine} & \small{Vélo} & \small{Tramway, Bus} & \small{Utrecht (Pays-Bas)}\\
        \hline
        \caption*{Corpus scientifique se rapportant à la taille des ménages, dans le cadre de la \acrshort{RSL}}
        \label{Corpus scientifique se rapportant à la taille des ménages, dans le cadre de la RSL}
        \begin{flushright}
        \scriptsize
    Auteur~: \textcopyright~Moinse 2023
        \end{flushright}
        \end{longtable}

    % Annexe D.25
    \newpage
\subsection{Corpus de la \acrshort{RSL} sur les résultats en lien avec l'influence des \acrshort{PCS}}
    \label{donnees-ouvertes:rsl_resultats_pcs}
    
    % Référence
Le présent tableau synthétise les résultats issus de la revue de littérature portant plus particulièrement sur la \hyperref[Caractéristiques socio-démographiques des usagers]{sous-partie consacrée aux caractéristiques socio-démographiques des usager·ère·s} (page \pageref{Caractéristiques socio-démographiques des usagers}).\par

    % Tableau résultats RSL (PCS)
        \begin{longtable}{p{3cm}p{4cm}p{1.5cm}p{1.8cm}p{2.3cm}}
        \hline
        \textcolor{blue}{\textbf{Références}} & \textcolor{blue}{\textbf{\acrshort{PCS}}} & \textcolor{blue}{\textbf{MIL}} & \textcolor{blue}{\textbf{TC}} & \textcolor{blue}{\textbf{Contexte}}
        \hline
        \endhead
    \small{\textcite{sherwin_practices_2011}}\index{Sherwin, Henrietta|pagebf} & \small{Surreprésentation des employé·e·s} & \small{Vélo} & \small{Train} & \small{Bristol (Angleterre)}\\
    \small{\textcite{halldorsdottir_home-end_2017}}\index{Halldórsdóttir, Katrín|pagebf} & \small{Surreprésentation des étudiant·e·s et des employé·e·s} & \small{Vélo} & \small{Train} & \small{Copenhague (Danemark)}\\
    \small{\textcite{jonkeren_bicycle-train_2021}}\index{Jonkeren, Olaf|pagebf} & \small{Surreprésentation des employé·e·s et des entrepreneur·se·s} & \small{Vélo} & \small{Train} & \small{Pays-Bas}\\
    \small{\textcite{quarshie_integrating_2007}}\index{Quarshie, Magnus|pagebf} & \small{Surreprésentation des étudiant·e·s et des artisan·ne·s qui détiennent une fréquence d'utilisation plus élevée} & \small{Vélo} & \small{Bus} & \small{Accra (Ghana)}\\
    \small{\textcite{cervero_bike-and-ride_2013}}\index{Cervero, Robert|pagebf} & \small{Surreprésentation des étudiant·e·s} & \small{Vélo} & \small{Métro} & \small{San Francisco (États-Unis)}\\
    \small{\textcite{lin_built_2018}}\index{Lin, Jen-Jia|pagebf} & \small{Surreprésentation des étudiant·e·s} & \small{VLS} & \small{Métro} & \small{Beijing (Chine), Taipei (Taïwan), Tokyo (Japon)}\\
    \small{\textcite{adnan_last-mile_2019}}\index{Adnan, Muhammad|pagebf} & \small{Surreprésentation des étudiant·e·s} & \small{VLS} & \small{Train, Métro, Tramway, Bus} & \small{Villes entre 30~000 et 200~000 habitants (Belgique)}\\
    \small{\textcite{hu_examining_2022}}\index{Hu, Songhua|pagebf} & \small{Surreprésentation des étudiant·e·s} & \small{VFF} & \small{Métro} & \small{Shanghai (Chine)}\\
    \small{\textcite{cheng_comparison_2023}}\index{Cheng, Long|pagebf} & \small{Concentration des flux autour des universités} & \small{VLS, VFF} & \small{Métro} & \small{Nanjing (Chine)}\\
    \small{\textcite{liu_use_2020}}\index{Liu, Yang|pagebf} & \small{Sous-représentation des étudiant·e·s} & \small{VFF} & \small{Métro} & \small{Nanjing (Chine)}\\
    \small{\textcite{fearnley_patterns_2020}}\index{Fearnley, Nils|pagebf} & \small{Surreprésentation des étudiant·e·s et sous-représentation des employé·e·s} & \small{TEFF} & \small{Train, Métro, Tramway, Bus} & \small{Oslo (Norvège)}\\
    \small{\textcite{pages_les_2021}}\index{Pages, Thibaud|pagebf} & \small{Surreprésentation des cadres et des professions intellectuelles} & \small{TEP} & \small{Train, Métro, Tramway, Bus} & \small{Marseille, Montpellier (France)}\\
        \hline
        \caption*{Corpus scientifique se rapportant à la répartition des \acrshort{PCS}, dans le cadre de la \acrshort{RSL}}
        \label{Corpus scientifique se rapportant à la répartition des PCS, dans le cadre de la RSL}
        \begin{flushright}
        \scriptsize
    Auteur~: \textcopyright~Moinse 2023
        \end{flushright}
        \end{longtable}

    % Annexe D.26
    \newpage
\subsection{Corpus de la \acrshort{RSL} sur les résultats en lien avec l'influence des revenus}
    \label{donnees-ouvertes:rsl_resultats_revenus}

    % Référence
Le présent tableau synthétise les résultats issus de la revue de littérature portant plus particulièrement sur la \hyperref[Caractéristiques socio-démographiques des usagers]{sous-partie consacrée aux caractéristiques socio-démographiques des usager·ère·s} (page \pageref{Caractéristiques socio-démographiques des usagers}).\par
  
    % Tableau résultats RSL (revenus)
        \begin{longtable}{p{3cm}p{4cm}p{1.5cm}p{1.8cm}p{2.3cm}}
        \hline
        \textcolor{blue}{\textbf{Références}} & \textcolor{blue}{\textbf{Revenus}} & \textcolor{blue}{\textbf{MIL}} & \textcolor{blue}{\textbf{TC}} & \textcolor{blue}{\textbf{Contexte}}
        \hline
        \endhead
\multicolumn{5}{l}{\textbf{Intégration plébiscitée par les ménages défavorisés}}\\
    \small{\textcite{advani_bicycle_2006}}\index{Advani, Mukti|pagebf} & \small{Surreprésentation des groupes sociaux à faible revenu} & \small{Vélo} & \small{Bus} & \small{New Delhi (Inde)}\\ 
    \small{\textcite{chen_demand_2013}}\index{Chen, Jingxu|pagebf} & \small{Surreprésentation des groupes sociaux à faible revenu, en raison d'un coût relativement bas} & \small{Vélo} & \small{Métro} & \small{Nanjing (Chine)}\\
    \small{\textcite{luan_better_2020}}\index{Luan, Xin|pagebf} & \small{Surreprésentation des groupes sociaux à faible revenu, les ménages privilégiés favorisant l'usage de l'automobile} & \small{Vélo} & \small{Métro} & \small{Nanjing (Chine)}\\  
    \small{\textcite{bechstein_cycling_2010}}\index{Bechstein, Eva|pagebf} & \small{Surreprésentation des groupes sociaux à faible revenu} & \small{Vélo} & \small{Train} & \small{Mamelodi, Nellmapius (Afrique du Sud)}\\
    \small{\textcite{rastogi_willingness_2010}}\index{Rastogi, Rajat|pagebf} & \small{Surreprésentation des groupes sociaux à faible revenu résidant dans les territoires périurbains} & \small{Vélo} & \small{Train} & \small{Mumbai (Inde)}\\
    \small{\textcite{meng_influence_2016}}\index{Meng, Meng|pagebf} & \small{Surreprésentation des groupes sociaux disposant de moins de 2~000\$ par mois} & \small{Vélo} & \small{Métro} & \small{Singapour}\\ 
    \small{\textcite{cervero_bike-and-ride_2013}}\index{Cervero, Robert|pagebf} & \small{Utilisation intensifiée dans les quartiers défavorisés} & \small{Vélo} & \small{Métro} & \small{San Francisco (États-Unis)}\\ 
    \small{\textcite{rastogi_travel_2003}}\index{Rastogi, Rajat|pagebf} & \small{Surreprésentation des groupes sociaux à faible revenu qui parcourent de plus longues distances} & \small{Vélo} & \small{Train} & \small{Mumbai (Inde)}\\
    \small{\textcite{balya_integration_2016}}\index{Balya, Manjurali|pagebf} & \small{Préférence à l'égard des aménagements cyclables dédiés par les ménages disposant de revenus plus faibles} & \small{Vélo} & \small{Bus} & \small{Ahmedabad (Inde)}\\ 
    \small{\textcite{zhao_public_2022}}\index{Zhao, Pengjun|pagebf} & \small{Surreprésentation des groupes sociaux à faible et moyen revenu} & \small{VLS} & \small{Métro} & \small{Beijing (Chine)}\\ 
    \small{\textcite{guo_exploring_2023}}\index{Guo, Dongbo|pagebf} & \small{Surreprésentation des groupes sociaux à faible revenu} & \small{VFF} & \small{Métro, Bus} & \small{Beijing (Chine)}\\
    \small{\textcite{ji_exploring_2018}}\index{Ji, Yanjie|pagebf} & \small{Utilisation moindre dans les quartiers favorisés} & \small{VLS} & \small{Métro} & \small{Nanjing (Chine)}\\ 
    \small{\textcite{mohammadian_analyzing_2022}}\index{Mohammadian, Abolfazl|pagebf} & \small{Surreprésentation des groupes sociaux à faible revenu} & \small{TEFF} & \small{Train, Métro, Tramway, Bus} & \small{Chicago (États-Unis)}\\ 
    \small{\textcite{kuijk_preferences_2022}}\index{van Kuijk, R.J.|pagebf} & \small{Surreprésentation des groupes sociaux disposant de moins de 2~000\€ par mois} & \small{VLS} & \small{Tramway, Bus} & \small{Utrecht (Pays-Bas)}\\
    \small{\textcite{ma_understanding_2018}}\index{Ma, Xinwei|pagebf} & \small{Surreprésentation des personnes migrant·e·s résidant à moins de deux kilomètres des stations périurbaines} & \small{VLS} & \small{Métro} & \small{Nanjing (Chine)}\\
    \hline
\multicolumn{5}{l}{\textbf{Intégration plébiscitée par les ménages favorisés}}\\
    \small{\textcite{shelat_analysing_2018}}\index{Shelat, Sanmay|pagebf} & \small{Surreprésentation des groupes sociaux à revenu élevé} & \small{Vélo} & \small{Train} & \small{Pays-Bas}\\
    \small{\textcite{chan_factors_2020}}\index{Chan, Kevin|pagebf} & \small{Surreprésentation des groupes sociaux à revenu élevé} & \small{Vélo} & \small{Train} & \small{Toronto, Hamilton (Canada)}\\
    \small{\textcite{ravensbergen_biking_2018}}\index{Ravensbergen, Léa|pagebf} & \small{Surreprésentation des groupes sociaux disposant de plus de 150~000\€ par an} & \small{Vélo} & \small{Train} & \small{Toronto, Hamilton (Canada)}\\
    \small{\textcite{yang_bike-and-ride_2014}}\index{Yang, Liu|pagebf}\index{Yang, Liu|pagebf} & \small{Sous-représentation des groupes sociaux disposant de moins de 40~000 RMB par mois} & \small{Vélo, VLS} & \small{Train, Métro, Tramway, Bus} & \small{Xi'an (Chine)}\\
    \small{\textcite{kampen_bicycle_2021}}\index{van Kampen, Jullian|pagebf} & \small{Les groupes sociaux à revenu élevé sont plus susceptibles d'occuper les stationnements vélo} & \small{Vélo} & \small{Métro} & \small{Amsterdam (Pays-Bas)}\\
    \small{\textcite{martin_evaluating_2014}}\index{Martin, Elliot W.|pagebf} & \small{Surreprésentation des groupes sociaux à revenu élevé} & \small{VLS} & \small{Métro, Tramway} & \small{Washington D.C., Minneapolis (États-Unis)}\\  
    \small{\textcite{zhao_bicycle-metro_2017}}\index{Zhao, Pengjun|pagebf} & \small{Surreprésentation des groupes sociaux à revenu élevé} & \small{Vélo, VLS} & \small{Métro} & \small{Beijing (Chine)}\\ 
    \small{\textcite{yang_empirical_2016}}\index{Yang, Min|pagebf} & \small{Surreprésentation des groupes sociaux à revenu supérieur à la moyenne} & \small{VLS} & \small{Métro} & \small{Nanjing (Chine)}\\  
    \small{\textcite{lin_built_2018}}\index{Lin, Jen-Jia|pagebf} & \small{Surreprésentation des groupes sociaux à revenu élevé} & \small{VLS} & \small{Métro} & \small{Beijing (Chine), Taipei (Taïwan), Tokyo (Japon)}\\ 
    \small{\textcite{bocker_bike_2020}}\index{Böcker, Lars|pagebf} & \small{Utilisation moindre dans les quartiers résidentiels à faible revenu et dans les centres d'emploi industriels et logistiques} & \small{VLS} & \small{Métro} & \small{Oslo (Norvège)}\\
    \small{\textcite{cao_e-scooter_2021}}\index{Cao, Zhejing|pagebf} & \small{Surreprésentation des groupes sociaux à revenu élevé} & \small{TEFF} & \small{Métro} & \small{Singapour}\\ 
    \small{\textcite{cheng_expanding_2018}}\index{Cheng, Yung-Hsiang|pagebf} & \small{Faible perception des avantages comparés de l'intégration modale par les groupes sociaux disposant de 15~000 à 20~000 NTD par mois} & \small{} & \small{VLS} & \small{Métro} & \small{Kaohsiung (Taïwan)}\\
    \small{\textcite{ma_bicycle_2015}}\index{Ma, Ting|pagebf} & \small{Une augmentation de 10\% du salaire médian des ménages amène à une fréquentation accrue de 0,4\%} & \small{VLS} & \small{Train} & \small{Washington D.C. (États-Unis)}\\
    \hline
\multicolumn{5}{l}{\textbf{Facteur non significatif}}\\
    \small{\textcite{zuo_first-and-last_2020}}\index{Zuo, Ting|pagebf} & \small{Pas de différenciation sociale dans l'usage} & \small{Vélo} & \small{Bus} & \small{Hamilton (États-Unis)}\\
    \small{\textcite{cheng_evaluating_2012}}\index{Cheng, Yung-Hsiang|pagebf} & \small{Pas de différenciation sociale dans l'usage} & \small{Vélo} & \small{Métro} & \small{Kaohsiung (Taïwan)}\\
    \small{\textcite{zuo_incorporating_2021}}\index{Zuo, Ting|pagebf} & \small{Gains d'accessibilité en faveur des groupes sociaux à faible revenu et résidant dans le centre ou en périphérie urbaine} & \small{Vélo} & \small{Bus} & \small{Hamilton (États-Unis)}\\
        \hline
        \caption*{Corpus scientifique se rapportant à l'influence des revenus disponibles, dans le cadre de la \acrshort{RSL}}
        \label{Corpus scientifique se rapportant à l'influence des revenus disponibles, dans le cadre de la RSL}
        \begin{flushright}
        \scriptsize
    Auteur~: \textcopyright~Moinse 2023
        \end{flushright}
        \end{longtable}

    % Annexe D.27
    \newpage
\subsection{Corpus de la \acrshort{RSL} sur les résultats en lien avec l'influence du niveau d'éducation}
    \label{donnees-ouvertes:rsl_resultats_education}

    % Référence
Le présent tableau synthétise les résultats issus de la revue de littérature portant plus particulièrement sur la \hyperref[Caractéristiques socio-démographiques des usagers]{sous-partie consacrée aux caractéristiques socio-démographiques des usager·ère·s} (page \pageref{Caractéristiques socio-démographiques des usagers}).

    % Tableau résultats RSL (diplôme)
        \begin{longtable}{p{3cm}p{4cm}p{1.5cm}p{1.8cm}p{2.3cm}}
        \hline
        \textcolor{blue}{\textbf{Références}} & \textcolor{blue}{\textbf{Diplômes}} & \textcolor{blue}{\textbf{MIL}} & \textcolor{blue}{\textbf{TC}} & \textcolor{blue}{\textbf{Contexte}}
        \hline
        \endhead
    \small{\textcite{flamm_public_2014}}\index{Flamm, Bradley J.|pagebf} & \small{Surreprésentation des individus diplômés en études supérieures} & \small{Vélo} & \small{Train, Métro, Tramway, Bus} & \small{Philadelphie, San Francisco (États-Unis)}\\
    \small{\textcite{jonkeren_bicycle-train_2021}}\index{Jonkeren, Olaf|pagebf} & \small{Surreprésentation des individus diplômés en études supérieures} & \small{Vélo} & \small{Train} & \small{Utrecht, Rotterdam, Eindhoven (Pays-Bas)}\\
    \small{\textcite{heinen_multimodal_2014}}\index{Heinen, Eva|pagebf} & \small{Surreprésentation des individus diplômés en études supérieures} & \small{Vélo} & \small{Train, Métro, Tramway, Bus} & \small{Delft, Zwolle, Midden-Delfland, Pijnacker-Nootdorp (Pays-Bas)}\\
    \small{\textcite{shelat_analysing_2018}}\index{Shelat, Sanmay|pagebf} & \small{85\% des usager·ère·s sont titulaires d'un diplôme d'études secondaires} & \small{Vélo} & \small{Train} & \small{Pays-Bas}\\
    \small{\textcite{arbis_analysis_2016}}\index{Arbis, David|pagebf} & \small{Influence sur le taux d'occupation des casiers à vélo sécurisés} & \small{Vélo} & \small{Train} & \small{Nouvelle-Galles du Sud (Australie)}\\
    \small{\textcite{kampen_bicycle_2021}}\index{van Kampen, Jullian|pagebf} & \small{Influence sur le taux d'occupation des lieux de stationnement vélo autour des gares} & \small{Vélo} & \small{Métro} & \small{Amsterdam (Pays-Bas)}\\
    \small{\textcite{van_der_nat_bicycle_2018}}\index{van der Nat, Johanna Debóra|pagebf} & \small{Surreprésentation des individus diplômés en études supérieures} & \small{VLS, VFF} & \small{Train} & \small{Amsterdam (Pays-Bas)}\\
    \small{\textcite{yang_empirical_2016}}\index{Yang, Min|pagebf} & \small{Surreprésentation des individus diplômés en études supérieures} & \small{VLS} & \small{Métro} & \small{Nanjing (Chine)}\\
    \small{\textcite{guo_exploring_2023}}\index{Guo, Dongbo|pagebf} & \small{Surreprésentation des individus diplômés en études supérieures} & \small{VFF} & \small{Métro, Bus} & \small{Beijing (Chine)}\\
    \small{\textcite{zhao_public_2022}}\index{Zhao, Pengjun|pagebf} & \small{Préférence pour des itinéraires plus directs par les personnes diplômées} & \small{VLS} & \small{Métro} & \small{Beijing (Chine)}\\
    \small{\textcite{liu_use_2020}}\index{Liu, Yang|pagebf} & \small{Sous-représentation des individus diplômés} & \small{VFF} & \small{Métro} & \small{Nanjing (Chine)}\\
    \small{\textcite{yan_evaluating_2023}}\index{Yan, Xiang|pagebf} & \small{Sous-représentation des individus titulaires d'un diplôme en études supérieures} & \small{TEFF} & \small{Métro, Bus} & \small{Washington D.C., Los Angeles (États-Unis)}\\
    \small{\textcite{adnan_last-mile_2019}}\index{Adnan, Muhammad|pagebf} & \small{Facteur non significatif} & \small{VLS} & \small{Train, Métro, Tramway, Bus} & \small{Villes entre 30~000 et 200~000 habitants (Belgique)}\\
        \hline
        \caption*{Corpus scientifique se rapportant à l'influence du niveau d'éducation, dans le cadre de la \acrshort{RSL}}
        \label{Corpus scientifique se rapportant à l'influence du niveau d'éducation, dans le cadre de la RSL}
        \begin{flushright}
        \scriptsize
    Auteur~: \textcopyright~Moinse 2023
        \end{flushright}
        \end{longtable}

    % Annexe D.28
    \newpage
\subsection{Corpus de la \acrshort{RSL} sur les résultats en lien avec le taux de motorisation des ménages}
    \label{donnees-ouvertes:rsl_resultats_motorisation}

    % Référence
Le présent tableau synthétise les résultats issus de la revue de littérature portant plus particulièrement sur la \hyperref[Caractéristiques socio-démographiques des usagers]{sous-partie consacrée aux caractéristiques socio-démographiques des usager·ère·s} (page \pageref{Caractéristiques socio-démographiques des usagers}).\par
    
    % Tableau résultats RSL (motorisation)
        \begin{longtable}{p{3cm}p{4cm}p{1.5cm}p{1.8cm}p{2.3cm}}
        \hline
        \textcolor{blue}{\textbf{Références}} & \textcolor{blue}{\textbf{Motorisation}} & \textcolor{blue}{\textbf{MIL}} & \textcolor{blue}{\textbf{TC}} & \textcolor{blue}{\textbf{Contexte}}
        \hline
        \endhead
\multicolumn{5}{l}{\textbf{Effets positifs du permis de conduire}}\\
    \small{\textcite{hamidi_shaping_2020}}\index{Hamidi, Zahra|pagebf} & \small{Association positive} & \small{Vélo} & \small{Train, Métro, Tramway, Bus} & \small{Göteborg, Malmö (Suède), Beijing (Chine)}\\
    \small{\textcite{lin_built_2018}}\index{Lin, Jen-Jia|pagebf} & \small{Association positive} & \small{VLS} & \small{Métro} & \small{Beijing (Chine), Taipei (Taïwan), Tokyo (Japon)}\\
    \small{\textcite{bachand-marleau_much-anticipated_2011}}\index{Bachand-Marleau, Julie|pagebf} & \small{94\% et 87\% des utilisateur·rice·s ont respectivement un vélo et un permis de conduire valide} & \small{VLS} & \small{Train, Métro, Tramway, Bus} & \small{Montréal (Canada)}\\
    \small{\textcite{liu_use_2020}}\index{Liu, Yang|pagebf} & \small{Facteur non significatif} & \small{VFF} & \small{Métro} & \small{Nanjing (Chine)}\\
    \hline
\multicolumn{5}{l}{\textbf{Effets positifs de la motorisation des ménages}}\\
    \small{\textcite{givoni_access_2007}}\index{Givoni, Moshe|pagebf} & \small{La moitié des usager·ère·s possèdent une voiture} & \small{Vélo} & \small{Train} & \small{Pays-Bas}\\
    \small{\textcite{martens_bicycle_2004}}\index{Martens, Karel|pagebf} & \small{Association positive} & \small{Vélo} & \small{Train, Métro, Tramway, Bus} & \small{Pays-Bas, Allemagne, Grande-Bretagne}\\
    \small{\textcite{adnan_last-mile_2019}}\index{Adnan, Muhammad|pagebf} & \small{Association positive} & \small{VLS} & \small{Train, Métro, Tramway, Bus} & \small{Villes entre 30~000 et 200~000 habitants (Belgique)}\\
    \small{\textcite{yang_bike-and-ride_2014}}\index{Yang, Liu|pagebf}\index{Yang, Liu|pagebf} & \small{Association positive} & \small{Vélo, VLS} & \small{Train, Métro, Tramway, Bus} & \small{Xi'an (Chine)}\\
    \small{\textcite{yang_empirical_2016}}\index{Yang, Min|pagebf} & \small{Association positive} & \small{VLS} & \small{Métro} & \small{Nanjing (Chine)}\\
    \small{\textcite{cheng_expanding_2018}}\index{Cheng, Yung-Hsiang|pagebf} & \small{Association positive exclusivement pour une voiture par ménage} & \small{VLS} & \small{Métro} & \small{Kaohsiung (Taïwan)}\\
    \small{\textcite{wang_bicycle-transit_2013}}\index{Wang, Rui|pagebf} & \small{Facteur non significatif} & \small{Vélo} & \small{Train, Bus} & \small{États-Unis}\\
    \hline
\multicolumn{5}{l}{\textbf{Effets négatifs de la motorisation des ménages}}\\
    \small{\textcite{debrezion_modelling_2009}}\index{Debrezion, Ghebreegziabiher|pagebf} & \small{Association négative avec le taux de motorisation} & \small{Vélo} & \small{Train} & \small{Pays-Bas}\\
    \small{\textcite{heinen_multimodal_2014}}\index{Heinen, Eva|pagebf} & \small{Association négative avec le taux de motorisation} & \small{Vélo} & \small{Train, Métro, Tramway, Bus} & \small{Delft, Zwolle, Midden-Delfland, Pijnacker-Nootdorp (Pays-Bas)}\\
    \small{\textcite{de_souza_modelling_2017}}\index{Souza, Flavia de|pagebf} & \small{Association négative avec le taux de motorisation} & \small{Vélo} & \small{Train, Métro, Tramway, Bus} & \small{Rio de Janeiro (Brésil)}\\
    \small{\textcite{park_finding_2014}}\index{Park, Sungjin|pagebf} & \small{Association négative avec le taux de motorisation} & \small{Vélo} & \small{Train} & \small{Mountain View (États-Unis)}\\
    \small{\textcite{li_exploring_2017}}\index{Li, Wenxiang|pagebf} & \small{Association négative avec le taux de motorisation} & \small{Vélo} & \small{Train, Bus} & \small{Austin (États-Unis)}\\
    \small{\textcite{chan_factors_2020}}\index{Chan, Kevin|pagebf} & \small{Association négative avec le taux de motorisation} & \small{Vélo} & \small{Train} & \small{Toronto, Hamilton (Canada)}\\
    \small{\textcite{mohammadian_analyzing_2022}}\index{Mohammadian, Abolfazl|pagebf} & \small{Association négative avec le taux de motorisation} & \small{TEFF} & \small{Train, Métro, Tramway, Bus} & \small{Chicago (États-Unis)}\\ 
    \small{\textcite{basu_planning_2021}}\index{Basu, Rounaq|pagebf} & \small{Mobilité résidentielle des ménages non motorisés en faveur des territoires équipés de stations de VLS} & \small{VLS} & \small{Train, Métro, Tramway, Bus} & \small{Boston (États-Unis)}\\
        \hline
        \caption*{Corpus scientifique se rapportant à l'influence de la motorisation des ménages, dans le cadre de la \acrshort{RSL}}
        \label{Corpus scientifique se rapportant à l'influence de la motorisation des ménages, dans le cadre de la RSL}
        \begin{flushright}
        \scriptsize
    Auteur~: \textcopyright~Moinse 2023
        \end{flushright}
        \end{longtable}

    % Annexe D.29
    \newpage
\subsection{Corpus de la \acrshort{RSL} sur les résultats en lien avec les motifs de déplacement}
    \label{donnees-ouvertes:rsl_resultats_motifs_deplacement}

    % Référence
Le présent tableau synthétise les résultats issus de la revue de littérature portant plus particulièrement sur la \hyperref[Comportements de mobilité]{sous-partie consacrée aux comportements de mobilité} (page \pageref{Comportements de mobilité}).\par

    % Tableau résultats RSL (motifs de déplacement)
        \begin{longtable}{p{3cm}p{4cm}p{1.5cm}p{1.8cm}p{2.3cm}}
        \hline
        \textcolor{blue}{\textbf{Références}} & \textcolor{blue}{\textbf{Motifs}} & \textcolor{blue}{\textbf{MIL}} & \textcolor{blue}{\textbf{TC}} & \textcolor{blue}{\textbf{Contexte}}
        \hline
        \endhead
\multicolumn{5}{l}{\textbf{Prédominance des déplacements pendulaires impliquant l'usage du vélo et de la TEP}}\\
    \small{\textcite{shelat_analysing_2018}}\index{Shelat, Sanmay|pagebf} & \small{Majorité de déplacements pendulaires, principalement pendant les heures de pointe en matinée} & \small{Vélo} & \small{Train} & \small{Pays-Bas}\\
    \small{\textcite{la_paix_puello_Train_2016}}\index{La Paix Puello, Lissy|pagebf} & \small{Majorité de déplacements pendulaires} & \small{Vélo} & \small{Train} & \small{La Haye, Rotterdam (Pays-Bas)}\\
    \small{\textcite{jonkeren_bicycle_2021}}\index{Jonkeren, Olaf|pagebf} & \small{Majorité de déplacements pendulaires} & \small{Vélo} & \small{Train} & \small{Pays-Bas}\\
    \small{\textcite{oostendorp_combining_2018}}\index{Oostendorp, Rebekka|pagebf} & \small{Majorité de déplacements pendulaires} & \small{Vélo} & \small{Train, Métro, Tramway, Bus} & \small{Berlin (Allemagne)}\\
    \small{\textcite{papon_rapport_2015}}\index{Papon, Francis|pagebf} & \small{Combinaison modale attractive en heures de pointe} & \small{Vélo} & \small{Train} & \small{Amboise (France)}\\
    \small{\textcite{flamm_public_2014}}\index{Flamm, Bradley J.|pagebf} & \small{Majorité de déplacements pendulaires, dont la fréquence s'élève à quatre à cinq déplacements hebdomadaires, et complétés par des motifs secondaires liés aux achats et aux loisirs} & \small{Vélo} & \small{Train, Métro, Tramway, Bus} & \small{Philadelphie, San Francisco (États-Unis)}\\
    \small{\textcite{kampen_bicycle_2021}}\index{van Kampen, Jullian|pagebf} & \small{Saturation des lieux de stationnement dédiés au vélo en heures de pointe} & \small{Vélo} & \small{Métro} & \small{Amsterdam (Pays-Bas)}\\
    \small{\textcite{lee_strategies_2010}}\index{Lee, Jaeyeong|pagebf} & \small{Majorité de déplacements pendulaires} & \small{Vélo} & \small{Métro} & \small{Séoul, Daejeon (Corée du Sud)}\\
    \small{\textcite{martens_bicycle_2004}}\index{Martens, Karel|pagebf} & \small{Parts des déplacements pendulaires de 40\% à 66\% et de 21\% à 49\% respectivement pour le train, le métro et le bus} & \small{Vélo} & \small{Train, Métro, Tramway, Bus} & \small{Pays-Bas, Allemagne, Grande-Bretagne}\\
    \small{\textcite{kuijk_preferences_2022}}\index{van Kuijk, R.J.|pagebf} & \small{Préférence moindre pour le VLS, le VFF et la TEFF au profit du vélo pour les déplacements domicile-travail} & \small{Vélo} & \small{Tramway, Bus} & \small{Utrecht (Pays-Bas)}\\
    \small{\textcite{moinse_intermodal_2022}}\index{Moinse, Dylan|pagebf} & \small{55\% et 26\% des déplacements intermodaux en TEP sont réalisés pour se rendre aux lieux de travail et d'études} & \small{TEP} & \small{Train} & \small{Provence-Alpes-Côte d'Azur (France)}\\
    \small{\textcite{rabaud_quand_2022}}\index{Rabaud, Mathieu|pagebf} & \small{24\% et 30\% des déplacements intermodaux à vélo, en VLS et en TEP sont réalisés pour se rendre aux lieux de travail et d'études} & \small{VLS, TEP} & \small{Train, Métro, Tramway, Bus} & \small{France}\\
    \hline
\multicolumn{5}{l}{\textbf{Prédominance des déplacements pendulaires impliquant l'usage du vélo et de la \gls{micro-mobilité} partagés}}\\
    \small{\textcite{bocker_bike_2020}}\index{Böcker, Lars|pagebf} & \small{Majorité de déplacements pendulaires} & \small{VLS} & \small{Métro} & \small{Oslo (Norvège)}\\
    \small{\textcite{ma_understanding_2018}}\index{Ma, Xinwei|pagebf} & \small{Majorité de déplacements pendulaires} & \small{VLS} & \small{Métro} & \small{Nanjing (Chine)}\\
    \small{\textcite{ma_measuring_2018}}\index{Ma, Xinwei|pagebf} & \small{Majorité de déplacements pendulaires} & \small{VLS} & \small{Métro} & \small{Suzhou (Chine)}\\
    \small{\textcite{gu_measuring_2019}}\index{Gu, Tianqi|pagebf} & \small{Majorité de déplacements pendulaires} & \small{VLS} & \small{Métro} & \small{Suzhou (Chine)}\\
    \small{\textcite{yu_policy_2021}}\index{Yu, Qing|pagebf} & \small{Majorité de déplacements pendulaires} & \small{VLS} & \small{Métro} & \small{Shanghai (Chine)}\\
    \small{\textcite{kim_analysis_2021}}\index{Kim, Minjun|pagebf} & \small{Majorité de déplacements pendulaires} & \small{VLS} & \small{Métro, Bus} & \small{Séoul (Corée du Sud)}\\
    \small{\textcite{bi_analysis_2021}}\index{Bi, Hui|pagebf} & \small{Majorité de déplacements pendulaires autour des stations intermédiaires} & \small{VLS} & \small{Métro} & \small{Chengdu (Chine)}\\
    \small{\textcite{liu_understanding_2020}}\index{Liu, Yang|pagebf} & \small{Majorité de déplacements pendulaires, principalement pendant les heures de pointe en matinée} & \small{VLS} & \small{Métro} & \small{Nanjing (Chine)}\\
    \small{\textcite{fan_dockless_2020}}\index{Fan, Yichun|pagebf} & \small{Majorité de déplacements pendulaires} & \small{VFF} & \small{Métro} & \small{Beijing (Chine)}\\
    \small{\textcite{wang_spatiotemporal_2020}}\index{Wang, Zijia|pagebf} & \small{Majorité de déplacements pendulaires} & \small{VFF} & \small{Métro} & \small{Beijing (Chine)}\\
    \small{\textcite{yu_understanding_2021}}\index{Yu, Senbin|pagebf} & \small{Majorité de déplacements pendulaires} & \small{VFF} & \small{Métro} & \small{Beijing (Chine)}\\
    \small{\textcite{lin_analysis_2019}}\index{Lin, Diao|pagebf} & \small{Majorité de déplacements pendulaires} & \small{VFF} & \small{Métro} & \small{Shanghai (Chine)}\\
    \small{\textcite{qiu_interplay_2021}}\index{Qiu, Waishan|pagebf} & \small{Majorité de déplacements pendulaires} & \small{VFF} & \small{Bus} & \small{Ithaca (États-Unis)}\\
    \small{\textcite{yang_spatiotemporal_2019}}\index{Yang, Yuanxuan|pagebf} & \small{Majorité de déplacements pendulaires, principalement pendant les heures de pointe en matinée} & \small{VFF} & \small{Métro} & \small{Nanjing (Chine)}\\
    \small{\textcite{liu_concordance_2022}}\index{Liu, Siyang|pagebf} & \small{Majorité de déplacements pendulaires, principalement pendant les heures de pointe en matinée} & \small{VFF} & \small{Métro} & \small{Beijing (Chine)}\\
    \small{\textcite{heumann_spatiotemporal_2021}}\index{Heumann, Maximilian|pagebf} & \small{Majorité de déplacements pendulaires} & \small{TEFF} & \small{Train, Métro, Tramway} & \small{Berlin (Allemagne)}\\
    \small{\textcite{wang_relationship_2020}}\index{Wang, Ruoyu|pagebf} & \small{Majorité de déplacements pendulaires, mais un usage non négligeable le week-end} & \small{VFF} & \small{Métro} & \small{Shenzhen (Chine)}\\
    \small{\textcite{wu_identification_2023}}\index{Wu, Hao|pagebf}\index{Wu, Hao|pagebf} & \small{Majorité de déplacements pendulaires, mais un usage non négligeable le week-end} & \small{VFF} & \small{Métro} & \small{Shenzhen (Chine)}\\
    \small{\textcite{li_unbalanced_2022}}\index{Li, Lili|pagebf} & \small{Majorité de déplacements pendulaires, mais un usage non négligeable le week-end} & \small{VFF} & \small{Métro} & \small{Beijing (Chine)}\\
    \small{\textcite{chen_what_2022}}\index{Chen, Wendong|pagebf} & \small{Majorité de déplacements pendulaires pour le VLS et davantage liés aux loisirs pour le VFF} & \small{VLS, VFF} & \small{Métro} & \small{Nanjing (Chine)}\\
    \hline
\multicolumn{5}{l}{\textbf{Mixité des motifs de déplacement}}\\
    \small{\textcite{jonkeren_bicycle_2021}}\index{Jonkeren, Olaf|pagebf} & \small{Chaînage de déplacements au cours du voyage pendulaire, lié à des activités annexes comme les achats ou les opérations bancaires} & \small{Vélo} & \small{Train} & \small{Pays-Bas}\\
    \small{\textcite{kong_deciphering_2020}}\index{Kong, Hui|pagebf} & \small{Distinction entre les abonné·e·s du VLS à des fins professionnelles et les usager·ère·s occasionnel·le·s pour des activités récréatives le week-end} & \small{VLS} & \small{Train, Métro, Tramway, Bus} & \small{Boston, Chicago, Washington D.C., New York City (États-Unis)}\\
    \small{\textcite{tarpin-pitre_typology_2020}}\index{Tarpin-Pitre, Léandre|pagebf} & \small{Identification de six groupes d'utilisateur·rice·s en fonction des motifs de déplacement} & \small{VLS} & \small{Métro} & \small{Montréal (Canada)}\\
    \small{\textcite{chen_determinants_2012}}\index{Chen, Lijun|pagebf} & \small{Majorité de déplacements non utilitaires liés aux achats et aux visites amicales} & \small{Vélo} & \small{Métro} & \small{Nanjing (Chine)}\\
    \small{\textcite{glass_role_2020}}\index{Glass, Caroline|pagebf} & \small{44\% de déplacements non utilitaires liés aux loisirs} & \small{VLS} & \small{Bus} & \small{Birmingham (États-Unis)}\\
    \small{\textcite{radzimski_exploring_2021}}\index{Radzimski, Adam|pagebf} & \small{Majorité de déplacements non utilitaires liés aux loisirs} & \small{VLS} & \small{Tramway, Bus} & \small{Poznań (Pologne)}\\
    \small{\textcite{ma_connecting_2022}}\index{Ma, Qingyu|pagebf} & \small{Mixité de déplacements liés aux loisirs et ayant un motif scolaire} & \small{TEFF} & \small{Métro} & \small{Washington D.C. (États-Unis)}\\
    \small{\textcite{mcqueen_assessing_2022}}\index{McQueen, Michael|pagebf} & \small{Les déplacements pendulaires remplacent le bus, alors que les déplacements associés aux loisirs complètent le tramway} & \small{TEFF} & \small{Tramway} & \small{Portland (États-Unis)}\\
        \hline
        \caption*{Corpus scientifique se rapportant aux motifs de déplacement, dans le cadre de la \acrshort{RSL}}
        \label{Corpus scientifique se rapportant aux motifs de déplacement, dans le cadre de la RSL}
        \begin{flushright}
        \scriptsize
    Auteur~: \textcopyright~Moinse 2023
        \end{flushright}
        \end{longtable}

    % Annexe D.30
    \newpage
\subsection{Corpus de la \acrshort{RSL} sur les résultats en lien avec l'impact des conditions météorologiques et des variations saisonnières}
    \label{donnees-ouvertes:rsl_resultats_meteo_saisons}

    % Référence
Le présent tableau synthétise les résultats issus de la revue de littérature portant plus particulièrement sur la \hyperref[Comportements de mobilité]{sous-partie consacrée aux comportements de mobilité} (page \pageref{Comportements de mobilité}).\par

    % Tableau résultats RSL (genre)
        \begin{longtable}{p{3cm}p{4cm}p{1.5cm}p{1.8cm}p{2.3cm}}
        \hline
        \textcolor{blue}{\textbf{Références}} & \textcolor{blue}{\textbf{Climat}} & \textcolor{blue}{\textbf{MIL}} & \textcolor{blue}{\textbf{TC}} & \textcolor{blue}{\textbf{Contexte}}
        \hline
        \endhead
    \small{\textcite{flamm_determinants_2013}}\index{Flamm, Bradley J.|pagebf} & \small{Influence des conditions météorologiques et des variations saisonnières} & \small{Vélo} & \small{Bus} & \small{Cleveland (États-Unis)}\\
    \small{\textcite{ma_understanding_2018}}\index{Ma, Xinwei|pagebf} & \small{Influence négative des précipitations} & \small{VLS} & \small{Métro} & \small{Nanjing (Chine)}\\
    \small{\textcite{adnan_last-mile_2019}}\index{Adnan, Muhammad|pagebf} & \small{Influence négative des précipitations} & \small{VLS} & \small{Train, Métro, Tramway, Bus} & \small{Villes entre 30~000 et 200~000 habitants (Belgique)}\\
    \small{\textcite{bi_analysis_2021}}\index{Bi, Hui|pagebf} & \small{Réduction de 90\% de l'usage lors de précipitations et de chaleur intenses} & \small{VLS} & \small{Métro} & \small{Chengdu (Chine)}\\
    \small{\textcite{zuniga-garcia_evaluation_2022}}\index{Zuniga-Garcia, Natalia|pagebf} & \small{Influence négative des précipitations et des températures} & \small{TEFF} & \small{Bus} & \small{Austin (États-Unis)}\\
    \small{\textcite{bachand-marleau_much-anticipated_2011}}\index{Bachand-Marleau, Julie|pagebf} & \small{Influence des variations saisonnières} & \small{VLS} & \small{Train, Métro, Tramway, Bus} & \small{Montréal (Canada)}\\
    \small{\textcite{ma_estimating_2019}}\index{Ma, Ting|pagebf} & \small{Augmentation de l'usage du printemps jusqu'au mois d'août puis diminution jusqu'à atteindre le seuil le plus bas en décembre} & \small{VLS} & \small{Métro} & \small{Washington D.C. (États-Unis)}\\
        \hline
        \caption*{Corpus scientifique se rapportant aux conditions météorologiques et aux variations saisonnières, dans le cadre de la \acrshort{RSL}}
        \label{Corpus scientifique se rapportant aux conditions météorologiques et aux variations saisonnières, dans le cadre de la RSL}
        \begin{flushright}
        \scriptsize
    Auteur~: \textcopyright~Moinse 2023
        \end{flushright}
        \end{longtable}

    % Annexe D.31
    \newpage
\subsection{Corpus de la \acrshort{RSL} sur les résultats en lien avec l'expérience et les représentations sociales}
    \label{donnees-ouvertes:rsl_resultats_experience_mobilite}
    
    % Référence
Le présent tableau synthétise les résultats issus de la revue de littérature portant plus particulièrement sur la \hyperref[Comportements de mobilité]{sous-partie consacrée aux comportements de mobilité} (page \pageref{Comportements de mobilité}).\par

    % Tableau résultats RSL (expérience et représentations sociales)
        \begin{longtable}{p{3cm}p{4cm}p{1.5cm}p{1.8cm}p{2.3cm}}
        \hline
        \textcolor{blue}{\textbf{Références}} & \textcolor{blue}{\textbf{Expérience}} & \textcolor{blue}{\textbf{MIL}} & \textcolor{blue}{\textbf{TC}} & \textcolor{blue}{\textbf{Contexte}}
        \hline
        \endhead
\multicolumn{5}{l}{\textbf{Influence positive de l'expérience antérieure de mobilité}}\\
    \small{\textcite{hamidi_shaping_2020}}\index{Hamidi, Zahra|pagebf} & \small{Influence positive du capital lié à la «~motilité~»} & \small{Vélo} & \small{Train, Métro, Tramway, Bus} & \small{Göteborg, Malmö (Suède), Beijing (Chine)}\\
    \small{\textcite{hamidi_inequalities_2019}}\index{Hamidi, Zahra|pagebf} & \small{Présence de barrières immatérielles liées aux compétences et à l'accès au vélo} & \small{Vélo} & \small{Train, Bus} & \small{Malmö (Suède)}\\
    \small{\textcite{montes_shared_2023}}\index{Montes, Alejandro|pagebf} & \small{Influence positive de l'expérience avec le VLS} & \small{Vélo, VLS} & \small{Métro, Tramway, Bus} & \small{Rotterdam (Pays-Bas)}\\
    \small{\textcite{cheng_expanding_2018}}\index{Cheng, Yung-Hsiang|pagebf} & \small{Influence positive de l'expérience avec le VLS} & \small{VLS} & \small{Métro} & \small{Kaohsiung (Taïwan)}\\
    \small{\textcite{baek_electric_2021}}\index{Baek, Kwangho|pagebf} & \small{Influence positive de l'expérience avec la TEFF} & \small{TEFF} & \small{Métro} & \small{Séoul (Corée du Sud)}\\
    \small{\textcite{ji_public_2017}}\index{Ji, Yanjie|pagebf} & \small{Influence positive de l'expérience liée au vol de vélo en faveur du VLS} & \small{Vélo, VLS} & \small{Métro} & \small{Nanjing (Chine)}\\
    \small{\textcite{liu_use_2020}}\index{Liu, Yang|pagebf} & \small{Influence positive de l'expérience avec le métro} & \small{VFF} & \small{Métro} & \small{Nanjing (Chine)}\\
    \small{\textcite{la_paix_puello_role_2021}}\index{La Paix Puello, Lissy|pagebf} & \small{Influence positive de l'expérience avec le train} & \small{Vélo} & \small{Train} & \small{La Haye, Rotterdam (Pays-Bas)}\\
    \small{\textcite{singleton_exploring_2014}}\index{Singleton, Patrick A.|pagebf} & \small{Influence positive des ménages utilisant régulièrement le métro} & \small{Vélo} & \small{Métro} & \small{Portland (États-Unis)}\\
    \small{\textcite{taylor_analysis_1996}}\index{Taylor, Dean|pagebf} & \small{Facteur non significatif} & \small{Vélo} & \small{Bus} & \small{Texas (États-Unis)}\\
    \hline
\multicolumn{5}{l}{\textbf{Influence des représentations sociales}}\\
    \small{\textcite{zhao_public_2022}}\index{Zhao, Pengjun|pagebf} & \small{Rôle de l'efficacité perçue du VLS en intermodalité} & \small{VLS} & \small{Métro} & \small{Beijing (Chine)}\\
    \small{\textcite{guo_exploring_2023}}\index{Guo, Dongbo|pagebf} & \small{Rôle de la perception du statut et de la satifaction procurée par l'usage du VFF} & \small{VFF} & \small{Métro, Bus} & \small{Beijing (Chine)}\\
    \small{\textcite{bopp_examining_2015}}\index{Bopp, Melissa|pagebf} & \small{Rôle de la perception de l'auto-efficacité et de la maîtrise du véhicule, plus importante chez les usager·ère·s des transports en commun} & \small{Vélo} & \small{Train, Métro, Tramway} & \small{Delaware, New Jersey, Maryland, Virginie-Occidentale, Pennsylvanie, Ohio (États-Unis)}\\
    \small{\textcite{kostrzewska_towards_2017}} & \small{Rôle de la perception de l'espace urbain et du plaisir ressenti lors du déplacement} & \small{Trottinette mécanique} & \small{Métro, Tramway, Bus} & \small{Berlin (Allemagne), Szczecin (Pologne)}\\
    \hline
\multicolumn{5}{l}{\textbf{Influence de la sensibilité environnementale}}\\
    \small{\textcite{cheng_evaluating_2012}}\index{Cheng, Yung-Hsiang|pagebf} & \small{Perception moindre des inconvénients perçus de l'intégration par les individus ayant une conscience environnementale plus développée} & \small{Vélo} & \small{Métro} & \small{Kaohsiung (Taïwan)}\\
    \small{\textcite{yang_empirical_2016}}\index{Yang, Min|pagebf} & \small{La majorité des utilisateur·rice·s montre un intérêt pour les initiatives environnementales} & \small{VLS} & \small{Métro} & \small{Nanjing (Chine)}\\
    \small{\textcite{zhao_bicycle-metro_2017}}\index{Zhao, Pengjun|pagebf} & \small{Absence de lien entre la conscience environnementale et l'utilisation intermodale du vélo, au contraire des raisons économiques} & \small{Vélo, VLS} & \small{Métro} & \small{Beijing (Chine)}\\
        \hline
        \caption*{Corpus scientifique se rapportant à l'expérience et aux représentations sociales, dans le cadre de la \acrshort{RSL}}
        \label{Corpus scientifique se rapportant à l'expérience et aux représentations sociales, dans le cadre de la RSL}
        \begin{flushright}
        \scriptsize
    Auteur~: \textcopyright~Moinse 2023
        \end{flushright}
        \end{longtable}

    % Annexe D.32
    \newpage
\subsection{Corpus de la \acrshort{RSL} sur les résultats en lien avec le potentiel de report modal}
    \label{donnees-ouvertes:rsl_resultats_potentiel_report_modal}

    % Référence
Le présent tableau synthétise les résultats issus de la revue de littérature portant plus particulièrement sur la \hyperref[Comportements de mobilité]{sous-partie consacrée aux comportements de mobilité} (page \pageref{Comportements de mobilité}).\par

    % Tableau résultats RSL (potentiel de report modal)
        \begin{longtable}{p{3cm}p{4cm}p{1.5cm}p{1.8cm}p{2.3cm}}
        \hline
        \textcolor{blue}{\textbf{Références}} & \textcolor{blue}{\textbf{Report modal}} & \textcolor{blue}{\textbf{MIL}} & \textcolor{blue}{\textbf{TC}} & \textcolor{blue}{\textbf{Contexte}}
        \hline
        \endhead
    \small{\textcite{wang_bicycle-transit_2013}}\index{Wang, Rui|pagebf} & \small{La part modale du vélo avec le train et le bus a augmenté de manière significative dans les régions à forte densité} & \small{Vélo} & \small{Train, Bus} & \small{États-Unis}\\
    \small{\textcite{lee_bicycle-based_2016}}\index{Lee, Jaeyeong|pagebf} & \small{Potentiel de report modal depuis la marche (40,6\%) et depuis le bus (23,5\%) vers le vélo en intermodalité} & \small{Vélo} & \small{Métro} & \small{Séoul, Daejeon (Corée du Sud)}\\
    \small{\textcite{bachand-marleau_much-anticipated_2011}}\index{Bachand-Marleau, Julie|pagebf} & \small{Potentiel de captation de 63\% des participant·e·s} & \small{VLS} & \small{Train, Métro, Tramway, Bus} & \small{Montréal (Canada)}\\
    \small{\textcite{fan_how_2019}}\index{Fan, Aihua|pagebf} & \small{Potentiel de captation de 58\% des participant·e·s} & \small{VFF} & \small{Train, Métro, Tramway, Bus} & \small{Beijing (Chine)}\\
        \hline
        \caption*{Corpus scientifique se rapportant au potentiel de report modal, dans le cadre de la \acrshort{RSL}}
        \label{Corpus scientifique se rapportant au potentiel de report modal, dans le cadre de la RSL}
        \begin{flushright}
        \scriptsize
    Auteur~: \textcopyright~Moinse 2023
        \end{flushright}
        \end{longtable}

    % Annexe D.33
    \newpage
\subsection{Corpus de la \acrshort{RSL} sur les résultats en lien avec les impacts sur les systèmes de mobilité}
    \label{donnees-ouvertes:rsl_resultats_potentiel_impacts_mobilite}

    % Référence
Le présent tableau synthétise les résultats issus de la revue de littérature portant plus particulièrement sur la \hyperref[Impacts sur les systèmes urbains et la mobilité]{sous-partie consacrée aux impacts sur les systèmes urbains et sur la mobilité} (page \pageref{Impacts sur les systèmes urbains et la mobilité}).\par
    
    % Tableau résultats RSL (impacts mobilité)
        \begin{longtable}{p{3cm}p{4cm}p{1.5cm}p{1.8cm}p{2.3cm}}
        \hline
        \textcolor{blue}{\textbf{Références}} & \textcolor{blue}{\textbf{Impacts mobilité}} & \textcolor{blue}{\textbf{MIL}} & \textcolor{blue}{\textbf{TC}} & \textcolor{blue}{\textbf{Contexte}}
        \hline
        \endhead
\multicolumn{5}{l}{\textbf{Amélioration de la fréquentation des transports en commun}}\\
    \small{\textcite{ashraf_impacts_2021}}\index{Ashraf, Md Tanvir|pagebf} & \small{Une augmentation de 10\% du nombre de trajets en VLS augmente de 2,3\% l'utilisation du métro} & \small{VLS} & \small{Métro} & \small{New York City (États-Unis)}\\
    \small{\textcite{ma_bicycle_2015}}\index{Ma, Ting|pagebf} & \small{Une augmentation de 10\% du nombre de trajets en VLS augmente de 2,8\% l'utilisation du métro, et de 4,9\% en heures de pointe matinales} & \small{VLS} & \small{Métro} & \small{Washington D.C. (États-Unis)}\\
    \small{\textcite{fan_dockless_2020}}\index{Fan, Yichun|pagebf} & \small{Augmentation de 5,54\% de l'utilisation du métro et allant jusqu'à 8\%} & \small{VFF} & \small{Métro} & \small{Beijing (Chine)}\\
    \small{\textcite{yang_spatiotemporal_2019}}\index{Yang, Yuanxuan|pagebf} & \small{Augmentation de 28\% de l'utilisation du métro} & \small{VFF} & \small{Métro} & \small{Nanjing (Chine)}\\
    \small{\textcite{jappinen_modelling_2013}}\index{Jäppinen, Sakari|pagebf} & \small{Économies de distance temps de 5 à 7,5 minutes à destination du centre-ville} & \small{VLS} & \small{Train, Métro, Tramway, Bus, Ferry} & \small{Helsinki (Finlande)}\\
    \small{\textcite{zuo_promote_2020}}\index{Zuo, Ting|pagebf} & \small{Une augmentation de 10\% de la couverture des services de bus, grâce au vélo, augmente de 5,9\% leur utilisation} & \small{Vélo} & \small{Bus} & \small{Hamilton (États-Unis)}\\
    \small{\textcite{bearn_adaption_2018}}\index{Bearn, Cary|pagebf} & \small{Amélioration de la desserte de 116\% de la population grâce au vélo} & \small{Vélo} & \small{Métro} & \small{Atlanta (États-Unis)}\\
    \hline
\multicolumn{5}{l}{\textbf{Diminution de la possession et de l'usage de l'automobile}}\\
    \small{\textcite{papon_evaluation_2017}}\index{Papon, Francis|pagebf} & \small{Réduction de la dépendance automobile} & \small{Vélo} & \small{Train} & \small{Amboise (France)}\\
    \small{\textcite{balya_integration_2016}}\index{Balya, Manjurali|pagebf} & \small{Réduction de l'utilisation de l'automobile, de la congestion urbaine et de la pollution atmosphérique} & \small{Vélo} & \small{Bus} & \small{Ahmedabad (Inde)}\\
    \small{\textcite{bopp_examining_2015}}\index{Bopp, Melissa|pagebf} & \small{Réduction de l'utilisation de l'automobile et de la pollution atmosphérique} & \small{Vélo} & \small{Train, Métro, Tramway} & \small{Delaware, New Jersey, Maryland, Virginie-Occidentale, Pennsylvanie, Ohio (États-Unis)}\\
    \small{\textcite{glass_role_2020}}\index{Glass, Caroline|pagebf} & \small{Réduction de l'utilisation de l'automobile, notamment en soirée} & \small{VLS} & \small{Bus} & \small{Birmingham (États-Unis)}\\
    \small{\textcite{martin_evaluating_2014}}\index{Martin, Elliot W.|pagebf} & \small{Réduction de l'utilisation de l'automobile personnelle et partagée et une augmentation respective de l'intégration de 83\% et de 72\%} & \small{Washington D.C., Minneapolis (États-Unis)}\\
    \small{\textcite{bachand-marleau_much-anticipated_2011}}\index{Bachand-Marleau, Julie|pagebf} & \small{25\% des voyageur·se·s étaient des automobilistes} & \small{VLS} & \small{Train, Métro, Tramway, Bus} & \small{Montréal (Canada)}\\
    \small{\textcite{fan_how_2019}}\index{Fan, Aihua|pagebf} & \small{5,5\% des voyageur·se·s étaient des automobilistes} & \small{VFF} & \small{Train, Métro, Tramway, Bus} & \small{Beijing (Chine)}\\
    \small{\textcite{yan_evaluating_2023}}\index{Yan, Xiang|pagebf} & \small{Réduction de l'utilisation de l'automobile} & \small{TEFF} & \small{Métro, Bus} & \small{Washington D.C., Los Angeles (États-Unis)}\\
    \small{\textcite{basu_planning_2021}}\index{Basu, Rounaq|pagebf} & \small{Réduction de la propriété de 2,2\% et de l'usage de l'automobile de 3,3\% ainsi que de la pollution atmosphérique de 2,9\%, contre respectivement 9,8\%, 10\% et 10\% autour des stations de transport en commun} & \small{VLS} & \small{Train, Métro, Tramway, Bus} & \small{Boston (États-Unis)}\\
    \hline
\multicolumn{5}{l}{\textbf{Diminution de l'usage de la marche et du bus en \gls{rabattement} et en \gls{diffusion}}}\\
    \small{\textcite{lee_forecasting_2021}}\index{Lee, Mina|pagebf} & \small{Remplacement de 32\% des voyages en covoiturage et de 13\% des voyages à vélo.} & \small{TEFF} & \small{Métro} & \small{New York City (États-Unis)}\\
    \small{\textcite{song_investigating_2020}}\index{Song, Ying|pagebf} & \small{Effet de substitution à double tranchant} & \small{VLS} & \small{Métro} & \small{Minneapolis, Saint-Paul (États-Unis)}\\
    \small{\textcite{yen_how_2023}}\index{Yen, Barbara T.H.|pagebf} & \small{Préférence pour le VLS plutôt que pour la marche combinée, en dépit d'une distance similaire} & \small{VLS} & \small{Métro} & \small{Taipei (Taïwan)}\\
    \small{\textcite{rastogi_willingness_2010}}\index{Rastogi, Rajat|pagebf} & \small{Report modal depuis les voyageur·se·s en bus principalement} & \small{Vélo} & \small{Train} & \small{Mumbai (Inde)}\\
    \small{\textcite{li_investigating_2022}}\index{Li, Xiaofeng|pagebf} & \small{Diminution de la fréquentation du BHNS} & \small{VLS} & \small{Tramway, Bus} & \small{Tucson (États-Unis)}\\
    \small{\textcite{ma_impacts_2019}}\index{Ma, Xiaolei|pagebf} & \small{Une augmentation d'une unité de VFF entraîne une augmentation de 4,23 voyages en bus en semaine, mais une baisse de 0,56 voyage le week-end} & \small{VFF} & \small{Bus} & \small{Chengdu (Chine)}\\
    \small{\textcite{ziedan_complement_2021}}\index{Ziedan, Abubakr|pagebf} & \small{Absence d'impact significatif sur la fréquentation du bus} & \small{TEFF} & \small{Bus} & \small{Nashville (États-Unis)}\\
    \hline
\multicolumn{5}{l}{\textbf{Amélioration de la résilience du système de transport en commun}}\\
    \small{\textcite{flamm_changes_2014}}\index{Flamm, Bradley J.|pagebf} & \small{Augmentation de l'intégration de 9\% après la réduction du service de bus} & \small{Vélo} & \small{Train, Métro, Tramway, Bus} & \small{Nord-ouest de l'Ohio (États-Unis)}\\
    \small{\textcite{yan_spatiotemporal_2021}}\index{Yan Xiang|pagebf} & \small{Remplacement de trajets courts en transport en commun} & \small{TEFF} & \small{Métro, Bus} & \small{Washington D.C. (États-Unis)}\\
    \small{\textcite{cao_e-scooter_2021}}\index{Cao, Zhejing|pagebf} & \small{Remplacement de trajets courts en transport en commun, jugés inconfortables} & \small{TEFF} & \small{Métro} & \small{Singapour}\\
    \small{\textcite{jin_competition_2019}}\index{Jin, Haitao|pagebf} & \small{Remplacement de trajets en transport en commun de moins de 2 kilomètres} & \small{VFF} & \small{Métro, Bus} & \small{Beijing (Chine)}\\
    \small{\textcite{singleton_exploring_2014}}\index{Singleton, Patrick A.|pagebf} & \small{Réduction de la pression de la demande de mobilité en heures de pointe, rendant le service plus fiable} & \small{Vélo} & \small{Métro} & \small{Portland (États-Unis)}\\
    \small{\textcite{van_der_nat_bicycle_2018}}\index{van der Nat, Johanna Debóra|pagebf} & \small{Allègement de la demande de stationnement de vélos de 22 à 25\% en ville et de 37 à 50\% autour des gares} & \small{VLS, VFF} & \small{Train} & \small{Amsterdam (Pays-Bas)}\\
        \hline
        \caption*{Corpus scientifique se rapportant aux impacts sur les systèmes de mobilité, dans le cadre de la \acrshort{RSL}}
        \label{Corpus scientifique se rapportant aux impacts sur les systèmes de mobilité, dans le cadre de la RSL}
        \begin{flushright}
        \scriptsize
    Auteur~: \textcopyright~Moinse 2023
        \end{flushright}
        \end{longtable}

    % Annexe D.34
    \newpage
\subsection{Corpus de la \acrshort{RSL} sur les résultats en lien avec les impacts sur les systèmes urbains}
    \label{donnees-ouvertes:rsl_resultats_potentiel_impacts_urbanisme}
    
    % Référence
Le présent tableau synthétise les résultats issus de la revue de littérature portant plus particulièrement sur la \hyperref[Impacts sur les systèmes urbains et la mobilité]{sous-partie consacrée aux impacts sur les systèmes urbains et sur la mobilité} (page \pageref{Impacts sur les systèmes urbains et la mobilité}).\par
    
    % Tableau résultats RSL (impacts urbains)
        \begin{longtable}{p{3cm}p{4cm}p{1.5cm}p{1.8cm}p{2.3cm}}
        \hline
        \textcolor{blue}{\textbf{Références}} & \textcolor{blue}{\textbf{Impacts urbanisme}} & \textcolor{blue}{\textbf{MIL}} & \textcolor{blue}{\textbf{TC}} & \textcolor{blue}{\textbf{Contexte}}
        \hline
        \endhead
\multicolumn{5}{l}{\textbf{Effets sur les territoires}}\\
    \small{\textcite{fan_dockless_2020}}\index{Fan, Yichun|pagebf} & \small{Réduction de la congestion urbaine de 4\%} & \small{VFF} & \small{Métro} & \small{Beijing (Chine)}\\
    \small{\textcite{lu_improving_2018}}\index{Lu, Miaojia|pagebf} & \small{Réduction du risque de mortalité de 11\% à vélo et de 10\% à pied, ainsi que la prévention de 22 décès prématurés dans le scénario de gratuité du VLS} & \small{VLS} & \small{Métro, Bus} & \small{Taipei (Taïwan)}\\
    \small{\textcite{basu_planning_2021}}\index{Basu, Rounaq|pagebf} & \small{Phénomène de mobilité résidentielle des ménages non motorisés vers les territoires équipés de service de VLS} & \small{VLS} & \small{Train, Métro, Tramway, Bus} & \small{Boston (États-Unis)}\\
    \small{\textcite{yang_spatiotemporal_2019}}\index{Yang, Yuanxuan|pagebf} & \small{Transformation polycentrique du système de mobilité et de l'agglomération avec l'émergence de «~nouveaux hubs de mobilité~»} & \small{VFF} & \small{Métro} & \small{Nanjing (Chine)}\\
    \small{\textcite{yu_policy_2021}}\index{Yu, Qing|pagebf} & \small{Conflits d'usage liés au stationnement des flottes de vélo et de trottinette dans les espaces publics} & \small{VLS} & \small{Métro} & \small{Shanghai (Chine)}\\
    \hline
\multicolumn{5}{l}{\textbf{Effets sur les dynamiques économiques}}\\
    \small{\textcite{li_exploring_2017}}\index{Li, Wenxiang|pagebf} & \small{Effets synergiques sur les valeurs immobilières, la croissance économique, la santé publique et l'équité sociale} & \small{Vélo} & \small{Train, Bus} & \small{Austin (États-Unis)}\\
    \small{\textcite{welch_long-term_2016}}\index{Welch, Timothy F.|pagebf} & \small{Effets positifs de la présence d'arrêts de tramway et de pistes cyclables sur le marché immobilier} & \small{Vélo} & \small{Tramway} & \small{Portland (États-Unis)}\\
    \small{\textcite{li_exploring_2017}}\index{Li, Wenxiang|pagebf} & \small{Effets positifs des scores liés à la qualité des transports en commun et de cyclabilité sur la valeur foncière des propriétés} & \small{Vélo} & \small{Train, Bus} & \small{Austin (États-Unis)}\
    \small{\textcite{chu_last_2021}}\index{Chu, Junhong|pagebf} & \small{Meilleure répartition des prix immobiliers entre les logements situés à moins d'un kilomètre et ceux situés au-delà de ce rayon} & \small{VFF} & \small{Métro} & \small{Beijing, Chengdu, Chongqing, Dalian, Hangzhou, Nanjing, Shanghai, Shenzhen, Tianjin, Wuhan (Chine)}\\
    \small{\textcite{jonkeren_bicycle-train_2021}}\index{Jonkeren, Olaf|pagebf} & \small{Réalisation d'achats quotidiens par les cyclistes dans les commerces localisés à proximité des itinéraires empruntés} & \small{Vélo} & \small{Train} & \small{Utrecht, Rotterdam, Eindhoven (Pays-Bas)}\\
\multicolumn{5}{l}{\textbf{Effets sur l'environnement}}\\
    \small{\textcite{chen_study_2013}}\index{Chen, Wan|pagebf} & \small{Valorisation du vélo comme un mode de déplacement sobre} & \small{Vélo} & \small{Métro} & \small{Xi'an (Chine)}\\
    \small{\textcite{kostrzewska_towards_2017}} & \small{La micro-mobilité combinée aux transports en commun en adéquation avec la «~mobilité durable~»} & \small{Trottinette mécanique} & \small{Métro, Tramway, Bus} & \small{Berlin (Allemagne), Szczecin (Pologne)}\\
    \small{\textcite{cho_estimation_2022}}\index{Cho, Shin-Hyung|pagebf} & \small{Le VLS combiné aux transports en commun en adéquation avec la «~mobilité durable~»} & \small{VLS} & \small{Métro} & \small{Séoul (Corée du Sud)}\\
    \small{\textcite{basu_planning_2021}}\index{Basu, Rounaq|pagebf} & \small{Réduction trimestrielle des émissions de GES de 2,9\% grâce au VLS, allant jusqu'à 10\% autour des stations de transport en commun} & \small{VLS} & \small{Train, Métro, Tramway, Bus} & \small{Boston (États-Unis)}\\
    \small{\textcite{lu_improving_2018}}\index{Lu, Miaojia|pagebf} & \small{Scénario ayant le plus de bénéfices sur l'environnement lié à la mise en gratuité du service de VLS} & \small{VLS} & \small{Métro, Bus} & \small{Taipei (Taïwan)}\\
    \small{\textcite{papon_evaluation_2017}}\index{Papon, Francis|pagebf} & \small{Bénéfices significatifs en termes de santé publique, en réduisant l'accidentologie} & \small{Vélo} & \small{Train} & \small{Amboise (France)}\\
    \small{\textcite{bopp_examining_2015}}\index{Bopp, Melissa|pagebf} & \small{Bénéfices significatifs en termes de santé publique, en améliorant la forme physique} & \small{Vélo} & \small{Train, Métro, Tramway} & \small{Delaware, New Jersey, Maryland, Virginie-Occidentale, Pennsylvanie, Ohio (États-Unis)}\\
    \small{\textcite{adnan_last-mile_2019}}\index{Adnan, Muhammad|pagebf} & \small{Bénéfices significatifs en termes de santé publique, en améliorant la forme physique} & \small{VLS} & \small{Train, Métro, Tramway, Bus} & \small{Villes entre 30~000 et 200~000 habitants (Belgique)}\\
        \hline
        \caption*{Corpus scientifique se rapportant aux impacts sur les systèmes urbains, dans le cadre de la \acrshort{RSL}}
        \label{Corpus scientifique se rapportant aux impacts sur les systèmes urbains, dans le cadre de la RSL}
        \begin{flushright}
        \scriptsize
    Auteur~: \textcopyright~Moinse 2023
        \end{flushright}
        \end{longtable}

    % ___________________________________________
    % ANNEXE G~: Observation quantitative
    \newpage
\section{Résultats statistiques de l'observation quantitative}
    \label{donnees-ouvertes:resultats_observation_quantitative}
    \markboth{Annexes liées à la mise en œuvre et aux résultats de l'observation quantitative}{}
    \markright{Annexes liées à la mise en œuvre et aux résultats de l'observation quantitative}{}

    \newpage
    % Annexe G.4
\subsection{Données de l'observation quantitative (toutes gares confondues)}
    \label{donnees-ouvertes:resultats_observation_quantitative_toutes_gares}

    % Explications variables
    % Exclut du calcul les 6 personnes dont le genre est non-identifié
    
% ___________________________________________
% Tableau G4
        \begin{longtable}{p{3.7cm}p{0.9cm}p{0.9cm}p{0.9cm}p{0.9cm}p{0.9cm}p{0.9cm}p{0.9cm}p{0.9cm}}
         \textcolor{blue}{\textbf{Variables}} & \textcolor{blue}{\textbf{Moy.}} & \textcolor{blue}{\textbf{VSV}} & \textcolor{blue}{\textbf{MIL}} & \textcolor{blue}{\textbf{VC}} & \textcolor{blue}{\textbf{VP}} & \textcolor{blue}{\textbf{TEP}} & \textcolor{blue}{\textbf{TM}} & \textcolor{blue}{\textbf{A}}\\
        \hline
        \endhead
\multicolumn{9}{l}{\textbf{Toutes variables confondues}}\\
    \small{\textbf{Effectif total}} & \small{\textbf{15~435}} & \small{\textbf{14~400}} & \small{\textbf{1~035}} & \small{\textbf{329}} & \small{\textbf{123}} & \small{\textbf{460}} & \small{\textbf{104}} & \small{\textbf{19}}\\
    \small{Part totale} & \small{\textbf{100,00\%}} & \small{\textbf{93,29\%}} & \small{\textbf{6,71\%}} & \small{\textbf{2,13\%}} & \small{\textbf{0,80\%}} & \small{\textbf{2,98\%}} & \small{\textbf{0,67\%}} & \small{\textbf{0,12\%}}\\
    \hline
\multicolumn{9}{l}{\textbf{Genre observé (part en fonction du genre)}}\\
    \small{Usagères (effectif)} & \small{7~519} & \small{7~227} & \small{292} & \small{78} & \small{43} & \small{115} & \small{51} & \small{5}\\
    \small{Usagères (part)} & \small{48,71\%} & \small{50,19\%} & \small{28,21\%} & \small{23,71\%} & \small{34,96\%} & \small{25,00\%} & \small{49,04\%} & \small{26,32\%}\\
    \small{Usagers (effectif)} & \small{7~916} & \small{7~173} & \small{743} & \small{251} & \small{80} & \small{345} & \small{53} & \small{14}\\
    \small{Usagers (part)} & \small{51,29\%} & \small{49,81\%} & \small{71,79\%} & \small{76,29\%} & \small{65,04\%} & \small{75,00\%} & \small{50,96\%} & \small{73,68}\\
    \hline
\multicolumn{9}{l}{\textbf{Catégories d'âge observées (part en fonction du mode)}}\\
    \small{Enfants (effectif)} & \small{60} & \small{60} & \small{0} & \small{0} & \small{0} & \small{0} & \small{0} & \small{0}\\
    \small{Enfants (part)} & \small{0,39\%} & \small{0,42\%} & \small{0,00\%} & \small{0,00\%} & \small{0,00\%} & \small{0,00\%} & \small{0,00\%} & \small{0,00\%}\\
    \small{Jeunes adultes (effectif)} & \small{4~641} & \small{4~399} & \small{242} & \small{71} & \small{7} & \small{136} & \small{20} & \small{8}\\
    \small{Jeunes adultes (part)} & \small{30,07\%} & \small{30,55\%} & \small{23,38\%} & \small{21,58\%} & \small{5,69\%} & \small{29,57\%} & \small{19,23\%} & \small{42,11\%}\\
    \small{Adultes (effectif)} & \small{9~772} & \small{9~039} & \small{733} & \small{229} & \small{104} & \small{311} & \small{78} & \small{11}\\
    \small{Adultes (part)} & \small{63,31\%} & \small{62,77\%} & \small{70,82\%} & \small{69,60\%} & \small{84,55\%} & \small{67,61\%} & \small{75,23\%} & \small{57,89\%}\\
    \small{Personnes âgées (effectif)} & \small{962} & \small{902} & \small{60} & \small{29} & \small{12} & \small{13} & \small{6} & \small{0}\\
    \small{Personnes âgées (part)} & \small{6,23\%} & \small{6,26\%} & \small{5,80\%} & \small{8,81\%} & \small{9,76\%} & \small{2,83\%} & \small{5,77,23\%} & \small{0,00\%}\\
    \hline
\multicolumn{9}{l}{\textbf{Tranches d'âge croisées avec le genre observé (part en fonction du mode)}}\\
    \small{Usagères – Enfants} & \small{48,33\%} & \small{48,33\%} & \small{0,00\%} & \small{0,00\%} & \small{0,00\%} & \small{0,00\%} & \small{0,00\%} & \small{0,00\%}\\
    \small{Usagers – Enfants} & \small{51,67\%} & \small{51,67\%} & \small{0,00\%} & \small{0,00\%} & \small{0,00\%} & \small{0,00\%} & \small{0,00\%} & \small{0,00\%}\\
    \small{Usagères – Jeunes adultes} & \small{50,44\%} & \small{51,88\%} & \small{19,60\%} & \small{14,08\%} & \small{57,14\%} & \small{27,21\%} & \small{35,00\%} & \small{12,50\%}\\
    \small{Usagers – Jeunes adultes} & \small{49,56\%} & \small{48,12\%} & \small{80,40\%} & \small{85,92\%} & \small{42,86\%} & \small{72,79\%} & \small{65,00\%} & \small{87,50\%}\\
    \small{Usagères – Adultes} & \small{48,97\%} & \small{50,47\%} & \small{23,33\%} & \small{27,07\%} & \small{34,62\%} & \small{24,76\%} & \small{56,41\%} & \small{36,36\%}\\
    \small{Usagers – Adultes} & \small{51,03\%} & \small{49,53\%} & \small{76,67\%} & \small{72,93\%} & \small{65,38\%} & \small{75,24\%} & \small{43,59\%} & \small{63,64\%}\\
    \small{Usagères – Personnes âgées} & \small{37,84\%} & \small{39,25\%} & \small{14,29\%} & \small{20,69\%} & \small{25,00\%} & \small{7,69\%} & \small{0,00\%} & \small{0,00\%}\\
    \small{Usagers – Personnes âgées} & \small{62,16\%} & \small{60,75\%} & \small{85,71\%} & \small{79,31\%} & \small{75,00\%} & \small{92,31\%} & \small{100,00\%} & \small{0,00\%}\\
        \hline
        \caption*{}
        \label{Statistiques observation annexe toutes gares}
        \begin{flushright}
        \scriptsize
    Auteur~: \textcopyright~Moinse 2022
        \end{flushright}
        \end{longtable}

    \newpage
    % Annexe G.5
\subsection{Données de l'observation quantitative en gare de Lille Flandres (1)}
    \label{donnees-ouvertes:resultats_observation_quantitative_lille_flandres}

% ___________________________________________
% Tableau F5
        \begin{longtable}{p{3.7cm}p{0.9cm}p{0.9cm}p{0.9cm}p{0.9cm}p{0.9cm}p{0.9cm}p{0.9cm}p{0.9cm}}
         \textcolor{blue}{\textbf{Variables}} & \textcolor{blue}{\textbf{Moy.}} & \textcolor{blue}{\textbf{VSV}} & \textcolor{blue}{\textbf{MIL}} & \textcolor{blue}{\textbf{VC}} & \textcolor{blue}{\textbf{VP}} & \textcolor{blue}{\textbf{TEP}} & \textcolor{blue}{\textbf{TM}} & \textcolor{blue}{\textbf{A}}\\
        \hline
        \endhead
\multicolumn{9}{l}{\textbf{Toutes variables confondues}}\\
    \small{\textbf{Effectif total}} & \small{\textbf{5~836}} & \small{\textbf{5~649}} & \small{\textbf{287}} & \small{\textbf{92}} & \small{\textbf{39}} & \small{\textbf{127}} & \small{\textbf{28}} & \small{\textbf{1}}\\   
    \small{\textbf{Part totale}} & \small{\textbf{100,00\%}} & \small{\textbf{96,80\%}} & \small{\textbf{4,92\%}} & \small{\textbf{1,58\%}} & \small{\textbf{0,67\%}} & \small{\textbf{2,18\%}} & \small{\textbf{0,48\%}} & \small{\textbf{0,02\%}}\\
    \hline    
\multicolumn{9}{l}{\textbf{Genre observé (part en fonction du genre)}}\\
    \small{Usagères (effectif)} & \small{2~998} & \small{2~912} & \small{88} & \small{30} & \small{15} & \small{36} & \small{10} & \small{0}\\    
    \small{Usagères (part)} & \small{51,37\%} & \small{51,60\%} & \small{30,66} & \small{32,61\%} & \small{38,46\%} & \small{28,35\%} & \small{35,71\%} & \small{0,00\%}\\    
    \small{Usagers (effectif)} & \small{2~838} & \small{2~737} & \small{196} & \small{62} & \small{24} & \small{91} & \small{18} & \small{1}\\    
    \small{Usagers (part)} & \small{48,63\%} & \small{48,40\%} & \small{68,29\%} & \small{67,39\%} & \small{61,54\%} & \small{71,65\%} & \small{64,29\%} & \small{100,00\%}\\   
    \hline
\multicolumn{9}{l}{\textbf{Catégories d'âge observées (part en fonction du mode)}}\\
    \small{Enfants (effectif)} & \small{21} & \small{21} & \small{0} & \small{0} & \small{0} & \small{0} & \small{0} & \small{0}\\    
    \small{Enfants (part)} & \small{0,36\%} & \small{0,37\%} & \small{0,00\%} & \small{0,00\%} & \small{0,00\%} & \small{0,00\%} & \small{0,00\%} & \small{0,00\%}\\    
    \small{Jeunes adultes (effectif)} & \small{1~588} & \small{1~539} & \small{49} & \small{10} & \small{0} & \small{34} & \small{5} & \small{0}\\   
    \small{Jeunes adultes (part)} & \small{27,21\%} & \small{27,20\%} & \small{17,07\%} & \small{10,87\%} & \small{0,00\%} & \small{26,77\%} & \small{17,86\%} & \small{0,00\%}\\    
    \small{Adultes (effectif)} & \small{3~803} & \small{3~586} & \small{218} & \small{74} & \small{33} & \small{90} & \small{19} & \small{1}\\    
    \small{Adultes (part)} & \small{65,15\%} & \small{63,50\%} & \small{76,00\%} & \small{80,43\%} & \small{84,62\%} & \small{70,87\%} & \small{67,86\%} & \small{100,00\%}\\    
    \small{Personnes âgées (effectif)} & \small{421} & \small{403} & \small{15} & \small{8} & \small{6} & \small{3} & \small{4} & \small{0}\\    
    \small{Personnes âgées (part)} & \small{7,21\%} & \small{7,13\%} & \small{5,23\%} & \small{8,70\%} & \small{15,38\%} & \small{2,36\%} & \small{14,29\%} & \small{0,00\%}\\    
    \hline
\multicolumn{9}{l}{\textbf{Tranches d'âge croisées avec le genre observé (part en fonction du mode)}}\\
    \small{Usagères – Enfants} & \small{38,10\%} & \small{38,10\%} & \small{0,00\%} & \small{0,00\%} & \small{0,00\%} & \small{0,00\%} & \small{0,00\%} & \small{0,00\%}\\    
    \small{Usagers – Enfants} & \small{61,90\%} & \small{61,90\%} & \small{0,00\%} & \small{0,00\%} & \small{0,00\%} & \small{0,00\%} & \small{0,00\%} & \small{0,00\%}\\    
    \small{Usagères – Jeunes adultes} & \small{55,73\%} & \small{56,77\%} & \small{22,45\%} & \small{10,00\%} & \small{0,00\%} & \small{20,59\%} & \small{60\%} & \small{0,00\%}\\    
    \small{Usagers – Jeunes adultes} & \small{44,27\%} & \small{43,23\%} & \small{77,55\%} & \small{90,00\%} & \small{0,00\%} & \small{79,41\%} & \small{40\%} & \small{0,00\%}\\    
    \small{Usagères – Adultes} & \small{51,06\%} & \small{52,13\%} & \small{33,80\%} & \small{33,78\%} & \small{39,39\%} & \small{31,11\%} & \small{36,84\%} & \small{0,00\%}\\    
    \small{Usagers – Adultes} & \small{48,94\%} & \small{47,87\%} & \small{66,20\%} & \small{66,22\%} & \small{60,61\%} & \small{68,89\%} & \small{63,16\%} & \small{100,00\%}\\    
    \small{Usagères – Personnes âgées} & \small{39,67\%} & \small{39,95\%} & \small{33,33\%} & \small{50,00\%} & \small{33,33\%} & \small{33,33\%} & \small{0\%} & \small{0,00\%}\\    
    \small{Usagers – Personnes âgées} & \small{60,33\%} & \small{60,05\%} & \small{76,19\%} & \small{50,00\%} & \small{66,67\%} & \small{66,67\%} & \small{100\%} & \small{0,00\%}\\
        \hline
        \caption*{}
        \label{Statistiques observation annexe gare Lille Flandres}
        \begin{flushright}
        \scriptsize
    Auteur~: \textcopyright~Moinse 2022
        \end{flushright}
        \end{longtable}

    \newpage
    % Annexe G.6
\subsection{Données de l'observation quantitative en gare d'Armentières (2)}
    \label{donnees-ouvertes:resultats_observation_quantitative_armentieres}

% ___________________________________________
% Tableau F6
        \begin{longtable}{p{3.7cm}p{0.9cm}p{0.9cm}p{0.9cm}p{0.9cm}p{0.9cm}p{0.9cm}p{0.9cm}p{0.9cm}}
         \textcolor{blue}{\textbf{Variables}} & \textcolor{blue}{\textbf{Moy.}} & \textcolor{blue}{\textbf{VSV}} & \textcolor{blue}{\textbf{MIL}} & \textcolor{blue}{\textbf{VC}} & \textcolor{blue}{\textbf{VP}} & \textcolor{blue}{\textbf{TEP}} & \textcolor{blue}{\textbf{TM}} & \textcolor{blue}{\textbf{A}}\\
        \hline
        \endhead        
\multicolumn{9}{l}{\textbf{Toutes variables confondues}}\\
    \small{\textbf{Effectif total}} & \small{\textbf{2~324}} & \small{\textbf{2~179}} & \small{\textbf{145}} & \small{\textbf{46}} & \small{\textbf{15}} & \small{\textbf{58}} & \small{\textbf{16}} & \small{\textbf{0}}\\   
    \small{\textbf{Part totale}} & \small{\textbf{100,00\%}} & \small{\textbf{93,76\%}} & \small{\textbf{6,24\%}} & \small{\textbf{1,98\%}} & \small{\textbf{0,64\%}} & \small{\textbf{2,49\%}} & \small{\textbf{0,69\%}} & \small{\textbf{0,00\%}}\\
    \hline    
\multicolumn{9}{l}{\textbf{Genre observé (part en fonction du genre)}}\\
    \small{Usagères (effectif)} & \small{1~200} & \small{1~153} & \small{26} & \small{15} & \small{7} & \small{10} & \small{10} & \small{0}\\    
    \small{Usagères (part)} & \small{51,63\%} & \small{52,95\%} & \small{17,93\%} & \small{32,61\%} & \small{46,67\%} & \small{17,24\%} & \small{62,5\%} & \small{0,00\%}\\    
    \small{Usagers (effectif)} & \small{1~124} & \small{1~026} & \small{119} & \small{31} & \small{8} & \small{48} & \small{6} & \small{0}\\    
    \small{Usagers (part)} & \small{48,37\%} & \small{47,05\%} & \small{82,07\%} & \small{67,39\%} & \small{53,33\%} & \small{82,76\%} & \small{37,5\%} & \small{0,00\%}\\
    \hline    
\multicolumn{9}{l}{\textbf{Catégories d'âge observées (part en fonction du mode)}}\\
    \small{Enfants (effectif)} & \small{13} & \small{13} & \small{0} & \small{0} & \small{0} & \small{0} & \small{0} & \small{0}\\    
    \small{Enfants (part)} & \small{0,56\%} & \small{0,60\%} & \small{0,00\%} & \small{0,00\%} & \small{0,00\%} & \small{0,00\%} & \small{0,00\%} & \small{0,00\%}\\    
    \small{Jeunes adultes (effectif)} & \small{945} & \small{900} & \small{40} & \small{16} & \small{2} & \small{17} & \small{6} & \small{0}\\    
    \small{Jeunes adultes (part)} & \small{40,68\%} & \small{41,31\%} & \small{27,59\%} & \small{34,78\%} & \small{13,33\%} & \small{29,31\%} & \small{37,5\%} & \small{0,00\%}\\    
    \small{Adultes (effectif)} & \small{1~244} & \small{1~150} & \small{80} & \small{26} & \small{11} & \small{41} & \small{10} & \small{0}\\    
    \small{Adultes (part)} & \small{53,53\%} & \small{52,82\%} & \small{55,17\%} & \small{56,52\%} & \small{73,33\%} & \small{70,69\%} & \small{62,5\%} & \small{0,00\%}\\    
    \small{Personnes âgées (effectif)} & \small{122} & \small{116} & \small{25} & \small{4} & \small{2} & \small{0} & \small{0} & \small{0}\\    
    \small{Personnes âgées (part)} & \small{5,25\%} & \small{5,32\%} & \small{17,24\%} & \small{8,70\%} & \small{13,33\%} & \small{0,00\%} & \small{0,00\%} & \small{0,00\%}\\
    \hline
\multicolumn{9}{l}{\textbf{Tranches d'âge croisées avec le genre observé (part en fonction du mode)}}\\
    \small{Usagères – Enfants} & \small{0,42\%} & \small{0,43\%} & \small{0,00\%} & \small{0,00\%} & \small{0,00\%} & \small{0,00\%} & \small{0,00\%} & \small{0,00\%}\\
    \small{Usagers – Enfants} & \small{0,71\%} & \small{0,78\%} & \small{0,00\%} & \small{0,00\%} & \small{0,00\%} & \small{0,00\%} & \small{0,00\%} & \small{0,00\%}\\
    \small{Usagères – Jeunes adultes} & \small{38,20\%} & \small{38,77\%} & \small{23,91\%} & \small{33,33\%} & \small{14,29\%} & \small{40,00\%} & \small{10,00\%} & \small{0,00\%}\\
    \small{Usagers – Jeunes adultes} & \small{43,29\%} & \small{44,15\%} & \small{34,34\%} & \small{35,48\%} & \small{12,50\%} & \small{27,08\%} & \small{83,33\%} & \small{66,67\%}\\
    \small{Usagères – Adultes} & \small{57,71\%} & \small{56,98\%} & \small{76,09\%} & \small{66,67\%} & \small{85,71\%} & \small{60,00\%} & \small{90,00\%} & \small{100,00\%}\\
    \small{Usagers – Adultes} & \small{49,07\%} & \small{48,05\%} & \small{59,60\%} & \small{51,61\%} & \small{62,50\%} & \small{72,92\%} & \small{16,67\%} & \small{33,33\%}\\
    \small{Usagères – Personnes âgées} & \small{3,67\%} & \small{3,82\%} & \small{0,00\%} & \small{0,00\%} & \small{0,00\%} & \small{0,00\%} & \small{0,00\%} & \small{0,00\%}\\
    \small{Usagers – Personnes âgées} & \small{6,93\%} & \small{5,32\%} & \small{6,06\%} & \small{12,90\%} & \small{25,00\%} & \small{0,00\%} & \small{12,50\%} & \small{0,00\%}\\
        \hline
        \caption*{}
        \label{Statistiques observation annexe gare Armentières}
        \begin{flushright}
        \scriptsize
    Auteur~: \textcopyright~Moinse 2022
        \end{flushright}
        \end{longtable}

    \newpage
    % Annexe G.7
\subsection{Données de l'observation quantitative en gare de Béthune (3)}
    \label{donnees-ouvertes:resultats_observation_quantitative_bethune}

% ___________________________________________
% Tableau F7
        \begin{longtable}{p{3.7cm}p{0.9cm}p{0.9cm}p{0.9cm}p{0.9cm}p{0.9cm}p{0.9cm}p{0.9cm}p{0.9cm}}
         \textcolor{blue}{\textbf{Variables}} & \textcolor{blue}{\textbf{Moy.}} & \textcolor{blue}{\textbf{VSV}} & \textcolor{blue}{\textbf{MIL}} & \textcolor{blue}{\textbf{VC}} & \textcolor{blue}{\textbf{VP}} & \textcolor{blue}{\textbf{TEP}} & \textcolor{blue}{\textbf{TM}} & \textcolor{blue}{\textbf{A}}\\
        \hline
        \endhead   
\multicolumn{9}{l}{\textbf{Toutes variables confondues}}\\
    \small{\textbf{Effectif total}} & \small{\textbf{1~281}} & \small{\textbf{1~185}} & \small{\textbf{96}} & \small{\textbf{32}} & \small{\textbf{7}} & \small{\textbf{35}} & \small{\textbf{19}} & \small{\textbf{3}}\\
    \small{\textbf{Part totale}} & \small{\textbf{100\%}} & \small{\textbf{92,51\%}} & \small{\textbf{7,49\%}} & \small{\textbf{2,50\%}} & \small{\textbf{0,55\%}} & \small{\textbf{2,73\%}} & \small{\textbf{1,48\%}} & \small{\textbf{0,23\%}}\\
    \hline
\multicolumn{9}{l}{\textbf{Genre observé (part en fonction du genre)}}\\
    \small{Usagères (effectif)} & \small{663} & \small{630} & \small{33} & \small{7} & \small{3} & \small{11} & \small{11} & \small{1}\\
    \small{Usagères (part)} & \small{51,76\%} & \small{53,16\%} & \small{34,38\%} & \small{21,88\%} & \small{42,86\%} & \small{31,43\%} & \small{57,89\%} & \small{0,08\%}\\
    \small{Usagers (effectif)} & \small{618} & \small{555} & \small{63} & \small{25} & \small{4} & \small{24} & \small{8} & \small{2}\\
    \small{Usagers (part)} & \small{48,24\%} & \small{46,84\%} & \small{65,63\%} & \small{78,13\%} & \small{57,14\%} & \small{68,57\%} & \small{42,11\%} & \small{66,67\%}\\
    \hline
\multicolumn{9}{l}{\textbf{Catégories d'âge observées (part en fonction du mode)}}\\
    \small{Enfants (effectif)} & \small{2} & \small{2} & \small{0} & \small{0} & \small{0} & \small{0} & \small{0} & \small{0}\\
    \small{Enfants (part)} & \small{0,16\%} & \small{0,17\%} & \small{0\%} & \small{0\%} & \small{0\%} & \small{0\%} & \small{0\%} & \small{0\%}\\
    \small{Jeunes adultes (effectif)} & \small{406} & \small{376} & \small{30} & \small{12} & \small{1} & \small{12} & \small{3} & \small{2}\\
    \small{Jeunes adultes (part)} & \small{31,69\%} & \small{31,73\%} & \small{31,25\%} & \small{37,50\%} & \small{14,29\%} & \small{34,29\%} & \small{15,79\%} & \small{66,67\%}\\
    \small{Adultes (effectif)} & \small{829} & \small{764} & \small{65} & \small{20} & \small{6} & \small{23} & \small{15} & \small{1}\\
    \small{Adultes (part)} & \small{64,72\%} & \small{64,47\%} & \small{67,71\%} & \small{62,50\%} & \small{85,71\%} & \small{65,71\%} & \small{78,95\%} & \small{33,33\%}\\
    \small{Personnes âgées (effectif)} & \small{44} & \small{43} & \small{1} & \small{0} & \small{0} & \small{0} & \small{1} & \small{0}\\  
    \small{Personnes âgées (part)} & \small{3,43\%} & \small{3,63\%} & \small{1,04\%} & \small{0,00\%} & \small{0,00\%} & \small{0,00\%} & \small{5,26\%} & \small{0,00\%}\\
        \hline
\multicolumn{9}{l}{\textbf{Tranches d'âge croisées avec le genre observé (part en fonction du mode)}}\\
    \small{Usagères – Enfants} & \small{0,15\%} & \small{0,16\%} & \small{0,00\%} & \small{0,00\%} & \small{0,00\%} & \small{0,00\%} & \small{0,00\%} & \small{0,00\%}\\
    \small{Usagers – Enfants} & \small{0,16\%} & \small{0,18\%} & \small{0,00\%} & \small{0,00\%} & \small{0,00\%} & \small{0,00\%} & \small{0,00\%} & \small{0,00\%}\\
    \small{Usagères – Jeunes adultes} & \small{33,63\%} & \small{34,44\%} & \small{18,18\%} & \small{14,29\%} & \small{33,33\%} & \small{18,18\%} & \small{9,09\%} & \small{100\%}\\
    \small{Usagers – Jeunes adultes} & \small{29,61\%} & \small{28,65\%} & \small{38,10\%} & \small{44,00\%} & \small{0,00\%} & \small{41,67\%} & \small{25,00\%} & \small{50,00\%}\\
    \small{Usagères – Adultes} & \small{64,10\%} & \small{63,17\%} & \small{81,82\%} & \small{85,71\%} & \small{66,67\%} & \small{81,82\%} & \small{90,91\%} & \small{00,00\%}\\
    \small{Usagers – Adultes} & \small{65,37\%} & \small{65,95\%} & \small{60,32\%} & \small{56,00\%} & \small{100,00\%} & \small{58,33\%} & \small{62,50\%} & \small{50,00\%}\\
    \small{Usagères – Personnes âgées} & \small{2,11\%} & \small{0,00\%} & \small{0,00\%} & \small{0,00\%} & \small{0,00\%} & \small{0,00\%} & \small{0,00\%} & \small{0,00\%}\\
    \small{Usagers – Personnes âgées} & \small{4,85\%} & \small{5,23\%} & \small{1,59\%} & \small{0,00\%} & \small{0,00\%} & \small{0,00\%} & \small{12,50\%} & \small{0,00\%}\\
        \hline
        \caption*{}
        \label{Statistiques observation annexe gare Béthune}
        \begin{flushright}
        \scriptsize
    Auteur~: \textcopyright~Moinse 2022
        \end{flushright}
        \end{longtable}

    \newpage
    % Annexe G.8
\subsection{Données de l'observation quantitative en gare de Dunkerque (4)}
    \label{donnees-ouvertes:resultats_observation_quantitative_dunkerque}

% ___________________________________________
% Tableau F8
        \begin{longtable}{p{3.7cm}p{0.9cm}p{0.9cm}p{0.9cm}p{0.9cm}p{0.9cm}p{0.9cm}p{0.9cm}p{0.9cm}}
         \textcolor{blue}{\textbf{Variables}} & \textcolor{blue}{\textbf{Moy.}} & \textcolor{blue}{\textbf{VSV}} & \textcolor{blue}{\textbf{MIL}} & \textcolor{blue}{\textbf{VC}} & \textcolor{blue}{\textbf{VP}} & \textcolor{blue}{\textbf{TEP}} & \textcolor{blue}{\textbf{TM}} & \textcolor{blue}{\textbf{A}}\\
        \hline
        \endhead        
\multicolumn{9}{l}{\textbf{Toutes variables confondues}}\\
    \small{\textbf{Effectif total}} & \small{\textbf{2~221}} & \small{\textbf{2~034}} & \small{\textbf{187}} & \small{\textbf{49}} & \small{\textbf{25}} & \small{\textbf{112}} & \small{\textbf{21}} & \small{\textbf{1}}\\    
    \small{\textbf{Part totale}} & \small{\textbf{100\%}} & \small{\textbf{91,58\%}} & \small{\textbf{8,42\%}} & \small{\textbf{2,21\%}} & \small{\textbf{1,13\%}} & \small{\textbf{5,04\%}} & \small{\textbf{0,41\%}} & \small{\textbf{0,05\%}}\\
    \hline    
\multicolumn{9}{l}{\textbf{Genre observé (part en fonction du genre)}}\\
    \small{Usagères (effectif)} & \small{1~080} & \small{1~030} & \small{50} & \small{6} & \small{6} & \small{26} & \small{12} & \small{0}\\    
    \small{Usagères (part)} & \small{48,63\%} & \small{50,64\%} & \small{26,74\%} & \small{12,24\%} & \small{24,00\%} & \small{28,57\%} & \small{57,14\%} & \small{0,00\%}\\   
    \small{Usagers (effectif)} & \small{1~141} & \small{1~004} & \small{137} & \small{43} & \small{19} & \small{65} & \small{9} & \small{1}\\    
    \small{Usagers (part)} & \small{51,37\%} & \small{49,36\%} & \small{73,26\%} & \small{87,76\%} & \small{76,00\%} & \small{71,43\%} & \small{42,86\%} & \small{100,00\%}\\
    \hline    
\multicolumn{9}{l}{\textbf{Catégories d'âge observées (part en fonction du mode)}}\\
    \small{Enfants (effectif)} & \small{11} & \small{11} & \small{0} & \small{0} & \small{0} & \small{0} & \small{0} & \small{0}\\    
    \small{Enfants (part)} & \small{0,50\%} & \small{0,54\%} & \small{0,00\%} & \small{0,00\%} & \small{0,00\%} & \small{0,00\%} & \small{0,00\%} & \small{0,00\%}\\   
    \small{Jeunes adultes (effectif)} & \small{606} & \small{568} & \small{38} & \small{14} & \small{1} & \small{17} & \small{5} & \small{1}\\    
    \small{Jeunes adultes (part)} & \small{27,29\%} & \small{27,93\%} & \small{20,32\%} & \small{28,57\%} & \small{4,00\%} & \small{18,68\%} & \small{23,81\%} & \small{100,00\%}\\    
    \small{Adultes (effectif)} & \small{1~419} & \small{1~286} & \small{133} & \small{30} & \small{20} & \small{67} & \small{16} & \small{0}\\    
    \small{Adultes (part)} & \small{63,89\%} & \small{63,23\%} & \small{71,12\%} & \small{61,22\%} & \small{80,00\%} & \small{73,63\%} & \small{76,19\%} & \small{0,00\%}\\    
    \small{Personnes âgées (effectif)} & \small{185} & \small{169} & \small{16} & \small{5} & \small{4} & \small{7} & \small{0} & \small{0}\\   
    \small{Personnes âgées (part)} & \small{8,33\%} & \small{8,31\%} & \small{8,56\%} & \small{10,20\%} & \small{16,00\%} & \small{7,69\%} & \small{0,00\%} & \small{0,00\%}\\
    \hline
\multicolumn{9}{l}{\textbf{Tranches d'âge croisées avec le genre observé (part en fonction du mode)}}\\
    \small{Usagères – Enfants} & \small{0,74\%} & \small{0,78\%} & \small{0,00\%} & \small{0,00\%} & \small{0,00\%} & \small{0,00\%} & \small{0,00\%} & \small{0,00\%}\\
        \small{Usagers – Enfants} & \small{0,26\%} & \small{0,30\%} & \small{0,00\%} & \small{0,00\%} & \small{0,00\%} & \small{0,00\%} & \small{0,00\%} & \small{0,00\%}\\    
    \small{Usagères – Jeunes adultes} & \small{27,69\%} & \small{27,86\%} & \small{24,00\%} & \small{0,00\%} & \small{16,67\%} & \small{34,62\%} & \small{16,67\%} & \small{0,00\%}\\   
    \small{Usagers – Jeunes adultes} & \small{26,91\%} & \small{27,99\%} & \small{18,98\%} & \small{0,00\%} & \small{0,00\%} & \small{12,31\%} & \small{33,33\%} & \small{100,00\%}\\    
    \small{Usagères – Adultes} & \small{63,89\%} & \small{63,50\%} & \small{72,00\%} & \small{83,33\%} & \small{66,67\%} & \small{65,38\%} & \small{83,33\%} & \small{0,00\%}\\    
    \small{Usagers – Adultes} & \small{63,89\%} & \small{62,95\%} & \small{70,80\%} & \small{58,14\%} & \small{84,21\%} & \small{76,92\%} & \small{66,67\%} & \small{0,00\%}\\    
    \small{Usagères – Personnes âgées} & \small{7,69\%} & \small{7,86\%} & \small{4,00\%} & \small{16,67\%} & \small{16,67\%} & \small{0,00\%} & \small{0,00\%} & \small{0,00\%}\\
    \small{Usagers – Personnes âgées} & \small{8,94\%} & \small{8,76\%} & \small{10,22\%} & \small{9,30\%} & \small{15,79\%} & \small{10,77\%} & \small{0,00\%} & \small{0,00\%}\\
        \hline
        \caption*{}
        \label{Statistiques observation annexe gare Dunkerque}
        \begin{flushright}
        \scriptsize
    Auteur~: \textcopyright~Moinse 2022
        \end{flushright}
        \end{longtable}
        
    \newpage
    % Annexe G.9
\subsection{Données de l'observation quantitative en gare de Creil (5)}
    \label{donnees-ouvertes:resultats_observation_quantitative_creil}

% ___________________________________________
% Tableau F9
        \begin{longtable}{p{3.7cm}p{0.9cm}p{0.9cm}p{0.9cm}p{0.9cm}p{0.9cm}p{0.9cm}p{0.9cm}p{0.9cm}}
         \textcolor{blue}{\textbf{Variables}} & \textcolor{blue}{\textbf{Moy.}} & \textcolor{blue}{\textbf{VSV}} & \textcolor{blue}{\textbf{MIL}} & \textcolor{blue}{\textbf{VC}} & \textcolor{blue}{\textbf{VP}} & \textcolor{blue}{\textbf{TEP}} & \textcolor{blue}{\textbf{TM}} & \textcolor{blue}{\textbf{A}}\\
        \hline
        \endhead        
\multicolumn{9}{l}{\textbf{Toutes variables confondues}}\\
    \small{\textbf{Effectif total}} & \small{\textbf{2~159}} & \small{\textbf{2~059}} & \small{\textbf{100}} & \small{\textbf{32}} & \small{\textbf{6}} & \small{\textbf{54}} & \small{\textbf{6}} & \small{\textbf{2}}\\    
    \small{\textbf{Part totale}} & \small{\textbf{100\%}} & \small{\textbf{95,37\%}} & \small{\textbf{4,63\%}} & \small{\textbf{1,48\%}} & \small{\textbf{0,28\%}} & \small{\textbf{2,50\%}} & \small{\textbf{0,28\%}} & \small{\textbf{0,09\%}}\\
    \hline    
\multicolumn{9}{l}{\textbf{Genre observé (part en fonction du genre)}}\\
    \small{Usagères (effectif)} & \small{814} & \small{795} & \small{19} & \small{4} & \small{1} & \small{10} & \small{4} & \small{0}\\    
    \small{Usagères (part)} & \small{37,70\%} & \small{38,61\%} & \small{19,00\%} & \small{12,50\%} & \small{16,67\%} & \small{18,52\%} & \small{66,67\%} & \small{0,00\%}\\    
    \small{Usagers (effectif)} & \small{1~345} & \small{1~264} & \small{81} & \small{28} & \small{5} & \small{44} & \small{2} & \small{2}\\    
    \small{Usagers (part)} & \small{62,30\%} & \small{61,39\%} & \small{81,00\%} & \small{87,50\%} & \small{83,33\%} & \small{81,48\%} & \small{33,33\%} & \small{100,00\%}\\
    \hline    
\multicolumn{9}{l}{\textbf{Catégories d'âge observées (part en fonction du mode)}}\\
    \small{Enfants (effectif)} & \small{8} & \small{8} & \small{0} & \small{0} & \small{0} & \small{0} & \small{0} & \small{0}\\    
    \small{Enfants (part)} & \small{0,37\%} & \small{0,39\%} & \small{0,00\%} & \small{0,00\%} & \small{0,00\%} & \small{0,00\%} & \small{0,00\%} & \small{0,00\%}\\    
    \small{Jeunes adultes (effectif)} & \small{543} & \small{514} & \small{29} & \small{6} & \small{0} & \small{23} & \small{0} & \small{0}\\    
    \small{Jeunes adultes (part)} & \small{25,15\%} & \small{24,96\%} & \small{29,00\%} & \small{18,75\%} & \small{0,00\%} & \small{42,59\%} & \small{0,00\%} & \small{0,00\%}\\    
    \small{Adultes (effectif)} & \small{1~527} & \small{1~459} & \small{68} & \small{24} & \small{5} & \small{30} & \small{6} & \small{2}\\   
    \small{Adultes (part)} & \small{70,73\%} & \small{70,86\%} & \small{68,00\%} & \small{75,00\%} & \small{100,00\%} & \small{55,56\%} & \small{100,00\%} & \small{100,00\%}\\    
    \small{Personnes âgées (effectif)} & \small{81} & \small{78} & \small{3} & \small{2} & \small{0} & \small{1} & \small{0} & \small{0}\\    
    \small{Personnes âgées (part)} & \small{3,75\%} & \small{3,79\%} & \small{3,00\%} & \small{6,25\%} & \small{0,00\%} & \small{1,85\%} & \small{0,00\%} & \small{0,00\%}\\
    \hline
\multicolumn{9}{l}{\textbf{Tranches d'âge croisées avec le genre observé (part en fonction du mode)}}\\
    \small{Usagères – Enfants} & \small{0,37\%} & \small{0,38\%} & \small{0,00\%} & \small{0,00\%} & \small{0,00\%} & \small{0,00\%} & \small{0,00\%} & \small{0,00\%}\\    
    \small{Usagers – Enfants} & \small{0,37\%} & \small{0,40\%} & \small{0,00\%} & \small{0,00\%} & \small{0,00\%} & \small{0,00\%} & \small{0,00\%} & \small{0,00\%}\\    
    \small{Usagères – Jeunes adultes} & \small{25,92\%} & \small{25,79\%} & \small{31,58\%} & \small{0,00\%} & \small{0,00\%} & \small{60,00\%} & \small{0,00\%} & \small{0,00\%}\\    
    \small{Usagers – Jeunes adultes} & \small{24,68\%} & \small{24,45\%} & \small{28,40\%} & \small{21,43\%} & \small{0,00\%} & \small{38,64\%} & \small{0,00\%} & \small{0,00\%}\\    
    \small{Usagères – Adultes} & \small{71,01\%} & \small{71,19\%} & \small{63,16\%} & \small{75,00\%} & \small{100,00\%} & \small{40,00\%} & \small{100,00\%} & \small{0,00\%}\\
        \small{Usagers – Adultes} & \small{70,56\%} & \small{70,65\%} & \small{69,14\%} & \small{75,00\%} & \small{100,00\%} & \small{59,09\%} & \small{100,00\%} & \small{100,00\%}\\    
    \small{Usagères – Personnes âgées} & \small{2,70\%} & \small{2,64\%} & \small{5,26\%} & \small{25,00\%} & \small{0,00\%} & \small{0,00\%} & \small{0,00\%} & \small{0,00\%}\\    
    \small{Usagers – Personnes âgées} & \small{4,39\%} & \small{4,51\%} & \small{2,47\%} & \small{3,57\%} & \small{0,00\%} & \small{2,27\%} & \small{0,00\%} & \small{0,00\%}\\    
        \hline
        \caption*{}
        \label{Statistiques observation annexe gare Creil}
        \begin{flushright}
        \scriptsize
    Auteur~: \textcopyright~Moinse 2022
        \end{flushright}
        \end{longtable}
        
    \newpage
    % Annexe G.10
\subsection{Données de l'observation quantitative en gare de Lesquin (6)}
    \label{donnees-ouvertes:resultats_observation_quantitative_lesquin}

% ___________________________________________
% Tableau F10
        \begin{longtable}{p{3.7cm}p{0.9cm}p{0.9cm}p{0.9cm}p{0.9cm}p{0.9cm}p{0.9cm}p{0.9cm}p{0.9cm}}
         \textcolor{blue}{\textbf{Variables}} & \textcolor{blue}{\textbf{Moy.}} & \textcolor{blue}{\textbf{VSV}} & \textcolor{blue}{\textbf{MIL}} & \textcolor{blue}{\textbf{VC}} & \textcolor{blue}{\textbf{VP}} & \textcolor{blue}{\textbf{TEP}} & \textcolor{blue}{\textbf{TM}} & \textcolor{blue}{\textbf{A}}\\
        \hline
        \endhead       
\multicolumn{9}{l}{\textbf{Toutes variables confondues}}\\
    \small{\textbf{Effectif total}} & \small{\textbf{309}} & \small{\textbf{256}} & \small{\textbf{53}} & \small{\textbf{22}} & \small{\textbf{4}} & \small{\textbf{19}} & \small{\textbf{6}} & \small{\textbf{2}}\\    
    \small{\textbf{Part totale}} & \small{\textbf{100\%}} & \small{\textbf{82,85\%}} & \small{\textbf{17,15\%}} & \small{\textbf{7,12\%}} & \small{\textbf{1,29\%}} & \small{\textbf{6,15\%}} & \small{\textbf{1,94\%}} & \small{\textbf{0,65\%}}\\
    \hline    
\multicolumn{9}{l}{\textbf{Genre observé (part en fonction du genre)}}\\
    \small{Usagères (effectif)} & \small{130} & \small{119} & \small{11} & \small{3} & \small{3} & \small{3} & \small{2} & \small{0}\\    
    \small{Usagères (part)} & \small{42,07\%} & \small{46,48\%} & \small{20,75\%} & \small{13,64\%} & \small{75,00\%} & \small{15,79\%} & \small{33,33\%} & \small{0,00\%}\\    
    \small{Usagers (effectif)} & \small{179} & \small{137} & \small{42} & \small{19} & \small{1} & \small{16} & \small{4} & \small{2}\\    
    \small{Usagers (part)} & \small{57,93\%} & \small{53,52\%} & \small{79,25\%} & \small{86,36\%} & \small{25,00\%} & \small{84,21\%} & \small{66,67\%} & \small{100,00\%}\\
    \hline    
\multicolumn{9}{l}{\textbf{Catégories d'âge observées (part en fonction du mode)}}\\
    \small{Enfants (effectif)} & \small{1} & \small{1} & \small{0} & \small{0} & \small{0} & \small{0} & \small{0} & \small{0}\\    
    \small{Enfants (part)} & \small{0,32\%} & \small{0,39\%} & \small{0,00\%} & \small{0,00\%} & \small{0,00\%} & \small{0,00\%} & \small{0,00\%} & \small{0,00\%}\\    
    \small{Jeunes adultes (effectif)} & \small{133} & \small{121} & \small{12} & \small{7} & \small{0} & \small{4} & \small{0} & \small{1}\\    
    \small{Jeunes adultes (part)} & \small{43,04\%} & \small{47,27\%} & \small{22,64\%} & \small{31,82\%} & \small{0,00\%} & \small{21,05\%} & \small{0,00\%} & \small{50,00\%}\\    
    \small{Adultes (effectif)} & \small{154} & \small{118} & \small{36} & \small{10} & \small{4} & \small{15} & \small{6} & \small{1}\\    
    \small{Adultes (part)} & \small{49,84\%} & \small{46,09\%} & \small{67,92\%} & \small{45,45\%} & \small{100,00\%} & \small{78,95\%} & \small{100,00\%} & \small{50,00\%}\\   
    \small{Personnes âgées (effectif)} & \small{21} & \small{16} & \small{5} & \small{5} & \small{0} & \small{0} & \small{0} & \small{0}\\    
    \small{Personnes âgées (part)} & \small{6,80\%} & \small{6,25\%} & \small{9,43\%} & \small{22,73\%} & \small{0,00\%} & \small{0,00\%} & \small{0,00\%} & \small{0,00\%}\\
    \hline
\multicolumn{9}{l}{\textbf{Tranches d'âge croisées avec le genre observé (part en fonction du mode)}}\\
    \small{Usagères – Enfants} & \small{0,00\%} & \small{0,00\%} & \small{0,00\%} & \small{0,00\%} & \small{0,00\%} & \small{0,00\%} & \small{0,00\%} & \small{0,00\%}\\   
    \small{Usagers – Enfants} & \small{0,58\%} & \small{0,73\%} & \small{0,00\%} & \small{0,00\%} & \small{0,00\%} & \small{0,00\%} & \small{0,00\%} & \small{0,00\%}\\    
    \small{Usagères – Jeunes adultes} & \small{45,38\%} & \small{49,58\%} & \small{0,00\%} & \small{0,00\%} & \small{0,00\%} & \small{0,00\%} & \small{0,00\%} & \small{0,00\%}\\   
    \small{Usagers – Jeunes adultes} & \small{41,34\%} & \small{45,26\%} & \small{28,57\%} & \small{36,84\%} & \small{100,00\%} & \small{25,00\%} & \small{0,00\%} & \small{50,00\%}\\    
    \small{Usagères – Adultes} & \small{50,77\%} & \small{46,22\%} & \small{100,00\%} & \small{100,00\%} & \small{100,00\%} & \small{100,00\%} & \small{100,00\%} & \small{100,00\%}\\    
    \small{Usagers – Adultes} & \small{49,16\%} & \small{45,99\%} & \small{59,52\%} & \small{36,84\%} & \small{100,00\%} & \small{75,00\%} & \small{100,00\%} & \small{50,00\%}\\    
    \small{Usagères – Personnes âgées} & \small{3,85\%} & \small{4,20\%} & \small{0,00\%} & \small{0,00\%} & \small{0,00\%} & \small{0,00\%} & \small{0,00\%} & \small{0,00\%}\\    
    \small{Usagers – Personnes âgées} & \small{8,94\%} & \small{8,03\%} & \small{11,90\%} & \small{26,32\%} & \small{0,00\%} & \small{0,00\%} & \small{0,00\%} & \small{0,00\%}\\
        \hline
        \caption*{}
        \label{Statistiques observation annexe gare Lesquin}
        \begin{flushright}
        \scriptsize
    Auteur~: \textcopyright~Moinse 2022
        \end{flushright}
        \end{longtable}
        
    \newpage
    % Annexe G.11
\subsection{Données de l'observation quantitative en gare de Lille CHR (7)}
    \label{donnees-ouvertes:resultats_observation_quantitative_lille_chr}

% ___________________________________________
% Tableau F11
        \begin{longtable}{p{3.7cm}p{0.9cm}p{0.9cm}p{0.9cm}p{0.9cm}p{0.9cm}p{0.9cm}p{0.9cm}p{0.9cm}}
         \textcolor{blue}{\textbf{Variables}} & \textcolor{blue}{\textbf{Moy.}} & \textcolor{blue}{\textbf{VSV}} & \textcolor{blue}{\textbf{MIL}} & \textcolor{blue}{\textbf{VC}} & \textcolor{blue}{\textbf{VP}} & \textcolor{blue}{\textbf{TEP}} & \textcolor{blue}{\textbf{TM}} & \textcolor{blue}{\textbf{A}}\\
        \hline
        \endhead     
\multicolumn{9}{l}{\textbf{Toutes variables confondues}}\\
    \small{\textbf{Effectif total}} & \small{\textbf{1~025}} & \small{\textbf{903}} & \small{\textbf{122}} & \small{\textbf{42}} & \small{\textbf{21}} & \small{\textbf{57}} & \small{\textbf{2}} & \small{\textbf{0}}\\   
    \small{\textbf{Part totale}} & \small{\textbf{100\%}} & \small{\textbf{88,10\%}} & \small{\textbf{11,90\%}} & \small{\textbf{4,10\%}} & \small{\textbf{2,05\%}} & \small{\textbf{5,56\%}} & \small{\textbf{0,20\%}} & \small{\textbf{0,00\%}}\\
    \hline    
\multicolumn{9}{l}{\textbf{Genre observé (part en fonction du genre)}}\\
    \small{Usagères (effectif)} & \small{532} & \small{493} & \small{39} & \small{13} & \small{8} & \small{18} & \small{0} & \small{0}\\  
    \small{Usagères (part)} & \small{51,90\%} & \small{54,60\%} & \small{31,97\%} & \small{30,95\%} & \small{38,10\%} & \small{31,58\%} & \small{0,00\%} & \small{0,00\%}\\    
    \small{Usagers (effectif)} & \small{493} & \small{410} & \small{83} & \small{29} & \small{13} & \small{39} & \small{2} & \small{0}\\    
    \small{Usagers (part)} & \small{48,10\%} & \small{45,40\%} & \small{68,03\%} & \small{69,05\%} & \small{61,90\%} & \small{68,42\%} & \small{100\%} & \small{0,00\%}\\
    \hline   
\multicolumn{9}{l}{\textbf{Catégories d'âge observées (part en fonction du mode)}}\\
    \small{Enfants (effectif)} & \small{3} & \small{3} & \small{0} & \small{0} & \small{0} & \small{0} & \small{0} & \small{0}\\   
    \small{Enfants (part)} & \small{0,29\%} & \small{0,33\%} & \small{0,00\%} & \small{0,00\%} & \small{0,00\%} & \small{0,00\%} & \small{0,00\%} & \small{0,00\%}\\    
    \small{Jeunes adultes (effectif)} & \small{311} & \small{286} & \small{25} & \small{5} & \small{1} & \small{19} & \small{0} & \small{0}\\    
    \small{Jeunes adultes (part)} & \small{30,34\%} & \small{31,67\%} & \small{20,49\%} & \small{11,90\%} & \small{4,76\%} & \small{33,33\%} & \small{0,00\%} & \small{0,00\%}\\    
    \small{Adultes (effectif)} & \small{639} & \small{548} & \small{91} & \small{34} & \small{20} & \small{36} & \small{1} & \small{0}\\    
    \small{Adultes (part)} & \small{62,34\%} & \small{60,69\%} & \small{74,59\%} & \small{80,95\%} & \small{95,24\%} & \small{63,16\%} & \small{50,00\%} & \small{0,00\%}\\    
    \small{Personnes âgées (effectif)} & \small{72} & \small{66} & \small{6} & \small{3} & \small{0} & \small{2} & \small{1} & \small{0}\\    
    \small{Personnes âgées (part)} & \small{7,02\%} & \small{7,31\%} & \small{4,92\%} & \small{7,14\%} & \small{0,00\%} & \small{3,51\%} & \small{50,00\%} & \small{0,00\%}\\
    \hline
\multicolumn{9}{l}{\textbf{Tranches d'âge croisées avec le genre observé (part en fonction du mode)}}\\
    \small{Usagères – Enfants} & \small{0,56\%} & \small{0,61\%} & \small{0,00\%} & \small{0,00\%} & \small{0,00\%} & \small{0,00\%} & \small{0,00\%} & \small{0,00\%}\\    
    \small{Usagers – Enfants} & \small{0,00\%} & \small{0,00\%} & \small{0,00\%} & \small{0,00\%} & \small{0,00\%} & \small{0,00\%} & \small{0,00\%} & \small{0,00\%}\\    
    \small{Usagères – Jeunes adultes} & \small{31,95\%} & \small{32,05\%} & \small{30,77\%} & \small{23,08\%} & \small{12,50\%} & \small{44,44\%} & \small{0,00\%} & \small{0,00\%}\\    
    \small{Usagers – Jeunes adultes} & \small{28,60\%} & \small{31,22\%} & \small{15,66\%} & \small{6,90\%} & \small{0,00\%} & \small{28,21\%} & \small{0,00\%} & \small{0,00\%}\\    
    \small{Usagères – Adultes} & \small{62,97\%} & \small{62,47\%} & \small{69,23\%} & \small{76,92\%} & \small{87,50\%} & \small{55,56\%} & \small{00,0\%} & \small{0,00\%}\\   
    \small{Usagers – Adultes} & \small{61,66\%} & \small{58,54\%} & \small{77,11\%} & \small{82,76\%} & \small{100,00\%} & \small{66,67\%} & \small{50,00\%} & \small{0,00\%}\\    
    \small{Usagères – Personnes âgées} & \small{4,51\%} & \small{4,87\%} & \small{0,00\%} & \small{0,00\%} & \small{0,00\%} & \small{0,00\%} & \small{0,00\%} & \small{0,00\%}\\    
    \small{Usagers – Personnes âgées} & \small{9,74\%} & \small{10,24\%} & \small{7,23\%} & \small{10,34\%} & \small{0,00\%} & \small{5,13\%} & \small{50,00\%} & \small{0,00\%}\\
        \hline
        \caption*{}
        \label{Statistiques observation annexe gare Lille CHR}
        \begin{flushright}
        \scriptsize
    Auteur~: \textcopyright~Moinse 2022
        \end{flushright}
        \end{longtable}    

    \newpage
    % Annexe G.12
\subsection{Données de l'observation quantitative en gare du Poirier-Université (8)}
    \label{donnees-ouvertes:resultats_observation_quantitative_poirier_universite}

% ___________________________________________
% Tableau F12
        \begin{longtable}{p{3.7cm}p{0.9cm}p{0.9cm}p{0.9cm}p{0.9cm}p{0.9cm}p{0.9cm}p{0.9cm}p{0.9cm}}
         \textcolor{blue}{\textbf{Variables}} & \textcolor{blue}{\textbf{Moy.}} & \textcolor{blue}{\textbf{VSV}} & \textcolor{blue}{\textbf{MIL}} & \textcolor{blue}{\textbf{VC}} & \textcolor{blue}{\textbf{VP}} & \textcolor{blue}{\textbf{TEP}} & \textcolor{blue}{\textbf{TM}} & \textcolor{blue}{\textbf{A}}\\
        \hline
        \endhead      
\multicolumn{9}{l}{\textbf{Toutes variables confondues}}\\
    \small{\textbf{Effectif total}} & \small{\textbf{280}} & \small{\textbf{235}} & \small{\textbf{45}} & \small{\textbf{14}} & \small{\textbf{6}} & \small{\textbf{19}} & \small{\textbf{6}} & \small{\textbf{0}}\\   
    \small{\textbf{Part totale}} & \small{\textbf{100\%}} & \small{\textbf{83,93\%}} & \small{\textbf{16,07\%}} & \small{\textbf{5,00\%}} & \small{\textbf{2,14\%}} & \small{\textbf{6,79\%}} & \small{\textbf{2,14\%}} & \small{\textbf{0,00\%}}\\
    \hline    
\multicolumn{9}{l}{\textbf{Genre observé (part en fonction du genre)}}\\
    \small{Usagères (effectif)} & \small{98} & \small{95} & \small{3} & \small{0} & \small{0} & \small{1} & \small{2} & \small{0}\\   
    \small{Usagères (part)} & \small{35,00\%} & \small{40,43\%} & \small{6,67\%} & \small{0,00\%} & \small{0,00\%} & \small{5,26\%} & \small{33,33\%} & \small{0,00\%}\\    
    \small{Usagers (effectif)} & \small{182} & \small{140} & \small{42} & \small{14} & \small{6} & \small{18} & \small{4} & \small{0}\\    
    \small{Usagers (part)} & \small{65,00\%} & \small{59,57\%} & \small{93,33\%} & \small{100,00\%} & \small{100,00\%} & \small{94,74\%} & \small{66,67\%} & \small{0,00\%}\\
    \hline    
\multicolumn{9}{l}{\textbf{Catégories d'âge observées (part en fonction du mode)}}\\
    \small{Enfants (effectif)} & \small{1} & \small{1} & \small{0} & \small{0} & \small{0} & \small{0} & \small{0} & \small{0}\\    
    \small{Enfants (part)} & \small{0,36\%} & \small{0,43\%} & \small{0,00\%} & \small{0,00\%} & \small{0,00\%} & \small{0,00\%} & \small{0,00\%} & \small{0,00\%}\\    
    \small{Jeunes adultes (effectif)} & \small{109} & \small{95} & \small{14} & \small{1} & \small{2} & \small{10} & \small{1} & \small{0}\\    
    \small{Jeunes adultes (part)} & \small{38,93\%} & \small{40,43\%} & \small{31,11\%} & \small{7,14\%} & \small{33,33\%} & \small{52,63\%} & \small{16,67\%} & \small{0,00\%}\\    
    \small{Adultes (effectif)} & \small{157} & \small{128} & \small{29} & \small{11} & \small{4} & \small{9} & \small{5} & \small{0}\\    
    \small{Adultes (part)} & \small{56,07\%} & \small{54,47\%} & \small{64,44\%} & \small{78,57\%} & \small{66,67\%} & \small{47,37\%} & \small{83,33\%} & \small{0,00\%}\\   
    \small{Personnes âgées (effectif)} & \small{13} & \small{11} & \small{2} & \small{2} & \small{0} & \small{0} & \small{0} & \small{0}\\   
    \small{Personnes âgées (part)} & \small{4,64\%} & \small{4,68\%} & \small{4,44\%} & \small{14,29\%} & \small{0,00\%} & \small{0,00\%} & \small{0,00\%} & \small{0,00\%}\\
    \hline
\multicolumn{9}{l}{\textbf{Tranches d'âge croisées avec le genre observé (part en fonction du mode)}}\\
    \small{Usagères – Enfants} & \small{1,02\%} & \small{1,05\%} & \small{0,00\%} & \small{0,00\%} & \small{0,00\%} & \small{0,00\%} & \small{0,00\%} & \small{0,00\%}\\    
    \small{Usagers – Enfants} & \small{0,00\%} & \small{0,00\%} & \small{0,00\%} & \small{0,00\%} & \small{0,00\%} & \small{0,00\%} & \small{0,00\%} & \small{0,00\%}\\   
    \small{Usagères – Jeunes adultes} & \small{36,73\%} & \small{36,84\%} & \small{33,33\%} & \small{0,00\%} & \small{0,00\%} & \small{100,00\%} & \small{0,00\%} & \small{0,00\%}\\    
    \small{Usagers – Jeunes adultes} & \small{40,11\%} & \small{42,86\%} & \small{30,95\%} & \small{7,14\%} & \small{33,33\%} & \small{50,00\%} & \small{25,00\%} & \small{0,00\%}\\    
    \small{Usagères – Adultes} & \small{58,16\%} & \small{57,89\%} & \small{66,67\%} & \small{0,00\%} & \small{0,00\%} & \small{0,00\%} & \small{100,00\%} & \small{0,00\%}\\   
    \small{Usagers – Adultes} & \small{54,95\%} & \small{52,14\%} & \small{64,29\%} & \small{78,57\%} & \small{66,67\%} & \small{50,00\%} & \small{75,00\%} & \small{0,00\%}\\    
    \small{Usagères – Personnes âgées} & \small{4,08\%} & \small{4,21\%} & \small{0,00\%} & \small{0,00\%} & \small{0,00\%} & \small{0,00\%} & \small{0,00\%} & \small{0,00\%}\\    
    \small{Usagers – Personnes âgées} & \small{4,95\%} & \small{5,00\%} & \small{4,76\%} & \small{14,29\%} & \small{0,00\%} & \small{0,00\%} & \small{0,00\%} & \small{0,00\%}\\
        \hline
        \caption*{}
        \label{Statistiques observation annexe gare Le Poirier Université}
        \begin{flushright}
        \scriptsize
    Auteur~: \textcopyright~Moinse 2022
        \end{flushright}
        \end{longtable}

    % ___________________________________________
    % ANNEXE H~: Protocole méthodologique du questionnaire auprès des usagers
    \newpage
\section{Résultats statistiques du questionnaire auprès des usager·ère·s}
    \label{donnees-ouvertes:questionnaire_usagers}
    \markboth{Annexes liées à la mise en œuvre et aux résultats du questionnaire auprès des usager·ère·s}{}
    \markright{Annexes liées à la mise en œuvre et aux résultats du questionnaire auprès des usager·ère·s}{}

    \newpage
    % Annexe H4
\subsection{Statistiques descriptives des principales caractéristiques des déplacements intermodaux déclarés}
    \label{donnees-ouvertes:statistiques_deplacements_questionnaire_usagers}

    \newpage
    % Annexe H5
\subsection{Statistiques descriptives des distances estimées pour chaque déplacement intermodal déclaré}
    \label{donnees-ouvertes:distances_questionnaire_usagers}

    % Variables
Liste des catégories dans le tableau suivant~:
\begin{itemize}
    \item \textcolor{blue}{$Mode_{TC}$}~: Type de transport en commun principal ;
    \item \textcolor{blue}{$Mode_{A}$}~: Mode de déplacement en \gls{rabattement} ;
    \item \textcolor{blue}{$Mode_{E}$}~: Mode de déplacement en \gls{diffusion} ;
    \item \textcolor{blue}{$DS_{A}$}~: Distance spatiale en \gls{rabattement} (en mètres) ;
    \item \textcolor{blue}{$DT_{A}$}~: Distance temps en \gls{rabattement} (en minutes) ;
    \item \textcolor{blue}{$DS_{E}$}~: Distance spatiale en \gls{diffusion} (en mètres) ;
    \item \textcolor{blue}{$DT_{E}$}~: Distance temps en \gls{diffusion} (en minutes) ;
    \item \textcolor{blue}{$DS_{A+E}$}~: Distance spatiale regroupant les deux trajets (en mètres) ;
    \item \textcolor{blue}{$DT_{A+E}$}~: Distance temps regroupant les deux trajets (en minutes).
\end{itemize}\par

Liste des variables utilisées dans le tableau suivant~:
\begin{itemize}
    \item VC~: Vélo classique ;
    \item VAE~: Vélo à assistance électrique ;
    \item TEP~: Trottinette électrique personnelle ;
    \item TM~: Trottinette mécanique ;
    \item VLS~: Vélo en libre-service ;
    \item e-VLS~: Vélo à assistance électrique en libre-service ;
    \item TEFF~: Trottinette électrique en \textsl{free-floating} ;
    \item \textcolor{blue}{TCU}~: Transports en commun urbains ;
    \item \textcolor{blue}{VP (C)}~: Voiture personnelle (conducteur·rice) ; 
    \item \textcolor{blue}{VP (P)}~: Voiture personnelle (passager·ère).
\end{itemize}\par
    
% ___________________________________________
% Tableau H5
        \begin{longtable}{p{0.7cm}p{1.4cm}p{1.4cm}p{1.6cm}p{0.8cm}p{0.8cm}p{0.8cm}p{0.8cm}p{1.1cm}p{1.1cm}}
         \textcolor{blue}{\small{ID}} & \small{\textcolor{blue}{$Mode_{TC}$}} & \small{\textcolor{blue}{$Mode_{A}$}} & \small{\textcolor{blue}{$Mode_{E}$}} & \small{\textcolor{blue}{$DS_{A}$}} & \small{\textcolor{blue}{$DT_{A}$}} & \small{\textcolor{blue}{$DS_{E}$}} & \small{\textcolor{blue}{$DT_{E}$}} & \small{\textcolor{blue}{$DS_{A+E}$}} & \small{\textcolor{blue}{$DT_{A+E}$}}
        \hline
        \endhead
    \small{1} & \small{TER} & \small{TM} & \small{TM} & \small{2~360} & \small{10} & \small{1~990} & \small{7} & \small{4~350} & \small{17}\\
    \small{2} & \small{TER} & \small{VC} & \small{VC} & \small{1~770} & \small{9} & \small{1~450} & \small{5} & \small{3~220} & \small{14}\\
    \small{3} & \small{TER} & \small{TEP} & \small{\textcolor{blue}{VP (P)}} & \small{2~140} & \small{8} & \small{1~800} & \small{4} & \small{3~940} & \small{12}\\
    \small{4} & \small{TER} & \small{VC} & \small{VC} & \small{9~440} & \small{34} & \small{1~440} & \small{6} & \small{10~880} & \small{40}\\
    \small{5} & \small{TER} & \small{\textcolor{blue}{VP (C)}} & \small{TEP} & \small{9~100} & \small{12} & \small{1~650} & \small{8} & \small{10~750} & \small{20}\\
    \small{6} & \small{TER} & \small{Vélo pliant} & \small{Vélo pliant} & \small{1~020} & \small{4} & \small{1~800} & \small{9} & \small{2~820} & \small{13}\\
    \small{7} & \small{TER} & \small{Vélo pliant} & \small{Vélo pliant} & \small{2~430} & \small{11} & \small{990} & \small{7} & \small{3~420} & \small{18}\\
    \small{8} & \small{TER} & \small{VC} & \small{VC} & \small{1~360} & \small{7} & \small{5~340} & \small{23} & \small{6~700} & \small{30}\\
    \small{9} & \small{TER} & \small{VC} & \small{\textcolor{blue}{Marche}} & \small{3~900} & \small{15} & \small{487} & \small{5} & \small{4~387} & \small{20}\\
    \small{10} & \small{TER} & \small{TEP} & \small{TEP} & \small{1~380} & \small{6} & \small{2~050} & \small{9} & \small{3~430} & \small{15}\\
    \small{11} & \small{TGV} & \small{VC} & \small{VC} & \small{3~240} & \small{13} & \small{6~050} & \small{22} & \small{9~290} & \small{35}\\
    \small{12} & \small{TER} & \small{VC} & \small{VC} & \small{2~340} & \small{11} & \small{1~670} & \small{7} & \small{4~010} & \small{18}\\
    \small{13} & \small{TER} & \small{TEP} & \small{TEP} & \small{1~660} & \small{6} & \small{1~060} & \small{7} & \small{2~720} & \small{13}\\
    \small{14} & \small{TER} & \small{VC} & \small{VC} & \small{1~250} & \small{5} & \small{1~000} & \small{6} & \small{2~250} & \small{11}\\
    \small{15} & \small{TER} & \small{TEP} & \small{TEP} & \small{3~560} & \small{15} & \small{4~470} & \small{17} & \small{8~030} & \small{32}\\
    \small{16} & \small{TER} & \small{VC} & \small{VC} & \small{1~460} & \small{5} & \small{1~350} & \small{8} & \small{2~810} & \small{13}\\
    \small{17} & \small{TERGV} & \small{\textcolor{blue}{VP (C)}} & \small{Vélo pliant} & \small{2~500} & \small{5} & \small{1~510} & \small{6} & \small{4~010} & \small{11}\\
    \small{18} & \small{TER} & \small{VC} & \small{VC} & \small{2~730} & \small{12} & \small{2~450} & \small{11} & \small{5~180} & \small{23}\\
    \small{19} & \small{TER} & \small{VC} & \small{VC} & \small{1~920} & \small{7} & \small{2~190} & \small{11} & \small{4~110} & \small{18}\\
    \small{20} & \small{TER} & \small{VC} & \small{VC} & \small{2~240} & \small{8} & \small{2~300} & \small{10} & \small{4~540} & \small{18}\\
    \small{21} & \small{TER} & \small{VC} & \small{VC} & \small{1~100} & \small{7} & \small{2~630} & \small{10} & \small{3~730} & \small{17}\\
    \small{22} & \small{TER} & \small{\textcolor{blue}{VP (P)}} & \small{TEP} & \small{5~900} & \small{8} & \small{1~240} & \small{7} & \small{7~140} & \small{15}\\
    \small{23} & \small{TER} & \small{VC} & \small{VC} & \small{2~990} & \small{13} & \small{1~520} & \small{6} & \small{4~510} & \small{19}\\
    \small{24} & \small{TER} & \small{Vélo pliant} & \small{Vélo pliant} & \small{2~150} & \small{10} & \small{1~660} & \small{6} & \small{3~810} & \small{16}\\
    \small{25} & \small{TER} & \small{VC} & \small{VC} & \small{1~000} & \small{5} & \small{3~050} & \small{14} & \small{4~050} & \small{19}\\
    \small{26} & \small{TER} & \small{VC} & \small{VC} & \small{2~310} & \small{10} & \small{3~580} & \small{14} & \small{5~890} & \small{24}\\
    \small{27} & \small{TER} & \small{Vélo pliant} & \small{Vélo pliant} & \small{1~510} & \small{8} & \small{2~670} & \small{11} & \small{4~180} & \small{19}\\
    \small{28} & \small{TER} & \small{VC} & \small{VC} & \small{1~230} & \small{6} & \small{3~020} & \small{13} & \small{4~250} & \small{19}\\
    \small{29} & \small{TER} & \small{VC} & \small{VC} & \small{2~140} & \small{10} & \small{8~550} & \small{31} & \small{10~690} & \small{41}\\
    \small{30} & \small{TER} & \small{\textcolor{blue}{TCU}} & \small{Skateboard} & \small{4~670} & \small{17} & \small{1~990} & \small{8} & \small{6~660} & \small{25}\\
    \small{31} & \small{TER} & \small{TEP} & \small{TEP} & \small{5~250} & \small{20} & \small{2~460} & \small{11} & \small{7~710} & \small{31}\\
    \small{32} & \small{TER} & \small{VC} & \small{VC} & \small{6~200} & \small{24} & \small{853} & \small{4} & \small{7~053} & \small{28}\\
    \small{33} & \small{TER} & \small{VC} & \small{VC} & \small{1~930} & \small{8} & \small{2~290} & \small{11} & \small{4~220} & \small{19}\\
    \small{34} & \small{TER} & \textcolor{blue}{\small{VP (C)}} & \small{TEP} & \small{7~000} & \small{12} & \small{1~860} & \small{9} & \small{8~860} & \small{21} \\
    \small{35} & \small{TER} & \small{VC} & \small{VC} & \small{2~920} & \small{13} & \small{2~460} & \small{11} & \small{5~380} & \small{24} \\
    \small{36} & \small{TER} & \small{VC} & \small{VC} & \small{1~270} & \small{7} & \small{1~660} & \small{6} & \small{2~930} & \small{13} \\
    \small{37} & \small{TER} & \small{TEP} & \small{TEP} & \small{1~880} & \small{11} & \small{6~580} & \small{24} & \small{8~460} & \small{35} \\
    \small{38} & \small{TER} & \small{VC} & \small{VC} & \small{4~670} & \small{19} & \small{1~700} & \small{7} & \small{6~370} & \small{26} \\
    \small{39} & \small{TER} & \small{TEP} & \small{TEP} & \small{2~100} & \small{9} & \small{2~620} & \small{13} & \small{4~720} & \small{22} \\
    \small{40} & \small{TER} & \small{TEP} & \small{TEP} & \small{1~340} & \small{5} & \small{1~280} & \small{5} & \small{2~620} & \small{10} \\
    \small{41} & \small{TER} & \small{TEP} & \small{TEP} & \small{1~140} & \small{5} & \small{1~710} & \small{9} & \small{2~850} & \small{14} \\
    \small{42} & \small{TER} & \small{VC} & \small{VC} & \small{7~100} & \small{26} & \small{1~920} & \small{9} & \small{9~020} & \small{35} \\
    \small{43} & \small{TER} & \small{VC} & \small{VC} & \small{1~630} & \small{9} & \small{1~420} & \small{6} & \small{3~050} & \small{15} \\
    \small{44} & \small{TER} & \small{Vélo pliant} & \small{Vélo pliant} & \small{990} & \small{4} & \small{1~720} & \small{7} & \small{2~710} & \small{11} \\
    \small{45} & \small{TER} & \small{TEP} & \small{TEP} & \small{2~220} & \small{8} & \small{1~790} & \small{8} & \small{4~010} & \small{16} \\
    \small{46} & \small{TER} & \small{VC} & \small{VC} & \small{1~110} & \small{4} & \small{1~260} & \small{4} & \small{2~370} & \small{8} \\
    \small{47} & \small{TER} & \small{TM} & \small{TM} & \small{3~110} & \small{15} & \small{1~770} & \small{7} & \small{4~880} & \small{22} \\
    \small{48} & \small{TER} & \small{TEP} & \small{TEP} & \small{2~650} & \small{10} & \small{1~940} & \small{8} & \small{4~590} & \small{18} \\
    \small{49} & \small{TER} & \small{VC} & \small{\textcolor{blue}{Marche}} & \small{1~250} & \small{6} & \small{858} & \small{10} & \small{2~108} & \small{16} \\
    \small{50} & \small{TER} & \small{TEP} & \small{TEP} & \small{2~620} & \small{10} & \small{1~920} & \small{8} & \small{4~540} & \small{18} \\
    \small{51} & \small{TER} & \small{VC} & \small{VC} & \small{1~430} & \small{6} & \small{3~100} & \small{13} & \small{4~530} & \small{19} \\
    \small{52} & \small{TER} & \small{Vélo pliant} & \small{Vélo pliant} & \small{3~340} & \small{12} & \small{1~450} & \small{7} & \small{4~790} & \small{19} \\
    \small{53} & \small{TER} & \small{VC} & \small{VC} & \small{2~100} & \small{9} & \small{1~700} & \small{9} & \small{3~800} & \small{18} \\
    \small{54} & \small{TER} & \small{TEP} & \small{TEP} & \small{1~650} & \small{6} & \small{1~200} & \small{8} & \small{2~850} & \small{14} \\
    \small{55} & \small{TER} & \textcolor{blue}{\small{VP (C)}} & \small{TEP} & \small{8~600} & \small{8} & \small{1~860} & \small{9} & \small{10~460} & \small{17} \\
    \small{56} & \small{TER} & \textcolor{blue}{\small{VP (C)}} & \small{TM} & \small{3~600} & \small{7} & \small{1~080} & \small{5} & \small{4~680} & \small{12} \\
    \small{57} & \small{TERGV} & \small{Vélo pliant} & \small{Vélo pliant} & \small{3~500} & \small{12} & \small{1~300} & \small{7} & \small{4~800} & \small{19} \\
    \small{58} & \small{TERGV} & \small{TEP} & \small{TEP} & \small{561} & \small{4} & \small{1~730} & \small{7} & \small{2~291} & \small{11} \\
    \small{59} & \small{TERGV} & \small{TEP} & \small{TEP} & \small{3~400} & \small{14} & \small{797} & \small{4} & \small{4~197} & \small{18} \\
    \small{60} & \small{TER} & \small{TM} & \textcolor{blue}{\small{VP (C)}} & \small{2~870} & \small{12} & \small{1~400} & \small{4} & \small{4~270} & \small{16} \\
    \small{61} & \small{TERGV} & \small{Vélo pliant} & \small{Vélo pliant} & \small{3~980} & \small{16} & \small{1~770} & \small{10} & \small{5~750} & \small{26} \\
    \small{62} & \small{TER} & \small{TEP} & \small{TEP} & \small{498} & \small{3} & \small{800} & \small{3} & \small{1~298} & \small{6} \\
    \small{63} & \small{TER} & \small{VC} & \small{VC} & \small{1~810} & \small{8} & \small{1~230} & \small{5} & \small{3~040} & \small{13} \\
    \small{64} & \small{TER} & \small{TEP} & \small{TEP} & \small{2~700} & \small{12} & \small{1~900} & \small{10} & \small{4~600} & \small{22} \\
    \small{65} & \small{Métro} & \small{TEP} & \small{TEP} & \small{1~000} & \small{3} & \small{1~800} & \small{6} & \small{2~800} & \small{9} \\
    \small{66} & \small{TER} & \small{Vélo pliant} & \small{Vélo pliant} & \small{2~800} & \small{11} & \small{5~600} & \small{23} & \small{8~400} & \small{34} \\
    \small{67} & \small{TER} & \small{TEP} & \small{TEP} & \small{1~600} & \small{9} & \small{1~800} & \small{10} & \small{3~400} & \small{19} \\
    \small{68} & \small{TGV} & \small{TEP} & \small{TEP} & \small{2~100} & \small{9} & \small{4~800} & \small{21} & \small{6~900} & \small{30} \\
    \small{69} & \small{TER} & \small{\textcolor{blue}{VP (P)}} & \small{VLS} & \small{5~600} & \small{10} & \small{2~900} & \small{10} & \small{8~500} & \small{20} \\
    \small{70} & \small{TER} & \small{\textcolor{blue}{VP (P)}} & \small{VC} & \small{3~200} & \small{8} & \small{3~100} & \small{10} & \small{6~300} & \small{18} \\
    \small{71} & \small{Métro} & \small{\textcolor{blue}{Marche}} & \small{TM} & \small{91} & \small{1} & \small{905} & \small{10} & \small{996} & \small{11} \\
    \small{72} & \small{Métro} & \small{TEP} & \small{TEP} & \small{1~100} & \small{3} & \small{300} & \small{1} & \small{1~400} & \small{4} \\
    \small{73} & \small{Bus} & \small{\textcolor{blue}{Marche}} & \small{VLS} & \small{117} & \small{1} & \small{1~800} & \small{16} & \small{1~917} & \small{17} \\
    \small{74} & \small{TER} & \small{VC} & \small{VC} & \small{2~300} & \small{7} & \small{2~000} & \small{6} & \small{4~300} & \small{13} \\
    \small{75} & \small{TER} & \small{VC} & \small{VC} & \small{6~700} & \small{25} & \small{1~700} & \small{6} & \small{8~400} & \small{31} \\
    \small{76} & \small{TERGV} & \small{TM} & \small{TM} & \small{1~400} & \small{7} & \small{989} & \small{3} & \small{2~389} & \small{10} \\
    \small{77} & \small{TER} & \small{VC} & \small{VC} & \small{2~500} & \small{10} & \small{2~800} & \small{17} & \small{5~300} & \small{27} \\
    \small{78} & \small{TER} & \small{VC} & \small{VC} & \small{1~300} & \small{6} & \small{2~000} & \small{7} & \small{3~300} & \small{13} \\
    \small{79} & \small{TERGV} & \small{VC} & \small{VC} & \small{4~400} & \small{19} & \small{2~700} & \small{12} & \small{7~100} & \small{31} \\
    \small{80} & \small{TGV} & \small{VC} & \small{\textcolor{blue}{Marche}} & \small{2~400} & \small{11} & \small{978} & \small{11} & \small{3~378} & \small{22} \\
    \small{81} & \small{TER} & \small{VC} & \small{VC} & \small{7~900} & \small{32} & \small{1~400} & \small{5} & \small{9~300} & \small{37} \\
    \small{82} & \small{Transilien} & \small{VC} & \small{VC} & \small{1~800} & \small{7} & \small{2~600} & \small{12} & \small{4~400} & \small{19} \\
    \small{83} & \small{TER} & \small{Vélo pliant} & \small{Vélo pliant} & \small{1~900} & \small{11} & \small{10~400} & \small{38} & \small{12~300} & \small{49} \\
    \small{84} & \small{Métro} & \small{\textcolor{blue}{Marche}} & \small{VC} & \small{2~200} & \small{26} & \small{368} & \small{3} & \small{2~568} & \small{29} \\
    \small{85} & \small{TER} & \small{VC} & \small{VC} & \small{7~600} & \small{29} & \small{2~400} & \small{10} & \small{10~000} & \small{39} \\
    \small{86} & \small{TER} & \small{VAE} & \small{VAE} & \small{2~700} & \small{9} & \small{9~500} & \small{36} & \small{12~200} & \small{45} \\
    \small{87} & \small{TGV} & \small{VC} & \small{VC} & \small{3~200} & \small{14} & \small{3~700} & \small{20} & \small{6~900} & \small{34} \\
    \small{88} & \small{Intercités} & \small{VC} & \small{VC} & \small{937} & \small{3} & \small{18~700} & \small{68} & \small{19~637} & \small{71} \\
    \small{89} & \small{TER} & \small{VC} & \small{VC} & \small{7~100} & \small{25} & \small{1~600} & \small{8} & \small{8~700} & \small{33} \\
    \small{90} & \small{TER} & \small{VC} & \small{VC} & \small{1~500} & \small{7} & \small{2~200} & \small{8} & \small{3~700} & \small{15} \\
    \small{91} & \small{Métro} & \small{\textcolor{blue}{Marche}} & \small{VLS} & \small{351} & \small{4} & \small{1~800} & \small{8} & \small{2~151} & \small{12} \\
    \small{92} & \small{TER} & \small{VC} & \small{VC} & \small{2~900} & \small{16} & \small{46~100} & \small{170} & \small{49~000} & \small{186} \\
    \small{93} & \small{TER} & \small{VC} & \small{\textcolor{blue}{Marche}} & \small{2~900} & \small{12} & \small{876} & \small{10} & \small{3~776} & \small{22} \\
    \small{94} & \small{TER} & \small{VAE} & \small{VAE} & \small{22~700} & \small{88} & \small{303~600} & \small{1127} & \small{326~300} & \small{1215} \\
    \small{95} & \small{Transilien} & \small{VAE} & \small{VAE} & \small{765} & \small{5} & \small{4~600} & \small{26} & \small{5~365} & \small{31} \\
    \small{96} & \small{Transilien} & \small{VC} & \small{\textcolor{blue}{Marche}} & \small{2~100} & \small{8} & \small{270} & \small{3} & \small{2~370} & \small{11}\\
    \small{97} & \small{TER} & \small{\textcolor{blue}{Marche}} & \small{VLS} & \small{787} & \small{9} & \small{1~800} & \small{7} & \small{2~587} & \small{16}\\
    \small{98} & \small{Métro} & \small{Monoroue} & \small{Monoroue} & \small{2~500} & \small{10} & \small{2~700} & \small{10} & \small{5~200} & \small{20}\\
    \small{99} & \small{RER} & \small{Vélo cargo} & \small{\textcolor{blue}{Marche}} & \small{2~500} & \small{10} & \small{1~500} & \small{18} & \small{4~000} & \small{28}\\
    \small{100} & \small{TER} & \small{VC} & \small{VC} & \small{1~400} & \small{5} & \small{857} & \small{3} & \small{2~257} & \small{8}\\
    \small{101} & \small{Intercités} & \small{VC} & \small{VC} & \small{7~600} & \small{29} & \small{6~400} & \small{24} & \small{14~000} & \small{53}\\
    \small{102} & \small{TER} & \small{VC} & \small{VC} & \small{3~800} & \small{18} & \small{3~800} & \small{16} & \small{7~600} & \small{34}\\
    \small{103} & \small{TGV} & \small{VLS} & \small{\textcolor{blue}{TCU}} & \small{3~500} & \small{18} & \small{4~500} & \small{19} & \small{8~000} & \small{37}\\
    \small{104} & \small{Tramway} & \small{VC} & \small{\textcolor{blue}{Marche}} & \small{948} & \small{3} & \small{869} & \small{10} & \small{1~817} & \small{13}\\
    \small{105} & \small{TGV} & \small{CV} & \small{\textcolor{blue}{Marche}} & \small{5~100} & \small{40} & \small{5~400} & \small{65} & \small{10~500} & \small{105}\\
    \small{106} & \small{TER} & \small{TEP} & \small{TEP} & \small{1~900} & \small{6} & \small{2~200} & \small{9} & \small{4~100} & \small{15}\\
    \small{107} & \small{Métro} & \small{Vélo pliant} & \small{\textcolor{blue}{Marche}} & \small{1~600} & \small{9} & \small{285} & \small{6} & \small{1~885} & \small{15}\\
    \small{108} & \small{RER} & \small{VAE} & \small{\textcolor{blue}{Marche}} & \small{4~000} & \small{15} & \small{684} & \small{8} & \small{4~684} & \small{23}\\
    \small{109} & \small{TER} & \small{Vélo pliant} & \small{Vélo pliant} & \small{2~500} & \small{10} & \small{2~500} & \small{9} & \small{5~000} & \small{19}\\
    \small{110} & \small{Bus} & \small{VC} & \small{\textcolor{blue}{Marche}} & \small{2~400} & \small{9} & \small{1~000} & \small{12} & \small{3~400} & \small{21}\\
    \small{111} & \small{RER} & \small{VC} & \small{VC} & \small{8~700} & \small{34} & \small{6~000} & \small{22} & \small{14~700} & \small{56}\\
    \small{112} & \small{RER} & \small{VAE} & \small{\textcolor{blue}{Marche}} & \small{4~600} & \small{19} & \small{289} & \small{3} & \small{4~889} & \small{22}\\
    \small{113} & \small{TER} & \small{\textcolor{blue}{Marche}} & \small{VC} & \small{398} & \small{4} & \small{3~000} & \small{11} & \small{3~398} & \small{15}\\
    \small{114} & \small{TER} & \small{VC} & \small{VC} & \small{946} & \small{4} & \small{14~200} & \small{52} & \small{15~146} & \small{56}\\
    \small{115} & \small{TER} & \small{VLS} & \small{\textcolor{blue}{Marche}} & \small{1~600} & \small{7} & \small{580} & \small{7} & \small{2~180} & \small{14}\\
    \small{116} & \small{TER} & \small{VC} & \small{\textcolor{blue}{Marche}} & \small{3~400} & \small{17} & \small{812} & \small{9} & \small{4~212} & \small{26}\\
    \small{117} & \small{TGV} & \small{VC} & \small{VC} & \small{423} & \small{2} & \small{7~100} & \small{26} & \small{7~523} & \small{28}\\
    \small{118} & \small{TER} & \small{VC} & \small{VC} & \small{1~200} & \small{5} & \small{1~100} & \small{3} & \small{2~300} & \small{8}\\
    \small{119} & \small{TER} & \small{\textcolor{blue}{TCU}} & \small{VC} & \small{2~000} & \small{10} & \small{8~400} & \small{30} & \small{10~400} & \small{40}\\
    \small{120} & \small{Métro} & \small{TM} & \small{TM} & \small{895} & \small{4} & \small{721} & \small{3} & \small{1~616} & \small{7}\\
    \small{121} & \small{TER} & \small{VC} & \small{VC} & \small{1~600} & \small{6} & \small{2~000} & \small{10} & \small{3~600} & \small{16}\\
    \small{122} & \small{RER} & \small{VC} & \small{\textcolor{blue}{Marche}} & \small{913} & \small{6} & \small{1~100} & \small{13} & \small{2~013} & \small{19}\\
    \small{123} & \small{RER} & \small{VC} & \small{\textcolor{blue}{Marche}} & \small{704} & \small{15} & \small{595} & \small{7} & \small{1~299} & \small{22}\\
    \small{124} & \small{TER} & \small{VC} & \small{\textcolor{blue}{March}e} & \small{3~000} & \small{7} & \small{644} & \small{7} & \small{3~644} & \small{14}\\
    \small{125} & \small{TGV} & \small{VLS} & \small\textcolor{blue}{{TCU}} & \small{3~700} & \small{15} & \small{4~500} & \small{33} & \small{8~200} & \small{48}\\
    \small{126} & \small{Bus} & \small{VC} & \small{\textcolor{blue}{Marche}} & \small{7~200} & \small{26} & \small{237} & \small{2} & \small{7~437} & \small{28}\\
    \small{127} & \small{RER} & \small{VC} & \small{VC} & \small{1~300} & \small{5} & \small{2~100} & \small{9} & \small{3~400} & \small{14}\\
    \small{128} & \small{TER} & \small{VC} & \small{VC} & \small{1~300} & \small{5} & \small{7~900} & \small{34} & \small{9~200} & \small{39}\\
    \small{129} & \small{Métro} & \small{\textcolor{blue}{VP (P)}} & \small{VC} & \small{4~700} & \small{10} & \small{847} & \small{5} & \small{5~547} & \small{15}\\
    \small{130} & \small{TER} & \small{\textcolor{blue}{VP (C)}} & \small{TM} & \small{13~600} & \small{14} & \small{1~300} & \small{5} & \small{14~900} & \small{19}\\
    \small{131} & \small{TERGV} & \small{TEP} & \small{TEP} & \small{2~500} & \small{9} & \small{6~700} & \small{25} & \small{9~200} & \small{34}\\
    \small{132} & \small{TER} & \small{\textcolor{blue}{VP (C)}} & \small{TM} & \small{2~200} & \small{6} & \small{1~500} & \small{7} & \small{3~700} & \small{13}\\
    \small{133} & \small{Tramway} & \small{VC} & \small{VC} & \small{3~700} & \small{16} & \small{2~300} & \small{9} & \small{6~000} & \small{25}\\
    \small{134} & \small{TER} & \small{VC} & \small{VC} & \small{2~600} & \small{9} & \small{959} & \small{5} & \small{3~559} & \small{14}\\
    \small{135} & \small{RER} & \small{VC} & \small{\textcolor{blue}{Marche}} & \small{4~200} & \small{17} & \small{1~200} & \small{14} & \small{5~400} & \small{31}\\
    \small{136} & \small{Bus} & \small{\textcolor{blue}{Marche}} & \small{TM} & \small{177} & \small{2} & \small{443} & \small{5} & \small{620} & \small{7}\\
    \small{137} & \small{RER} & \small{e-VLS} & \small{\textcolor{blue}{Marche}} & \small{1~900} & \small{7} & \small{206} & \small{2} & \small{2~106} & \small{9}\\
    \small{138} & \small{TER} & \small{\textcolor{blue}{Marche}} & \small{VLS} & \small{1~200} & \small{14} & \small{2~200} & \small{10} & \small{3~400} & \small{24}\\
    \small{139} & \small{TGV} & \small{\textcolor{blue}{TCU}} & \small{Vélo pliant} & \small{25~700} & \small{63} & \small{2~800} & \small{10} & \small{28~500} & \small{73}\\
    \small{140} & \small{TGV} & \small{VC} & \small{VC} & \small{4~400} & \small{18} & \small{39~900} & \small{169} & \small{44300} & \small{187}\\
    \small{141} & \small{Métro} & \small{VC} & \small{\textcolor{blue}{Marche}} & \small{6~500} & \small{24} & \small{502} & \small{6} & \small{7~002} & \small{30} \\
    \small{142} & \small{TER} & \small{VC} & \small{\textcolor{blue}{Marche}} & \small{4~100} & \small{16} & \small{1~600} & \small{18} & \small{5~700} & \small{34} \\
    \small{143} & \small{TGV} & \small{\textcolor{blue}{TCU}} & \small{VLS} & \small{3~000} & \small{12} & \small{2~000} & \small{9} & \small{5~000} & \small{21} \\
    \small{144} & \small{TER} & \small{VC} & \small{VC} & \small{1~300} & \small{7} & \small{2~900} & \small{11} & \small{4~200} & \small{18} \\
    \small{145} & \small{TER} & \small{VC} & \small{VC} & \small{11~300} & \small{46} & \small{6~300} & \small{26} & \small{17~600} & \small{72} \\
    \small{146} & \small{Transilien} & \small{\textcolor{blue}{TCU}} & \small{VLS} & \small{6~300} & \small{20} & \small{4~200} & \small{17} & \small{10~500} & \small{37} \\
    \small{147} & \small{Tramway} & \small{TEFF} & \small{\textcolor{blue}{Marche}} & \small{1~800} & \small{6} & \small{890} & \small{10} & \small{2~690} & \small{16} \\
    \small{148} & \small{TGV} & \small{TEP} & \small{\textcolor{blue}{Taxi/VTC}} & \small{1~500} & \small{8} & \small{4~700} & \small{14} & \small{6~200} & \small{22} \\
    \small{149} & \small{TER} & \small{VC} & \small{VC} & \small{5~900} & \small{23} & \small{4~200} & \small{16} & \small{10~100} & \small{39} \\
    \small{150} & \small{Métro} & \small{Vélo pliant} & \small{Vélo pliant} & \small{428} & \small{1} & \small{9~100} & \small{42} & \small{9~528} & \small{43} \\
    \small{151} & \small{TER} & \small{VC} & \small{VC} & \small{1~600} & \small{7} & \small{3~200} & \small{31} & \small{4~800} & \small{38} \\
    \small{152} & \small{TER} & \small{VP (C)} & \small{VC} & \small{13~500} & \small{18} & \small{4~000} & \small{16} & \small{17~500} & \small{34} \\
    \small{153} & \small{Bus} & \small{TEP} & \small{TEP} & \small{530} & \small{2} & \small{2~000} & \small{7} & \small{2~530} & \small{9} \\
    \small{154} & \small{TER} & \small{TEP} & \small{TEP} & \small{1~900} & \small{8} & \small{2~300} & \small{8} & \small{4~200} & \small{16} \\
    \small{155} & \small{RER} & \small{VC} & \small{\textcolor{blue}{Marche}} & \small{1~200} & \small{4} & \small{2~600} & \small{31} & \small{3~800} & \small{35} \\
    \small{156} & \small{TER} & \small{\textcolor{blue}{Marche}} & \small{VLS} & \small{846} & \small{10} & \small{2~800} & \small{9} & \small{3~646} & \small{19} \\
    \small{157} & \small{TGV} & \small{VC} & \small{\textcolor{blue}{Marche}} & \small{6~600} & \small{25} & \small{775} & \small{9} & \small{7~375} & \small{34} \\
    \small{158} & \small{Métro} & \small{Skateboard} & \small{Skateboard} & \small{753} & \small{2} & \small{668} & \small{2} & \small{1~421} & \small{4} \\
    \small{159} & \small{TER} & \small{VC} & \small{VC} & \small{2~300} & \small{11} & \small{2~700} & \small{13} & \small{5~000} & \small{24} \\
    \small{160} & \small{TER} & \small{VC} & \small{VC} & \small{1~800} & \small{9} & \small{1~500} & \small{7} & \small{3~300} & \small{16} \\
    \small{161} & \small{TER} & \small{\textcolor{blue}{VP (C)}} & \small{VLS} & \small{1~900} & \small{6} & \small{1~900} & \small{9} & \small{3~800} & \small{15}\\
    \small{162} & \small{TER} & \small{VC} & \small{VC} & \small{2~600} & \small{9} & \small{1~400} & \small{5} & \small{4~000} & \small{14}\\
    \small{163} & \small{TER} & \small{VC} & \small{\textcolor{blue}{VP (P)}} & \small{3~500} & \small{17} & \small{2~800} & \small{5} & \small{6~300} & \small{22}\\
    \small{164} & \small{Métro} & \small{TM} & \small{TM} & \small{199} & \small{1} & \small{1~300} & \small{7} & \small{1~499} & \small{8}\\
    \small{165} & \small{TER} & \small{VAE pliant} & \small{VAE pliant} & \small{1~600} & \small{6} & \small{5~000} & \small{19} & \small{6~600} & \small{25}\\
    \small{166} & \small{TGV} & \small{TEP} & \small{TEP} & \small{3~400} & \small{17} & \small{2~200} & \small{9} & \small{5~600} & \small{26}\\
    \small{167} & \small{TGV} & \small{Vélo pliant} & \small{Vélo pliant} & \small{3~600} & \small{16} & \small{2~100} & \small{11} & \small{5~700} & \small{27}\\
    \small{168} & \small{TER} & \small{VAE pliant} & \small{VAE pliant} & \small{2~000} & \small{7} & \small{3~200} & \small{16} & \small{5~200} & \small{23}\\
    \small{169} & \small{TGV} & \small{\textcolor{blue}{TCU}} & \small{TEP} & \small{4~800} & \small{27} & \small{1~800} & \small{8} & \small{6~600} & \small{35}\\
    \small{170} & \small{TER} & \small{VC} & \small{\textcolor{blue}{Marche}} & \small{4~000} & \small{16} & \small{2~500} & \small{30} & \small{6~500} & \small{46}\\
    \small{171} & \small{Transilien} & \small{VC} & \small{VC} & \small{996} & \small{5} & \small{5~300} & \small{20} & \small{6~296} & \small{25}\\
    \small{172} & \small{TER} & \small{\textcolor{blue}{VP (P)}} & \small{VAE pliant} & \small{3~700} & \small{8} & \small{2~400} & \small{9} & \small{6~100} & \small{17}\\
    \small{173} & \small{TER} & \small{\textcolor{blue}{VP (C)}} & \small{VAE pliant} & \small{1~800} & \small{3} & \small{2~900} & \small{11} & \small{4~700} & \small{14}\\
    \small{174} & \small{TER} & \small{\textcolor{blue}{Marche}} & \small{VLS} & \small{1~100} & \small{13} & \small{4~100} & \small{16} & \small{5~200} & \small{29}\\
    \small{175} & \small{TER} & \small{VC} & \small{VC} & \small{1~100} & \small{4} & \small{371} & \small{4} & \small{1~471} & \small{8}\\
    \small{176} & \small{Bus} & \small{\textcolor{blue}{Marche}} & \small{VLS} & \small{328} & \small{3} & \small{2~900} & \small{10} & \small{3~228} & \small{13}\\
    \small{177} & \small{TER} & \small{\textcolor{blue}{VP (C)}} & \small{Vélo pliant} & \small{6~400} & \small{11} & \small{1~100} & \small{6} & \small{7~500} & \small{17}\\
    \small{178} & \small{Transilien} & \small{\textcolor{blue}{VP (C)}} & \small{TEP} & \small{2~700} & \small{7} & \small{3~500} & \small{13} & \small{6~200} & \small{20}\\
    \small{179} & \small{TER} & \small{VC} & \small{VC} & \small{941} & \small{3} & \small{330} & \small{1} & \small{1~271} & \small{4}\\
    \small{180} & \small{TER} & \small{TEP} & \small{TEP} & \small{3~500} & \small{14} & \small{619} & \small{2} & \small{4~119} & \small{16}\\
    \small{181} & \small{TER} & \small{VC} & \small{VC} & \small{2~400} & \small{9} & \small{2~500} & \small{9} & \small{4~900} & \small{18}\\
    \small{182} & \small{Métro} & \small{\textcolor{blue}{Marche}} & \small{TEP} & \small{111} & \small{1} & \small{1~100} & \small{4} & \small{1~211} & \small{5}\\
    \small{183} & \small{TER} & \small{Vélo pliant} & \small{Vélo pliant} & \small{2~900} & \small{12} & \small{2~200} & \small{8} & \small{5~100} & \small{20}\\
    \small{184} & \small{TER} & \small{VAE pliant} & \small{VAE pliant} & \small{1~700} & \small{5} & \small{4~700} & \small{15} & \small{6~400} & \small{20}\\
    \small{185} & \small{TER} & \small{VC} & \small{VC} & \small{13~500} & \small{48} & \small{1~200} & \small{5} & \small{14~700} & \small{53}\\
    \small{186} & \small{TER} & \small{VC} & \small{VC} & \small{2~500} & \small{17} & \small{1~500} & \small{6} & \small{4~000} & \small{23}\\
    \small{187} & \small{TER} & \small{TEP} & \small{TEP} & \small{3~000} & \small{10} & \small{1~100} & \small{6} & \small{4~100} & \small{16}\\
    \small{188} & \small{TGV} & \small{Vélo pliant} & \small{Vélo pliant} & \small{1~600} & \small{7} & \small{2~700} & \small{13} & \small{4~300} & \small{20}\\
    \small{189} & \small{TER} & \small{VC} & \small{VC} & \small{1~400} & \small{13} & \small{1~200} & \small{4} & \small{2~600} & \small{17}\\
    \small{190} & \small{TER} & \small{VC} & \small{VC} & \small{2~400} & \small{10} & \small{8~000} & \small{28} & \small{10~400} & \small{38}\\
    \small{191} & \small{TER} & \small{Vélo pliant} & \small{Vélo pliant} & \small{1~600} & \small{5} & \small{1~600} & \small{6} & \small{3~200} & \small{11}\\
    \small{192} & \small{TGV} & \small{TEP} & \small{TEP} & \small{2~900} & \small{11} & \small{3~500} & \small{13} & \small{6~400} & \small{24}\\
    \small{193} & \small{TER} & \small{VC} & \small{VC} & \small{1~700} & \small{7} & \small{12~100} & \small{42} & \small{13~800} & \small{49}\\
    \small{194} & \small{TER} & \small{VC} & \small{VC} & \small{1~800} & \small{7} & \small{2~800} & \small{12} & \small{4~600} & \small{19}\\
    \small{195} & \small{TER} & \small{VAE pliant} & \small{VAE pliant} & \small{1~900} & \small{9} & \small{1~800} & \small{7} & \small{3~700} & \small{16}\\
    \small{196} & \small{Tramway} & \small{VC} & \small{\textcolor{blue}{TCU}} & \small{398} & \small{2} & \small{1~200} & \small{4} & \small{1~598} & \small{6}\\
    \small{197} & \small{TER} & \small{VC} & \small{VC} & \small{3~100} & \small{10} & \small{3~700} & \small{14} & \small{6~800} & \small{24}\\
    \small{198} & \small{Tramway} & \small{TEP} & \small{TEP} & \small{420} & \small{2} & \small{1~200} & \small{8} & \small{1~620} & \small{10}\\
    \small{199} & \small{RER} & \small{TEP} & \small{TEP} & \small{4~900} & \small{22} & \small{1~400} & \small{6} & \small{6v300} & \small{28}\\
    \small{200} & \small{TER} & \small{VC} & \small{VC} & \small{517} & \small{3} & \small{3~500} & \small{13} & \small{4~017} & \small{16}\\
    \small{201} & \small{Bus} & \small{Vélo pliant} & \small{Vélo pliant} & \small{6~700} & \small{23} & \small{408} & \small{1} & \small{7~108} & \small{24}\\
    \small{202} & \small{TER} & \small{VC} & \small{VC} & \small{2v400} & \small{9} & \small{2~000} & \small{9} & \small{4~400} & \small{18}\\
    \small{203} & \small{TER} & \small{\textcolor{blue}{VP (C)}} & \small{VC} & \small{32~000} & \small{25} & \small{4~700} & \small{18} & \small{36~700} & \small{43}\\
    \small{204} & \small{TER} & \small{TM} & \small{TM} & \small{880} & \small{3} & \small{2~000} & \small{8} & \small{2~880} & \small{11}\\
    \small{205} & \small{TER} & \small{\textcolor{blue}{Marche}} & \small{VC} & \small{998} & \small{11} & \small{1~800} & \small{6} & \small{2~798} & \small{17}\\
    \small{206} & \small{TER} & \small{VC} & \small{VC} & \small{3~100} & \small{14} & \small{1~200} & \small{4} & \small{4~300} & \small{18}\\
    \small{207} & \small{TER} & \small{VC} & \small{VC} & \small{2~900} & \small{11} & \small{865} & \small{3} & \small{3~765} & \small{14}\\
    \small{208} & \small{TER} & \small{Vélo pliant} & \small{Vélo pliant} & \small{2~300} & \small{10} & \small{29~800} & \small{105} & \small{32~100} & \small{115}\\
    \small{209} & \small{Tramway} & \small{TEP} & \small{TEP} & \small{662} & \small{2} & \small{1~100} & \small{6} & \small{1~762} & \small{8}\\
    \small{210} & \small{Métro} & \small{\textcolor{blue}{Marche}} & \small{TM} & \small{380} & \small{4} & \small{711} & \small{5} & \small{1~091} & \small{9}\\
    \small{211} & \small{Tramway} & \small{VC} & \small{VC} & \small{2~100} & \small{8} & \small{1~900} & \small{7} & \small{4~000} & \small{15}\\
    \small{212} & \small{TER} & \small{VLS} & \small{\textcolor{blue}{Marche}} & \small{2~900} & \small{9} & \small{1~000} & \small{12} & \small{3~900} & \small{21}\\
    \small{213} & \small{TER} & \small{VC} & \small{VC} & \small{10~900} & \small{38} & \small{2~000} & \small{9} & \small{12~900} & \small{47}\\
    \small{214} & \small{Métro} & \small{\textcolor{blue}{Marche}} & \small{Skateboard} & \small{352} & \small{4} & \small{856} & \small{10} & \small{1~208} & \small{14}\\
    \small{215} & \small{Tramway} & \small{VC} & \small{\textcolor{blue}{Marche}} & \small{846} & \small{4} & \small{1~700} & \small{21} & \small{2~546} & \small{25}\\
    \small{216} & \small{TGV} & \small{Vélo pliant} & \small{Vélo pliant} & \small{4~700} & \small{16} & \small{4~300} & \small{18} & \small{9~000} & \small{34}\\
    \small{217} & \small{TER} & \small{\textcolor{blue}{VP (P)}} & \small{VC} & \small{36~900} & \small{28} & \small{16~800} & \small{56} & \small{53~700} & \small{84}\\
        \hline
        \caption*{}
        \label{Tableau statistiques descriptives des distances estimées pour chaque déplacement intermodal}
        \begin{flushright}
        \scriptsize
    Auteur~: \textcopyright~Moinse 2023
        \end{flushright}
        \end{longtable}

    % ___________________________________________
    % ANNEXE L~: Détours
    \newpage
\section{Calcul détaillé des détours \acrshort{E-TVS} identifiés}
    \label{donnees-ouvertes:calcul_detours}
    \markboth{Calcul détaillé des détours E-TVS identifiés}{}
    \markright{Calcul détaillé des détours E-TVS identifiés}{}

    \newpage
    % Annexe L2
\subsection{Statistiques descriptives des distances estimées des \acrshort{E-TVS}}
    \label{donnees-ouvertes:statistiques_detours}
    
    % Variables
Liste des catégories dans le tableau suivant~:
\begin{itemize}
    \item \textcolor{blue}{$S^{E-TVS}_{A}$}~: Stratégie de détour des trajets en \gls{rabattement} présentant un détour ;
    \item \textcolor{blue}{$S^{E-TVS}_{E}$}~: Stratégie de détour des trajets en \gls{diffusion} présentant un détour ;
    \item \textcolor{blue}{$R_{km}$}~: Ratio d'optimisation spatiale ;
    \item \textcolor{blue}{$R_{t_O}$}~: Ratio d'optimisation temporelle objective ;
    \item \textcolor{blue}{$R_{t_P}$}~: Ratio d'optimisation temporelle subjective ;
    \item \textcolor{blue}{$C_{R}$}~: Clusters (profils) des déplacements \acrshort{E-TVS} ;
    \item \textcolor{blue}{$ GRDI $}~: Indice de circuité ($eff/eud$) ;
    \item \textcolor{blue}{\alpha$_A$}~: Angle effectif des trajets en \gls{rabattement} présentant un détour ;
    \item \textcolor{blue}{\alpha$_E$}~: Angle effectif des trajets en \gls{diffusion} présentant un détour.
\end{itemize}
    
    % Détails variables
Les stratégies d'optimisation par le biais de détours (\textcolor{blue}{$S^{E-TVS}_{A}$}) sont les suivantes~: l'«~évitement des correspondances~» (1), le choix d'une «~station de transport en commun plus attractive~» (2) et la pratique visant une «~réduction du temps à bord des transports en commun~» (3) (voir la \hyperref[Typologie des stratégies d'optimisation basées sur les détours et les pauses]{sous-section présentant la typologie des stratégies d'optimisation}, page \pageref{Typologie des stratégies d'optimisation basées sur les détours et les pauses}).\par

    % Détails variables
Les ratios d'optimisation spatiale (\textcolor{blue}{$R_{km}$}) et temporelle (\textcolor{blue}{$R_{t_0}$} et \textcolor{blue}{$R_{t_P}$}) se lisent de la manière suivante~: un indicateur supérieur à 1 signifie que le déplacement basé sur des détours est caractérisé par des économies de distance spatiale ou/et des gains de distance temps, tandis qu'un ratio inférieur à 1 revient à considérer un accroissement de la distance spatiale ou/et de distance temps (voir la \hyperref[Les détours, catalyseurs d'économie de distance]{sous-section consacrée aux économies de distances générées par les détours}, page \pageref{Les détours, catalyseurs d'économie de distance}).\par

    % Détails variables
Le regroupement, sous la forme d'une clustérisation (\textcolor{blue}{$C_{R}$}), des déplacements \acrshort{E-TVS} prend la forme de quatre profils~: le profil A se caractérise par des économies de distance temps, mais par un ratio de distance spatiale défavorable, tandis que le profil B privilégie les gains de distance spatiale au détriment de la distance temps, le profil C combine les avantages en termes de distance temps et spatiale, alors que le profil D cumule un ratio d'optimisation spatiale et temporelle négatif (voir la \hyperref[Clustérisation des déplacements intermodaux E-TVS du point de vue de l'optimisation spatio-temporelle]{sous-section dédiée à la clustérisation des déplacements E-TVS au prisme de l'optimisation}, page \pageref{Clustérisation des déplacements intermodaux E-TVS du point de vue de l'optimisation spatio-temporelle}).\par

    % Détails variables
L'indice de circuité (\textcolor{blue}{$ GRDI $}) consiste à comparer la distance spatiale effective ($eff$) avec la distance à vol d'oiseau, dite Euclidienne ($eud$)~: si la valeur est supérieure à 1, l'indice de circuité suggère que la distance spatiale du déplacement \acrshort{E-TVS} est plus longue que la distance Euclidienne, alors qu'un indice égal à 1 signifie que le déplacement \acrshort{E-TVS} suit une ligne droite (voir la \hyperref[Les détours, catalyseurs d'économie de distance]{sous-section consacrée aux économies de distances générées par les détours}, page \pageref{Les détours, catalyseurs d'économie de distance}).\par

    % Détails variables
Les angles effectifs du trajet en \gls{rabattement} (\textcolor{blue}{\alpha$_A$}) et du trajet en \gls{diffusion} (\textcolor{blue}{\alpha$_E$}) d'un déplacement \acrshort{E-TVS} correspondent à la mesure de l'angle entre les vecteurs $AB$ et $BC$. Si \alpha se situe entre 0° et 90°, alors le détour analysé détient une géométrie s'apparentant à une inversion spatiale. Inversement, si \alpha est compris entre 90° et 180°, le détour prend une forme plus directe à destination du point $C$ (voir la \hyperref[L'inversion spatiale, moteur de l'optimisation espace-temps]{sous-section examinant l'impact de la géométrie des détours}, page \pageref{L'inversion spatiale, moteur de l'optimisation espace-temps}).\par
    
% ___________________________________________
% Tableau K2
        \begin{longtable}{p{0.7cm}p{1.4cm}p{1.4cm}p{1cm}p{1cm}p{1cm}p{1cm}p{1cm}p{1cm}p{1cm}}
         \textcolor{blue}{\small{ID}} & \small{\textcolor{blue}{$S^{E-TVS}_{A}$}} & \small{\textcolor{blue}{$S^{E-TVS}_{E}$}} & \small{\textcolor{blue}{$R_{km}$}} & \small{\textcolor{blue}{$R_{t_O}$}} & \small{\textcolor{blue}{$R_{t_P}$}} & \small{\textcolor{blue}{$C_{R}$}} & \small{\textcolor{blue}{$ GRDI $}} & \small{\textcolor{blue}{\alpha$_A$}} & \small{\textcolor{blue}{\alpha$_E$}}
        \hline
        \endhead
\small{1} & \small{1} & \small{1} & \small{1,08} & \small{0,97} & \small{0,99} & \small{B} & \small{1,23} & \small{54,09°} & \small{134,96°}\\
\small{2} & \small{1} & \small{1} & \small{1,11} & \small{1,07} & \small{1,13} & \small{C} & \small{1,18} & \small{32,04°} & \small{118,74°}\\
\small{3} & \small{2} & \small{-} & \small{0,98} & \small{0,96} & \small{0,95} & \small{D} & \small{4,10} & \small{153,01°} & \small{-}\\
\small{4} & \small{2} & \small{1} & \small{1,23} & \small{1,13} & \small{1,21} & \small{C} & \small{1,49} & \small{22,71°} & \small{154,56°}\\
\small{5} & \small{-} & \small{1} & \small{1,01} & \small{0,96} & \small{1,02} & \small{C} & \small{1,61} & \small{-} & \small{82,55°}\\
\small{6} & \small{1} & \small{-} & \small{1,05} & \small{1,03} & \small{1,05} & \small{C} & \small{1,34} & \small{41,48°} & \small{-}\\
\small{7} & \small{1} & \small{1} & \small{1,05} & \small{0,88} & \small{0,88} & \small{B} & \small{1,48} & \small{34,29°} & \small{130,10°}\\
\small{8} & \small{1} & \small{-} & \small{0,96} & \small{1,12} & \small{1,19} & \small{A} & \small{1,41} & \small{11,50°} & \small{-}\\
\small{9} & \small{-} & \small{1} & \small{1,02} & \small{1,00} & \small{1,06} & \small{C} & \small{1,19} & \small{-} & \small{143,37°}\\
\small{10} & \small{1} & \small{-} & \small{0,99} & \small{0,94} & \small{0,98} & \small{D} & \small{1,48} & \small{19,18°} & \small{-}\\
\small{11} & \small{3} & \small{1} & \small{0,97} & \small{1,29} & \small{1,22} & \small{A} & \small{1,79} & \small{11,36°} & \small{11,73°}\\
\small{12} & \small{-} & \small{1} & \small{0,94} & \small{1,02} & \small{1,07} & \small{A} & \small{1,43} & \small{-} & \small{80,47°}\\
\small{13} & \small{1} & \small{2} & \small{0,99} & \small{1,08} & \small{1,04} & \small{A} & \small{1,56} & \small{16,65°} & \small{42,92°}\\
\small{14} & \small{1} & \small{1} & \small{0,90} & \small{1,13} & \small{1,12} & \small{A} & \small{1,45} & \small{31,40°} & \small{99,58°}\\
\small{15} & \small{-} & \small{1} & \small{1,01} & \small{1,04} & \small{1,11} & \small{C} & \small{1,56} & \small{-} & \small{155,85°}\\
\small{16} & \small{1} & \small{1} & \small{1,01} & \small{1,15} & \small{1,25} & \small{C} & \small{1,18} & \small{31,73°} & \small{32,79°}\\
\small{17} & \small{2} & \small{1} & \small{0,93} & \small{1,62} & \small{1,56} & \small{A} & \small{1,55} & \small{149,54°} & \small{43,61°}\\
\small{18} & \small{-} & \small{1} & \small{1,00} & \small{1,00} & \small{1,02} & \small{C} & \small{1,25} & \small{-} & \small{167,83°}\\
\small{19} & \small{1} & \small{1} & \small{1,01} & \small{0,93} & \small{0,89} & \small{B} & \small{1,10} & \small{153,93°} & \small{162,80°}\\
\small{20} & \small{1} & \small{-} & \small{0,86} & \small{0,91} & \small{0,98} & \small{D} & \small{1,62} & \small{21,40°} & \small{-}\\
\small{21} & \small{1} & \small{1} & \small{1,10} & \small{1,64} & \small{1,70} & \small{C} & \small{1,27} & \small{16,40°} & \small{28,24°}\\
\small{22} & \small{3} & \small{-} & \small{1,04} & \small{1,12} & \small{1,10} & \small{C} & \small{1,28} & \small{28,97°} & \small{-}\\
\small{23} & \small{1} & \small{-} & \small{1,01} & \small{0,98} & \small{0,97} & \small{B} & \small{1,69} & \small{12,94°} & \small{-}\\
\small{24} & \small{-} & \small{1} & \small{0,99} & \small{1,32} & \small{1,22} & \small{A} & \small{1,30} & \small{-} & \small{159,40°}\\
\small{25} & \small{1} & \small{2} & \small{0,82} & \small{1,12} & \small{1,25} & \small{A} & \small{1,59} & \small{91,79°} & \small{64,41°}\\
\small{26} & \small{1} & \small{-} & \small{1,03} & \small{1,19} & \small{1,29} & \small{C} & \small{1,41} & \small{52,25°} & \small{-}\\
\small{27} & \small{-} & \small{1} & \small{1,06} & \small{0,98} & \small{0,98} & \small{B} & \small{1,30} & \small{-} & \small{43,28°}\\
\small{28} & \small{1} & \small{1} & \small{1,10} & \small{1,14} & \small{1,18} & \small{C} & \small{1,46} & \small{15,07°} & \small{127,53°}\\
\small{29} & \small{1} & \small{-} & \small{0,94} & \small{1,14} & \small{1,17} & \small{A} & \small{1,45} & \small{87,78°} & \small{-}\\
\small{30} & \small{1} & \small{1} & \small{0,97} & \small{1,18} & \small{1,20} & \small{A} & \small{1,60} & \small{19,83°} & \small{12,19°}\\
\small{31} & \small{1} & \small{-} & \small{1,08} & \small{1,12} & \small{1,13} & \small{C} & \small{1,27} & \small{43,13°} & \small{-}\\
\small{32} & \small{-} & \small{1} & \small{1,02} & \small{1,30} & \small{1,21} & \small{C} & \small{1,27} & \small{-} & \small{33,98°}\\
\small{33} & \small{1} & \small{3} & \small{0,99} & \small{1,22} & \small{1,23} & \small{A} & \small{1,39} & \small{13,29°} & \small{90,42°}\\
\small{34} & \small{1} & \small{2} & \small{0,88} & \small{1,51} & \small{1,65} & \small{A} & \small{1,44} & \small{138,71°} & \small{99,65°}\\
\small{35} & \small{1} & \small{-} & \small{0,93} & \small{0,95} & \small{0,97} & \small{D} & \small{1,54} & \small{33,13°} & \small{-}\\
\small{36} & \small{1} & \small{-} & \small{1,03} & \small{0,89} & \small{0,92} & \small{B} & \small{1,24} & \small{96,50°} & \small{-}\\
\small{37} & \small{-} & \small{1} & \small{1,00} & \small{1,25} & \small{1,13} & \small{C} & \small{1,45} & \small{-} & \small{5,45°}\\
\small{38} & \small{1} & \small{-} & \small{1,04} & \small{1,14} & \small{1,14} & \small{C} & \small{1,33} & \small{164,28°} & \small{-}\\
\small{39} & \small{-} & \small{1} & \small{1,06} & \small{1,16} & \small{1,15} & \small{C} & \small{1,51} & \small{-} & \small{138,81°}\\
\small{40} & \small{-} & \small{1} & \small{0,88} & \small{0,95} & \small{1,03} & \small{A} & \small{1,57} & \small{-} & \small{175,00°}\\
\small{41} & \small{1} & \small{1} & \small{1,10} & \small{0,94} & \small{1,00} & \small{B} & \small{1,69} & \small{160,35°} & \small{60,81°}\\
\small{42} & \small{1} & \small{-} & \small{1,00} & \small{1,03} & \small{1,09} & \small{C} & \small{2,16} & \small{16,83°} & \small{-}\\
\small{43} & \small{1} & \small{-} & \small{1,05} & \small{1,24} & \small{1,14} & \small{C} & \small{2,17} & \small{41,40°} & \small{-}\\
\small{44} & \small{1} & \small{-} & \small{1,02} & \small{1,07} & \small{1,10} & \small{C} & \small{1,65} & \small{40,27°} & \small{-}\\
\small{45} & \small{-} & \small{1} & \small{1,07} & \small{1,20} & \small{1,24} & \small{C} & \small{1,83} & \small{-} & \small{103,82°}\\
\small{46} & \small{1} & \small{-} & \small{1,05} & \small{1,45} & \small{1,32} & \small{C} & \small{1,83} & \small{105,28°} & \small{-}\\
\small{47} & \small{1} & \small{1} & \small{1,07} & \small{1,11} & \small{1,13} & \small{C} & \small{1,48} & \small{20,37°} & \small{132,73°}\\
\small{48} & \small{1} & \small{-} & \small{1,01} & \small{1,24} & \small{1,18} & \small{C} & \small{1,97} & \small{24,33°} & \small{-}\\
\small{49} & \small{-} & \small{1} & \small{1,03} & \small{1,14} & \small{1,08} & \small{C} & \small{2,42} & \small{-} & \small{72,45°}\\
\small{50} & \small{2} & \small{-} & \small{1,18} & \small{1,36} & \small{1,24} & \small{C} & \small{1,35} & \small{108,21°} & \small{-}\\
\small{51} & \small{1} & \small{-} & \small{1,01} & \small{0,97} & \small{1,01} & \small{C} & \small{1,41} & \small{78,65°} & \small{-}\\
\small{52} & \small{-} & \small{1} & \small{1,00} & \small{1,02} & \small{1,06} & \small{C} & \small{1,43} & \small{-} & \small{129,00°}\\
\small{53} & \small{2} & \small{1} & \small{0,91} & \small{1,05} & \small{1,12} & \small{A} & \small{1,83} & \small{20,62°} & \small{62,19°}\\
\small{54} & \small{-} & \small{1} & \small{1,00} & \small{1,67} & \small{1,49} & \small{C} & \small{1,32} & \small{-} & \small{13,32°}\\
\small{55} & \small{-} & \small{1} & \small{1,00} & \small{1,75} & \small{1,56} & \small{A} & \small{1,28} & \small{-} & \small{52,93°}\\
\small{56} & \small{1} & \small{-} & \small{1,03} & \small{1,67} & \small{1,43} & \small{C} & \small{1,34} & \small{110,07°} & \small{-}\\
\small{57} & \small{1} & \small{-} & \small{1,01} & \small{0,97} & \small{1,02} & \small{C} & \small{1,43} & \small{122,67°} & \small{-}\\
\small{58} & \small{1} & \small{-} & \small{1,02} & \small{1,76} & \small{1,56} & \small{C} & \small{1,27} & \small{101,90°} & \small{-}\\
\small{59} & \small{1} & \small{-} & \small{0,99} & \small{0,76} & \small{0,88} & \small{D} & \small{2,98} & \small{161,42°} & \small{-}\\
\small{60} & \small{1} & \small{-} & \small{1,01} & \small{1,10} & \small{1,15} & \small{C} & \small{2,04} & \small{133,01°} & \small{-}\\
\small{61} & \small{1} & \small{-} & \small{1,00} & \small{1,27} & \small{1,13} & \small{C} & \small{1,32} & \small{137,89°} & \small{-}\\
\small{62} & \small{1} & \small{-} & \small{1,01} & \small{0,78} & \small{0,78} & \small{B} & \small{1,18} & \small{114,35°} & \small{-}\\
\small{63} & \small{1} & \small{1} & \small{0,90} & \small{1,03} & \small{0,93} & \small{D} & \small{1,44} & \small{87,37°} & \small{55,02°}\\
\small{64} & \small{-} & \small{1} & \small{1,01} & \small{1,35} & \small{1,24} & \small{C} & \small{1,78} & \small{-} & \small{86,35°}\\
\small{65} & \small{-} & \small{2} & \small{1,04} & \small{1,17} & \small{1,21} & \small{C} & \small{1,58} & \small{-} & \small{49,42°}\\
\small{66} & \small{1} & \small{-} & \small{1,08} & \small{1,47} & \small{1,52} & \small{C} & \small{1,25} & \small{142,64°} & \small{-}\\
\small{67} & \small{-} & \small{1} & \small{1,02} & \small{1,22} & \small{1,23} & \small{C} & \small{1,70} & \small{-} & \small{5,12°}\\
\small{68} & \small{1} & \small{1} & \small{1,42} & \small{1,38} & \small{1,23} & \small{C} & \small{1,46} & \small{16,15°} & \small{37,81°}\\
\small{69} & \small{1} & \small{3} & \small{1,21} & \small{1,51} & \small{1,43} & \small{C} & \small{1,22} & \small{11,36°} & \small{85,13°}\\
\small{70} & \small{1} & \small{1} & \small{1,02} & \small{1,11} & \small{1,12} & \small{C} & \small{1,23} & \small{91,72°} & \small{147,27°}\\
\small{71} & \small{-} & \small{1} & \small{1,11} & \small{1,28} & \small{1,22} & \small{C} & \small{1,39} & \small{-} & \small{111,22°}\\
\small{72} & \small{1} & \small{1} & \small{1,00} & \small{1,09} & \small{1,13} & \small{A} & \small{1,21} & \small{88,12°} & \small{102,05°}\\
\small{73} & \small{1} & \small{-} & \small{1,01} & \small{1,11} & \small{1,18} & \small{C} & \small{1,66} & \small{131,68°} & \small{-}\\
\small{74} & \small{-} & \small{1} & \small{1,28} & \small{0,94} & \small{0,93} & \small{B} & \small{1,84} & \small{-} & \small{154,09°}\\
\small{75} & \small{1} & \small{-} & \small{1,40} & \small{1,52} & \small{1,48} & \small{C} & \small{1,52} & \small{74,35°} & \small{-}\\
\small{76} & \small{1} & \small{1} & \small{1,15} & \small{0,88} & \small{0,78} & \small{B} & \small{1,39} & \small{13,09°} & \small{54,93°}\\
\small{77} & \small{1} & \small{-} & \small{0,98} & \small{1,27} & \small{1,25} & \small{A} & \small{1,45} & \small{98,64°} & \small{-}\\
\small{78} & \small{-} & \small{1} & \small{1,02} & \small{1,18} & \small{1,34} & \small{C} & \small{1,26} & \small{-} & \small{51,13°}\\
\small{79} & \small{1} & \small{-} & \small{1,03} & \small{1,38} & \small{1,27} & \small{C} & \small{1,66} & \small{33,14°} & \small{-}\\
\small{80} & \small{-} & \small{1} & \small{1,00} & \small{1,38} & \small{1,23} & \small{C} & \small{1,49} & \small{-} & \small{28,46°}\\
\small{81} & \small{1} & \small{1} & \small{1,00} & \small{1,61} & \small{1,50} & \small{C} & \small{1,32} & \small{51,36°} & \small{43,36°}\\
\small{82} & \small{3} & \small{1} & \small{1,04} & \small{0,91} & \small{0,88} & \small{B} & \small{1,91} & \small{5,46°} & \small{57,89°}\\
\small{83} & \small{1} & \small{1} & \small{1,03} & \small{1,18} & \small{1,21} & \small{C} & \small{1,35} & \small{131,25°} & \small{157,12°}\\
\small{84} & \small{1} & \small{-} & \small{1,03} & \small{1,08} & \small{1,09} & \small{C} & \small{1,63} & \small{58,17°} & \small{-}\\
\small{85} & \small{1} & \small{1} & \small{1,01} & \small{1,15} & \small{1,21} & \small{C} & \small{1,18} & \small{75,63°} & \small{162,07°}\\
\small{86} & \small{1} & \small{-} & \small{1,05} & \small{1,37} & \small{1,33} & \small{C} & \small{1,76} & \small{12,20°} & \small{-}\\
\small{87} & \small{-} & \small{1} & \small{1,03} & \small{1,05} & \small{1,15} & \small{C} & \small{1,25} & \small{-} & \small{124,44°}\\
\small{88} & \small{1} & \small{1} & \small{1,04} & \small{1,14} & \small{1,09} & \small{C} & \small{1,23} & \small{28,24°} & \small{155,03°}\\
\small{89} & \small{-} & \small{1} & \small{1,00} & \small{1,07} & \small{1,12} & \small{A} & \small{1,38} & \small{-} & \small{22,79°}\\
\small{90} & \small{-} & \small{1} & \small{1,00} & \small{1,26} & \small{1,09} & \small{A} & \small{1,30} & \small{-} & \small{74,26°}\\
\small{91} & \small{1} & \small{3} & \small{1,03} & \small{0,93} & \small{0,83} & \small{B} & \small{1,92} & \small{95,79°} & \small{84,67°}\\
\small{92} & \small{-} & \small{1} & \small{1,01} & \small{1,06} & \small{1,11} & \small{C} & \small{1,10} & \small{-} & \small{51,04°}\\
\small{93} & \small{1} & \small{-} & \small{1,08} & \small{1,28} & \small{1,15} & \small{C} & \small{1,75} & \small{96,78°} & \small{-}\\
\small{94} & \small{1} & \small{-} & \small{1,15} & \small{0,78} & \small{0,79} & \small{B} & \small{1,92} & \small{126,79°} & \small{-}\\
\small{95} & \small{1} & \small{1} & \small{1,01} & \small{1,63} & \small{1,53} & \small{C} & \small{1,26} & \small{149,04°} & \small{29,81°}\\
\small{96} & \small{1} & \small{1} & \small{1,02} & \small{1,19} & \small{1,28} & \small{C} & \small{1,09} & \small{79,27°} & \small{175,79°}\\
\small{97} & \small{-} & \small{1} & \small{1,00} & \small{1,07} & \small{1,09} & \small{C} & \small{1,57} & \small{-} & \small{42,03°}\\
\small{98} & \small{1} & \small{1} & \small{1,01} & \small{1,36} & \small{1,31} & \small{C} & \small{1,26} & \small{117,33°} & \small{14,81°}\\
\small{99} & \small{-} & \small{1} & \small{1,00} & \small{1,05} & \small{1,01} & \small{C} & \small{1,64} & \small{-} & \small{7,41°}\\
\small{100} & \small{1} & \small{1} & \small{1,31} & \small{2,04} & \small{2,05} & \small{C} & \small{1,83} & \small{70,73°} & \small{48,71°}\\
\small{101} & \small{-} & \small{1} & \small{1,02} & \small{1,46} & \small{1,37} & \small{C} & \small{1,78} & \small{-} & \small{15,66°}\\
\small{102} & \small{1} & \small{-} & \small{1,16} & \small{1,43} & \small{1,57} & \small{C} & \small{2,12} & \small{29,68°} & \small{-}\\
\small{103} & \small{1} & \small{-} & \small{1,02} & \small{1,24} & \small{1,17} & \small{C} & \small{1,51} & \small{31,64°} & \small{-}\\
\small{104} & \small{-} & \small{1} & \small{1,01} & \small{1,04} & \small{1,08} & \small{C} & \small{1,58} & \small{-} & \small{171,37°}\\
\small{105} & \small{-} & \small{1} & \small{1,01} & \small{1,15} & \small{1,15} & \small{C} & \small{1,33} & \small{-} & \small{15,32°}\\
\small{106} & \small{-} & \small{1} & \small{1,02} & \small{1,28} & \small{1,36} & \small{C} & \small{1,49} & \small{-} & \small{54,04°}\\
\small{107} & \small{-} & \small{1} & \small{1,05} & \small{1,28} & \small{1,09} & \small{C} & \small{1,67} & \small{-} & \small{90,66°}\\
\small{108} & \small{1} & \small{1} & \small{1,00} & \small{1,03} & \small{0,84} & \small{D} & \small{1,18} & \small{64,55°} & \small{63,04°}\\
\small{109} & \small{1} & \small{1} & \small{1,01} & \small{2,02} & \small{2,03} & \small{C} & \small{1,21} & \small{57,69°} & \small{63,92°}\\
\small{110} & \small{-} & \small{1} & \small{1,00} & \small{0,98} & \small{0,97} & \small{D} & \small{1,25} & \small{-} & \small{159,44°}\\
\small{111} & \small{-} & \small{1} & \small{1,01} & \small{1,13} & \small{1,11} & \small{C} & \small{1,49} & \small{-} & \small{25,97°}\\
\small{112} & \small{2} & \small{-} & \small{1,06} & \small{1,32} & \small{1,31} & \small{C} & \small{1,28} & \small{148,57°} & \small{-}\\
\small{113} & \small{-} & \small{1} & \small{1,02} & \small{1,39} & \small{1,27} & \small{C} & \small{1,54} & \small{-} & \small{34,25°}\\
\small{114} & \small{1} & \small{2} & \small{1,01} & \small{1,15} & \small{1,32} & \small{C} & \small{1,10} & \small{149,38°} & \small{149,33°}\\
\small{115} & \small{-} & \small{1} & \small{1,10} & \small{1,52} & \small{1,37} & \small{C} & \small{2,39} & \small{-} & \small{152,83°}\\
\small{116} & \small{2} & \small{-} & \small{0,87} & \small{1,43} & \small{1,60} & \small{A} & \small{1,81} & \small{129,78°} & \small{-}\\
\small{117} & \small{1} & \small{1} & \small{1,13} & \small{1,27} & \small{1,05} & \small{C} & \small{2,14} & \small{3,50°} & \small{27,87°}\\
\small{118} & \small{1} & \small{-} & \small{1,01} & \small{0,84} & \small{0,78} & \small{B} & \small{1,38} & \small{69,72°} & \small{-}\\
\small{119} & \small{1} & \small{-} & \small{1,01} & \small{1,21} & \small{1,35} & \small{C} & \small{1,26} & \small{129,12°} & \small{-}\\
\small{120} & \small{1} & \small{-} & \small{1,01} & \small{0,99} & \small{1,00} & \small{C} & \small{1,28} & \small{59,86°} & \small{-}\\
\small{121} & \small{1} & \small{1} & \small{1,13} & \small{0,91} & \small{0,76} & \small{B} & \small{1,33} & \small{9,74°} & \small{113,78°}\\
\small{122} & \small{-} & \small{1} & \small{1,00} & \small{1,05} & \small{1,09} & \small{C} & \small{1,27} & \small{-} & \small{4,08°}\\
\small{123} & \small{1} & \small{-} & \small{1,03} & \small{0,90} & \small{0,95} & \small{B} & \small{1,69} & \small{25,24°} & \small{-}\\
\small{124} & \small{1} & \small{-} & \small{1,15} & \small{1,81} & \small{1,64} & \small{C} & \small{2,00} & \small{135,28°} & \small{-}\\
\small{125} & \small{1} & \small{1} & \small{1,03} & \small{1,07} & \small{0,94} & \small{B} & \small{1,65} & \small{81,59°} & \small{142,24°}\\
\small{126} & \small{1} & \small{-} & \small{1,02} & \small{1,09} & \small{1,14} & \small{C} & \small{1,55} & \small{106,66°} & \small{-}\\
\small{127} & \small{-} & \small{3} & \small{1,10} & \small{1,32} & \small{1,39} & \small{C} & \small{1,81} & \small{-} & \small{75,73°}\\
\small{128} & \small{1} & \small{-} & \small{1,00} & \small{1,06} & \small{1,16} & \small{C} & \small{1,70} & \small{57,76°} & \small{-}\\
\small{129} & \small{1} & \small{-} & \small{1,10} & \small{1,33} & \small{1,27} & \small{C} & \small{1,79} & \small{24,08°} & \small{-}\\
    \hline
        \caption*{}
        \label{Tableau statistiques descriptives annexe des distances des E-TVS}
        \begin{flushright}
        \scriptsize
    Auteur~: \textcopyright~Moinse 2023
        \end{flushright}
        \end{longtable}

    % ___________________________________________
    % ANNEXE M~: Questionnaire FUB
    \newpage
\section{Annexes liées à l'association positive entre l'usage inclusif de la \gls{mobilité individuelle légère} et la \gls{cyclabilité}}
    \label{donnees-ouvertes:cyclabilite_genre}
    \markboth{Annexes liées à l'association positive entre l'usage inclusif de la mobilité individuelle légère et la cyclabilité}{}
    \markright{Annexes liées à l'association positive entre l'usage inclusif de la mobilité individuelle légère et la cyclabilité}{}

    \newpage
    % Annexe M2
\subsection{Statistiques descriptives des divers indicateurs mesurés en lien avec le genre et la mobilité}
    \label{donnees-ouvertes:calcul_cyclabilite_genre}

La présente annexe se réfère à la \hyperref[La micro-mobilité et la cyclabilité des territoires mises en relation avec une perspective sur le genre]{sous-section consacrée à l'association positive entre la part modale genrée de la mobilité individuelle légère et la cyclabilité des cinquante-trois villes françaises étudiées} (page \pageref{La micro-mobilité et la cyclabilité des territoires mises en relation avec une perspective sur le genre}), dans le cadre du \hyperref[chap6:titre]{chapitre 6} (page \pageref{chap6:titre}). Le tableau, ci-après, présente de manière synthétique les principaux résultats statistiques obtenus, en détaillant les principaux indicateurs mobilisés.\par

Les indicateurs mesurés sont les suivants~:
\begin{itemize}
    \item $I_{gcb}$~: le score agréageant la proportion d'usagères à vélo, la cyclabilité et la part modale du vélo dans la commune ;
    \item $I_{g}$~: la proportion d'usagères à vélo dans la commune (sur 50\%) ;
    \item $I_{b}$~: la cyclabilité de la commune (sur 6) ;
    \item $I_{c}$~: la part modale du vélo dans la commune (sur 25\%) ;
    \item PMV~: la part modale du vélo dans la commune (en \%) ;
    \item Densité~: la densité de population dans la commune (en hab/km²) ;
    \item AC~: le taux d'aménagements cyclables au sein du réseau viaire de la commune (en \%) ;
    \item AC30~: le taux d'aménagements cyclables et de Zones 30 au sien du réseau viaire de la commune (en \%) ;
\end{itemize}
    
% Tableau D
        \begin{longtable}{p{0.5cm}p{4cm}p{0.5cm}p{0.5cm}p{0.5cm}p{0.5cm}p{1cm}p{1cm}p{1cm}p{1cm}}
        \hline
        \small{\textcolor{blue}{\textbf{ID}}} & \small{\textcolor{blue}{\textbf{Ville-centre}}} & \small{\textcolor{blue}{\textbf{$I_{gcb}$}}} & \small{\textcolor{blue}{\textbf{$I_{g}$}}} & \small{\textcolor{blue}{\textbf{$I_{b}$}}} & \small{\textcolor{blue}{\textbf{$I_{c}$}}} & \small{\textcolor{blue}{\textbf{PMV}}} & \small{\textcolor{blue}{\textbf{Densité}}} & \small{\textcolor{blue}{\textbf{AC}}} & \small{\textcolor{blue}{\textbf{AC30}}}\\
        \hline
        \endhead
    \small{1} & \small{Strasbourg} & \small{\textbf{0,47}} & \small{0,98} & \small{0,70} & \small{0,68} & \small{17\%} & \small{3 713} & \small{26\%} & \small{37\%}\\
    \small{2} & \small{Grenoble} & \small{\textbf{0,44}} & \small{0,90} & \small{0,70} & \small{0,70} & \small{17\%} & \small{8 728} & \small{36\%} & \small{85\%}\\
    \small{3} & \small{La Rochelle} & \small{\textbf{0,34}} & \small{1,00} & \small{0,69} & \small{0,49} & \small{12\%} & \small{2 716} & \small{18\%} & \small{26\%}\\
    \small{4} & \small{Bordeaux} & \small{\textbf{0,32}} & \small{1,00} & \small{0,57} & \small{0,57} & \small{14\%} & \small{5 264} & \small{23\%} & \small{41\%}\\
    \small{5} & \small{Rennes} & \small{\textbf{0,22}} & \small{0,87} & \small{0,62} & \small{0,40} & \small{10\%} & \small{4 415} & \small{33\%} & \small{60\%}\\
    \small{6} & \small{Nantes} & \small{\textbf{0,21}} & \small{0,88} & \small{0,61} & \small{0,40} & \small{10\%} & \small{4 920} & \small{27\%} & \small{86\%}\\
    \small{7} & \small{Angers} & \small{\textbf{0,17}} & \small{0,90} & \small{0,57} & \small{0,33} & \small{8\%} & \small{3 650} & \small{20\%} & \small{53\%}\\
    \small{8} & \small{Lyon} & \small{\textbf{0,17}} & \small{0,81} & \small{0,59} & \small{0,35} & \small{9\%} & \small{10 909} & \small{26\%} & \small{47\%}\\
    \small{9} & \small{Chambéry} & \small{\textbf{0,16}} & \small{0,80} & \small{0,63} & \small{0,31} & \small{8\%} & \small{2 819} & \small{19\%} & \small{35\%}\\
    \small{10} & \small{Annecy} & \small{\textbf{0,15}} & \small{0,91} & \small{0,62} & \small{0,26} & \small{7\%} & \small{1 969} & \small{15\%} & \small{34\%}\\
    \small{11} & \small{Tours} & \small{\textbf{0,15}} & \small{0,90} & \small{0,57} & \small{0,28} & \small{7\%} & \small{3 976} & \small{22\%} & \small{71\%}\\
    \small{12} & \small{Toulouse} & \small{\textbf{0,14}} & \small{0,83} & \small{0,49} & \small{0,35} & \small{9\%} & \small{4 210} & \small{21\%} & \small{44\%}\\
    \small{13} & \small{Avignon} & \small{\textbf{0,14}} & \small{1,00} & \small{0,52} & \small{0,27} & \small{7\%} & \small{1 396} & \small{22\%} & \small{26\%}\\
    \small{14} & \small{Montpellier} & \small{\textbf{0,13}} & \small{0,79} & \small{0,52} & \small{0,32} & \small{8\%} & \small{5 285} & \small{19\%} & \small{84\%}\\
    \small{15} & \small{Valence} & \small{\textbf{0,11}} & \small{0,85} & \small{0,54} & \small{0,23} & \small{6\%} & \small{1 756} & \small{18\%} & \small{34\%}\\
    \small{16} & \small{Caen} & \small{\textbf{0,11}} & \small{0,79} & \small{0,57} & \small{0,24} & \small{6\%} & \small{4 173} & \small{21\%} & \small{29\%}\\
    \small{17} & \small{Dijon} & \small{\textbf{0,10}} & \small{0,78} & \small{0,53} & \small{0,25} & \small{6\%} & \small{3 937} & \small{14\%} & \small{21\%}\\
    \small{18} & \small{Lorient} & \small{\textbf{0,10}} & \small{0,69} & \small{0,61} & \small{0,24} & \small{6\%} & \small{3 284} & \small{18\%} & \small{79\%}\\
    \small{19} & \small{Lille} & \small{\textbf{0,10}} & \small{0,77} & \small{0,51} & \small{0,25} & \small{6\%} & \small{6 783} & \small{23\%} & \small{73\%}\\
    \small{20} & \small{Paris} & \small{\textbf{0,10}} & \small{0,82} & \small{0,55} & \small{0,22} & \small{5\%} & \small{20 360} & \small{28\%} & \small{91\%}\\
    \small{21} & \small{Orléans} & \small{\textbf{0,09}} & \small{0,83} & \small{0,50} & \small{0,22} & \small{5\%} & \small{4 259} & \small{22\%} & \small{38\%}\\
    \small{22} & \small{Poitiers} & \small{\textbf{0,09}} & \small{0,90} & \small{0,55} & \small{0,18} & \small{5\%} & \small{2 138} & \small{14\%} & \small{18\%}\\
    \small{23} & \small{Le Mans} & \small{\textbf{0,08}} & \small{0,79} & \small{0,52} & \small{0,20} & \small{5\%} & \small{2 749} & \small{13\%} & \small{22\%}\\
    \small{24} & \small{Besançon} & \small{\textbf{0,08}} & \small{0,71} & \small{0,53} & \small{0,20} & \small{5\%} & \small{1 818} & \small{21\%} & \small{41\%}\\
    \small{25} & \small{Troyes} & \small{\textbf{0,07}} & \small{0,77} & \small{0,47} & \small{0,20} & \small{5\%} & \small{4 742} & \small{16\%} & \small{25\%}\\
    \small{26} & \small{Nancy} & \small{\textbf{0,07}} & \small{0,76} & \small{0,50} & \small{0,18} & \small{4\%} & \small{6 956} & \small{19\%} & \small{46\%}\\
    \small{27} & \small{Amiens} & \small{\textbf{0,07}} & \small{0,72} & \small{0,44} & \small{0,21} & \small{5\%} & \small{2 696} & \small{14\%} & \small{21\%}\\
    \small{28} & \small{Dunkerque} & \small{\textbf{0,06}} & \small{0,77} & \small{0,61} & \small{0,14} & \small{3\%} & \small{1 972} & \small{17\%} & \small{34\%}\\
    \small{29} & \small{Saint-Nazaire} & \small{\textbf{0,06}} & \small{0,63} & \small{0,55} & \small{0,18} & \small{4\%} & \small{1 536} & \small{11\%} & \small{13\%}\\
    \small{30} & \small{Rouen} & \small{\textbf{0,06}} & \small{0,80} & \small{0,52} & \small{0,15} & \small{4\%} & \small{5 341} & \small{22\%} & \small{53\%}\\
    \small{31} & \small{Mulhouse} & \small{\textbf{0,06}} & \small{0,71} & \small{0,51} & \small{0,16} & \small{4\%} & \small{4 871} & \small{15\%} & \small{42\%}\\
    \small{32} & \small{Nîmes} & \small{\textbf{0,05}} & \small{0,78} & \small{0,38} & \small{0,18} & \small{4\%} & \small{911} & \small{5\%} & \small{8\%}\\
    \small{33} & \small{Bayonne} & \small{\textbf{0,05}} & \small{0,74} & \small{0,52} & \small{0,12} & \small{3\%} & \small{2 399} & \small{19\%} & \small{47\%}\\
    \small{34} & \small{Clermont-Ferrand} & \small{\textbf{0,05}} & \small{0,65} & \small{0,46} & \small{0,15} & \small{4\%} & \small{3 452} & \small{9\%} & \small{29\%}\\
    \small{35} & \small{Reims} & \small{\textbf{0,04}} & \small{0,66} & \small{0,49} & \small{0,14} & \small{3\%} & \small{3 845} & \small{12\%} & \small{27\%}\\
    \small{36} & \small{Le Havre} & \small{\textbf{0,04}} & \small{0,69} & \small{0,57} & \small{0,11} & \small{3\%} & \small{3 532} & \small{13\%} & \small{21\%}\\
    \small{37} & \small{Valenciennes} & \small{\textbf{0,04}} & \small{0,84} & \small{0,36} & \small{0,12} & \small{3\%} & \small{3 093} & \small{13\%} & \small{32\%}\\
    \small{38} & \small{Brest} & \small{\textbf{0,03}} & \small{0,50} & \small{0,52} & \small{0,12} & \small{3\%} & \small{2 817} & \small{10\%} & \small{39\%}\\
    \small{39} & \small{Metz} & \small{\textbf{0,03}} & \small{0,67} & \small{0,48} & \small{0,09} & \small{2\%} & \small{2 866} & \small{16\%} & \small{30\%}\\
    \small{40} & \small{Perpignan} & \small{\textbf{0,03}} & \small{0,69} & \small{0,42} & \small{0,10} & \small{3\%} & \small{1 734} & \small{9\%} & \small{37\%}\\
    \small{41} & \small{Douai} & \small{\textbf{0,03}} & \small{0,63} & \small{0,57} & \small{0,08} & \small{2\%} & \small{2 360} & \small{22\%} & \small{40\%}\\
    \small{42} & \small{Aix-en-Provence} & \small{\textbf{0,03}} & \small{0,51} & \small{0,43} & \small{0,13} & \small{3\%} & \small{791} & \small{8\%} & \small{16\%}\\
    \small{43} & \small{Nice} & \small{\textbf{0,03}} & \small{0,69} & \small{0,43} & \small{0,09} & \small{2\%} & \small{4 776} & \small{5\%} & \small{16\%}\\
    \small{44} & \small{Toulon} & \small{\textbf{0,02}} & \small{0,40} & \small{0,38} & \small{0,15} & \small{4\%} & \small{4 194} & \small{6\%} & \small{10\%}\\
    \small{45} & \small{Saint-Paul (La Réunion)} & \small{\textbf{0,02}} & \small{0,44} & \small{0,43} & \small{0,12} & \small{3\%} & \small{432} & \small{NC} & \small{NC}\\
    \small{46} & \small{Beauvais} & \small{\textbf{0,02}} & \small{0,56} & \small{0,46} & \small{0,09} & \small{2\%} & \small{1 708} & \small{12\%} & \small{22\%}\\
    \small{47} & \small{Limoges} & \small{\textbf{0,02}} & \small{0,55} & \small{0,45} & \small{0,06} & \small{2\%} & \small{1 674} & \small{6\%} & \small{10\%}\\
    \small{48} & \small{Marseille} & \small{\textbf{0,02}} & \small{0,67} & \small{0,32} & \small{0,07} & \small{2\%} & \small{3 617} & \small{6\%} & \small{13\%}\\
    \small{49} & \small{Saint-Etienne} & \small{\textbf{0,02}} & \small{0,57} & \small{0,41} & \small{0,06} & \small{2\%} & \small{2 177} & \small{8\%} & \small{12\%}\\
    \small{50} & \small{Saint-Denis (La Réunion)} & \small{\textbf{0,01}} & \small{0,51} & \small{0,40} & \small{0,06} & \small{2\%} & \small{1 072} & \small{NC} & \small{NC}\\
    \small{51} & \small{Angoulême} & \small{\textbf{0,01}} & \small{0,69} & \small{0,45} & \small{0,04} & \small{1\%} & \small{1 895} & \small{8\%} & \small{12\%}\\
    \small{52} & \small{Arras} & \small{\textbf{0,01}} & \small{0,57} & \small{0,55} & \small{0,04} & \small{1\%} & \small{3 640} & \small{14\%} & \small{59\%}\\
    \small{53} & \small{Saint-Pierre (La Réunion)} & \small{\textbf{0,01}} & \small{0,49} & \small{0,35} & \small{0,06} & \small{2\%} & \small{874} & \small{NC} & \small{NC}\\
        \hline
        \caption*{}
        \label{Annexe résultats analyse MOBPro FUB}
        \begin{flushright}
        \scriptsize
        Jeux de données~: \textcolor{blue}{\textcite{fub_barometre_2021, insee_documentation_2023, insee_grille_2021, velo__territoires_atlas_2023}}, Auteur~: \textcopyright~Moinse 2023
        \end{flushright}
        \end{longtable}

    % ___________________________________________
    % ANNEXE N~: Texte et dessin MT180
    \newpage
\section{Vulgarisation scientifique de la recherche doctorale lors du concours \textsl{Ma Thèse en 180 Secondes} (MT180)}
    \label{donnees-ouvertes:mt180}
    \markboth{Vulgarisation scientifique de la recherche doctorale lors du concours Ma Thèse en 180 Secondes (MT180)}{}
    \markright{Vulgarisation scientifique de la recherche doctorale lors du concours Ma Thèse en 180 Secondes (MT180)}{}
    
%\end{refsegment}