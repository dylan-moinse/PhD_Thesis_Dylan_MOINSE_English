%------------------------------%
%% ✎ Dylan (V1) %%%%%%%%% ✅ %%
%% ✎ Alain (V2) %%%%%%%%% ✅ %%
%% ✎ Dylan (V3) %%%%%%%%% ✅ %%
%------------------------------%

%\cleardoublepage
    \needspace{1\baselineskip} % Reserve space
\chapter*{Reading Guide of the Manuscript
    \label{body:guide-lecture-manuscrit}
    }
    \markboth{Reading Guide of the Manuscript}{}
    \markright{Preface}{}
    \addcontentsline{toc}{part}{Reading Guide of the Manuscript}

% --- %
    % Ecriture inclusive
% \section*{Adoption de l'écriture inclusive
%     \label{subbody:adoption-ecriture-inclusive}
%     }
%     \addcontentsline{toc}{section}{Adoption de l'écriture inclusive}

% \lettrine[lines=3, findent=8pt, nindent=0pt]{\lettrinefont A}{dopter} une approche d'écriture inclusive vise à promouvoir une représentation équilibrée des genres dans la langue écrite, afin de lutter contre les stéréotypes de genre tout en mettant en valeur la diversité des personnes. L'écriture inclusive ne prétend pas complexifier la langue, mais plutôt rendre l'écriture plus représentative de la réalité sociale en nous invitant à repenser notre façon de communiquer. Au lieu d'alourdir la lecture, l'écriture inclusive offre bien au contraire des tournures grammaticales plus fluides et concises. D'après le manuel d'écriture inclusive publié par \textcolor{blue}{Raphaël} \textcolor{blue}{\textcite[19-23]{haddad_manuel_2019}}\index{Haddad, Raphaël|pagebf}, l'écriture inclusive permet de déconstruire dix arguments majeurs: (i) elle n'aurait pas d'influence sur nos représentations; (ii) serait trop compliquée à utiliser; (iii) encombrerait, et même (iv) défigurerait le texte; (v) elle serait interdite par certaines institutions, (vi) car elle menacerait la langue française, (vii) en promouvant la \textsl{novlangue}, (viii) et en ayant des règles hétérogènes; (ix) tandis que le masculin serait le marqueur du neutre, (x) et qu'il serait signe de prestige en français.%%Rédigé%%

% À cet égard, nous avons fait le choix d'inclure le point médian (·) qui intègre à la fois la forme féminine et masculine. Conscient·e·s que cette démarche nécessite un temps d'adaptation, nous avons fait le choix d'énumérer les quelques mots spécifiquement affectés par l'adoption de l'écriture inclusive afin de guider la lecture pour tou·te·s.%%Rédigé%%

% --- %
    % Types of Scientific Productions
    \needspace{1\baselineskip} % Reserve space
\section*{Nomenclature of Scientific Productions
    \label{subbody:nomenclature-productions-scientifiques}
    }
    \addcontentsline{toc}{section}{Nomenclature of Scientific Productions in Human and Social Sciences}

\acrfull{Hcéres} has established, in France, a nomenclature for scientific productions in \acrfull{HSS}. This framework aims to categorize the different types of scientific outputs produced by researchers, in the context of evaluating research institutions. The proposed coding is present throughout the doctoral thesis, within the sections listing scientific productions related to each chapter. The nomenclature is as follows \textcolor{blue}{\autocite{ministere_de_leducation_nationale_de_lenseignement_superieur_et_de_la_recherche_nomenclatures_nodate}}\index{Ministère de l'Éducation Nationale, de l'Enseignement Supérieur et de la Recherche@\textsl{Ministère de l'Éducation Nationale, de l'Enseignement Supérieur et de la Recherche}|pagebf}:%%Translated%%

    \needspace{1\baselineskip} % Reserve space
\subsubsection*{Scientific Publications:}
    \begin{customitemize}
\item \textbf{ACL}: Articles in international or national peer-reviewed journals listed by \acrshort{Hcéres} or in international databases;
\item \textbf{ACLN}: Articles in peer-reviewed journals not listed by \acrshort{Hcéres} or in international databases;
\item \textbf{ASCL}: Articles in journals without peer review;
\item \textbf{OS}: Scientific books (including critical editions and scientific translations);
\item \textbf{PT}: Transfer publications.
    \end{customitemize}

    \needspace{1\baselineskip} % Reserve space
\subsubsection*{Scientific Events:}
    \begin{customitemize}
\item \textbf{C-INV}: Keynote lectures given at the invitation of the Organizing Committee at a national or international conference;
\item \textbf{C-ACTI}: Presentations with proceedings at an international conference;
\item \textbf{C-ACTN}: Presentations with proceedings at a national conference;
\item \textbf{C-COM}: Oral presentations without proceedings at a national or international conference;
\item \textbf{C-AFF}: Poster presentations at a national or international conference.
    \end{customitemize}

    \needspace{1\baselineskip} % Reserve space
\subsubsection*{Dissemination of Scientific Culture:}
    \begin{customitemize}
\item \textbf{PV}: Popular science publications;
\item \textbf{PAT}: Theorized artistic productions.
    \end{customitemize}

    \needspace{1\baselineskip} % Reserve space
\subsubsection*{Other Productions:}
    \begin{customitemize}
\item \textbf{BRE}: Patents;
\item \textbf{DO}: Editorial direction of books or journals;
\item \textbf{OR}: Research tools;
\item \textbf{TH}: Doctoral theses;
\item \textbf{AP}: Other productions.
    \end{customitemize}

% --- %
    % LaTeX
    \needspace{1\baselineskip} % Reserve space
\section*{The Benefits of \latexword{\LaTeX} in \acrlong{HSS}
    \label{subbody:interet-latex}
    }
    \addcontentsline{toc}{section}{The Benefits of \latexword{\LaTeX} in Human and Social Sciences}

    % Writing in LaTeX
\lettrine[lines=3, findent=8pt, nindent=0pt]{\lettrinefont I}{n} closing this section dedicated to the reading recommendations of this manuscript, it is appropriate to inform the reader that this doctoral thesis has been written using the document preparation system \latexword{\LaTeX}. The decision to opt for this computer language, known as \Commas{light markup}, or \textsl{balisage léger} \textcolor{blue}{\autocite[16]{pochet_markdown_2023}}\index{Pochet, Bernard|pagebf}, should not remain unexplained, as a brief outline of its advantages may, we hope, encourage its adoption in our disciplines. Indeed, the use of \latexword{\LaTeX}, far beyond a mere stylistic preference, aligns with a commitment to academic excellence and the renewal of research practices. However, although \latexword{\LaTeX}-written books are widespread, few are specifically dedicated to \acrshort{HSS} \textcolor{blue}{\autocite[7]{rouquette_xelatex_2012}}\index{Rouquette, Maïeul|pagebf}.%%Translated%%

    % Advantages of LaTeX
Building on the argumentation developed by \textcolor{blue}{\textcite[8-9]{rouquette_xelatex_2012}}\index{Rouquette, Maïeul|pagebf} in his work titled \textsl{(Xe)LaTeX Applied to the Humanities} (\textsl{(Xe)LaTeX appliqué aux sciences humaines}), it appears that \acrfull{WYSIWYG} software, such as \Marque{Microsoft Office Word}\footnote{
    \Marque{Microsoft Office Word} (\url{https://www.microsoft.com/fr-fr/microsoft-365/word}) is a word processing software, distributed since 1983 and is now part of the \textsl{Microsoft Office} suite.
} or \Marque{LibreOffice}\footnote{
    \Marque{LibreOffice} (\url{https://www.libreoffice.org/}) is a free and open-source office suite, developed by the non-profit organization \Marque{Document Foundation}.
}, face various challenges in word processing, especially for long documents. In contrast, the \latexword{\LaTeX} system and language stand out for their typographic precision, facilitating the management of large documents and the handling of bibliographic references. The separation between the text editor and the compiler in \latexword{\LaTeX} allows for a clear distinction between content and formatting, with the document being produced in an open PDF format (\textsl{Portable Document Format}). From our perspective, the use of \latexword{\LaTeX} is justified by numerous advantages in the context of our doctoral research:
    \begin{customitemize}
\item Reproducibility and longevity of the manuscript, independent of software evolution;
\item Free and open-source computer language, reliable for over three decades, compatible with all operating systems;
\item Easy access via the PDF format;
\item Optimized management and time-saving for long documents, such as a doctoral thesis;
\item Compliance with academic publisher standards, who favor this format for scientific publications;
\item Collaborative nature facilitated by an online text editor such as \Marque{Overleaf}, which offers proofreading and commenting features\footnote{
    \Marque{Overleaf} (\url{https://www.overleaf.com/}) is a \latexword{\LaTeX} editor that combines a code editor with a preview, while enabling real-time collaboration with sharing and versioning features. The online platform integrates reference management software like \Marque{Zotero} and \Marque{Mendeley}.
};
\item Recognition of professional typography due to full support for typographic rules;
\item Simplified management of bibliographies and mathematical formulas;
\item Advanced customization through commands, enhanced by additional modules (\textsl{packages});
\item Ability to include content and annotations not visible in the final output;
\item Support and contribution from a particularly active \latexword{\LaTeX} community.
    \end{customitemize}%%Translated%%

    % Acknowledgements LaTeX
From a more personal perspective, I would like to express my deep gratitude to \textcolor{blue}{Jorge Mariano}\index{Mariano, Jorge|pagebf}, Research Engineer at Gustave Eiffel University, for his significant contribution to the design of a \latexword{\LaTeX} template that meets the visual standards of Gustave Eiffel University, as well as for his ongoing support. This manuscript partially incorporates this template and has been enhanced with modifications aimed at better customization for the specific needs of our discipline and our individual preferences. The adoption of \latexword{\LaTeX} was also made possible thanks to the valuable advice of my thesis supervisor, \textcolor{blue}{Alain L'Hostis}\index{L'Hostis, Alain|pagebf}, who, from the beginning of my doctoral journey, convinced me of the merits of this system and programming language. My thoughts go to my colleague and friend \textcolor{blue}{Iñigo Aguas Ardaiz}\index{Aguas Ardaiz, Iñigo|pagebf}, who, at the end of my research, generously gave me his precious time and provided invaluable help in resolving the code issues. Finally, I would like to thank Gustave Eiffel University for supporting my professional access to the \latexword{\LaTeX} collaboration environment \Marque{Overleaf}.%%Translated%%

    % GitHub Link
With the aim of open science and, consequently, the sharing of knowledge, reuse, and collaboration, we have made this manuscript, along with the customized template and code used, accessible through a \Marque{GitHub} repository\footnote{
    \Marque{GitHub} (\url{https://github.com/}) is an online platform for software development management and a hosting service for open access projects.
}.%%Translated%%

    \bigskip
    \begin{tcolorbox}[colback=white!5!white,
                      colframe=blue!75!blue,
                      title=
                      \bigskip
                      \center{\Marque{GitHub} Repository}
                      \bigskip]
\center{\normalsize{\url{https://github.com/dylan-moinse/PhD_Thesis_Dylan_MOINSE_English}}}
    \end{tcolorbox}