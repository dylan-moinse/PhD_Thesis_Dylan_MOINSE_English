%------------------------------%
%% ✎ Dylan (V1) %%%%%%%%% ✅ %%
%% ✎ Alain (V2) %%%%%%%%% ✅ %%
%% ✎ Dylan (V3) %%%%%%%%% ✅ %%
%------------------------------%

% Conclusion of Part II
\cleardoublepage
\section*{Conclusion of Part~II
    \label{part2:conclusion}
    }
    \addcontentsline{toc}{chapter}{Conclusion of Part~II}

    % Transition
\lettrine[lines=3, findent=8pt, nindent=0pt]{\lettrinefont D}{iverse} lessons emerge from this second part to rethink \acrshort{M-TOD}. The study of intermodal practices highlights that light individual mobility serves as a strategic lever for accessibility to stations, particularly for spatial-temporal distances ranging from one to five kilometers, or up to twenty minutes, where public transport is not competitive with cars. Its development thus strengthens the attractiveness of public transport by offering an efficient and flexible feeder and distribution solution, better suited to the realities of daily commuting. The spatial analysis reveals that the integration of light individual mobility redefines the functional boundaries of station districts by expanding the areas and thus the points of interest served by rail, optimizing the connection between transport networks. This evolution calls for a renewed understanding of \acrshort{TOD}, where the structuring of territories no longer relies solely on immediate proximity to railway infrastructures but incorporates a broader reflection on intermodal continuities. These findings reveal that the conditions for integrating light individual mobility remain uneven. They emphasize the need for a more systemic approach to intermodality. These lessons thus set the stage for the third and final part of this thesis, which focuses on formalizing the conceptualization of an \acrshort{M-TOD} in the form of proposed urban strategies. Based on the results from geostatistical modeling, the goal will be to propose an operational framework that organizes the proper deployment of light individual mobility integration in station districts and identifies action levers for the effective implementation of this expanded rail-based urbanism model.%%Translated%%
