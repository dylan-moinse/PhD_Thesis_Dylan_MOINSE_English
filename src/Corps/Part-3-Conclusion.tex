%------------------------------%
%% ✎ Dylan (V1) %%%%%%%%% ✅ %%
%% ✎ Alain (V2) %%%%%%%%% ✅ %%
%% ✎ Dylan (V3) %%%%%%%%% ✅ %%
%------------------------------%

% Conclusion of Part III
\cleardoublepage
\section*{Conclusion of Part~III
    \label{part3:conclusion}
    }
    \addcontentsline{toc}{chapter}{Conclusion of Part~III}

    % Transition
\lettrine[lines=3, findent=8pt, nindent=0pt]{\lettrinefont T}{hrough} this third and final part, we have proposed a formalization of \acrshort{M-TOD}, aiming to confront this adaptation of \acrshort{TOD} with the urban and social realities of the studied territory. By structuring these contributions around a reflection on infrastructure, mobility services, and the territorial configurations of station districts, this formalization helps to renew the interpretation of rail-based urbanism and the strategies for planning territories served by rail. This final part thus represents the culmination of the work done in this thesis, translating theoretical and empirical findings into a structured and operational proposal for intermodal rail-based urbanism. By fully integrating light individual mobility into the design of station districts and transit hubs, \acrshort{M-TOD} opens a new perspective on urban mobility planning. It goes beyond a segmented view of transportation modes to promote a high-performance intermodal system in alignment with the urban system. These reflections are part of a broader debate on the evolution of \acrshort{TOD}, within a context marked by ecological transition and the transformation of urban mobility. They also provide concrete insights for urban planners wishing to integrate light individual mobility into their planning and public transport strategies. The findings of this doctoral research thus affirm that \acrshort{M-TOD} offers a relevant solution to the challenges of intermodal accessibility and the structuring of station districts. These elements will be addressed in the general conclusion of this thesis, which will revisit the main contributions of this work, its scientific and technical implications, as well as the research and action perspectives to be considered in supporting the transition towards a rail-based urbanism fully integrated with these emerging mobility practices.%%Translated%%
