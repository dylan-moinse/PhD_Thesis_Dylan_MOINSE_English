%------------------------------%
%% ✎ Dylan (V1) %%%%%%%%% ✅ %%
%% ✎ Alain (V2) %%%%%%%%% ✅ %%
%% ✎ Dylan (V3) %%%%%%%%% ✅ %%
%------------------------------%

\cleardoublepage
\section*{Conclusion of Part~I
    \label{part1:conclusion}
    }
    \addcontentsline{toc}{chapter}{Conclusion of Part~I}

    % Transition
\lettrine[lines=3, findent=8pt, nindent=0pt]{\lettrinefont T}{he} exploration of the theoretical and methodological foundations conducted in this first part has enabled this thesis to be grounded in a reflection on the dynamics of mobility and territoriality associated with \acrshort{M-TOD} and the proximities generated by the rise of light individual mobility. This approach helped define the conceptual framework within which this research is situated, while identifying the methodological requirements crucial for investigating the interactions between public transport and light individual mobility in station districts, within an integrated regional system. By establishing a solid theoretical and methodological foundation, this part serves as a prerequisite for the subsequent empirical stage. Several key lessons emerge: \acrshort{TOD} must be updated to incorporate a deeper reflection on the local and intermodal proximities generated by light individual mobility and their impact on forms of accessibility; the integration of light individual mobility into \acrshort{TOD} remains incomplete, both in academic research and in public policies; empirical analysis should help fill these gaps by enhancing knowledge about effective intermodal practices and their effects on station districts. Thus, this first part lays the conceptual and methodological groundwork for the rest of the thesis. It highlights the limitations of \acrshort{TOD} and paves the way for a case study on the dynamics and the potential use of light individual mobility in station districts. The second part will further develop this investigation by utilizing the methodological tools defined here to assess the contribution of \acrshort{M-TOD} in terms of accessibility.%%Translated%%
